\chapter{Verification of the constitutive model}
\noindent
In this chapter, we will report on the performance of the proposed model and its capability to fit as well as to predict the real stress-strain response of the nickel-base superalloy Inconel 718.
The model has been validated for many more isothermal as well as thermomechanical fatigue tests than the ones presented in the following.

\section{Identification of the model parameters}
\noindent
The model parameters can be divided into the following categories: Elastic-plastic parameters, isotropic hardening parameters, kinematic hardening parameters and cyclic hardening/softening parameters. The parameters for non-proportional loading hardening is included in the isotropic hardening.
Table \ref{tab:ParametersCollection} lists the parameters of the constitutive model.
% According to the functions of these parameters, they can be divided into the following categories: general elastic-plastic parameters, isotropic hardening parameters, kinematic hardening parameters and cyclic hardening/softening parameters. The parameters to describe the hardening induced by non-proportional loading path is introduced in the isotropic hardening.

The material parameters of the constitutive model can be determined from the monotonic and cyclic experiments since several parameters are less interdependent on the others.
Generally Young's modulus $E$, Poisson's ratio $\nu$ and initial yield stress $\sigma_{\rm p=0.2}$ can be identified from the monotonic tension and torsion tests;
parameters $\zeta^k$ and $r_0^k$ are determined from monotonic tension test using the method provided by Jiang and Kurath (1996)\cite{Jiang1996387};
parameters $r^k$ are determined from the uniaxial stable hysteresis loop curve using the same method;
parameters $a_1^k$, $b_1^k$, $b_2^k$ are determined from the history curve of peak stress under uniaxial cyclic loading;
%parameters $Y_{\Delta pts}$, $t$, $m$ are determined from the phase transfer experiments;
parameters $Y_{\Delta \rm{sat}}$, $\gamma_{\rm{p}}$ and $\gamma_{\rm{q}}$ are fitted from the stabilized non-proportional cyclic stress-strain curve;
parameters $\mu$ and $c_c$ are suggested as the values proposed by Chaboche (1986)\cite{Chaboche1986149} and Tanaka (1994)\cite{tanaka1994nonproportionality}.

\begin{table}[htbp]
  \centering
  \caption{Material parameters of the proposed constitutive model.}
    \begin{tabular}{llll}
    \toprule
    General & Kinematic hardening & Isotropic hardening & Cyclic hardening \\
    \midrule
    $E$   & $\zeta^k$ & $c_c$ & $r_{\Delta \rm{s}}^k$ \\
    $\nu$ & $r_0^k$ & $Y_{\Delta \rm{sat}}$,$\gamma_{\rm{p}}$,$\gamma_{\rm{q}}$ & $a_1^k$,$b_1^k$,$b_2^k$ \\
    $\sigma_{\rm p=0.2}$ & $\mu$ &  & \\
    \bottomrule
    \end{tabular}%
  \label{tab:ParametersCollection}%
\end{table}%


\subsection{Determination of elastic constants}
\noindent
%In homogeneous and isotropic materials, these define Hooke's law in 3D can be written as:
%\begin{equation}
%\bm{\upsigma }=2\mu \bm{\upepsilon }+ \lambda \rm{Tr} (\bm{\upepsilon })\bf{I},
%\end{equation}
%$\lambda$ and $\mu$ are individually referred to as Lam\'{e}'s first parameter and Lam\'{e}'s second parameter, respectively.
As introduced above, for small deformations, below the proportional limit, most elastic materials such as metals exhibit linear elasticity and can be described by the theory of elasticity.
The modern theory of elasticity generalizes Hooke's law to say that the strain (deformation) of an elastic object or material is proportional to the stress applied to it.
The general proportionality relation between stress and strain in three dimensions can be expressed as:
\begin{equation}
{\bm{\upsigma }} = {\mathbb{D}^{\rm{e}}}:{{\bm{\upepsilon }}^{\rm{e}}},
\label{Equ:HookesLaw}
\end{equation}
where ${\mathbb{D}^{\rm{e}}}$ is a 4th-order tensor called elastic stiffness.

Because of the symmetry, Hooke's law for isotropic materials can then be expressed in terms of Young's modulus and Poisson's ratio as:
\begin{eqnarray}
   \boldsymbol{\upepsilon} = \frac{1}{E}(\boldsymbol{\sigma} - \nu[\mathrm{Tr}(\boldsymbol{\sigma})~\mathbf{I} - \boldsymbol{\sigma}]),
\label{Equ:HookesLawInEandNu}
\end{eqnarray}
where $E$ is the Young's modulus and $\nu$ is Poisson's ratio.
In matrix form, Equation \ref{Equ:HookesLawInEandNu} can be written as:
\begin{eqnarray}
   \begin{bmatrix}\varepsilon_{11} \\ \varepsilon_{22} \\ \varepsilon_{33} \\ 2\varepsilon_{23} \\ 2\varepsilon_{31} \\ 2\varepsilon_{12} \end{bmatrix} =
   \begin{bmatrix}\varepsilon_{11} \\ \varepsilon_{22} \\ \varepsilon_{33} \\ \gamma_{23} \\ \gamma_{31} \\ \gamma_{12} \end{bmatrix} =
   \cfrac{1}{E}
   \begin{bmatrix} 1 & -\nu & -\nu & 0 & 0 & 0 \\
                   -\nu & 1 & -\nu & 0 & 0 & 0 \\
                   -\nu & -\nu & 1 & 0 & 0 & 0 \\
                   0 & 0 & 0 & 2(1+\nu) & 0 & 0 \\
                   0 & 0 & 0 & 0 & 2(1+\nu) & 0 \\
                   0 & 0 & 0 & 0 & 0 & 2(1+\nu) \end{bmatrix}
    \begin{bmatrix}\sigma_{11} \\ \sigma_{22} \\ \sigma_{33} \\ \sigma_{23} \\ \sigma_{31} \\ \sigma_{12} \end{bmatrix},
 \end{eqnarray}
where $\gamma_{ij} = 2\varepsilon_{ij}$ is the engineering shear strain.
The two parameters $E$ and $\nu$ together constitute a parameterization of the elastic moduli for homogeneous isotropic material, and are thus related to the other elastic moduli; for instance, the shear modulus can be expressed as $G={E}/{2(1+\nu)}$, bulk modulus can be expressed as $K = {E}/{3(1-2\nu)}$.

The inverse relation can be simplified as:
\begin{eqnarray}
   \begin{bmatrix}\sigma_{11} \\ \sigma_{22} \\ \sigma_{33} \\ \sigma_{23} \\ \sigma_{31} \\ \sigma_{12} \end{bmatrix}
   =
   \begin{bmatrix} 2\mu+\lambda & \lambda & \lambda & 0 & 0 & 0 \\
                   \lambda & 2\mu+\lambda & \lambda & 0 & 0 & 0 \\
                   \lambda & \lambda & 2\mu+\lambda & 0 & 0 & 0 \\
                   0 & 0 & 0 & \mu & 0 & 0 \\
                   0 & 0 & 0 & 0 & \mu & 0 \\
                   0 & 0 & 0 & 0 & 0 & \mu \end{bmatrix}
    \begin{bmatrix}\varepsilon_{11} \\ \varepsilon_{22} \\ \varepsilon_{33} \\ 2\varepsilon_{23} \\ 2\varepsilon_{31} \\ 2\varepsilon_{12} \end{bmatrix},
\end{eqnarray}
where $\lambda$ and $\mu$ are individually referred to as Lam\'{e}'s constants.

Once Young's modulus and Poisson's ratio are obtained, the elastic stiffness tensor is positively determined.
Therefore, it is necessary to obtain these two elastic constants of Inconel 718 at different temperatures.

Young's modulus can be obtained by uniaxial tests, and the experiential procedure is as followed.
The specimen was heated to a specific temperature and held an hour to ensure that the temperature is uniform in the gauge length.
For each temperature, the specimen was loaded and unloaded below one-third of the yield stress.
The stress-strain curves at different temperatures are shown in \ref{Fig:E_Test_Show}, where the horizontal axial is the total logarithmic strain and the vertical axial is the true stress.
Young's modulus can be obtained by computing the slope of the stress-strain curves.

Shear modulus can be obtained by pure torsional tests, and the solid specimen was used.
The experiential procedure was similar to the test procedure for measuring Young's modulus at different temperatures.
Under elastic loading conditions, shear stress varies linearly through the section of the specimen.
The maximum shear stress occurs at the outer diameter of the specimen can be obtained as:
% \begin{equation}
% \tau=\frac{{16T}}{{\pi d_{out}^3\left( {1 - {{\left( {{d_{in}}/{d_{out}}} \right)}^4}} \right)}},
% \end{equation}
\begin{equation}
\tau=\frac{{16T}}{\pi d^3},
\end{equation}
where $d$ is the diameter of the specimen, $T$ is the torsional moment.
The torsional moment was measured by the torsional load cell.
The shear strain $\gamma$ was measured by the high-temperature extensometer.
Then the shear modulus can be obtained as:
\begin{equation}
G=\frac{\tau}{\gamma}
\end{equation}

Poisson's ratio can be expressed by:
\begin{equation}
\nu  = \frac{E}{{2G}} - 1
\end{equation}

For experiments which are conducted at different temperatures, Young's modulus, shear modulus and Poisson's ratio are obtained, and the results are listed in Table \ref{Tab:EandG}.
% As shown in \ref{Fig:E_G_PoissonsRatio}, they are plot versus temperature.
% It is obvious that with the increase of temperature, the Young's modulus and shear modulus show a certain decline.
\ref{Fig:plot_elastic_by_temperature_in718} shows variations of elasticity modulus, shear modulus and Poisson's ratio as functions of temperature. The curves are independent of mechanical loading. The Young's modulus and shear modulus are observed to decrease with the increasing temperature. However, the Poisson's ratio does not vary monotonically with temperature.

Cubic polynomials are proposed to fit the Young's modulus and Poisson's ratio (see the solid and dash line in the \ref{Fig:plot_elastic_by_temperature_in718}).
The advantage of the cubic polynomial is the good precision and the efficiency in programming.
In the temperature range 20$^{\circ}$C to 700$^{\circ}$C, the values of Young's modulus $E$ and Poisson's ratio $\nu$ of Inconel 718 can be calculated as:
\begin{equation}
E=206308.7+(-51.20)T+(0.0111)T^2+(-3.84\times10^{-5})T^3,
\label{Equ:polynomial_of_E}
\end{equation}
\begin{equation}
\nu=0.29+(1.46\times10^{-5})T+(-2.07\times10^{-7})T^2+(-2.78\times10^{-10})T^3,
\label{Equ:polynomial_of_nu}
\end{equation}
noting that the calculated Young's modulus $E$ is in the unit of MPa and Poisson's ratio $\nu$ is dimensionless.

\begin{figure}[!htp]
\centering
\includegraphics[width=14cm]{E_Test_Show.pdf}
\caption{Experimental results to determine Young's modulus $E$ at different temperatures.}
\label{Fig:E_Test_Show}
\end{figure}

\begin{table}[htbp]
  \centering
  \caption{Young's modulus $E$, shear modulus $G$ and Poisson's ratio of Inconel 718 at different temperatures.}
    \begin{tabular}{p{3cm}<{\centering}p{3cm}<{\centering}p{3cm}<{\centering}p{3cm}<{\centering}}
    \toprule
    Temperature & Young's modulus & Shear modulus & Poisson's ratio \\
    $T$ [$^{\circ}$C] & $E$ [GPa] & $G$ [GPa] & $\mu$ [-] \\
    \midrule
    16    & 208.2  & 80.8  & 0.289 \\
    100   & 205.7  & 79.7  & 0.290 \\
    200   & 198.6  & 77.1  & 0.288 \\
    300   & 193.3  & 75.3  & 0.283 \\
    400   & 185.0  & 72.3  & 0.280 \\
    500   & 177.7  & 69.4  & 0.279 \\
    600   & 170.8  & 66.5  & 0.285 \\
    700   & 157.4  & 60.8  & 0.295 \\
    \bottomrule
    \end{tabular}%
  \label{Tab:EandG}%
\end{table}%

\begin{figure}[!htp]
  \centering
  \includegraphics[width=14cm]{plot_elastic_by_temperature_in718.pdf}
  \caption{Young's modulus $E$, shear modulus $G$ and Poisson's Ratio at different temperatures.}
  \label{Fig:plot_elastic_by_temperature_in718}
\end{figure}

\subsection{Determination of $\zeta^k$ and $r_0^k$ from the monotonic stress-strain curves}
\noindent
The cyclic softening and multiaxial effect are absent in the monotonic tensile test.
By neglecting the plastic strain amplitude effect, as introduced by Jiang and Kurath\cite{Jiang1996387} and Jiang and Sehitoglu\cite{jiang1996modeling}, material parameters ${\zeta ^k}( k = 1,2,3,...,M )$ and $r_0^k( k = 1,2,3,...,M )$ can be determined using the following equations, if no isotropic hardening is assumed:
\begin{equation}
{\zeta ^k} = \frac{1}{{\varepsilon _{\rm{p}}^k}},
\label{Eqn:zetak}
\end{equation}
\begin{equation}
r_0^k = \left( {\frac{{\sigma _{}^k - \sigma _{}^{k - 1}}}{{\varepsilon _{\rm{p}}^k - \varepsilon _{\rm{p}}^{k - 1}}} - \frac{{\sigma _{}^{k + 1} - \sigma _{}^k}}{{\varepsilon _{\rm{p}}^{k + 1} - \varepsilon _{\rm{p}}^k}}} \right)\varepsilon _{\rm{p}}^k,
\end{equation}
% where $\sigma^k( k = 0,1,2,...,M+1 )$ and $\varepsilon_p^k( k = 0,1,2,...,M+1 )$ denote the stress and plastic strain at the $k$-th point on the monotonic tensile stress vs. plastic strain curve, $\sigma^0$ is the initial yield stress related to the zero plastic strain $\varepsilon_p^0=0$.
% For example, we select M points from the uniaxial tensile stress strain curve at 650$^\circ$C (see \ref{Fig:Determine_the_basic_material_constants}), and M=9 is found to represent the curve well.
where $\sigma^k$ and $\varepsilon_p^k( k = 0,1,2,...,M+1 )$ denote the stress and plastic strain at the $k$-th point on the monotonic  stress-plastic strain curve, respectively. $\sigma^0$ is the initial yield stress related to the zero plastic strain $\varepsilon_p^0=0$. Generally, the uniaxial tensile stress-strain curve is represented in $M$ intervals, as shown in \ref{Fig:Determine_the_basic_material_constants}. In the present work, $M=9$ is proposed.

\begin{figure}[!htp]
  \centering
  \includegraphics[width=14cm]{Determine_the_basic_material_constants.pdf}
  \caption{The material constants $\zeta^k$ and $r_0^k$ are determined by monotonic tensile stress-strain curves of Inconel 718 at 650$^\circ$C.}
  \label{Fig:Determine_the_basic_material_constants}
\end{figure}

\subsection{Determination of $r_{\Delta \rm{s}}^k$ from the cyclic stress-strain curves}
\noindent
In the constitutive model, the saturated value of $r^k$ has to be determined from the cyclic stress-strain curve, which can be obtained from the experimental stabilized hysteresis loop.
The half fatigue life ($N_f/2$) cycle is assumed to be the stabilized hysteresis loop.
% In the constitutive model, $r^k$ is the saturated value.
% To determine the values of $r^k$ cyclic stress-strain curve is needed.
% The cyclic stress-strain curve can be obtained from the experimental stabilized hysteresis loop.
%The cyclic fatigue curve used as input for the determination of $r^k$ and $r_{\Delta}^k$ is the alternating values of stress and strain.
% The half-life ($N_f/2$) cycle is suggested as the stabilized hysteresis loop.
The data in the stabilized hysteresis loop can be divided into a stress increasing half cycle and a stress decreasing half cycle.
The alternating values of stress and strain were calculated by taking half of the absolute value between a data point and the starting point.
The alternating stress and strain relation of the increasing stress part can be expressed in the function $f_{\rm{inc}}(\sigma,\varepsilon)$ as:
\begin{equation}
f_{\rm{inc}}(\sigma,\varepsilon)=|2\sigma-\sigma_{\min}|-|2\varepsilon-\varepsilon_{\min}|=0,
\label{Equ:f_inc}
\end{equation}
and function $f_{\rm{dec}}(\sigma,\varepsilon)$ is the alternating stress and strain relation of the decreasing stress part:
\begin{equation}
f_{\rm{dec}}(\sigma,\varepsilon)=|2\sigma-\sigma_{\max}|-|2\varepsilon-\varepsilon_{\max}|=0,
\label{Equ:f_dec}
\end{equation}
where $\sigma_{\min}$ and $\sigma_{\max}$ are minimum value for increasing stress part and maximum value for decreasing stress part.

By combining Equations (\ref{Equ:f_inc}) and (\ref{Equ:f_dec}) together we obtain the final alternating stress and strain curve, as illustrated in \ref{Fig:Alternating_Curve}:
\begin{equation}
f(\sigma,\varepsilon)=\frac{1}{2}(f_{\rm{inc}}\left(\sigma,\varepsilon)+f_{\rm{dec}}(\sigma,\varepsilon)\right).
\end{equation}
\begin{figure}[!htp]
\centering
  \includegraphics[width=14cm]{plot_alternating_curve.pdf}
  \caption{Calculation of the alternating stress-strain curve from the hysteresis loop at 650$^\circ$C.}
  \label{Fig:Alternating_Curve}
\end{figure}
% By combining Equation \ref{Equ:rk1} and \ref{Equ:rdeltak1}, $r^k$ can be expressed as:
% \begin{equation}
% {r^k} = r_0^k + r_{\Delta \rm{s}}^k\left[ {1 - a_1^k{e^{ - b_1^kp}} - (1-a_1^k){e^{ - b_2^kp}} }\right],
% \end{equation}
% where $r_{\Delta \rm{s}}^k$, $a_1^k$, $b_1^k$ and $b_2^k$ are material constants and $p$ is the equivalent plastic strain.
% The saturated value of $r^k$ is denoted as $r^k_{\rm{s}}$ and the saturated increment value of $r^k$ is denoted as $r^k_{\Delta \rm{s}}$.
% A simple procedure for determining the material constants $r_{\Delta \rm{s}}^k$ was by assuming that $p$ is infinitely large.
% Then we have:
% \begin{equation}
% r_{\rm{s}}^k=\mathop {\lim }\limits_{{\rm{p}} \to \infty } {r^k} = r_0^k + r_{\Delta \rm{s}}^k.
% \end{equation}
% Generally, when the cyclic hardening/softening is stabilized, the equivalent plastic strain is large.
% Consequently, the data in the stabilized hysteresis loop can be used to determine the saturated values $r^k_{\rm{s}}$ and $r^k_{\Delta \rm{s}}$.
\begin{figure}[!htp]
\centering
\includegraphics[width=14cm]{Determine_rsk.pdf}
\caption{The material constants $r_{\rm{s}}^k$ and $r_{\Delta \rm{s}}^k$ are determined by monotonic tensile and calculated alternating stress-strain curves of Inconel 718 at 650$^\circ$C.}
\label{Fig:Determine_rsk}
\end{figure}
% From the calculated alternating stress-strain curve the same procedure for determining the material constants $\zeta^k$ and $r_0^k$ are used to determine the saturated values $r^k_{\rm{s}}$ and $r^k_{\Delta \rm{s}}$.
% The calculated alternating stress-strain curve was also divided into M=9 segments with the same plastic strain (see \ref{Fig:Determine_rsk}).
% Because we use the same plastic strain to divide the alternating stress-strain curve, by using the Equation \ref{Eqn:zetak}, the parameters ${\zeta ^k}( k = 1,2,3,...,M )$ are the same as the parameters determined in the last section.

By combining Equations (\ref{Equ:rk1}) and (\ref{Equ:rdeltak}), $r^k$ can be expressed as
\begin{equation}\label{rk}
{r^k} = r_0^k + r_{\Delta \rm{s}}^k\left[ {1 - a_1^k{e^{ - b_1^kp}} - (1-a_1^k){e^{ - b_2^kp}} }\right],
\end{equation}
where $r_{\Delta \rm{s}}^k$, $a_1^k$, $b_1^k$ and $b_2^k$ are constants and $p$ is the accumulated plastic strain.
The saturated value of $r^k$ is denoted as $r^k_{\rm{s}}$ and the saturated increment value of $r^k$ is denoted as $r^k_{\Delta \rm{s}}$.
A simple procedure for determining the material constants $r_{\Delta \rm{s}}^k$ was by assuming that $p$ is infinitely large,
\begin{equation}
r_{\rm{s}}^k=\mathop {\lim }\limits_{{\rm{p}} \to \infty } {r^k} = r_0^k + r_{\Delta \rm{s}}^k.
\label{Equ:rsk}
\end{equation}
Generally, when the cyclic hardening/softening is stabilized, the accumulated plastic strain is large.
Consequently, the data in the stabilized hysteresis loop can be used to determine the saturated values $r^k_{\rm{s}}$ and $r^k_{\Delta \rm{s}}$.

The calculated alternating stress-strain curve was divided into $M=9$ segments with the same plastic strain for both monotonic and cyclic stress-strain curve.  $\zeta^k$ and $r_0^k$ can be determined from monotonic tension test as suggested by Jiang and Kurath \cite{Jiang1996387}. From the calculated alternating stress-strain curve the same procedure for $\zeta^k$ and $r_0^k$ is used to determine the saturated values $r^k_{\rm{s}}$ and $r^k_{\Delta \rm{s}}$ as:
\begin{equation}
r_{\rm{s}}^k = \left( {\frac{{\sigma _{\rm{alt}}^k - \sigma _{\rm{alt}}^{k - 1}}}{{\varepsilon _{\rm{p}}^k - \varepsilon _{\rm{p}}^{k - 1}}} - \frac{{\sigma _{\rm{alt}}^{k + 1} - \sigma _{\rm{alt}}^k}}{{\varepsilon _{\rm{p}}^{k + 1} - \varepsilon _{\rm{p}}^k}}} \right)\varepsilon _{\rm{p}}^k,
\end{equation}
where $\sigma _{\rm{alt}}^k$ is the $k$-th stress at the calculated alternating stress-strain curve, and,
\begin{equation}
r_{\Delta \rm{s}}^k = r_{\rm{s}}^k - r_{0}^k.
\end{equation}
$r^k$ is determined from the uniaxial stabilized hysteresis loop curve following Equation (\ref{rk}). ${\zeta ^k}$ for the cyclic stress-strain relation is the same as that for the monotonic case.

\subsection{Determination of $a_{1}^k$, $b_{1}^k$, $a_{2}^k$ from the cyclic peak stress curves}
\noindent
% As discussed in the last section, we have two different stress-strain curves, one is the monotonic tension curve and the other is the stable alternating stress-strain curve.
% In the constitutive model, $r_{\Delta \rm{s}}^k( k = 1,2,3,...,M )$ is the parameter which describes the deviation magnitude of the two curves and the parameters $a_{1}^k$, $b_{1}^k$ and $a_{2}^k$ are used to describe the softening rate of the material.
% We assume that all of the cyclic softening is kinematic under uniaxial loading conditions.
% A simplification is that from $k=1$ to $M$, $a_{1}^k$, $b_{1}^k$ and $a_{2}^k$ have the same values which means $a_{1}=a_{1}^k( k = 1,2,3,...,M )$, $b_{a}=b_{1}^k( k = 1,2,3,...,M )$ and $b_{2}=b_{2}^k( k = 1,2,3,...,M )$.
% Then the parameters can be determined from the cyclic stress-strain curve, with the relation:
% \begin{equation}
% 1 - \frac{{\sigma  - {\sigma _0}}}{{{\sigma _0} - {\sigma _{\min }}}} = 1 - a_1{e^{ - b_1 p}} - \left( {1 - a_1} \right){e^{ - b_2 p}}
% \end{equation}
As discussed above, both monotonic tension curve and the stabilized alternating stress-strain curve are used for identifying the model parameters.
In the constitutive model, $r_{\Delta \rm{s}}^k$ is used to describe the deviation magnitude of the two curves, and $a_{1}^k$, $b_{1}^k$ and $b_{2}^k$ are used to describe the softening rate of the material.
$a_{1}^k$, $b_{1}^k$ and $b_{2}^k$ are determined from the history curve of the peak stress under uniaxial cyclic loading. For simplicity, the cyclic softening is assumed to be purely kinematic under uniaxial loading conditions, as introduced by Ohno et al. \cite{Ohno1993375}.
For the nickel based superalloy Inconel 718, $a_{1}^k$, $b_{1}^k$ and $b_{2}^k$ are independent of $k$, i.e. all segments share the same values, $a_1, b_2$ and $b_2$.
Thus, the parameters can be determined from the cyclic stress-strain curve,
\begin{equation}
1 - \frac{{\sigma  - {\sigma _0}}}{{{\sigma _0} - {\sigma _{\min }}}} = 1 - a_1{e^{ - b_1 p}} - \left( {1 - a_1} \right){e^{ - b_2 p}}.
\end{equation}
The relation provides optimal agreement between experimental curve and model, as shown in \ref{Fig:plot_determination_of_rk}. Fang \cite{fang2015cyclic} proposed a simplified model as
\begin{equation}
r_\Delta ^k = r_{\Delta \rm{s}}^k\left( 1 - a_1^k{e^{ - b_1^kp}}\right).
\label{Equ:fangrdeltak}
\end{equation}
Results in \ref{Fig:plot_determination_of_rk} reveal significant deviations between the two different expressions above at small plastic strain region.

\begin{figure}
  \centering
  \begin{overpic}[width=8.0cm]{plot_determination_of_rk_1.pdf}
    \put(84,65){\fcolorbox{white}{white}{(a)}}
  \end{overpic}
  \begin{overpic}[width=8.0cm]{plot_determination_of_rk_2.pdf}
    \put(84,65){\fcolorbox{white}{white}{(b)}}
  \end{overpic}
  \caption{Determination of $r_k$. (a) Axial peak stress versus accumulated plastic strain. (b) Kinematic hardening from the full reversed strain cycling tests with comparison to different hardening models of $r_k$.}
  \label{Fig:plot_determination_of_rk}
\end{figure}


\subsection{Model parameters of Inconel 718}
\noindent
% In this section, the model parameters are determined under isothermal conditions.
% Similar to most constitutive models, we need series of parameters in describing the mechanical behavior of the material under different temperatures.
% According to the isothermal uniaxial monotonic tension tests and the cyclic stress-strain hysteresis curve, we can obtain the material constants of Inconel 718 at 300$^\circ$C, 550$^\circ$C and 650$^\circ$C as shown in Table \ref{tab:General_material_mechanical_properties} and Table \ref{tab:3}, where $k=9$ is prescribed in order to simulate the cyclic hard/softening more accurately.
In this section, the model parameters are determined under isothermal conditions.
Temperature dependency of material parameters should be considered when anisothermal cyclic tests are calculated.
According to isothermal uniaxial monotonic tension tests and cyclic stress-strain hysteresis curves at different temperatures, the material constants at 300$^\circ$C, 550$^\circ$C and 650$^\circ$C are identified and summarized and Table \ref{Tab:ModelParameters} with $M=9$. 
In the present constitutive model, the parameters $\mu$, $\gamma_{\rm{p}}$, $\gamma_{\rm{q}}$, $Y_{\Delta {\rm{sat}}}$ and $c_c$ are assumed to be temperature independent, while the parameters $E$, $\nu$, $r_0^k$, $r_{\Delta \rm{s}}^k$, $a_1^k$, $b_1^k$ and $b_2^k$ are considered as the temperature dependent parameters.
As introduced above, the Young's modulus $E$ and Poisson's ratio $\nu$ of Inconel 718 can be calculated by the cubic polynomials (see Equation (\ref{Equ:polynomial_of_E}) and (\ref{Equ:polynomial_of_nu})).

The temperature dependent parameters $r_0^k$, $r_{\Delta \rm{s}}^k$, $a_1^k$, $b_1^k$ and $b_2^k$ in the present constitutive model are plotted in \ref{Fig:plot_umat_parameters}. In \ref{Fig:plot_umat_parameters}(a) and (b), the parameters $r_0^k$ and $r_{\Delta \rm{s}}^k$ show a monotonic decrease with the temperature increase. And as shown in \ref{Fig:plot_umat_parameters}(c) and (d), $a_1^k$ increases with the increasing temperature while $b_1^k$ decreases with the increasing temperature and the variation of $b_2^k$ with temperature are not monotonous.

Becker and Hackenberg \cite{Becker2011596} proposed that the temperature dependent parameters $p_T$ can be interpolated between $T_1$ and $T_2$ according to
\[p_T = {10^{{{\log }_{10}}\left[ {{p_{{T_1}}}} \right] + \frac{{{{\log }_{10}}\left[ {{p_{{T_1}}}/{p_{{T_2}}}} \right]}}{{{{\log }_{10}}\left[ {{T_2}/{T_1}} \right]}}{{\log }_{10}}\left[ {T/{T_1}} \right]}},\]
where ${{p_{{T_1}}}}$ is the parameter value at $T_1$ and ${{p_{{T_2}}}}$ is the parameter value at $T_2$.

However, in the present constitutive model, the linear relationship can be observed between the most parameters and temperature, as shown in \ref{Fig:plot_umat_parameters}. Therefore, the temperature dependence of the parameters is suggested to be linearly interpolated in intervals. For an intermediate temperature $T$, the temperature dependent parameters $p_T$ are given by
\[{p_T} = {p_{{T_1}}} + \frac{{{p_{{T_2}}} - {p_{{T_1}}}}}{{{T_2} - {T_1}}}{T}.\]

In order to verify whether the model parameters calculated by the linear interpolation method are appropriate, a tension-compression isothermal fatigue test was carried out at 400$^\circ$C with the strain range of $\pm1.0\%$.
\ref{Fig:plot_cmp_TCIF400} shows the computational stress-strain response and peak-valley stress curve at the intermediate temperature $T=400^\circ$C by using the linearly interpolated temperature dependent parameters. The computational result shows a good agreement with the experimental result. Consequently, the linear interpolation method is used in the following computations.

% All material parameters are determined under isothermal condition.
% To determine the material parameters for Inconel 718 test data has been available at 3 temperatures ranging from 300$^\circ$C to 650$^\circ$C.
% The temperature dependent parameters $E$, $\nu$, $r_0^k$, $r_{\Delta \rm{s}}^k$ are interpolated according to the linear relationship.
% Alternatively, we assume the other material parameters are temperature independent.

% \begin{table}[htbp]
%   \centering
%   \caption{Material properties of the proposed constitutive model.}
%     \begin{tabular}{rr|rr|rr|rr}
%     \toprule
%           &       & \multicolumn{2}{c}{300$^\circ$C} & \multicolumn{2}{c}{550$^\circ$C} & \multicolumn{2}{c}{650$^\circ$C} \\
%     \midrule
%     $k$   & $\zeta^k$ & $r_0^k$ & $r_{\Delta \rm{s}}^k$ & $r_0^k$ & $r_{\Delta \rm{s}}^k$ & $r_0^k$ & $r_{\Delta \rm{s}}^k$ \\
%     1     & 20000 & 292.65  & -37.83  & 242.66  & -100.14  & 227.08  & -139.00  \\
%     2     & 10000 & 60.11  & -10.89  & 57.06  & -13.56  & 54.94  & -15.16  \\
%     3     & 5000  & 74.70  & -14.30  & 71.17  & -17.29  & 68.47  & -18.98  \\
%     4     & 2000  & 89.45  & -18.06  & 85.54  & -21.23  & 82.22  & -22.89  \\
%     5     & 1000  & 82.86  & -17.20  & 79.54  & -20.15  & 76.38  & -21.36  \\
%     6     & 500   & 91.27  & -20.20  & 87.90  & -22.68  & 84.34  & -23.68  \\
%     7     & 250   & 113.42  & -26.21  & 109.64  & -28.84  & 105.11  & -29.65  \\
%     8     & 100   & 135.81  & -32.72  & 131.78  & -35.34  & 126.21  & -35.76  \\
%     9     & 50    & 140.01  & -32.38  & 136.93  & -33.02  & 130.89  & -32.75  \\
%     \midrule
%     \multicolumn{8}{l}{$\mu=0.2$,$a_1^k$=1,$b_1^k$=10,$a_2^k$=0,$b_2^k$=0,$b_3^k$=0} \\
%     \multicolumn{8}{l}{$\gamma_{\rm{p}}$=10,$\gamma_{\rm{q}}$=50,$Y_{\Delta nonps}$=100,$c_c$=50} \\
%     \bottomrule
%     \end{tabular}%
%   \label{tab:3}%
% \end{table}%

\begin{table*}[htbp]
  \centering
  \caption{Model parameters (stress unit: MPa, strain unit: mm/mm).}
    \begin{tabular}{rrrrrrrrrrr}
    \hline
          & $k$   & 1     & 2     & 3     & 4     & 5     & 6     & 7     & 8     & 9 \\
          & $\zeta^k$ & 20000 & 10000 & 5000  & 2000  & 1000  & 500   & 250   & 100   & 50 \\
    \hline
    300$^\circ$C & $r_0^k$ & 292.65  & 60.11  & 74.70  & 89.45  & 82.86  & 91.27  & 113.42  & 135.81  & 140.01  \\
          & $r_{\Delta \rm{s}}^k$ & -37.83  & -10.89  & -14.30  & -18.06  & -17.20  & -20.20  & -26.21  & -32.72  & -32.38  \\
          & & \multicolumn{9}{l}{$a_1^k$=0.4, $b_1^k$=1.27, $b_2^k$=9.45, $k$=1 to 9} \\
    \hline
    550$^\circ$C & $r_0^k$ & 242.66  & 57.06  & 71.17  & 85.54  & 79.54  & 87.90  & 109.64  & 131.78  & 136.93  \\
          & $r_{\Delta \rm{s}}^k$ & -100.14  & -13.56  & -17.29  & -21.23  & -20.15  & -22.68  & -28.84  & -35.34  & -33.02  \\
          & & \multicolumn{9}{l}{$a_1^k$=0.41, $b_1^k$=0.87, $b_2^k$=11.8, $k$=1 to 9} \\
    \hline
    650$^\circ$C & $r_0^k$ & 227.08  & 54.94  & 68.47  & 82.22  & 76.38  & 84.34  & 105.11  & 126.21  & 130.89  \\
          & $r_{\Delta \rm{s}}^k$ & -139.00  & -15.16  & -18.98  & -22.89  & -21.36  & -23.68  & -29.65  & -35.76  & -32.75  \\
          & & \multicolumn{9}{l}{$a_1^k$=0.42, $b_1^k$=0.49, $b_2^k$=9.18, $k$=1 to 9} \\
    \hline
          & & \multicolumn{9}{l}{$\mu=0.2$,$\gamma_{\rm{p}}$=10,$\gamma_{\rm{q}}$=50,$Y_{\Delta {\rm{sat}}}$=200,$c_c$=50} \\
    \hline
    \end{tabular}%
  \label{Tab:ModelParameters}%
\end{table*}%

\begin{figure}
  \centering
  \begin{overpic}[width=8.0cm]{plot_umat_parameters_r0k.pdf}
    \put(20,14){(a)}
  \end{overpic}
  \begin{overpic}[width=8.0cm]{plot_umat_parameters_rdeltak.pdf}
    \put(20,14){(b)}
  \end{overpic}
  \begin{overpic}[width=8.0cm]{plot_umat_parameters_a1.pdf}
    \put(20,14){(c)}
  \end{overpic}
  \begin{overpic}[width=8.0cm]{plot_umat_parameters_b1b2.pdf}
    \put(20,14){(d)}
  \end{overpic}
  \caption{Model parameters vs. temperature: (a) $r_0^k$, (b) $r_{\Delta \rm{s}}^k$, (c) $a_1^k$, (d) $b_1^k$ and $b_2^k$.}
  \label{Fig:plot_umat_parameters}
\end{figure}

\begin{figure}
  \centering
  \begin{overpic}[width=8.0cm]{plot_cmp_half_life_cycle_TCIF400.pdf}
    \put(84,14){(a)}
  \end{overpic}
  \begin{overpic}[width=8.0cm]{plot_cmp_pv_TCIF400.pdf}
    \put(84,14){(b)}
  \end{overpic}
  \caption{Comparison of the experimental and computational stress-strain response at 400$^\circ$C by using the interpolated temperature dependent parameters. (a) The stress-strain hysteresis loop at the $N_{\rm{f}}/2$ cycle. (b) Peak and valley stress values.}
  \label{Fig:plot_cmp_TCIF400}
\end{figure}



\section{Application to Inconel 718 numerical calculations}

\subsection{ABAQUS user interface UMAT}
\noindent
ABAQUS user subroutines are usually used to extend the ABAQUS FEM solver to include user-defined functionalities not implemented in the ABAQUS source code.
The UMAT (User MATerial) interface provided by the ABAQUS solver for user-defined material laws and plasticity models is described in detail in \cite{abaqus20106}. In the following, we briefly discuss important variables and functions.

A UMAT routine is called at least once per time step during a FEM calculation for each integration point.
Inputs of the UMAT routine are the stress-strain state at the current time, all state variables and constants of the plasticity model, and the strain increment to be applied in the current time step.
Within the UMAT routine, the constitutive equations of the plasticity model are solved in the form of a differential equation system using a corresponding numerical solution method.
Subsequently, the new stress-strain state and the updated state variables are returned to the ABAQUS solver.
If the expansion increment is too large, and a solution of the constitutive equations with the necessary accuracy is not possible, the variable PNEWDT is returned to a new smaller step.
A UMAT routine thus operates on an incremental basis.
When it is viewed as a "black box", the corresponding conversion increment is simply expressed for each expansion increment entered into the UMAT routine:

\begin{equation}
\Delta \sigma_{ij}=UMAT(\Delta \varepsilon_{ij}).
\end{equation}

Since the present UMAT routines are programmed in FORTRAN and are not based on internal functions of the ABAQUS solver, it is possible to use these UMAT routines and thus also the included plasticity models "stand-alone" without the ABAQUS solver in a separate FORTRAN based environment.
It is only necessary to ensure that the plasticity models receive the necessary input variables via the UMAT interface and the existing status variables and output variables are buffered accordingly.
%This possibility is used here for multiaxial non-proportional notch strain simulation.
Significant variables of the ABAQUS-UMAT interface used are listed in Table \ref{tab:ABAQUS-UMAT_interface}.

\begin{figure}[!htp]
  \centering
  \includegraphics[width=16cm]{abaqus_umat_subroutine.pdf}
  \caption{A flow diagram outlining the relation between the ABAQUS standard solver and UMAT subroutine.}
  \label{Fig:ABAQUS_UMAT_SUBROUTINE}
\end{figure}

\begin{table}[htbp]
  \centering
  \caption{Used variables of the ABAQUS-UMAT interface}
    \begin{tabular}{p{2cm}p{2cm}p{2cm}p{8cm}}
    \toprule
    Variable & Type  & Dimension & Description \\
    \midrule
    NDI   & INTEGER & scalar & Number of direct stress components \\
    NSHR  & INTEGER & scalar & Number of engineering shear stress components \\
    NTENS & INTEGER & scalar & Size of the stress or strain component array (NDI+NSHR) \\
    NSTATV & INTEGER & scalar & Number of solution-dependent state variables \\
    NPROPS & INTEGER & scalar & User-defined number of material constants associated with this user material \\
    DTIME & DOUBLE & scalar & Time increment $\Delta t$ \\
    TEMP  & DOUBLE & scalar & Temperature at the start of the increment $T_n$ \\
    DTEMP & DOUBLE & scalar & Increment of temperature $\Delta T$ \\
    PNEWDT & DOUBLE & scalar & Ratio of suggested new time increment to the time increment being used \\
    STRAN & DOUBLE & (NTENS) & Total mechanical strains at the beginning of the increment $\bm{\upepsilon}_n$ \\
    DSTRAN & DOUBLE & (NTENS) & Mechanical strain increments $\Delta \bm{\upepsilon}$ \\
    DDSDDE & DOUBLE & (NTENS, NTENS) & Jacobian matrix of the constitutive model $\partial\Delta {\bm{\upsigma }} /\partial\Delta {\bm{\upepsilon }}$ \\
    STRESS & DOUBLE & (NTENS) & Stress tensor at the end of the increment $\bm{\upsigma}_{n+1}$ \\
    STATEV & DOUBLE & (NSTATV) & Internal state variables of the model \\
    PROPS & DOUBLE & (NPROPS) & Material constants and model parameters \\
    DFGRD0 & DOUBLE & (3,3) & Deformation gradient at the beginning of the increment $\bm{F}_n$ \\
    DFGRD1 & DOUBLE & (3,3) & Deformation gradient at the end of the increment $\bm{F}_{n+1}$ \\
    \bottomrule
    \end{tabular}%
  \label{tab:ABAQUS-UMAT_interface}%
\end{table}%

\subsection{FE-model}
\noindent
In this section, we create the uniaxial tension calculation model and the axisymmetric tension-torsion calculation model in ABAQUS.
ABAQUS/Standard is a general-purpose finite element program, and the user-defined material model is implemented in user subroutine UMAT.
As introduced above, the user subroutine UMAT which implemented by the implicit integral algorithm is not only can be used in 1D elements but also can be used in 2D or 3D elements.
As shown in \ref{Fig:Compuation_model}, simplifications are introduced.
%that one 3D element is used in the uniaxial model and 25 2D elements are used in the axisymmetric model.
The uniaxial calculation can be treated as the one-dimensional problem so only one 3D element is used and the size of the element is $1 \times 1 \times 1$, considering the length unit as the millimeter.
Similarly, the axisymmetric tension-torsion calculation is a typical biaxial stress state. Therefore the 8-node axisymmetric displacement/temperature element is suitable for this simulation.
We take out a cross-section in the specimen gauge length with the same inner and outer radius as the hollow round specimen.

\begin{figure}[!htp]
  \centering
  \includegraphics[width=16cm]{compuation_model.pdf}
  \caption{Computation model with ABAQUS.}
  \label{Fig:Compuation_model}
\end{figure}

The small number of elements will speed up the computation and simultaneously provide sufficient precision.
The elements used in both models include the degree of freedoms: displacement and temperature.
A primary goal is to simulate the experiments with thermomechanical loading condition.
%The material model and parameters used in the following simulation models was taken from the above sections.

\ref{Fig:plot_cmp_monotonic_tensile} shows the experimental and computational monotonic tensile responses at 300$^\circ$C, 550$^\circ$C and 650$^\circ$C.
The computational stress-strain curves are simulated by ABAQUS/Standard using the subroutine UMAT.
Because of the initial kinematic hardening parameters $r_0^k$ are directly obtained from the monotonic tensile curves, the computational and experimental results are almost the same. Meanwhile, the results also show that the number of $r_0^k$ ($M=9$) is suitable for the simulation.

\begin{figure}[!htp]
  \centering
  \includegraphics[width=14.0cm]{plot_cmp_monotonic_tensile.pdf}
  \caption{Comparison of experimental and computational stress-strain responses of monotonic tensile test at 300$^\circ$C, 550$^\circ$C and 650$^\circ$C.}
  \label{Fig:plot_cmp_monotonic_tensile}
\end{figure}

\section{Computational analysis}

\subsection{Simulations of isothermal fatigue tests}
\noindent
The constitutive model described in the previous sections has been implemented into the commercial finite element code  ABAQUS via the user-defined material model interface UMAT.
The stress-strain hysteresis loops of the uniaxial tests are simulated and predicted by the proposed model under the symmetrical strain path.
Comparison between computational and experimental stress-strain loops is shown in \ref{Fig:200th_Exp_Sim}. The uniaxial tension and torsion test can be modelled using a single axisymmetric tension-torsion element.
The figure illustrates results for first loading cycles and the 200th cycles at 300$^\circ$C, 550$^\circ$C and 650$^\circ$C, respectively.

\begin{figure}
  \centering
    \begin{overpic}[width=8.0cm]{plot_cmp_1st_life_cycle_TCIF_1d0.pdf}
      \put(84,13){\fcolorbox{white}{white}{(a)}}
    \end{overpic}
    \begin{overpic}[width=8.0cm]{plot_cmp_half_life_cycle_TCIF_1d0.pdf}
      \put(84,13){\fcolorbox{white}{white}{(b)}}
    \end{overpic}
\caption{Comparison of the stress-strain hysteresis loops from experiments and computations  under isothermal uniaxial tensile loading conditions at 300$^\circ$C, 550$^\circ$C and 650$^\circ$C, respectively. (a) The first loading cycle. (b) The 200th cycle.}
\label{Fig:200th_Exp_Sim}
\end{figure}

The strain controlled loop consists of initial loading, unloading and reverse loading.
As shown in \ref{Fig:200th_Exp_Sim}(a), in the initial loading cycle the computational and experimental loops are in reasonable agreement, but deviations are observed during reverse plastic loading for the lower temperatures because the kinematic hardening variable $r^k$ approximates ${r_0^k}$ in the first loading loop.
Note that the kinematic hardening variable ${r_0^k}$ is obtained from monotonic tension tests and evolution of $r^k$ depends on the accumulated plastic strain $p$.
During the first reverse plastic loading, the accumulated plastic strain $p$ is small, so the first simulated stress-strain hysteresis loops are mainly based on the monotonic curve. It leads to the deviation in the computation from the experimental result.
Along loading and unloading cycles, with evolution of $r_{\rm{s}}^k$ (Equation (\ref{Equ:rsk})), the predicted responses approach the experimental curves and the agreement of the 200th cycle becomes optimal, as shown in \ref{Fig:200th_Exp_Sim}(b). Since the fatigue is determined mainly by the peak and valley values, such deviations will not change fatigue life assessment.

% \begin{figure}
% \centering{\includegraphics[width=8.0cm]{IN718_Isothermal_Axial+-1_PV_Exp_vs_Sim.pdf}}
% \caption{Comparison of experimental and computational peak and valley stress values under isothermal cyclic uniaxial tensile loading conditions at 300$^\circ$C, 550$^\circ$C and 650$^\circ$C, respectively.}
% \label{Fig:Compare_PACC-PV_stress_temperature}
% \end{figure}

\begin{figure}
  \centering
    \begin{overpic}[width=8.0cm]{plot_cmp_1st_life_cycle_TCIF_Circle.pdf}
      \put(84,65){\fcolorbox{white}{white}{(a)}}
    \end{overpic}
    \begin{overpic}[width=8.0cm]{plot_cmp_150th_life_cycle_TCIF_Circle.pdf}
      \put(84,65){\fcolorbox{white}{white}{(b)}}
    \end{overpic}
\caption{Comparison between experiments and computations for the isothermal non-proportional cyclic loading path CIRC at 300$^\circ$C, 550$^\circ$C, 650$^\circ$C, respectively. (a) The first loading cycle. (b) The 200th cycle.}
\label{Fig:Circle_Exp_Sim}
\end{figure}

More details from the experiments and computations reveal that the material is cyclic softening, as shown in \ref{Fig:IN718_Isothermal_Axial+-1_PV_Exp_vs_Sim_1}(a), where the stress peak and valley values from the uniaxial cyclic tension test are summarized. The softening behavior becomes more severe as temperature increases.
As mentioned earlier, in the present constitutive relation, cyclic softening is modeled by the kinematic hardening variable $r^k$, with its saturation value $r_{\rm{s}}^k$ smaller than its initial value $r_0$.
The figure confirms a good agreement between experiments and the constitutive model prediction in the whole loading and unloading cycles.

The multiaxial non-proportional loading results are summarized in \ref{Fig:Circle_Exp_Sim} for the circular tension-torsion loading path CIRC under strain controlling at 300$^\circ$C, 550$^\circ$C and 650$^\circ$C, respectively. The cyclic stress curves are constructed by plotting the shear stress versus the tensile stress in the first cycle and the 200th cycle with the total strain amplitude of $1\%$, respectively. At 650$^\circ$C the computational prediction shows slight over-estimate near the maximum normal stress, while others reveal optimal agreement for both initial and cyclic loading.

The present constitutive can build different cyclic softening features. \ref{Fig:IN718_Isothermal_Axial+-1_PV_Exp_vs_Sim_1}(a) depicts the peak stress from the circular loading path at the different temperatures.
The alloy exhibits an obvious continuous cyclic softening at the higher temperature, while the cyclic softening may reach a saturation stage at low temperature. Such observation can be critical for mechanical components of high temperature.
\ref{Fig:IN718_Isothermal_Axial+-1_PV_Exp_vs_Sim_1}(a) illustrates effects of kinematic hardening variables. While the Fang's model  \cite{fang2015cyclic}  in Equation (\ref{Equ:fangrdeltak}) disagrees to the experiment in high cycle regime, as shown in \ref{Fig:IN718_Isothermal_Axial+-1_PV_Exp_vs_Sim_1}(b), the present model in Equation (\ref{Equ:rdeltak}) provides a consistent cyclic behavior in the whole loading history. One term expression of Fang cannot describe the different softening behavior properly.

\begin{figure}[!htp]
  \nonumber
    \centering
    \begin{overpic}[width=8.0cm]{plot_cmp_pv_TCIF.pdf}
      \put(84,65){\fcolorbox{white}{white}{(a)}}
    \end{overpic}
    \centering
    \begin{overpic}[width=8.0cm]{plot_cmp_pv_TCIF_2.pdf}
      \put(84,65){\fcolorbox{white}{white}{(b)}}
    \end{overpic}
\caption{Peak stress variations for the circular loading path CIRC at 300$^\circ$C, 550$^\circ$C and 650$^\circ$C, respectively. (a) Prediction based on Equation (\ref{Equ:fangrdeltak}) \cite{fang2015cyclic}. (b) The present model.}
\label{Fig:IN718_Isothermal_Axial+-1_PV_Exp_vs_Sim_1}
\end{figure}

To study effects of the complex non-proportional loading path, stress variations for the loading cross path CROS (\ref{Fig:LoadPath}(d)) are plotted in \ref{Fig:IN718_Isothermal_300C_7049_XPath_Exp_vs_Sim}. Symbols denote experimental record, and solid curves stand for computational predictions. Due to prompt changes in the loading direction, the curves become non-symmetric. In the figure both computational results with and without non-proportional hardening term are present. The obvious disagreement between computations and experiments is observed at the stress peaks. However, the improvement of the non-proportional hardening in the model is significant, as shown in \ref{Fig:IN718_Isothermal_300C_7049_XPath_Exp_vs_Sim}. The non-proportional hardening term is meaningful for the constitutive modeling.

\begin{figure}[!htp]
  \centering
    \begin{overpic}[width=8.0cm]{plot_cmp_half_life_cycle_XPath_Y0.pdf}
      \put(16,65){\fcolorbox{white}{white}{(a)}}
    \end{overpic}
    \begin{overpic}[width=8.0cm]{plot_cmp_half_life_cycle_XPath_Y200.pdf}
      \put(16,65){\fcolorbox{white}{white}{(b)}}
    \end{overpic}
    \begin{overpic}[width=8.0cm]{plot_cmp_half_life_cycle_XPath_Y0_Mises.pdf}
      \put(16,65){\fcolorbox{white}{white}{(c)}}
    \end{overpic}
    \begin{overpic}[width=8.0cm]{plot_cmp_half_life_cycle_XPath_Y200_Mises.pdf}
      \put(16,65){\fcolorbox{white}{white}{(d)}}
    \end{overpic}
\caption{Comparison between experimental results and computational predictions for the cross loading path CROS at 300$^\circ$C. (a)(c) Without non-proportional hardening term. (b)(d) With the non-proportional hardening factor $Y_{\Delta{\rm{sat}}}$=200MPa.}
\label{Fig:IN718_Isothermal_300C_7049_XPath_Exp_vs_Sim}
\end{figure}

%\begin{figure*}[!htp]
%\centering{\includegraphics[width=8.0cm]{MisesStressWithoutNPHardening.pdf}}
%\centering{\includegraphics[width=8.0cm]{MisesStressWithNPHardening.pdf}}
%\caption{Comparison between experiment at 300$^\circ$C under the cross loading path and computations by using different non-proportional hardening factors. The Mises stress is compared directly. (a) Without non-proportional hardening. (b) Non-proportional hardening factor = 200MPa.}
%\label{Fig:IN718_Isothermal_300C_7049_XPath_Exp_vs_Sim_MisesStress}
%\end{figure*}

%\begin{figure*}
%  \begin{minipage}[t]{0.5\linewidth}
%  \nonumber
%    \centering
%    \includegraphics[width=3.5in]{IN718_Isothermal_300C_7049_XPath_Exp_vs_Sim_0.pdf}
%    \centerline{(a) Without proportional hardening.}
%    \label{Fig:IN718_Isothermal_300C_7049_XPath_Exp_vs_Sim_0}
%  \end{minipage}%
%  \begin{minipage}[t]{0.5\linewidth}
%    \centering
%    \includegraphics[width=3.5in]{IN718_Isothermal_300C_7049_XPath_Exp_vs_Sim_200.pdf}
%    \centerline{(b) Non-proportional hardening factor = 200MPa.}
%    \label{Fig:IN718_Isothermal_300C_7049_XPath_Exp_vs_Sim_200}
%  \end{minipage}
%  \caption{Experiments at 300$^\circ$C under the cross loading path, comparing with computations by using different non-proportional hardening factors.}
%  \label{Fig:IN718_Isothermal_300C_7049_XPath_Exp_vs_Sim}
%\end{figure*}

%\begin{figure*}
%  \begin{minipage}[t]{0.5\linewidth}
%  \nonumber
%    \centering
%    \includegraphics[width=3.5in]{MisesStressWithoutNPHardening.pdf}
%    \centerline{(a) Without non-proportional hardening.}
%    \label{Fig:MisesStressWithoutNPHardening}
%  \end{minipage}%
%  \begin{minipage}[t]{0.5\linewidth}
%    \centering
%    \includegraphics[width=3.5in]{MisesStressWithNPHardening.pdf}
%    \centerline{(b) Non-proportional hardening factor = 200MPa.}
%    \label{Fig:MisesStressWithNPHardening}
%  \end{minipage}
%  \caption{Comparison between experiment at 300$^\circ$C under the cross loading path and computations by using different non-proportional hardening factors. The Mises stress is compared directly.}
%  \label{Fig:IN718_Isothermal_300C_7049_XPath_Exp_vs_Sim_MisesStress}
%\end{figure*}

\begin{figure}
  \centering
    \begin{overpic}[width=8.0cm]{plot_cmp_10th_cycle_TCTMF_1d0.pdf}
      \put(84,65){\fcolorbox{white}{white}{(a)}}
    \end{overpic}
    \centering
    \begin{overpic}[width=8.0cm]{plot_cmp_pv_TCTMF_1d0.pdf}
      \put(84,65){\fcolorbox{white}{white}{(b)}}
    \end{overpic}
\caption{Experimental and computational results under the uniaxial thermomechanical loading conditions with varying temperature between 300$^\circ$C and 650$^\circ$C. Three different phase angles of the TMF loads are summarized. (a) The hysteresis loops. (b) Peak and valley stress values.}
\label{Fig:TMF_IP}
\end{figure}

\subsection{Uniaxial thermomechanical fatigue tests}
\noindent
Under thermomechanical loading conditions, both mechanical loads, and temperature, vary, which cause different deformation and damage mechanisms in the material.
In the present section both experimental and computational results for the varying temperature between 300$^\circ$C and 650$^\circ$C are considered. Tests are controlled by the tensile and shear strains simultaneously, by a constant given equivalent strain amplitude of 1\%. In computations the model parameters determined for the isothermal cyclic tests are used, that is, the thermomechanical coupling is assumed not to affect the constitutive model explicitly.

In \ref{Fig:TMF_IP} the hysteresis loops of the stabilized stress-strain cycles are shown, in which the in-phase (IP), the out-of-phase (OP) and 90$^\circ$-phase temperature loads are considered, respectively.
The IP-phase loops show similar features as those of the isothermal loading.
The stress-strain curve reveals cyclic softening, as observed in isothermal tests. Interesting is the valley stress of the IP test seems higher than the computational prediction, which is related to the thermomechanical effects. However, the effect seems not significant.

As soon as the temperature does not change proportionally to the mechanical load, the stress-strain loops become asymmetric, as shown in \ref{Fig:TMF_IP}(a).
The stress-strain loops for the stabilized cycles, i.e., the $N_f$/2-th cycle, in \ref{Fig:TMF_IP}(a) confirm that the constitutive model can reasonably predict the thermomechanical behavior of the material adequately.
Furthermore, \ref{Fig:TMF_IP}(b) shows that the peak and valley stress of the computational prediction in comparing with the experiments.
The peak stresses of all temperature loading phases, i.e., IP, OP and 90$^\circ$-phase, confirm a good agreement between the experiments and computations.
However, for all valley stresses, the experimental results are observed approximately 50 MPa lower than the predictions.
The reason is that the yield surface is assumed to be circular in the present constitutive model and the yield surface seems to stiff.
As shown by Gil \cite{Gil1998}, the yield surface of Inconel 718 is a shift to the compression direction as temperature increase from room temperature to 650$^\circ$C.

%The agreement between the experiments and the computations is obvious.

%\begin{figure*}[!htp]
%\centering{\includegraphics[width=8.0cm]{7018_TMF_+-1_IP_10th.pdf}}
%\centering{\includegraphics[width=8.0cm]{IN718_TMF_Axial+-1_PV_Exp_vs_Sim_IP.pdf}}
%\centering{\includegraphics[width=8.0cm]{7017_TMF_+-1_OP_10th.pdf}}
%\centering{\includegraphics[width=8.0cm]{IN718_TMF_Axial+-1_PV_Exp_vs_Sim_OP.pdf}}
%\centering{\includegraphics[width=8.0cm]{7025_TMF_+-1_90_10th.pdf}}
%\centering{\includegraphics[width=8.0cm]{IN718_TMF_Axial+-1_PV_Exp_vs_Sim_90.pdf}}
%\caption{Experimental and computational results under the multiaxial thermomechanical loading conditions with temperature between 300 and 650$^\circ$C. (a) The hysteresis loop in the in-phase loading. (b) Peak-valley stresses as function of the loading cycles in the in-phase loading. (c) The stress-strain hysteresis loop in the out-of-phase loading. (d) Peak-valley stress as a function of loading cycles in the out-of-phase loading. (e) The  stress-strain hysteresis loop with a phase angle of 90$^\circ$ . (f) Peak-valley stress as a function of loading cycles with a phase angle of 90$^\circ$ . \marked{(I would suggest to combine loops into one figure and peak-valleys into one figure. These 6 figures do not give so different informations. They are too similar to give explanations.)}}
%\label{Fig:TMF_IP}
%\end{figure*}

%\begin{figure*}[!htp]
%\centering{\includegraphics[width=8.0cm]{7017_TMF_+-1_OP_10th.pdf}}
%\centering{\includegraphics[width=8.0cm]{IN718_TMF_Axial+-1_PV_Exp_vs_Sim_OP.pdf}}
%\caption{Experimental and computational results under the out-of-phase multiaxial thermomechanical loading conditions with temperature 300-650$^\circ$C. (a) The stabilized stress-strain hysteresis loop. (b) Peak-valley stress as a function of loading cycles.}
%\label{Fig:TMF_OP}
%\end{figure*}

%\begin{figure*}[!htp]
%\centering{\includegraphics[width=8.0cm]{7025_TMF_+-1_90_10th.pdf}}
%\centering{\includegraphics[width=8.0cm]{IN718_TMF_Axial+-1_PV_Exp_vs_Sim_90.pdf}}
%\caption{Experimental and computational results under the multiaxial thermomechanical loading conditions with a phase angle of 90$^\circ$ and temperature of 300-650$^\circ$C. (a) The stabilized stress-strain hysteresis loop. (b) Peak-valley stress as a function of loading cycles.}
%\label{Fig:TMF_90}
%\end{figure*}

\begin{figure}[!htp]
\centering{\begin{overpic}[width=8.0cm]{IN718_TMF_IP_Prop45+-1_1st.pdf}
\put(0,65){\fcolorbox{white}{white}{(a)}}
\end{overpic}}
\centering{\begin{overpic}[width=8.0cm]{IN718_TMF_IP_Prop45+-1_50th.pdf}
\put(0,65){\fcolorbox{white}{white}{(b)}}
\end{overpic}}
\centering{\begin{overpic}[width=8.0cm]{IN718_TMF_IP_Prop45+-1_Exp_vs_Sim_PV_Axial.pdf}
\put(0,65){\fcolorbox{white}{white}{(c)}}
\end{overpic}}
\centering{\begin{overpic}[width=8.0cm]{IN718_TMF_IP_Prop45+-1_Exp_vs_Sim_PV_Torsional.pdf}
\put(0,65){\fcolorbox{white}{white}{(d)}}
\end{overpic}}
\caption{Experimental and computational results for the proportional in-phase thermomechanical loading with varying temperature between 300$^\circ$C and 650$^\circ$C. (a) The first loading cycle. (b) The half life loading cycle. (c) Axial peak-valley stresses as function of the loading cycles. (d) Torsional peak-valley stresses as function of the loading cycles.}
\label{Fig:TMF_Prop45}
\end{figure}

\begin{figure}
\centering{\begin{overpic}[width=8.0cm]{IN718_TMF_7037_Diamond+-1_1st.pdf}
\put(0,65){\fcolorbox{white}{white}{(a)}}
\end{overpic}}
\centering{\begin{overpic}[width=8.0cm]{IN718_TMF_7037_Diamond+-1_20th.pdf}
\put(0,65){\fcolorbox{white}{white}{(b)}}
\end{overpic}}
\centering{\begin{overpic}[width=8.0cm]{IN718_TMF_IP_Diamond+-1_Exp_vs_Sim_PV_Axial.pdf}
\put(0,65){\fcolorbox{white}{white}{(c)}}
\end{overpic}}
\centering{\begin{overpic}[width=8.0cm]{IN718_TMF_IP_Diamond+-1_Exp_vs_Sim_PV_Torsional.pdf}
\put(0,65){\fcolorbox{white}{white}{(d)}}
\end{overpic}}
\caption{Experimental and computational results for the diamond strain path under in-phase thermomechanical loading conditions with varying temperature between  300$^\circ$C and 650$^\circ$C. (a) The first loading cycle. (b) The half life loading cycle. (c) Axial peak-valley normal stress as function of loading cycles. (d) Torsional peak-valley shear stress as function of  loading cycles.}
\label{Fig:TMF_Diamond}
\end{figure}

\subsection{Multiaxial thermomechanical fatigue tests}
\noindent
Multiaxial thermomechanical material behavior was not studied systematically in literature. Effects of multiaxial loads and temperature were not discussed. In the article both proportional loading path PRO (\ref{Fig:LoadPath}(a)) and non-proportional diamond loading path NPR (\ref{Fig:LoadPath}(c)) are investigated. Experimental results are shown in \ref{Fig:TMF_Prop45} and \ref{Fig:TMF_Diamond}, together with computational predictions.

In both tests, the temperature varies between 300$^\circ$C and 650$^\circ$C, the equivalent mechanical strain amplitude $\Delta\varepsilon_{\rm{eq}}/2=1\%$, the phase angle between axial strain and temperature is zero degree for the in-phase loading (IP).
The torsional and axial stress curve reveals cyclic softening for all tested multiaxial loading cases.
Computations were performed with the non-proportional hardening factor $Y_{\Delta {\rm{sat}}}=0$MPa and $Y_{\Delta {\rm{sat}}}=200$MPa, respectively, to demonstrate the influence of the non-proportional correction to the computational results.

As shown in \ref{Fig:TMF_Prop45}, the proportional loading case provides good agreement between the computation and experiment in the peak value distribution, while the valley values show a similar deviation as observed in isothermal loading. The shifting in the shear stress-normal stress loop is caused by the under-estimate of the compressive normal stress. Generally, computations agree with the experiments reasonably. The non-proportional hardening factor does not affect the computational results, as expected.

Under the non-proportional loading condition, the diamond loading path shows significant effects of the non-proportional hardening factor in the constitutive model, as shown in \ref{Fig:TMF_Diamond}. The stress loops of the first loading cycles and the stabilized cycles are illustrated in \ref{Fig:TMF_Diamond}(a) and (b), respectively. In the figures, both computational results with and without non-proportional hardening factor are illustrated. In the first loading cycle and the stabilized cycle, the non-proportional hardening factor improves computational predictions obviously. The model can describe additional hardening due to the non-proportional loading.

The axial and torsional peak-valley stresses are illustrated in  \ref{Fig:TMF_Diamond}(c) and (d).
In both tests, the slight deviations between computational predictions and experiments change after loading direction. The stress amplitude at the low temperature seems to be under-estimated in comparing with the higher temperature point. However, it is confirmed that the constitutive model can predict the thermomechanical behavior of the material properly.

% \subsection{Simulations of isothermal fatigue tests}
% \label{}
% \begin{figure}
%   \begin{minipage}[t]{0.5\linewidth}
%   \nonumber
%     \centering
%     \begin{overpic}[width=8.0cm]{plot_cmp_1st_life_cycle_TCIF_1d0.pdf}
%       \put(84,13){\fcolorbox{white}{white}{(a)}}
%     \end{overpic}
%   \end{minipage}%
%   \begin{minipage}[t]{0.5\linewidth}
%     \centering
%     \begin{overpic}[width=8.0cm]{plot_cmp_half_life_cycle_TCIF_1d0.pdf}
%       \put(84,13){\fcolorbox{white}{white}{(b)}}
%     \end{overpic}
%   \end{minipage}
%   \caption{Comparison of the stress-strain hysteresis loops between experiments and computations  under isothermal uniaxial tensile loading conditions at 300$^\circ$C, 550$^\circ$C and 650$^\circ$C, respectively. (a) The first loading cycle. (b) The half life cycle.}
%   \label{Fig:plot_cmp_life_cycle_TCIF_1d0}
% \end{figure}

% The uniaxial tension and torsion test can be modeled using a single axisymmetric tension-torsion element.
% Details of the ABAQUS use-defined material model UMAT has been reported in the previous sections.
% The stress-strain hysteresis loops of the uniaxial tests are simulated and predicted by the proposed model under the symmetrical strain path.
% Comparison between  computational stress-strain and  experimental loops is shown in \ref{Fig:200th_Exp_Sim}.
% The figure illustrates results for first loading cycles and the 200th cycles at 300$^\circ$C, 550$^\circ$C and 650$^\circ$C, respectively.

% The first strain controlled loop consists of initial loading, unloading and reverse loading.
% As shown in \ref{Fig:200th_Exp_Sim}(a), in the initial loading cycle the predicted and experimental loops are in reasonable agreement, but deviations are observed during reverse plastic loading for the lower temperatures because the kinematic hardening variable $r^k$ approximates ${r_0^k}$ in the first loading loop.
% Note that the kinematic hardening variable ${r_0^k}$ is obtained from monotonic tension tests and evolution of $r^k$ depends on the accumulated plastic strain $p$.
% During the first reverse plastic loading, the accumulated plastic strain $p$ is small so the first simulated stress-strain hysteresis loops are mainly based on the monotonic curve. It leads to the deviation of the computation from the experimental result in the initial loading cycle.
% Along loading and unloading cycles, with evolution of $r_{\rm{s}}^k$ (see Equation (\ref{Equ:rsk})), the predicted responses approaches experimental curve and agreement at the 200th cycle becomes optimal, as shown in \ref{Fig:200th_Exp_Sim}(b). Since the fatigue is determined mainly by the peak and valley values, such deviations will not change fatigue life assessment.


% %\begin{figure}
% %\centering{\includegraphics[width=8.0cm]{Chapter6Figs/IN718_Isothermal_Axial+-1_PV_Exp_vs_Sim.pdf}}
% %\caption{Comparison of experimental and predicted peak and valley stresses under isothermal uniaxial tensile loading conditions at 300$^\circ$C, 550$^\circ$C and 650$^\circ$C, respectively.}
% %\label{Fig:Compare_PACC-PV_stress_temperature}
% %\end{figure}

% \begin{figure}
%   \begin{minipage}[t]{0.5\linewidth}
%   \nonumber
%     \centering
%     \begin{overpic}[width=8.0cm]{plot_cmp_1st_life_cycle_TCIF_Circle.pdf}
%       \put(84,65){\fcolorbox{white}{white}{(a)}}
%     \end{overpic}
%   \end{minipage}%
%   \begin{minipage}[t]{0.5\linewidth}
%     \centering
%     \begin{overpic}[width=8.0cm]{plot_cmp_150th_life_cycle_TCIF_Circle.pdf}
%       \put(84,65){\fcolorbox{white}{white}{(b)}}
%     \end{overpic}
%   \end{minipage}
%   \caption{Comparison between experiments and computations  under isothermal non-proportional loading conditions at  300, 550, 650$^\circ$C. (a) The first loading cycle. (b) The half life cycle.}
%   \label{Fig:Circle_Exp_Sim}
% \end{figure}

% More details from the experiments and computations reveal that the material is cyclic softening, as shown in \ref{Fig:Compare_PACC-PV_stress_temperature}. The softening behavior becomes more severe as  temperature increases.
% As mentioned earlier, in the present constitutive relation, cyclic softening is modeled by the kinematic hardening variable $r^k$, with its saturation value $r_{\rm{s}}^k$ smaller than its initial value $r_0$.
% The figure confirms a good agreement between experiments and the model in the whole loading and unloading processes.


% The multiaxial non-proportional loading results are summarized in \ref{Fig:Circle_Exp_Sim}, for the circular tension-torsion loading cases at 300$^\circ$C, 550$^\circ$C and 650$^\circ$C, respectively. The cyclic deformation curves were constructed by plotting the amplitudes from peak tensile to peak compressive stress for the first cycle and the 200th cycle with total strain range $\pm1\%$, respectively. At 650$^\circ$C the computational prediction shows slight over-estimate near the maximum normal stress, while others reveal optimal agreement for both initial and cyclic loading.

% \ref{Fig:IN718_Isothermal_Axial+-1_PV_Exp_vs_Sim_1} depicts the peak stress from the circular loading at the different temperatures.
% The alloy exhibits an obvious continuous cyclic softening at higher temperature, while the cyclic softening may reach a saturation stage at low temperature. Such observation can be critical for mechanical components of high temperature.
% \ref{Fig:IN718_Isothermal_Axial+-1_PV_Exp_vs_Sim_1}(a) illustrates effects of kinematic hardening variables. While the Fang's model \cite{fang2015cyclic} in Equation (\ref{Equ:fangrdeltak}) shows a clear disagreement to the experiments in \ref{Fig:IN718_Isothermal_Axial+-1_PV_Exp_vs_Sim_1}(b), the present prediction in Equation (\ref{Equ:rdeltak}) provides a consistent cyclic behavior in the whole loading history. One term expression of Fang cannot describe the softening behavior properly.

% \begin{figure}
%   \begin{minipage}[t]{0.5\linewidth}
%   \nonumber
%     \centering
%     \begin{overpic}[width=8.0cm]{plot_cmp_pv_TCIF.pdf}
%       \put(84,65){\fcolorbox{white}{white}{(a)}}
%     \end{overpic}
% %    \includegraphics[width=3.2in]{plot_cmp_pv_TCIF.pdf}
% %    \centerline{(a)}
%   \end{minipage}%
%   \begin{minipage}[t]{0.5\linewidth}
%     \centering
%     \begin{overpic}[width=8.0cm]{plot_cmp_pv_TCIF_2.pdf}
%       \put(84,65){\fcolorbox{white}{white}{(b)}}
%     \end{overpic}
% %    \includegraphics[width=3.2in]{plot_cmp_pv_TCIF_2.pdf}
% %    \centerline{(b)}
%   \end{minipage}
%   \caption{Comparison of the peak stress under isothermal circular non-proportional loading conditions at 300$^\circ$C, 550$^\circ$C and 650$^\circ$C, respectively. (a) Fang's model \cite{fang2015cyclic}. (b) The proposed model.}
%   \label{Fig:IN718_Isothermal_Axial+-1_PV_Exp_vs_Sim_1}
% \end{figure}

% To study effects of complex non-proportional loading path, an experiment with loading cross path (\ref{Fig:LoadPath}(d)) is plotted in \ref{Fig:IN718_Isothermal_300C_7049_XPath_Exp_vs_Sim}, together with computational predictions. Due to prompt changes in loading direction, the stress-strain curves become non-symmetric. In the figure both computational results with and without non-proportional hardening term are present. The obvious disagreement between computations and experiments is observed at the peaks. However, the improvement of the non-proportional hardening in the model is significant, as shown in \ref{Fig:IN718_Isothermal_300C_7049_XPath_Exp_vs_Sim}. The non-proportional hardening term is meaningful for the constitutive modeling.

% \begin{figure}
%   \begin{minipage}[t]{0.5\linewidth}
%   \nonumber
%     \centering
%     \begin{overpic}[width=8.0cm]{plot_cmp_half_life_cycle_XPath_Y0.pdf}
%       \put(16,65){\fcolorbox{white}{white}{(a)}}
%     \end{overpic}
%   \end{minipage}%
%   \begin{minipage}[t]{0.5\linewidth}
%     \centering
%     \begin{overpic}[width=8.0cm]{plot_cmp_half_life_cycle_XPath_Y200.pdf}
%       \put(16,65){\fcolorbox{white}{white}{(b)}}
%     \end{overpic}
%   \end{minipage}

%   \begin{minipage}[t]{0.5\linewidth}
%   \nonumber
%     \centering
%     \begin{overpic}[width=8.0cm]{plot_cmp_half_life_cycle_XPath_Y0_Mises.pdf}
%       \put(16,65){\fcolorbox{white}{white}{(c)}}
%     \end{overpic}
%   \end{minipage}%
%   \begin{minipage}[t]{0.5\linewidth}
%     \centering
%     \begin{overpic}[width=8.0cm]{plot_cmp_half_life_cycle_XPath_Y200_Mises.pdf}
%       \put(16,65){\fcolorbox{white}{white}{(d)}}
%     \end{overpic}
%   \end{minipage}

%   \caption{Comparison between experiments at 300$^\circ$C under the cross loading path with computational predictions. (a) Without non-proportional hardening term. (b) With the non-proportional hardening factor $Y_{sat}$ = 200MPa.}
%   \label{Fig:IN718_Isothermal_300C_7049_XPath_Exp_vs_Sim}
% \end{figure}

% %\begin{figure*}
% %  \begin{minipage}[t]{0.5\linewidth}
% %  \nonumber
% %    \centering
% %    \includegraphics[width=3.5in]{Chapter6Figs/IN718_Isothermal_300C_7049_XPath_Exp_vs_Sim_0.pdf}
% %    \centerline{(a) Without proportional hardening.}
% %    \label{Fig:IN718_Isothermal_300C_7049_XPath_Exp_vs_Sim_0}
% %  \end{minipage}%
% %  \begin{minipage}[t]{0.5\linewidth}
% %    \centering
% %    \includegraphics[width=3.5in]{Chapter6Figs/IN718_Isothermal_300C_7049_XPath_Exp_vs_Sim_200.pdf}
% %    \centerline{(b) Non-proportional hardening factor = 200MPa.}
% %    \label{Fig:IN718_Isothermal_300C_7049_XPath_Exp_vs_Sim_200}
% %  \end{minipage}
% %  \caption{Experiments at 300$^\circ$C under the cross loading path, comparing with computations by using different non-proportional hardening factors.}
% %  \label{Fig:IN718_Isothermal_300C_7049_XPath_Exp_vs_Sim}
% %\end{figure*}

% %\begin{figure*}
% %  \begin{minipage}[t]{0.5\linewidth}
% %  \nonumber
% %    \centering
% %    \includegraphics[width=3.5in]{Chapter6Figs/MisesStressWithoutNPHardening.pdf}
% %    \centerline{(a) Without non-proportional hardening.}
% %    \label{Fig:MisesStressWithoutNPHardening}
% %  \end{minipage}%
% %  \begin{minipage}[t]{0.5\linewidth}
% %    \centering
% %    \includegraphics[width=3.5in]{Chapter6Figs/MisesStressWithNPHardening.pdf}
% %    \centerline{(b) Non-proportional hardening factor = 200MPa.}
% %    \label{Fig:MisesStressWithNPHardening}
% %  \end{minipage}
% %  \caption{Comparison between experiment at 300$^\circ$C under the cross loading path and computations by using different non-proportional hardening factors. The Mises stress is compared directly.}
% %  \label{Fig:IN718_Isothermal_300C_7049_XPath_Exp_vs_Sim_MisesStress}
% %\end{figure*}

% \begin{figure}
%   \begin{minipage}[t]{0.5\linewidth}
%   \nonumber
%     \centering
%     \begin{overpic}[width=8.0cm]{plot_cmp_10th_cycle_TCTMF_1d0.pdf}
%       \put(84,65){\fcolorbox{white}{white}{(a)}}
%     \end{overpic}
%     %\includegraphics[width=3.2in]{Chapter6Figs/All_TMF_+-1_IP_10th.pdf}
%   \end{minipage}%
%   \begin{minipage}[t]{0.5\linewidth}
%     \centering
%     \begin{overpic}[width=8.0cm]{plot_cmp_pv_TCTMF_1d0.pdf}
%       \put(84,65){\fcolorbox{white}{white}{(b)}}
%     \end{overpic}
%     %\includegraphics[width=3.2in]{Chapter6Figs/IN718_TMF_Axial+-1_PV_Exp_vs_Sim_ALL.pdf}
%   \end{minipage}
%   \caption{Experimental and computational results under the multiaxial thermomechanical loading conditions with temperature between 300 and 650$^\circ$C. (a) The hysteresis loops in the in-phase loading, out-of-phase loading and the loading with a phase angle of 90$^\circ$. (b) Peak-valley stresses as function of the loading cycles in the in-phase loading, out-of-phase loading and and the loading with a phase angle of 90$^\circ$.}
%   \label{Fig:TMF_IP}
% \end{figure}

% %\begin{figure*}
% %\centering{\includegraphics[width=8.0cm]{Chapter6Figs/All_TMF_+-1_IP_10th.pdf}}
% %\centering{\includegraphics[width=8.0cm]{Chapter6Figs/IN718_TMF_Axial+-1_PV_Exp_vs_Sim_ALL.pdf}}
% %\caption{Experimental and computational results under the multiaxial thermomechanical loading conditions with temperature between 300 and 650$^\circ$C. (a) The hysteresis loops in the in-phase loading, out-of-phase loading and the loading with a phase angle of 90$^\circ$. (b) Peak-valley stresses as function of the loading cycles in the in-phase loading, out-of-phase loading and and the loading with a phase angle of 90$^\circ$.}
% %\label{Fig:TMF_IP}
% %\end{figure*}


% \subsection{Uniaxial thermomechanical fatigue tests}
% Under thermomechanical loading conditions both mechanical loads as well as specimen temperature vary, which cause different deformation and damage mechanisms in the material.
% In the present section both experimental and computational results for the varying temperature between 300-650$^\circ$C are considered, where tests are controlled by the tensile and shear strains simultaneously, by a constant given equivalent strain amplitude, 1\%. In the computations the model parameters have been determined from the isothermal analysis, that is, the thermomechanical coupling is not considered in the constitutive modeling explicitly.

% In \ref{Fig:TMF_IP} the hysteresis loops of the stabilized cycles are shown, in which the in-phase (IP), the out-of-phase (OP) and 90$^\circ$-phase temperature loads are considered, respectively.
% The IP-phase loops show similar features as those of the isothermal loading.
% The stress-strain curve reveals cyclic softening, as observed in isothermal tests.Interesting is the valley stress of the IP test seems higher than the computational prediction, which may be related to the thermomechanical effects. However, the effect seems not significant.

% As soon as the temperature does not change proportionally to the mechanical load, the stress-strain loops become asymmetric, as shown in \ref{Fig:TMF_IP}(a).
% %The stress-strain loops for the stabilized cycles, i.e. $N_f$/2th cycle, in \ref{Fig:TMF_IP}(a), \ref{Fig:TMF_OP}(a) and \ref{Fig:TMF_90}(a) confirm that the constitutive model can predict the thermomechanical behavior of the material properly.
% Furthermore, \ref{Fig:TMF_IP}(b) shows that the peak and valley stress of the computational prediction in comparing with the experiments.
% The peak stresses of all temperature loading phases, i.e. IP, OP and 90$^\circ$-phase, are confirm a good agreement between the experiments and computations.
% However, for all valley stresses, the experimental results are observed approximately 50 MPa lower than the predictions.
% The reason is that the yield surface is assumed to be circular in the present constitutive model.
% As shown by Gil \cite{Gil1998}, the yield surface of Inconel 718 is shift to the compression direction as temperature increase from room temperature to 650$^\circ$C.
% Because of the small deviations in the valley stresses, the yield surface is not further modified in the present work.

% \begin{figure}
%   \begin{minipage}[t]{0.5\linewidth}
%     \centering
%     \begin{overpic}[width=8.0cm]{Chapter6Figs/IN718_TMF_IP_Prop45+-1_1st.pdf}
%       \put(0,65){\fcolorbox{white}{white}{(a)}}
%     \end{overpic}
%   \end{minipage}%
%   \begin{minipage}[t]{0.5\linewidth}
%     \centering
%     \begin{overpic}[width=8.0cm]{Chapter6Figs/IN718_TMF_IP_Prop45+-1_50th.pdf}
%       \put(0,65){\fcolorbox{white}{white}{(b)}}
%     \end{overpic}
%   \end{minipage}

%   \begin{minipage}[t]{0.5\linewidth}
%   \nonumber
%     \centering
%     \begin{overpic}[width=8.0cm]{Chapter6Figs/IN718_TMF_IP_Prop45+-1_Exp_vs_Sim_PV_Axial.pdf}
%       \put(0,65){\fcolorbox{white}{white}{(c)}}
%     \end{overpic}
%   \end{minipage}%
%   \begin{minipage}[t]{0.5\linewidth}
%     \centering
%     \begin{overpic}[width=8.0cm]{Chapter6Figs/IN718_TMF_IP_Prop45+-1_Exp_vs_Sim_PV_Torsional.pdf}
%       \put(0,65){\fcolorbox{white}{white}{(d)}}
%     \end{overpic}
%   \end{minipage}

%   \caption{Experimental and computational results under the proportional in-phase thermomechanical loading conditions with temperature of 300-650$^\circ$C. (a) The first loading cycle. (b) The half life loading cycle. (c) Axial peak-valley stresses as function of the loading cycles. (d) Torsional peak-valley stresses as function of the loading cycles.}
%   \label{Fig:TMF_Prop45}
% \end{figure}

% \begin{figure}
%   \begin{minipage}[t]{0.5\linewidth}
%   \nonumber
%     \centering
%     \begin{overpic}[width=8.0cm]{Chapter6Figs/IN718_TMF_7037_Diamond+-1_1st.pdf}
%       \put(0,65){\fcolorbox{white}{white}{(a)}}
%     \end{overpic}
%   \end{minipage}%
%   \begin{minipage}[t]{0.5\linewidth}
%     \centering
%     \begin{overpic}[width=8.0cm]{Chapter6Figs/IN718_TMF_7037_Diamond+-1_20th.pdf}
%       \put(0,65){\fcolorbox{white}{white}{(b)}}
%     \end{overpic}
%   \end{minipage}

%   \begin{minipage}[t]{0.5\linewidth}
%   \nonumber
%     \centering
%     \begin{overpic}[width=8.0cm]{Chapter6Figs/IN718_TMF_IP_Diamond+-1_Exp_vs_Sim_PV_Axial.pdf}
%       \put(0,65){\fcolorbox{white}{white}{(c)}}
%     \end{overpic}
%   \end{minipage}%
%   \begin{minipage}[t]{0.5\linewidth}
%     \centering
%     \begin{overpic}[width=8.0cm]{Chapter6Figs/IN718_TMF_IP_Diamond+-1_Exp_vs_Sim_PV_Torsional.pdf}
%       \put(0,65){\fcolorbox{white}{white}{(d)}}
%     \end{overpic}
%   \end{minipage}

%   \caption{Experimental and computational results under the diamond strain path in-phase thermomechanical loading conditions with temperature of 300-650$^\circ$C. (a) The first loading cycle. (b) The half life loading cycle. (c) Axial peak-valley stresses as function of the loading cycles. (d) Torsional peak-valley stresses as function of the loading cycles.}
%   \label{Fig:TMF_Diamond}
% \end{figure}

% \subsection{Multi-axial thermomechanical fatigue tests}
% Multi-axial thermomechanical material behavior have not been studied systematically in literature. Effects of multiaxial loads and temperature are not discussed. In the present work both proportional loading path (\ref{Fig:LoadPath}(b)) and non-proportional diamond loading path (\ref{Fig:LoadPath}(c)) are investgated. Experimental results are shown in \ref{Fig:TMF_Prop45} and \ref{Fig:TMF_Diamond}, together with computational predictions.

% In both tests, the temperature ranges between 300 and 650$^\circ$C, the equivalent mechanical strain amplitude $\varepsilon_{eq,a}=1\%$, the phase angle between axial strain and temperature is zero degree, the in-phase loading (IP).
% The torsional and axial stress curve reveals cyclic softening for all tested multiaxial loading cases.
% Computations were performed with the non-proportional hardening factor $Y_{sat}=0$MPa and $Y_{sat}=200$MPa, respectively, to demonstrate the influence of the non-proportional correction to the computational results.

% As shown in \ref{Fig:TMF_Prop45}, the proportional loading case provides good agreement between the computation and experiment in peak value distribution, while the valley show a similar deviation as observed in isothermal loading. The shifting in the shear stress-normal stress loop is caused by the under-estimate of the compressive normal stress. Generally, computations agree with the experiment reasonably. Additionally, the non-proportional hardening factor do not effect the computational results, as expected.

% Under the non-proportional loading condition, the diamond loading path shows significant effects of the non-proportional hardening factor  in the constitutive model, as shown in \ref{Fig:TMF_Diamond}. The stress loops of the first loading cycles and the stabilized cycles are illustrated in \ref{Fig:TMF_Diamond}(a) and (b). In both first loading cycle and stabilized cycle the non-proportional hardening factor improves a much better computational prediction in comparing with the conventional model. Additional hardening due to the non-proportional loading can be observed and effects of the non-proportional hardening in the model are significant.

% The axial and torsional peak-valley stresses are illustrated in  \ref{Fig:TMF_Diamond}(c) and (d).
% In both tests one observes slight deviations between computational predictions and experiments after  loading direction changes. The stress amplitude at the low temperature seems to be under-estimated in comparing with the higher temperature point. However, it confirm that the constitutive model can predict the thermomechanical behavior of the material properly.

%The stress-strain hysteresis loops of the uniaxial tests are simulated and predicted by the proposed model under the symmetrical strain-controlled cyclic loading conditions and at 300$^\circ$C, 550$^\circ$C and 650$^\circ$C.
%The results are shown in \ref{Fig:Axial_Exp_Sim}.
%The cyclic softening behavior is more obviously as the temperature increases.
%After 50th cycle, the stress-strain loops become stabilized.
%In order to observe the stress-strain hysteresis clearly, we compare the computational and experimental results at the 200th cycle as shown in \ref{Fig:200th_Exp_Sim}.
%It is concluded from the figures that the simulations and predictions are in good agreement with the corresponding experimental results at all temperatures.
%More importantly, the cyclic softening occurred in each simulation case.
%Comparison of experimental and predicted peak stresses are shown in \ref{Fig:Compare_PACC-PV_stress_temperature}, cyclic softening behavior can be observed clearly.
%The experimental and predicted peak stresses are completely fitting after 100 cycles.
%Also at the beginning the results are in good agreement
%
%\begin{figure}[!htp]
%\centering\scalebox{0.9}{\includegraphics{eps/Axial_Compare_1.pdf}}
%\caption{Comparison of experimental and computational results under uniaxial loading in different temperatures.}
%\label{Fig:Axial_Exp_Sim}
%\end{figure}
%
%\begin{figure}[!htp]
%\centering\scalebox{0.5}{\includegraphics{IN718_Isothermal_200thCycle_Exp_vs_Sim}}
%\caption{A comparison of stress-strain curves between experimental results and simulation results at 200th cycle.}
%\label{Fig:200th_Exp_Sim}
%\end{figure}
%
%The cyclic deformation curves were constructed by plotting the amplitudes from peak tensile to peak compressive stress as a function of logarithm of the number of cycles.
%\ref{Fig:IN718_Isothermal_Axial+-1_PV_Exp_vs_Sim_2} depicts the cyclic deformation results for Inconel 718 at different temperatures with total strain range $\pm1\%$.
%It is observed that the alloy exhibits a continuous fatigue softening to failure at each temperature, while at low temperature amplitudes the fatigue softening is followed by a well-defined saturation stage.
%\begin{figure}[!htp]
%\centering\scalebox{0.6}{\includegraphics{eps/Compare_PACC-PV_stress_temperature.eps}}
%\caption{Comparison of experimental and predicted peak stresses.}
%\label{Fig:Compare_PACC-PV_stress_temperature}
%\end{figure}
%
%\begin{figure}[!htp]
%\centering\scalebox{0.9}{\includegraphics{eps/Circle_Compare_1.pdf}}
%\caption{Comparison of experimental and computational results of circle path strain controlled hysteresis in different temperatures.}
%\label{Fig:Circle_Exp_Sim}
%\end{figure}
%
%\begin{figure}[!htp]
%\centering\scalebox{0.5}{\includegraphics{IN718_Isothermal_Axial+-1_PV_Exp_vs_Sim_1.pdf}}
%\caption{Comparison of experimental and computational results of circle path strain controlled hysteresis in different temperatures.}
%\label{Fig:IN718_Isothermal_Axial+-1_PV_Exp_vs_Sim_1}
%\end{figure}
%
%\begin{figure}[!htp]
%\centering\scalebox{0.5}{\includegraphics{IN718_Isothermal_Axial+-1_PV_Exp_vs_Sim_2.pdf}}
%\caption{Comparison of experimental and computational results of circle path strain controlled hysteresis in different temperatures.}
%\label{Fig:IN718_Isothermal_Axial+-1_PV_Exp_vs_Sim_2}
%\end{figure}
%
%\begin{figure}[!htp]
%\centering\scalebox{0.12}{\includegraphics{TMF_Speceimen_After_Experiment.png}}
%\caption{IN718.}
%\label{Fig:TMF_Speceimen_After_Experiment}
%\end{figure}
%
%%\begin{figure}[!htp]
%%\centering\scalebox{0.5}{\includegraphics{MisesStressWithoutNPHardening.pdf}}
%%\caption{300$^\circ$Cê±£?XDí?·??£?ê??é?á1?ó??????£?aμ?μèD§ó|á|±è??£?·?±èày???ˉ2?êyμèóú0.}
%%\label{Fig:MisesStressWithoutNPHardening}
%%\end{figure}
%%
%%\begin{figure}[!htp]
%%\centering\scalebox{0.5}{\includegraphics{MisesStressWithNPHardening.pdf}}
%%\caption{300$^\circ$Cê±£?XDí?·??£?ê??é?á1?ó??????£?aμ?μèD§ó|á|±è??£?·?±èày???ˉ2?êyμèóú200.}
%%\label{Fig:MisesStressWithNPHardening}
%%\end{figure}
%\subsection{Multiaxial simulation under isothermal condition}
%\begin{figure}
%  \begin{minipage}[t]{0.5\linewidth}
%  \nonumber
%    \centering
%    \includegraphics[width=3.5in]{IN718_Isothermal_300C_7049_XPath_Exp_vs_Sim_0.pdf}
%    \centerline{(a) ·?±èày???ˉ2?êyμèóú0MPa.}
%    \label{Fig:IN718_Isothermal_300C_7049_XPath_Exp_vs_Sim_0}
%  \end{minipage}%
%  \begin{minipage}[t]{0.5\linewidth}
%    \centering
%    \includegraphics[width=3.5in]{IN718_Isothermal_300C_7049_XPath_Exp_vs_Sim_200.pdf}
%    \centerline{(b) ·?±èày???ˉ2?êyμèóú200MPa.}
%    \label{Fig:IN718_Isothermal_300C_7049_XPath_Exp_vs_Sim_200}
%  \end{minipage}
%  \caption{300$^\circ$Cê±£?XDí?·??£?ê??é?á1?ó??????£?aμ??á?òó|á|ó???ó|á|.}
%  \label{Fig:IN718_Isothermal_300C_7049_XPath_Exp_vs_Sim}
%\end{figure}
%
%\begin{figure}
%  \begin{minipage}[t]{0.5\linewidth}
%  \nonumber
%    \centering
%    \includegraphics[width=3.5in]{MisesStressWithoutNPHardening.pdf}
%    \centerline{(a) ·?±èày???ˉ2?êyμèóú0MPa.}
%    \label{Fig:MisesStressWithoutNPHardening}
%  \end{minipage}%
%  \begin{minipage}[t]{0.5\linewidth}
%    \centering
%    \includegraphics[width=3.5in]{MisesStressWithNPHardening.pdf}
%    \centerline{(b) ·?±èày???ˉ2?êyμèóú200MPa.}
%    \label{Fig:MisesStressWithNPHardening}
%  \end{minipage}
%  \caption{300$^\circ$Cê±£?XDí?·??£?ê??é?á1?ó??????£?aμ?μèD§ó|á|±è??.}
%  \label{Fig:IN718_Isothermal_300C_7049_XPath_Exp_vs_Sim_MisesStress}
%\end{figure}
%
%\subsection{Uniaxial simulation under thermomechanical condition}
%\begin{figure}
%  \begin{minipage}[t]{0.5\linewidth}
%  \nonumber
%    \centering
%    \includegraphics[width=3.5in]{TMF/TMF_IP_Path.pdf}
%    \centerline{(a) ???è?-?·ó?ó|±??-?·.}
%    \label{Fig:TMF_IP_Path}
%  \end{minipage}%
%  \begin{minipage}[t]{0.5\linewidth}
%    \centering
%    \includegraphics[width=3.5in]{TMF/7018_TMF_+-1_IP_EXP_TIMED.pdf}
%    \centerline{(b) ê??éó|á|ó|±??ú??.}
%    \label{Fig:7018_TMF_+-1_IP_EXP_TIMED}
%  \end{minipage}
%
%  \begin{minipage}[t]{0.5\linewidth}
%  \nonumber
%    \centering
%    \includegraphics[width=3.5in]{TMF/IN718_TMF_Axial+-1_PV_Exp_vs_Sim_IP.pdf}
%    \centerline{(c) ó|á|·?1è?μ±è??.}
%    \label{Fig:IN718_TMF_Axial+-1_PV_Exp_vs_Sim_IP}
%  \end{minipage}%
%  \begin{minipage}[t]{0.5\linewidth}
%    \centering
%    \includegraphics[width=3.5in]{TMF/7018_TMF_+-1_IP_SIM_TIMED.pdf}
%    \centerline{(d) êy?μ?£?aó|á|ó|±??ú??.}
%    \label{Fig:7018_TMF_+-1_IP_SIM_TIMED}
%  \end{minipage}
%  \caption{í??à??μ?èè?ú?£àíê??éó?êy?μ?£?a£????è300-650$^\circ$Cê±.}
%  \label{Fig:TMF_IP}
%\end{figure}
%
%\begin{figure}
%  \begin{minipage}[t]{0.5\linewidth}
%  \nonumber
%    \centering
%    \includegraphics[width=3.5in]{TMF/TMF_OP_Path.pdf}
%    \centerline{(a) ???è?-?·ó?ó|±??-?·.}
%    \label{Fig:TMF_OP_Path}
%  \end{minipage}%
%  \begin{minipage}[t]{0.5\linewidth}
%    \centering
%    \includegraphics[width=3.5in]{TMF/7017_TMF_+-1_OP_EXP_TIMED.pdf}
%    \centerline{(b) ê??éó|á|ó|±??ú??.}
%    \label{Fig:7017_TMF_+-1_OP_EXP_TIMED}
%  \end{minipage}
%
%  \begin{minipage}[t]{0.5\linewidth}
%  \nonumber
%    \centering
%    \includegraphics[width=3.5in]{TMF/IN718_TMF_Axial+-1_PV_Exp_vs_Sim_OP.pdf}
%    \centerline{(c) ó|á|·?1è?μ±è??.}
%    \label{Fig:IN718_TMF_Axial+-1_PV_Exp_vs_Sim_OP}
%  \end{minipage}%
%  \begin{minipage}[t]{0.5\linewidth}
%    \centering
%    \includegraphics[width=3.5in]{TMF/7017_TMF_+-1_OP_SIM_TIMED.pdf}
%    \centerline{(d) êy?μ?£?aó|á|ó|±??ú??.}
%    \label{Fig:7017_TMF_+-1_OP_SIM_TIMED}
%  \end{minipage}
%  \caption{·′?à??μ?èè?ú?£àíê??éó?êy?μ?£?a£????è300-650$^\circ$Cê±.}
%  \label{Fig:TMF_OP}
%\end{figure}
%
%\begin{figure}
%  \begin{minipage}[t]{0.5\linewidth}
%  \nonumber
%    \centering
%    \includegraphics[width=3.5in]{TMF/TMF_90_Path.pdf}
%    \centerline{(a) ???è?-?·ó?ó|±??-?·.}
%    \label{Fig:TMF_90_Path}
%  \end{minipage}%
%  \begin{minipage}[t]{0.5\linewidth}
%    \centering
%    \includegraphics[width=3.5in]{TMF/7025_TMF_+-1_90_EXP_TIMED.pdf}
%    \centerline{(b) ê??éó|á|ó|±??ú??.}
%    \label{Fig:7025_TMF_+-1_90_EXP_TIMED}
%  \end{minipage}
%
%  \begin{minipage}[t]{0.5\linewidth}
%  \nonumber
%    \centering
%    \includegraphics[width=3.5in]{TMF/IN718_TMF_Axial+-1_PV_Exp_vs_Sim_90.pdf}
%    \centerline{(c) ó|á|·?1è?μ±è??.}
%    \label{Fig:IN718_TMF_Axial+-1_PV_Exp_vs_Sim_90}
%  \end{minipage}%
%  \begin{minipage}[t]{0.5\linewidth}
%    \centering
%    \includegraphics[width=3.5in]{TMF/7025_TMF_+-1_90_SIM_TIMED.pdf}
%    \centerline{(d) êy?μ?£?aó|á|ó|±??ú??.}
%    \label{Fig:TMF/7025_TMF_+-1_90_SIM_TIMED}
%  \end{minipage}
%  \caption{90$^{\circ}$?à??μ?èè?ú?£àíê??éó?êy?μ?£?a£????è300-650$^\circ$Cê±.}
%  \label{Fig:TMF_90}
%\end{figure}

\section{Summaries}
\noindent
In the above chapters, the multiaxial thermomechanical behavior of the nickel-based superalloy Inconel 718 has been investigated experimentally and computationally.
Based on the Ohno-Wang's model, a constitutive model has been developed for multiaxial thermomechanical cyclic plasticity.
An implicit computational integration algorithm for the constitutive model has been developed and implemented into the general purpose commercial finite element code ABAQUS. The computational predictions confirm that the model can describe elastic-plastic mechanical behavior under most different thermomechanical loading conditions.
From the present experimental and computational investigation the following conclusions can be drawn:
\begin{itemize}

\item {The kinematic hardening can represent the cyclic loading, and the isotropic hardening can consider the non-proportional loading path effects in the constitutive model.
Complex variations in the peak and valley stresses can be described by the kinematic hardening model properly. The numerical predictions present a good agreement with the experiments in varying temperatures.}

\item {Computations confirm the significance of the non-proportional hardening under multiaxial loading conditions and agree with experiments under different loading conditions. Generally, the prediction of the constitutive model is reasonable under both proportional and non-proportional loadings.}

\item {The temperature-dependent material parameters are determined under isothermal conditions. The results reveal the constitutive model can approach thermomechanical behavior reasonably under both axial and multiaxial thermomechanical loadings.}

\item{The complex loading path may induce additional strain hardening in the compressive normal stress, which cannot be caught by the present constitutive model. To quantify this effect, more detailed experiments are necessary.} 

\end{itemize}