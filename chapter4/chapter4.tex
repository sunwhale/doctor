\chapter{Constitutive modeling under theromechanical fatigue loading}

%当前国际上通用的循环塑性模型大多在屈服面方程、流动法则方面都采用一般塑性通用的法则。
%这些法则既适合循环荷载,也适合非循环荷载。各
%种塑性模型的主要不同点在于硬化准则。
%并且随着加载方式的逐渐复杂化,循环塑性模型的硬化准则也被不断的修正。
%最初的硬化准则是线性硬化准则,随着研究的深入,Armstrong-Frederic提出了著名的非线性硬化准则。
%随着加载的复杂化,后人又在Armstrong-Frederick 模型基础上进行了一系列的修正,以期满足在循环加载过程中出现的棘轮效应以及非比例硬化等现象。
%循环塑性的基本模型包含三个部分:(1)屈服准则,(2)流动法则,(3)硬化准则。
%循环塑性模型和单调塑性模型相比,除去硬化准则不同外,其屈服准则和流动法则是完全相同的。下
%面就分别就三个准则进行论述。
%我们一般认为,当材料的塑性应变为零时满足弹性应力-应变关系,直至应力达到屈服条件。
%两个最常用的屈服准则是:Tresca 应力屈服准则和 Von Mises应力屈服准则。
%由于 Tresca 应力屈服准则在数学描述上出现了拐点,其连续性受到影响,所以现在多数研究者使用 Von Mises 应力屈服准则。
%\begin{equation}
%F = \sqrt {\frac{3}{2}\left( {{\mathbf{s}} - {\mathbf{a}}} \right):\left( {{\mathbf{s}} - {\mathbf{a}}} \right)}  - Y
%\end{equation}
%当屈服表面的形状不变时,即背应力值为零,只有屈服表面的半径是变化的时候,一般称为等向强化。
%当屈服表面的半径不变时,屈服表面可以移动但是不会改变形状和大小,即k 是不变的,生变化时,则称为随动强化。
%对于循环塑性来说,一般假定屈服表面可以移动但是不会改变形状和大小,即为随动强化。
%“流动法则”将塑性应变增量和偏应力增量联系起来。金属材料通常应用的法则是相关的流动法则或者正态条件(Drucker公设)。
%根据假定,该应力状态下的塑性应变增量是和此应力状态下的屈服表面的外法向共线的。其数学表达式为:
%\begin{equation}
%{{\bm{\dot \upepsilon }}^{\rm{p}}} = \dot \lambda \frac{{\partial F}}{{\partial {\bm{\upsigma }}}}
%\end{equation}

\section{Introduction}
\noindent
Gas turbine components are exposed in high temperature and experience cyclic multi-axial thermo-mechanical loadings. The increasing performance requirements to modern gas turbines challenge  the material operating limits. Quantified characterization of deformation and fatigue behavior of materials under realistic conditions becomes increasingly of importance for industrial application \cite{harrison1996modelling}, including variations of structure temperature, mechanical loads as well as environmental conditions. Inconel 718 is the most popular nickel-based superalloy in turbine industry. It is an oxidation- and corrosion-resistant material possessing optimal thermal and mechanical property. It is used for both rotors and casings, so that the low cycle fatigue characterization of the material is of great significance for mechanical design. The operating temperature of Inconel 718 is limited to 923K for the life limited parts, such as discs and shafts.
A turbine suffers from high temperature low cycle fatigue (LCF) as well as thermo-mechanical fatigue (TMF).
The LCF and TMF behaviors of the nickel-based superalloy are the major concern for the safety and reliability of gas turbine engines.

Predicting fatigue performance of mechanical components needs a reliable constitutive description of the material. There were numerous studies on low cycle fatigue of the nickel-based superalloy published. In the past decades numerous constitutive models for cyclic inelasticity have been published \cite{ohno1993kinematic, Pun2014138, AbdelKarim2000225, Kang2004299}, based on the concept introduced by Armstrong and Frederick \cite{armstrong1966mathematical}. Kobayashi et al. \cite{Kobayashi2008389} and Pereira and Lerch \cite{Pereira2001715} studied Inconel 718 behavior using the Johnson-Cook constitutive relation and, however, did not take the cyclic softening behavior into account.
Bai and Wierzbicki \cite{Bai20081071} modified the classical metal plasticity by taking the dependency of the Lode angle and stress triaxiality into account, which was extended by Algarni et al.  \cite{Algarni2015140} to describe the evolution of yield surface under monotonic loading conditions. Becker and Hackenberg \cite{Becker2011596} suggested a limit surface concept and described the material behavior for a wide temperature range under monotonic and cyclic loading.
Farrahi  \cite{Farrahi2014245} applied two plasticity approaches including the spring-slider rule of Nagode and the hardening rule of Chaboche to simulate cyclic behaviors.
Ohmenh\"{a}user (2014) \cite{Ohmenhauser2014631} adopted a two-layer rheological constitutive model for cyclic thermo-mechanical loading conditions.
However, neither non-proportionality nor multi-axiality of the applied loads was considered under thermo-mechanical cyclic loading conditions.
The alloy Inconel 718 exhibited a pronounced initial cyclic hardening then became continuous cyclic softening at high strain amplitudes. Experiments revealed that the conventional constitutive equations are not suitable to give correct stress and strain predictions for Inconel 718. The conventional constitutive models have to be improved to take the multi-axial, thermo-mechanical and cyclic loading conditions into account.

Zhu et al. \cite{ZHU2016} introduced a cyclic elasto-viscoplastic constitutive model for thermo-mechanically coupled problems. The model introduced the energy conservation in viscoplastic deformations and considered heating exchange due to high strain rate as well as its effects to mechanical behavior of the material. The work is interesting for metal forming, but not related to thermo-mechanical fatigue, in which the description of materials mechanical behavior under complex loading condition is of importance. 
Development of a reliable cyclic plasticity with experimental verification is necessary for predicting thermo-mechanical fatigue performance.

The present work focuses on the multi-axial cyclic behavior and the stress relaxations of Inconel 718 at varying temperatures between 573K and 923K. Effects of non-proportional cyclic loading as well as thermo-mechanical coupling are studied. Inconel 718 was proven to be hardly influenced by the strain rate below 923K under realistic loading speed \cite{kim1988elevated, Schlesinger2017}, and the constitutive model is assumed to be rate-independent. However,  experiments reveal complex constitutive behavior of the material, such as cyclic hardening/softening, non-proportional hardening, thermo-mechanical phase effect, non-masing effect etc. Quantitative description of the material behavior needs a constitutive model based on extensive experimental investigation. In the present work a modified cyclic plasticity model is suggested for the nickel-based superalloy Inconel 718 under multi-axial thermo-mechanical cyclic loading conditions and can give a uniform description of the material modeling for both isothermal and thermo-mechanical fatigue loading components.

\section{Continuum plasticity}
\noindent
Continuum plasticity is a solid mechanics theory that is used to describe the plastic behavior of materials.
It is characterized by the assumption that a flow rule exists that can be used to determine the amount of plastic strain in the material.
In the section, the fundamentals of time-independent phenomenological plasticity are introduced, such as invariants of the stress tensor, strain decomposition, yield surface, flow rule, Drucker postulates, consistency condition, isotropic and kinematic hardening.

\subsection{Invariants of the stress tensor}
\noindent
In continuum mechanics, the stress state at any material point can be represented by the stress tensor $\bm{\sigma}$.
The stress tensor can be represented by a matrix of second order in an arbitrary coordinate system as:
\begin{equation}
\bm{\sigma} =
  \begin{bmatrix}
    \sigma_{1\,1}& \sigma_{1\,2}& \sigma_{1\,3}\\
    \sigma_{2\,1}& \sigma_{2\,2}& \sigma_{2\,3}\\
    \sigma_{3\,1}& \sigma_{3\,2}& \sigma_{3\,3}
  \end{bmatrix}.
\end{equation}
Since the stress tensor is symmetric, only six stress components axe independent.
Thus, six independent stress components determine a stress state uniquely and visa versa.
Using the above stress tensor, the three principal stresses can be determined using the characteristic equation
\begin{equation}
\det \left( {{\bm{\sigma }} - \sigma {\bf{I}}} \right) = 0,
\end{equation}
which can be expanded to
\begin{equation}
{\sigma ^3} - {I_1}{\sigma ^2} + {I_2}\sigma  - {I_3} = 0.
\end{equation}
where $\bf{I}$ is the second order identity tensor, ${I_1}$, ${I_2}$, and ${I_3}$ are the invariants of the stress tensor.
In terms of the principal stresses, $\sigma_1$, $\sigma_2$ and $\sigma_3$, the invariants are
\begin{equation}
\begin{array}{*{20}{l}}
{{I_1} = {\sigma _1} + {\sigma _2} + {\sigma _3}},\\
{{I_2} = {\sigma _1}{\sigma _2} + {\sigma _2}{\sigma _3} + {\sigma _3}{\sigma _1}},\\
{{I_3} = {\sigma _1}{\sigma _2}{\sigma _3}}.
\end{array}
\end{equation}
In plasticity theory, it is customary to decompose the stress tensor as
\begin{equation}
{\bm{\sigma }} = {\bf{s}} + {\sigma _m}{\bf{I}},
\end{equation}
where $\sigma_m$ is the hydrostatic stress given by
\begin{equation}
{\sigma _m} = {\rm{Tr}}\left( {\bm{\sigma }} \right) = \frac{1}{3}\left( {{\sigma _1} + {\sigma _2} + {\sigma _3}} \right),
\end{equation}
and $\bf{s}$ is the deviatoric stress tensor.
Using the deviatoric stress tensor, the deviatoric stress invariants can be defined as
\begin{equation}
\begin{array}{*{20}{l}}
{{J_1} = {\rm{Tr}}\left( {\bf{s}} \right) = 0},\\
{{J_2} = \frac{1}{2}{\bf{s}}:{\bf{s}} = \frac{1}{6}\left[ {{{\left( {{\sigma _1} - {\sigma _2}} \right)}^2} + {{\left( {{\sigma _2} - {\sigma _3}} \right)}^2} + {{\left( {{\sigma _3} - {\sigma _1}} \right)}^2}} \right]},\\
{{J_3} = \det \left( {\bf{s}} \right) = {s_1}{s_2}{s_3}}.
\end{array}
\end{equation}

In plasticity mechanics, the first principal invariant $I_1$ of the Cauchy stress $\boldsymbol{\sigma}$, and the second and third principal invariants $J_2$, $J_3$ of the deviatoric part $\boldsymbol{s}$ of the Cauchy stress are usually used to express the yield surface.


\subsection{Strain decomposition}
\noindent
It is shown in \ref{Fig:strain_decomposition} that the stress-strain curve can be obtained from a uniaxial tensile test.
Both elastic and inelastic regions are indicated.
In elastic region, the stress is proportional to the strain if the stress in the specimen is below a certain value, i.e. the yield stress $\sigma_{y0}$, which is the elastic limit.
As a material is loaded beyond its elastic limit, Hooke's law does not apply, the material no longer exhibits elastic behavior and the stress-strain relation becomes nonlinear. The material yields, begins to flow and residual, permanent deformation results after unloading.

For strain hardening materials the yield stress increases with increasing plastic deformation to a value of $\sigma_y$.
It is called hardening because the stress is increasing relative to perfect plastic behavior, also shown in the figure.
\begin{figure}[ht]
	\centering
	\includegraphics[width=14cm]{strain_decomposition.pdf}
	\caption{The classical decomposition of strain into elastic and plastic parts.}
	\label{Fig:strain_decomposition}
\end{figure}
If, at a strain of $\varepsilon$, the loading were to be reversed, the material would cease to deform plastically (at least in the absence of time-dependent effects) and would show a linearly decreasing stress with strain such that the gradient of this part of the stress-strain curve would again be the Young's modulus, E, shown in \ref{Fig:strain_decomposition}.
Once a stress of zero is achieved (provided the material remains elastic on full reversal of the load), the strain remaining in the test specimen is the plastic strain, $\varepsilon_{\rm{p}}$.
The recovered strain, $\varepsilon_e$, is the elastic strain and it can be seen that the total strain, $\varepsilon$, is the sum of the two
\begin{equation}
\varepsilon = \varepsilon_{\rm{e}} + \varepsilon_{\rm{p}}
\end{equation}
This is called the classical additive decomposition of strain.
It is also apparent from \ref{Fig:strain_decomposition} that the stress achieved at a strain of $\varepsilon$ is given by
\begin{equation}
\sigma = E\varepsilon_{\rm{e}} = E(\varepsilon - \varepsilon_{\rm{p}})
\end{equation}

\subsection{Yield criterion}
\noindent
A yield criterion is a hypothesis defining the limit of elasticity in a material and the onset of plastic deformation under any possible combination of stresses.
The yield criterion can be described by a yield surface which is a super surface in the space of stresses.
The yield surface is usually convex and the state of stress of inside the yield surface is elastic.
When the stress state lies on the surface the material is said to have reached its yield point and the material is said to have become plastic.
Further deformation of the material causes the stress state to remain on the yield surface, even though the shape and size of the surface may change as the plastic deformation evolves.

In the stress space, the yield surface can be expressed as
\begin{equation}
F\left( {{\sigma _{ij}}} \right) = k,
\label{Equ:yield_function_1}
\end{equation}
where $F\left( {{\sigma _{ij}}} \right)$ is a function of the stress tensor and $k$ is a critical value. Generally, Eq. (\ref{Equ:yield_function_1}) is called yield function. In the principal stress space, the Eq. (\ref{Equ:yield_function_1}) can be expressed as
\begin{equation}
F\left( {{\sigma _1},{\sigma _2},{\sigma _3}} \right) = k,
\end{equation}
where ${\sigma _1},{\sigma _2},{\sigma _3}$ are the principal stresses.

There are several possible yield criteria. The Tresca and von Mises yield criterions are commonly used for metallic materials. The Tresca yield criterion is also known as the maximum shear stress theory. It suggests that yielding of a ductile material begins when the maximum shear stress reaches a critical value. In terms of the principal stresses, the Tresca criterion is expressed as
\begin{equation}
\frac{1}{2}\max \left( {\left| {{\sigma _1} - {\sigma _2}} \right|,\left| {{\sigma _2} - {\sigma _3}} \right|,\left| {{\sigma _3} - {\sigma _1}} \right|} \right) = \tau_y,
\end{equation}
where $\tau_y$ is the yield strength in shear.

The von Mises yield criterion is also known as the maximum distortion energy criterion. It suggests that yielding of a ductile material begins when the second deviatoric stress invariant $J_2$ reaches a critical value. In terms of the principal stresses, the von Mises yield criterion is expressed as
\begin{equation}
{\left( {{\sigma _1} - {\sigma _2}} \right)^2} + {\left( {{\sigma _2} - {\sigma _3}} \right)^2} + {\left( {{\sigma _3} - {\sigma _1}} \right)^2} = 2\sigma _y^2,
\end{equation}
where $\sigma_y$ is the yield strength in tension.

\ref{Fig:YieldSurface3D} shows the Tresca and von Mises yield surfaces in principal stress coordinates. 
% There are many others including that of Drucker and the Gurson model for porous materials.
In the study, the von Mises yield criteriona are used.

\begin{figure}[!htp]
\centering
\includegraphics[width=8cm]{yield_surface_2D.pdf}
\caption{The von Mises and Tresca yield surfaces in principal stress coordinates.}
\label{Fig:YieldSurface3D}
\end{figure}

% \begin{figure}[!htp]
% \centering\scalebox{0.5}{\includegraphics{yield_surface_3D.pdf}}
% \caption{The von Mises and Tresca yield surfaces in principal stress coordinates.}
% \label{Fig:YieldSurface2D}
% \end{figure}

% Let $f$ be a yield function such that $f = 0$ is our yield criterion. Then:

% 1.Yield is independent of the hydrostatic stress.
% Since $f$ is independent of hydrostatic stress, it must be expressible in terms of the deviatoric stresses alone.

% 2.Yield in polycrystalline metals can be taken to be isotropic (provided we are concerned with yield in a volume of material containing many grains) and must therefore be independent of the labelling of the axes.

% 3.Yield stresses measured in compression have the same magnitude as yield stresses measured in tension.

% The von Mises yield function is defined by
% \begin{equation}
% f = \sigma_e-\sigma_y =\sqrt{3J_2}-\sigma_y
% \end{equation}

% The von Mises yield criterion suggests that the yielding of materials begins when the second deviatoric stress invariant $J_2$ reaches a critical value.
% For this reason, it is sometimes called the $J_2$-plasticity or $J_2$ flow theory.
% The yield criterion is given by
% \[f<0\rm{: Elastic\ deformation}\]
% \[f=0\rm{: Plastic\ deformation}\]

%Let us consider the yield function in two-dimensional principal stress space by putting $\sigma_3 = 0$ and so imposing conditions of plane stress.
%Geometrically, this corresponds to finding the intersection between the von Mises cylinder and the plane $\sigma_3 = 0$.

\subsection{Flow rule}
\noindent
% The general mathematical treatment of the constitutive equation for plastic deformation was proposed by von Mises in 1928.
% He noticed that in elasticity theory, the strain tensor was related to the stress through an elastic potential function, the complementary strain energy $U$ such that
% \begin{equation}
% {\bm{\upepsilon }} = \frac{{\partial U}}{{\partial {\bm{\sigma }}}}
% \end{equation}
In elasticity theory, the strain tensor was related to the stress through an elastic potential function.
By generalizing and applying this idea to plasticity theory, Mises proposed that there exists a plastic potential $g(\bm{\sigma})$, such that the plastic strain rate $\dot{{\bm{\upepsilon }}^{\rm{p}}}$ could be derived from the following flow rule:
\begin{equation}
\dot{{\bm{\upepsilon }}^{\rm{p}}} = \dot\lambda \frac{\partial g}{\partial \bm{\sigma}},
\label{Equ:dgdsigma}
\end{equation}
where $\dot\lambda$ is a proportional positive scalar factor which can be determined from the yield criterion. 
% Geometrically, the plastic potential $g(\bm{\sigma})=0$, represents a surface in the stress space and $\dot{{\bm{\upepsilon }}^{\rm{p}}}$ can be represented by a vector in this space.
% The plastic strain rate vector is normal to $g(\bm{\sigma})=0$.
% Therefore, Equation \ref{Equ:dgdsigma} is also referred to as the normality flow rule in plasticity theory.
Common approach in plasticity theory is to assume that the plastic potential function $g(\bm{\sigma})$ is the same as the yield function $f(\bm{\sigma})$.
It is typically assumed that the plastic strain increment and the normal to the yield surface have the same direction, so that the flow rule can be written as:
%In metal plasticity, the assumption that the plastic strain increment and deviatoric stress tensor have the same principal directions is encapsulated in a relation called the flow rule.
\begin{equation}
\dot{{\bm{\upepsilon }}^{\rm{p}}} = \dot\lambda \frac{\partial f}{\partial \bm{\sigma}}.
\end{equation}
This form of the flow rule is called the associated flow rule.
% On the other hand, if $g \neq f$, the flow rule is called nonassociated.
In general, experimental observations show that inelastic deformation of metals can be characterized quite well by an associated flow rule, but for some porous materials a nonassociated flow rule provides a better representation of inelastic deformation.

% The above flow rule is easily justified for perfectly plastic deformations for which $d\boldsymbol{\sigma} = 0$  when $d{{\bm{\upepsilon }}^{\rm{p}}} > 0$, i.e., the yield surface remains constant under increasing plastic deformation. This implies that the increment of elastic strain is also zero, $d{{\bm{\upepsilon }}^{\rm{e}}} = 0$, because of Hooke's law.
% Therefore,
% \begin{equation}
% d\boldsymbol{\sigma}:\frac{\partial f}{\partial \boldsymbol{\sigma}} = 0 \quad \text{and} \quad d\boldsymbol{\sigma}:d{{\bm{\upepsilon }}^{\rm{p}}} = 0 \,.
% \end{equation}
% Hence, both the normal to the yield surface and the plastic strain tensor are perpendicular to the stress tensor and must have the same direction.
For a work hardening material, the yield surface can expand with increasing stress.
We assume Drucker's second stability postulate which states that for an infinitesimal stress cycle this plastic work is positive, i.e.,
\begin{equation}
d\boldsymbol{\sigma}: d{{\bm{\upepsilon }}^{\rm{p}}} \ge 0 \,.
\end{equation}
The above quantity is equal to zero for purely elastic cycles.
Examination of the work done over a cycle of plastic loading-unloading can be used to justify the validity of the associated flow rule.
% The Prager consistency condition is needed to close the set of constitutive equations and to eliminate the unknown parameter $d\lambda$ from the system of equations.
% The consistency condition states that $df = 0$  at yield because  $f(\boldsymbol{\sigma},{{\bm{\upepsilon }}^{\rm{p}}}) = 0$ , and hence
% \begin{equation}
% df = \frac{\partial f}{\partial \boldsymbol{\sigma}}:d\boldsymbol{\sigma} + \frac{\partial f}{\partial \boldsymbol{\varepsilon}_p}:d\boldsymbol{\varepsilon}_p = 0 \,.
% \end{equation}

\subsection{Hardening rule}
\noindent
The hardening rule describes how the yield surface changes in position and size.
Evolution equations of the back stress tensor allows by the change in position of the kinematic hardening into account.
With this shift in the yield surface, the size and shape of the yield surface remain unchanged.

\begin{figure}[htbp]
	\centering
	\includegraphics[width=14cm]{isotropic_hardening.pdf}
	\caption{Isotropic hardening, in which the yield surface expands with plastic deformation, and the corresponding uniaxial stress–strain curve.}
	\label{Fig:isotropic_hardening}
\end{figure}

\begin{figure}[htbp]
	\centering
	\includegraphics[width=14cm]{kinematic_hardening.pdf}
	\caption{Kinematic hardening, in which the yield surface translates with plastic deformation, and the corresponding uniaxial stress–strain curve.}
	\label{Fig:kinematic_hardening}
\end{figure}

In material models in which isotropic hardening is depicted, it is done by evolution equations describing an increase in the radius $k$ of the yield surface.
A modeling that represents isotropic hardening is not suitable for cyclic stresses.
In such cases, the yield surface is expanded as far as it finally only purely elastic state occurs (elastic "Shake Down").
The differences of the material models described in the next section consist kinematic hardening only in the respectively used hardening rules to take into account.

\section{A review of some cyclic plasticity theories}
\subsection{Kinematic hardening models}
\noindent
In the case of monotonically increasing loading, it is often reasonable to assume that any hardening that occurs is isotropic.
For the case of cyclic loading (increasing loading and reversed loading), however, this is often not appropriate.
Because isotropic hardening leads to a very large elastic region, on reversed loading, which is often not to be observed in experiments.
In fact, a much smaller elastic region is expected and this results from what is often called the Bauschinger effect, and
kinematic hardening.
In kinematic hardening, the yield surface translates in stress space, rather than expanding.
The yield function describing the yield surface must now also depend on the location of the surface in stress space, i.e.,
\begin{equation}
F = \sqrt {\frac{3}{2}\left( {{\bf{s}} - {\bf{a}}} \right):\left( {{\bf{s}} - {\bf{a}}} \right)}  - Y,
\end{equation}
in which $\bf{a}$ is the kinematic hardening variable and is often called the back stress.
Because it is a variable defined in stress space, it has the same components as stress.
A kinematic hardening model must include the evolution functions of back stress.
A review of some kinematic hardening models is introduced as following.

%It is generally suggested that in the case of cyclic loading, isotropic hardening rules are not suitable to simulate the Bauschinger effect.
%Hence the kinematic hardening model is used to represent Bauschinger effect.
%It is assumed that the size of yield surface
\subsection{Prager's model}
\noindent
Prager (1949) \cite{prager1949recent} introduce a simple linear kinematic hardening model, in which the evolution of the back-stress is collinear with the evolution of the plastic strain:
\begin{equation}
{\dot{\bf a}} = \frac{2}{3}c{\bm{\dot \upepsilon }}^{\rm{p}},
\end{equation}
and
\begin{equation}
{\bf{ a}} = \frac{2}{3}c{\bm{ \upepsilon }}^{\rm{p}},
\end{equation}
in which $c$ is a material constant, the increment in kinematic hardening to be proportional to the increment
in plastic strain.

\subsection{Armstrong and Frederick's model}
\noindent
A better description is given by the model proposed initially by Armstrong and Frederick (1966) \cite{armstrong1966mathematical} introducing a recall term, called dynamic recovery:
\begin{equation}
{\dot {\bf a}} = \frac{2}{3}c{\bm{\dot \upepsilon }}^{\rm{p}} - \gamma {\bf{a}}\dot p,
\end{equation}
in which $c,\gamma$ are material constants.
The second term on the right hand side is called the recall term which is collinear with back stress $\bf{a}$ and is proportional to the norm of the plastic strain rate. The evolution of $\bf{a}$ instead of being linear, is then exponential for a monotonic uniaxial loading, with a saturation for a value $c/\gamma $.

\subsection{Chaboche's model}
\noindent
Nonlinear hardening rules based on the Armstrong and Frederick (A-F) relation have been expressed in the form of a series expansion of the back stress.
It was postulated by Chaboche (1983) \cite{chaboche1983plastic} that the total back stress is composed of additive parts:
\begin{equation}
{\bf{a}} = \sum\limits_{i = 1}^M {{{\bf{a}}_i}},
\end{equation}
where ${\bf{a}}$ is the total back stress, ${{{\bf{a}}_i}}$ is a part of the total back stress, $i = 1\sim M$ , and $M$ is the number of back stress parts considered. Each back stress, ${{{\bf{a}}_i}}$ follows an A-F type relation:
\begin{equation}
{{\dot{\bf a}}_i} = \frac{2}{3}{c_i}{{\bm{\dot \upepsilon }}^{\rm{p}}} - {\gamma _i}{{\bf{a}}_i}\dot p.
\end{equation}

Instead of using a nonlinear dependency (of the back-stress norm) in the dynamic recovery term, it was proposed by Chaboche (1991) \cite{Chaboche1991661} to introduce a threshold. The structure of the additional back-stress evolution equation is then:
\begin{equation}
{\dot{\bf a}}_i = \frac{2}{3}c_i{{\bm{\dot \upepsilon }}^{\rm{p}}} - {\gamma _i} \left\langle {1 - \frac{{\bf{a}}_i}{{\left\| {\bf{a}}_i \right\|}}} \right\rangle {\bf{a}}_i\dot p,
\end{equation}
where
\begin{equation}
\left\| {\bf{a}}_i \right\| = \sqrt {{\bf{a}}_i:{\bf{a}}_i}.
\end{equation}

\subsection{Ohno and Wang's model}
\noindent
After the model with a threshold in the dynamic recovery term, it was proposed by Ohno and Wang (1993) \cite{ohno1993kinematic} a slightly different approach, introducing a critical state of dynamic recovery.
The total back stress is composed of additive parts:
\begin{equation}
{\dot{\bf a}} = \sum\limits_{i = 1}^M {{{{\dot{\bf a}}}_i}},
\end{equation}
For each back stress, ${\bf{a}}_i$, it takes two forms as following.

\textbf{Model I}, the extreme case, corresponding to some multilinear model.
The decomposition of the kinematic hardening variable into components, are based on the assumption that each component has a critical state for its dynamic recovery to be activated fully.
The critical state is introduced by a surface ${f_i} = \left\| {{{\bf{a}}_i}} \right\| - {c_i}/{\gamma _i} = 0$.
The recovery term operates only when the back stress attains this surface.
In that case, under proportional loading, the back-stress rate vanishes. It writes:
\begin{equation}
{{\dot{\bf a}}_i} = \frac{2}{3}{c_i}{{\bm{\dot \upepsilon }}^{\rm{p}}} - {\gamma _i}H\left( {{f_i}} \right)\left\langle {{\bf{n}}:{{\bf{L}}_i}} \right\rangle {{\bf{a}}_i}\dot p,
\end{equation}
in which
\begin{equation}
{{\bf{L}}_i} = \frac{{{{\bf{a}}_i}}}{{\left\| {{{\bf{a}}_i}} \right\|}},
\end{equation}
or
\begin{equation}
{{\dot{\bf a}}_i} = {\zeta _i}\left[ {\frac{2}{3}{r_i}{{{\bm{\dot \upepsilon }}}^{\rm{p}}} - H\left( {{f_i}} \right)\left\langle {{{{\bm{\dot \upepsilon }}}^{\rm{p}}}:\frac{{{{\bf{a}}_i}}}{{{{\bar a}_i}}}} \right\rangle {{\bf{a}}_i}} \right],
\label{Eqn:Introduction2}
\end{equation}
where $H$ is the Heaviside step function, ${\bf{n}}$ the plastic strain rate direction.
The Macaulay bracket introduces the consistency condition for the critical state ${f_i} = {\dot f_i} = 0$.
This version will never predict ratchetting under stationary conditions.

\textbf{Model II}, the smooth version, more realistic, replaces the Heaviside step function by a power function:
\begin{equation}
{{\dot{\bf a}}_i} = \frac{2}{3}{c_i}{{\bm{\dot \upepsilon }}^{\rm{p}}} - {\gamma _i}{\left( {\frac{{\left\| {{{\bf{a}}_i}} \right\|}}{{{c^k}/{\gamma ^k}}}} \right)^{{m_i}}}\left\langle {{\bf{n}}:{{\bf{L}}_i}} \right\rangle {{\bf{a}}_i}\dot p,
\end{equation}
or
\begin{equation}
{{\dot{\bf a}}_i} = {\zeta _i}\left[ {\frac{2}{3}{r_i}{{{\bm{\dot \upepsilon }}}^{\rm{p}}} - {{\left( {\frac{{{{\bar a}_i}}}{r}} \right)}^{{m_i}}}\left\langle {{{{\bm{\dot \upepsilon }}}^{\rm{p}}}:\frac{{{{\bf{a}}_i}}}{{{{\bar a}_i}}}} \right\rangle {{\bf{a}}_i}} \right].
\end{equation}
Though not necessary, the Macaulay bracket $\left\langle {{\bf{n}}:{{\bf{L}}_i}} \right\rangle$ continues to operate in this smooth version, and will play a significant role in reducing ratchetting.

\subsection{Karim and Ohno's model}
\noindent
Abdel-Karim and Ohno (2000) \cite{Abdel2000} proposed a further modification of Ohno and Wang's model.
In the first version of the Ohno and Wang's model, the dynamic recovery of ${{\bf{a}}_i}$ is assumed to take place only in a critical state, which is taken to be a hyper sphere of radius $r_i$ in the space of ${{\bf{a}}_i}$ such as
\begin{equation}
{f_i} = \frac{3}{2}{{\bf{a}}_i}:{{\bf{a}}_i} - r_i^2 = 0.
\end{equation}
Let us assume the two kinds of dynamic recovery terms mentioned above. Then, with Heaviside's step function and Macaulay bracket, the evolution equation of ${{\bf{a}}_i}$ can be expressed as
\begin{equation}
{{\dot{\bf a}}_i} = {\zeta _i}\left[ {\frac{2}{3}{r_i}{{{\bm{\dot \upepsilon }}}^{\rm{p}}} - {\mu _i}{{\bf{a}}_i}\dot p - H\left( {{f_i}} \right)\left\langle {{{\dot \lambda }_i}} \right\rangle {{\bf{a}}_i}} \right],
\label{Equ:Introduction1}
\end{equation}
where $\zeta_i$ and $\mu_i$ are material parameters, and ${\dot \lambda }_i$ is determined to have the following form using the consistency condition ${\dot f_i} = 0$:
\begin{equation}
{\dot \lambda _i} = {\bm{\dot \upepsilon }}^{\rm{p}}:\frac{{{{\bf{a}}_i}}}{{{r_i}}} - {\mu _i}\dot p.
\end{equation}
The second and third terms in the right hand side in Equation (\ref{Equ:Introduction1}) express the dynamic recovery of ${\bf{a}}_i$ based on the Armstrong and Frederick model and the Ohno and Wang model, respectively, whereas the first term is responsible for strain hardening.
\begin{equation}
{{\dot{\bf a}}_i} = {\zeta _i}\left( {\frac{2}{3}{r_i}{{{\bm{\dot \upepsilon }}}^{\rm{p}}} - \left[ {{\mu _i} + H\left( {{f_i}} \right)\left( {1 - {\mu _i}} \right)} \right]\left\langle {{{{\bm{\dot \upepsilon }}}^{\rm{p}}}:\frac{{{{\bf{a}}_i}}}{{{r_i}}}} \right\rangle {{\bf{a}}_i}} \right),
\end{equation}
where $0 \le {\mu _i} \le 1$ Critical examination for this rule indicates that continuous ratchetting with constant rate occurs as long as ${\mu _i} > 0$ while complete shakedown takes place as ${\mu _i} = 0$ since the rule reduces to the original Ohno-Wang multi-linear rule defined by Equation (\ref{Eqn:Introduction2}).

\subsection{Jiang and Sehitoglu's model}
\noindent
Based on the the smooth version of Ohno-Wang's model, Jiang and Sehitoglu \cite{jiang1996modeling} suggested that the backstress can be expressed as
\begin{equation}
{{\dot{\bf a}}_i} = {c_i}{r_i}\left[ {{\bf{n}} - {{\left( {\frac{{\left\| {{{\bf{a}}_i}} \right\|}}{{{r_i}}}} \right)}^{{\chi _i} + 1}}{{\bf{L}}_i}} \right]\dot p,
\end{equation}
where ${{\bf{L}}_i} = {{{\bf{a}}_i}}/{\left\| {{{\bf{a}}_i}} \right\|}$.
Cyclic hardening/softening, progressive elevation or decrease in flow strength with cycles results in cyclic hardening or softening, respectively.
Cyclic hardening can be considered with $c_i$ in the hardening relation Equation (\ref{Eqn:Introduction3}) being a function of the accumulated plastic strain,
\begin{equation}
{c_i} = {c_{i0}}\left( {1 + {a_{i1}}{e^{ - {b_{i1}}p}} + {a_{i2}}{e^{ - {b_{i2}}p}}} \right),
\label{Eqn:Introduction3}
\end{equation}
where $c_{i0}$, $a_{i1}$, $b_{i1}$, $a_{i2}$ and $b_{i2}$ are material constants.
When the material displays monotonic hardening or monotonic softening, only two terms in Equation (\ref{Eqn:Introduction3}) are needed.
When the material behavior is mixed hardening/softening, all three terms are necessary.

\subsection{Kang's model}
\noindent
Kang \cite{kang2004uniaxial} considered the Abdel-Karim and Ohno rule and incorporated the accumulated plastic strain increment within dynamic recovery term rather than the plastic strain increment:
\begin{equation}
{{\dot{\bf a}}_i} = {\zeta _i}\left( {\frac{2}{3}{r_i}{{{\bm{\dot \upepsilon }}}^{\rm{p}}} - \left[ {{\mu _i} + H\left( {{f_i}} \right)\left( {1 - {\mu _i}} \right)} \right]\dot p{{\bf{a}}_i}} \right).
\end{equation}
It is clear that, as $\mu_i=0$ this rule reduces to the modified Ohno-Wang rule defined by Equation (\ref{Eqn:Introduction2}) while it reduces to the Armstrong-Frederick rule when $\mu_i>0$.

\subsection{Chen's model}
\noindent
Chen et al. \cite{chen2004modified} introduced a multiplying factor to the dynamic recovery term that considers non-proportionality of the plastic strain and back stress:
\begin{equation}
{{\dot{\bf a}}_i} = {\zeta _i}\left( {\frac{2}{3}{r_i}{{{\bm{\dot \upepsilon }}}^{\rm{p}}} - {{\left( {\frac{{{{\bar a}_i}}}{{{r_i}}}} \right)}^{{m_i}}}{{\left\langle {{\bf{n}}:\frac{{{{\bf{a}}_i}}}{{{{\bar a}_i}}}} \right\rangle }^{{\chi _i}}}\left\langle {{{{\bm{\dot \upepsilon }}}^{\rm{p}}}:\frac{{{{\bf{a}}_i}}}{{{{\bar a}_i}}}} \right\rangle {{\bf{a}}_i}} \right).
\end{equation}

\section{The cyclic plasticity model}

\subsection{Formulation of the constitutive model}
\noindent
In the framework of the rate independent and initially isotropic plasticity, the strain rate can be additively decomposed into the elastic and plastic  part, as
\begin{equation}
\dotbfepsilon = {\dotbfepsilon}^e + {\dotbfepsilon}^{\rm{p}}.
\end{equation}
The elastic strain and total stress are subjected to Hooke's law,
\begin{equation}
{\dot\bfsigma} = {\mathbb{D}^{\rm{e}}}:{\dot\bfepsilon^e},
\label{Equ:HookesLaw}
\end{equation}
where ${\mathbb{D}^{\rm{e}}}$ is the fourth-order isotropic elastic stiffness tensor. Hereafter, $(:)$ indicates the inner product between two tensors. $({\dot  \bullet })$ denotes the differentiation of $\bullet$ with respect to time.
The evolution of the plastic strain can be determined from the flow rule as
\begin{equation}
{\dotbfepsilon}^{\rm{p}} = \dot \lambda \frac{{\partial F}}{{\partial {\bfsigma}}},
\end{equation}
with
\begin{equation}
F = \sqrt {\frac{3}{2}\left( {\bfs - \bfa} \right):\left( {\bfs - \bfa} \right)}  - Y=0,
\end{equation}
as the yield function, where $\bfs$ denotes the deviatoric part of the stress ${\bfsigma}$, the deviatoric backstress $\bfa$ describes the center of the yield surface $F$ in the deviatoric stress space, the yield stress $Y$ is the radius of the yield surface and $\dot \lambda$ is the scalar to be determined using the consistency condition $\dot F = 0$.

The backstress is the essential ingredient of the cyclic plasticity, as initially introduced by Chaboche \cite{Chaboche1986149}, as
\begin{equation}
\bfa = \sum\limits_{k = 1}^M {{\bfa^k}},
\end{equation}
where $M$ denotes the number of partial backstresses.

Following suggestions by D\"orring et al. \cite{Doerring2003} and Abdel Karim et al. \cite{AbdelKarim20051303}, Fang \cite{fang2015cyclic} suggested that the evolution equation of the backstress can be expressed as
\begin{equation}
\label{Equ:dotak1}
{\dot\bfa^k} = {h^k}\frac{2}{3}{\dotbfepsilon^{\rm{p}}} - H\left( {{f^k}} \right){\dot \omega ^k}\frac{{{\bfa^k}}}{{{r^k}}},
\end{equation}
where ${{{h}}^k} $ and ${{{r}}^k} $ are the state variables introduced to describe cyclic plasticity of the material, $r^k$ is a function of accumulated plastic strain, i.e. ${r^k} = {r^k}\left( p \right)$.  Note no summation convention is applied for $k$. ${{{f}}^k} $ defines a series of the critical state surface in the deviatoric stress space,
\begin{equation}
\label{Equ:fk}
{f^k} = \frac{3}{2}{\bfa^k}:{\bfa^k} - {\left( {{r^k}} \right)^2}.
\end{equation}
Above  $H(\cdot)$ denotes the Heaviside step function. According to Eq. (\ref{Equ:fk}), the recovery term is non-zero only when ${f^k} = 0$. In the article $r^k$ is assumed to be a function of the accumulated plastic strain and ${\dot \omega ^k}$ is derived from the consistent condition ${\dot f^k} = 0$. It follows
\begin{eqnarray}
\label{Equ:dotfk}
{\dot f^k} &=& 3{\bfa^k}:{\dot\bfa^k} - 2{r^k}{\dot r^k} \\ \nonumber
&=& 3{\bfa^k}:\left[ {{h^k}\frac{2}{3}{{\dotbfepsilon}^{\rm{p}}} - H\left( {{f^k}} \right){{\dot \omega }^k}\frac{{{\bfa^k}}}{{{r^k}}}} \right] - 2{r^k}{\dot r^k} = 0.
\end{eqnarray}
When $H\left( {{f^k}} \right) = 1$, it results in
\begin{equation}
\label{Equ:dotomegak}
{\dot \omega ^k} = {h^k}\frac{{{\bfa^k}}}{{{r^k}}}:{\dotbfepsilon^{\rm{p}}} - {\dot r^k}.
\end{equation}
Substituting ${\dot \omega ^k}$ into Eq. (\ref{Equ:dotak1}) follows
\begin{eqnarray}
\label{Equ:dotak2}
{\dot\bfa^k} &=& \frac{2}{3}{h^k}{\dotbfepsilon^{\rm{p}}} - H\left( {{f^k}} \right){h^k}\left\langle {\frac{{{\bfa^k}}}{{{r^k}}}:{{\dotbfepsilon}^{\rm{p}}}} \right\rangle \frac{{{\bfa^k}}}{{{r^k}}} + H\left( {{f^k}} \right)\frac{{{\bfa^k}}}{{{r^k}}}{\dot r^k},
\end{eqnarray}
with $\left<\cdot\right>$ as the Macauley brackets.

Combining Eq. (\ref{Equ:dotak2}) with Armstrong and Frederick model as suggested by Adbel-Karim and Ohno  \cite{AbdelKarim20051303},  ${\bfa^k}$ obeys the following evolution rule, as
\begin{eqnarray}
\label{Equ:dotak3}
{\dot\bfa^k} &=& {r^k}{\zeta ^k}\left[ \frac{2}{3}{{\dotbfepsilon}^{\rm{p}}} - {\mu ^k}\frac{\bfa^k}{r^k}\dot p - H\left( {f^k} \right) \left\langle {\frac{\bfa^k}{r^k}:{\dotbfepsilon}^{\rm{p}}} - {\mu ^k}\dot p \right\rangle \frac{\bfa^k}{r^k} \right]
+ H\left( {f^k} \right)\frac{\bfa^k}{r^k}{\dot r^k},
\end{eqnarray}
where ${\zeta ^k}$ is the material constant related to ${h^k} = {r^k}{\zeta ^k}$. ${\mu ^k}$ is the combination parameter with $0 \leqslant {\mu ^k} \leqslant 1$. When ${\dot r^k} = 0$ and ${\mu ^k} = 0$, the modified model here is reduced to the known Ohno and Wang model. For ${\dot r^k} = 0$ and ${\mu ^k} = 1$, the  model above is reduced to the Armstrong and Frederick model. Thus, for large ${\mu ^k}$, the present model predicts more significant ratcheting and cyclic stress relaxation.

As shown by Kang \cite{Kang2004299}, it is convenient to normalize the variable $\bfa_k$ by $r^k$ as
\begin{equation}
\label{Equ:ak1}
{\bfa^k} = {r^k}{\bfb^k},
\end{equation}
with
\begin{equation}
{f^k} = \frac{3}{2}{\bfb^k}:{\bfb^k} - 1 = 0.
\end{equation}
It follows
\begin{equation}
{\dot\bfb^k} = {\zeta ^k}\left( {\frac{2}{3}{\dotbfepsilon^{\rm{p}}} - {\bfb^k}{{\dot p}^k}} \right) + \left[ {H\left( {{f^k}} \right) - 1} \right]\frac{{{{\dot r}^k}}}{{{r^k}}}{\bfb^k},
\end{equation}
\begin{equation}
{\dot p^k} = \left[ {{\mu ^k} + H\left( {{f^k}} \right)\left\langle {\sqrt {\frac{3}{2}} {\bfb^k}:\bfn - {\mu ^k}} \right\rangle } \right]\dot p,
\end{equation}
where
\begin{equation}
\bfn = \frac{{\bfs - \bfa}}{{\left\| {\bfs - \bfa} \right\|}},
\end{equation}
is the normal direction of the yield surface. Here $\parallel\cdot\parallel$ is the Euclidean norm of a second rank tensor.


\subsection{Cyclic hardening/softening}
\noindent
The characteristic stress-strain curve for cyclic loading is obtained from material testing subjected to alternating loads. Observations on stress-strain hysteresis loops of the austenite stainless steel showed that the peak stresses were not monotonic to plastic strain \cite{fang2015cyclic}. The steel demonstrated initial cyclic hardening behavior and then cyclic softening. Before the final failure occurs, the material showed secondary hardening. To describe such complex behavior $r^k$ should consist of two parts, $r_0^k$ and $r_{\Delta}^k$  \cite{fang2015cyclic}, that is,
\begin{equation}
\label{Equ:rk1}
{r^k} = r_0^k + r_\Delta ^k,
\end{equation}
which can be identified from the monotonic tension curve and cyclic tension-compression curve, respectively.
Above $r_\Delta ^k$ is assumed to depend on the accumulated plastic strain $p$,
\begin{equation}
% \label{Equ:rdeltak1}
%r_\Delta ^k = r_{\Delta s}^k\left[ {1 - a_1^k{e^{ - b_1^kp}} + a_2^k({e^{ - b_2^kp}} - {e^{ - b_3^kp}})} \right]
r_\Delta ^k = r_{\Delta s}^k\left[ {1 - a_1^k{e^{ - b_1^kp}} - (1-a_1^k){e^{ - b_2^kp}} }\right].
\label{Equ:rdeltak}
\end{equation}
The dependence on plastic strain allows more accurate description of the cyclic hardening as well as softening behavior of the material.

\subsection{Dynamic recovery and plastic strain memorization}
\noindent
The plastic strain based memory surface proposed by Chaboche \cite{Chaboche1986149} is defined as
\begin{equation}
\label{Equ:g1}
g = \sqrt {\frac{2}{3}\left( {{\bfepsilon^{\rm{p}}} - \bfbeta} \right):\left( {{\bfepsilon^{\rm{p}}} - \bfbeta} \right)}  - q,
\end{equation}
where $\bfbeta$ and $q$ represent the radius and center of the non-hardening surface, respectively.
Updating of the memory state is only available when the current plastic strain is on the surface (i.e. $g=0$) and the flow direction is outwards of the surface, that is,
\begin{equation}
\frac{{\partial g}}{{\partial {\bfepsilon^{\rm{p}}}}}:d{\bfepsilon^{\rm{p}}} > 0.
\end{equation}
Above $\bfbeta$ and $q$ respectively obey the following evolution rules,
\begin{equation}
\label{Equ:dotbeta1}
\dot\bfbeta  = \left( {1 - \eta } \right)H\left( g \right)\left\langle {\bfn:{\bfn^*}} \right\rangle \sqrt {\frac{3}{2}} {\bfn^*}\dot p,
\end{equation}
in which
\begin{equation}
\label{Equ:dotq1}
\dot q = \eta H\left( g \right)\left\langle {\bfn:{\bfn^*}} \right\rangle \dot p,
\end{equation}
with
\begin{equation}
\label{Equ:nstar}
{\bfn^*} = \frac{{\partial g}}{{\partial {\bfepsilon^{\rm{p}}}}} = \frac{{{\bfepsilon^{\rm{p}}} - \bfbeta}}{{\left\| {{\bfepsilon^{\rm{p}}} - \bfbeta} \right\|}}.
\end{equation}

% \subsection{Non-proportional hardening}
% \noindent
% In the article the non-proportional hardening is considered in an additional term of the isotropic hardening, as
% \begin{equation}
% Y = {Y_0} + {Y_{\Delta \rm np}},
% \end{equation}
% where $Y_0$ denotes the yield stress for proportional loading and ${Y_{\Delta \rm np}}$ is the non-proportional hardening defined by
% \begin{equation}
% {\dot Y_{\Delta \rm np}} = {\gamma _{\rm q}}\left( {{Y_{\rm sat}} - {Y_{\Delta \rm np}}} \right)\dot p.
% \end{equation}
% Integrating the above equation with the initial condition ${Y_{\Delta \rm np}}(0)=0$ follows
% \begin{equation}
% {Y_{\Delta \rm np}} = Y_{\rm sat}\left( 1-e^{- \gamma_{\rm p} p} \right),
% \end{equation}
% where $Y_{\rm sat}$ is the saturated value of ${Y_{\Delta \rm np}}$ under a non-proportional loading path.
% Tanaka introduced a non-proportionality factor \cite{tanaka1994nonproportionality},
% \begin{equation}
% \phi  = \sqrt {1 - \frac{{\bfn:\mathbb{C}:\mathbb{C}:\bfn}}{{\mathbb{C}::\mathbb{C}}}}
% \end{equation}
% to quantify influence of the non-proportional load path to the material behavior.  $\mathbb{C}$ is a fourth-rank tensor and represents the internal dislocation structure \cite{tanaka1994nonproportionality}, as
% \begin{equation}
% \dot {\mathbb{C}} = {c_c}\left( {\bfn \otimes \bfn - \mathbb{C}} \right)\dot p.
% \end{equation}
% Assuming ${Y_{\rm sat}}$ depending on the loading path follows the simplest form as
% \begin{equation}
% {Y_{\rm sat}} = \phi {Y_{\Delta NS}},
% \end{equation}
% with $0 \leq \phi \leq 1$.
% For the proportional loading $\phi = 0$, there is no non-proportional hardening, while $\phi = 1$ means ${Y_{\rm sat}} = {Y_{\Delta NS}}$ for the maximum non-proportional hardening.

% According to Ref. \cite{fang2015cyclic}, the evolution of ${Y_{\Delta NS}}$ can be defined as
% \begin{equation}
% {\dot Y_{\Delta NS}} = {\gamma _{\rm q}}\left( {{Y_{\Delta 0}} - {Y_{\Delta NS}}} \right)\dot q.
% \end{equation}
% Noting that the strain memory effect is taken into account, $q$ is the radius of strain memory surface, ${\gamma _{\rm q}}$ is a rate parameter.
% If the material satisfies the Masing postulate,  ${\gamma _{\rm q}}=0$. $Y_{\Delta 0} $ denotes the saturate value of $Y_{\Delta NS}$.

\subsection{Non-proportional hardening}
\noindent
In this paper, we consider the non-proportional hardening as an additional term of isotropic hardening.
Thus, the yield stress is suggested as:
\begin{equation}
Y = {Y_0} + {Y_{\Delta \rm np}}
\end{equation}
where $Y_0$ denotes the initial yield stress and ${Y_{\Delta \rm np}}$ is the non-proportional hardening term.

The evolution of ${Y_{\Delta \rm np}}$ is defined by:
\begin{equation}
{\dot Y_{\Delta \rm np}} = {\gamma _{\rm p}}\left( {{Y_{\Delta \rm nps}} - {Y_{\Delta \rm np}}} \right)\dot p
\end{equation}
we integrate the equation with initial condition ${Y_{\Delta \rm np}}(0)=0$ and it gives:
\begin{equation}
{Y_{\Delta \rm np}} = Y_{\Delta \rm nps}\left( 1-e^{- \gamma_{\rm p} p} \right)
\end{equation}
where $Y_{\Delta \rm nps}$ is the saturated value of ${Y_{\Delta \rm np}}$ under a certain loading path.

Because ${Y_{\Delta \rm nps}}$ depends on the loading path, we introduce a linear function:
\begin{equation}
{Y_{\Delta \rm nps}} = \phi {Y_{\Delta \rm nonps}}
\end{equation}
where $\phi$ is a non-proportionality parameter with the condition $0 \leq \phi \leq 1$.
When $\phi = 0$, we have no non-proportional hardening.
When $\phi = 1$, we have ${Y_{\Delta \rm nps}} = {Y_{\Delta \rm nonps}}$, then ${Y_{\Delta \rm nonps}}$ represents the maximum value corresponding to a certain loading path.

We use the definition suggested by \cite{tanaka1994nonproportionality}:
\begin{equation}
\phi  = \sqrt {1 - \frac{{{\bf{n}}:\mathbb{C}:\mathbb{C}:{\bf{n}}}}{{\mathbb{C}::\mathbb{C}}}}
\end{equation}
where $\mathbb{C}$ is a fourth-rank tensor and it represents the internal dislocation structure.
The evolution equation is defined as:
\begin{equation}
\dot {\mathbb{C}} = {c_c}\left( {{\bf{n}} \otimes {\bf{n}} - \mathbb{C}} \right)\dot p
\end{equation}

As discussed by \cite{fang2015cyclic}, the evolution of ${Y_{\Delta \rm nonps}}$ can be provide as:
\begin{equation}
{\dot Y_{\Delta \rm nonps}} = {\gamma _{\rm q}}\left( {{Y_{\Delta \rm sat}} - {Y_{\Delta \rm nonps}}} \right)\dot q
\end{equation}
noting that the account of the strain memory effect is counted, $q$ is the radius of strain memory surface, ${\gamma _{\rm q}}$ is a rate parameter.
Generally if the material performs Massing behavior, we can choose ${\gamma _{\rm q}}=0$.

Box 1 and Box 2 summarize the constitutive equations of the model as follows:

\begin{framed}
\label{Box:1}
Box 1: Summary of the constitutive equations.

1. Additive decomposition of the strain tensor:
\[{\bm{\upepsilon }} = {{\bm{\upepsilon }}^{\rm{e}}} + {{\bm{\upepsilon }}^{\rm{p}}}\]

2. Total stress:
\[{\bm{\upsigma }} = {\bf{s}} + \frac{1}{3}{\rm{Tr}}\left( {\bm{\upsigma }} \right){\bf{I}} = {\mathbb{D}^{\rm{e}}}:{{\bm{\upepsilon }}^{\rm{e}}}\]

3. Yield function:
\[F = \sqrt {\frac{3}{2}\left( {{\bf{s}} - {\bf{a}}} \right):\left( {{\bf{s}} - {\bf{a}}} \right)}  - Y = 0\]

4. Plastic flow rule:
\[{{\bm{\dot \upepsilon }}^{\rm{p}}} = \dot \lambda \frac{{\partial F}}{{\partial {\bm{\upsigma }}}}\]

5. Effective stress:
\[{\bf{n}} = \frac{{{\bf{s}} - {\bf{a}}}}{{\left\| {{\bf{s}} - {\bf{a}}} \right\|}}\]

6. Backstress:
\[{\bf{a}} = \sum\limits_{k = 1}^M {{{\bf{a}}^k}} \]
\[{{\dot{\bf a}}^k} = {r^k}{\zeta ^k}\left[ {\frac{2}{3}{{{\bm{\dot \upepsilon }}}^{\rm{p}}} - \mu \frac{{{{\bf{a}}^k}}}{{{r^k}}}\dot p - H\left( {{f^k}} \right)\left\langle {\frac{{{{\bf{a}}^k}}}{{{r^k}}}:{{{\bm{\dot \upepsilon }}}^{\rm{p}}} - \mu \dot p} \right\rangle \frac{{{{\bf{a}}^k}}}{{{r^k}}}} \right] + H\left( {{f^k}} \right)\frac{{{{\bf{a}}^k}}}{{{r^k}}}{\dot r^k}\]

Normalized form:
\[{{\dot{\bf b}}^k} = {\zeta ^k}\left( {\frac{2}{3}{{{\bm{\dot \upepsilon }}}^{\rm{p}}} - {{\bf{b}}^k}{{\dot p}^k}} \right) + \left[ {H\left( {{f^k}} \right) - 1} \right]\frac{{{{\dot r}^k}}}{{{r^k}}}{{\bf{b}}^k}\]

where
\[{{\bf{a}}^k} = {r^k}{{\bf{b}}^k}\]
\[{f^k} = \frac{3}{2}{{\bf{b}}^k}:{{\bf{b}}^k} - 1\]
\[{\dot p^k} = \left[ {\mu  + H\left( {{f^k}} \right)\left\langle {\sqrt {\frac{3}{2}} {{\bf{b}}^k}:{\bf{n}} - \mu } \right\rangle } \right]\dot p\]
\end{framed}


\begin{framed}
\label{Box:1}
Box 2: Specification of the material functions.

1. Dynamic recovery:
\[g = \sqrt {\frac{2}{3}\left( {{{\bm{\upepsilon }}^{\rm{p}}} - {\bm{\upbeta}}} \right):\left( {{{\bm{\upepsilon }}^{\rm{p}}} - {\bm{\upbeta}}} \right)}  - q\]
\[{\dot{\bm{\upbeta}}}  = \left( {1 - \eta } \right)H\left( g \right)\left\langle {{\bf{n}}:{{\bf{n}}^*}} \right\rangle \sqrt {\frac{3}{2}} {{\bf{n}}^*}\dot p\]
\[\dot q = \eta H\left( g \right)\left\langle {{\bf{n}}:{{\bf{n}}^*}} \right\rangle \dot p\]
\[{{\bf{n}}^*} = \frac{{\partial g}}{{\partial {{\bm{\upepsilon }}^{\rm{p}}}}} = \frac{{{{\bm{\upepsilon }}^{\rm{p}}} - {\bm{\upbeta}}}}{{\left\| {{{\bm{\upepsilon }}^{\rm{p}}} - {\bm{\upbeta}}} \right\|}}\]

2. Kinematic hard/softening:
\[{r^k} = r_0^k + r_\Delta ^k\]
\[r_\Delta ^k = r_{\Delta s}^k\left[ {1 - a_1^k{e^{ - b_1^kp}} + a_2^k({e^{ - b_2^kp}} - {e^{ - b_3^kp}})} \right]\]

3. Isotropic hard/softening:
\[Y = {Y_0} + {Y_{\Delta \rm np}}\]
\[{\dot Y_{\Delta \rm np}} = {\gamma _{\rm q}}\left( {{Y_{\Delta \rm nps}} - {Y_{\Delta \rm np}}} \right)\dot p\]
\[{Y_{\Delta \rm nps}} = \phi {Y_{\Delta \rm nonps}}\]
\[{\dot Y_{\Delta \rm nonps}} = {\gamma _{\rm q}}\left( {{Y_{\Delta \rm sat}} - {Y_{\Delta \rm nonps}}} \right)\dot q\]


4. Non-proportionality parameter:
\[\phi  = \sqrt {1 - \frac{{{\bf{n}}:\mathbb{C}:\mathbb{C}:{\bf{n}}}}{{\mathbb{C}::\mathbb{C}}}} \]
\[\dot {\mathbb{C}} = {c_c}\left( {{\bf{n}} \otimes {\bf{n}} - \mathbb{C}} \right)\dot p\]

\end{framed}

\section{Algorithmic formulation of the constitutive model}
\noindent
% In the following, a fully implicit strain driven integration algorithm is developed for the constitutive model discussed in the previous section. It is ready for an implementation in a general Finite Element (FE) context where the proposed algorithm is applied locally at every integration point during each global iteration step.
The present constitutive model has included Tanaka non-proportional term and modified backstress equations for complex thermo-mechanical loads. The implicit algorithm of the constitutive equation becomes necessary and should provide the stress increment and the plastic strain increment, based on the given strain increment. The integration method is to be implemented into the commercial finite element codes, such ABAQUS via the UMAT interface.

\subsection{Backward Euler discretization}
\noindent
The backward Euler method is a first order implicit method popularly used in the finite element computation. Considering the integration interval from Step $n$ to Step $n+1$, variables of Step $n$ are denoted by the index $n$. The symbol $\Delta$ indicates the increment of the variable from $n$ to $n+1$. The constitutive equations can be discretized as follows.
% Backward Euler method is a first order fully implicit method. Consider the interval from a state $n$ to $n+1$, in the following, any quantity evaluated at state $n$ will be indicated by a subscript $n$, for quantities evaluated at state $n+1$ subscript $n+1$ are omitted, and $\Delta$ indicates the increments in the interval from $n$ to $n+1$. The backward Euler method allows the constitutive equations to be discretized as the following.
%Hereafter, the subscripts $n$ and $n+1$ signify the values at $n$ and $n+1$ respectively, and $\Delta$ indicates the increments in the interval from $n$ to $n+1$.
Strain decomposition and total stress:
\begin{equation}
{{\bm{\upepsilon }}_{n + 1}} = {\bm{\upepsilon }}_{n + 1}^{\rm{e}} + {\bm{\upepsilon }}_{n + 1}^{\rm{p}},
\end{equation}
\begin{equation}
{\bm{\upepsilon }}_{n + 1}^{\rm{p}} = {\bm{\upepsilon }}_n^{\rm{p}} + \Delta {\bm{\upepsilon }}_{n + 1}^{\rm{p}},
\end{equation}
\begin{equation}
\label{Equ:sigman+1}
{{\bm{\upsigma }}_{n + 1}} = {\mathbb{D}^{\rm{e}}}:\left( {{{\bm{\upepsilon }}_{n + 1}} - {\bm{\upepsilon }}_{n + 1}^{\rm{p}}} \right).
\end{equation}
Yield function and plastic flow rule:
\begin{equation}
\label{Equ:Fn+1}
{F_{n + 1}} = \sqrt {\frac{3}{2}\left( {{{\bf{s}}_{n + 1}} - {{\bf{a}}_{n + 1}}} \right):\left( {{{\bf{s}}_{n + 1}} - {{\bf{a}}_{n + 1}}} \right)}  - {Y_{n + 1}},
\end{equation}
\begin{equation}
\label{Equ:epsilonpn+1}
\Delta {\bm{\upepsilon }}_{n + 1}^{\rm{p}} = \sqrt {\frac{3}{2}} \Delta {p_{n + 1}}{{\bf{n}}_{n + 1}},
\end{equation}
\begin{equation}
\label{Equ:pn+1}
\Delta {p_{n + 1}} = \sqrt {\frac{2}{3}\Delta {\bm{\upepsilon }}_{n + 1}^{\rm{p}}:\Delta {\bm{\upepsilon }}_{n + 1}^{\rm{p}}},
\end{equation}
\begin{equation}
\label{Equ:nn+1}
{{\bf{n}}_{n + 1}} = \frac{{{{\bf{s}}_{n + 1}} - {{\bf{a}}_{n + 1}}}}{{\left\| {{{\bf{s}}_{n + 1}} - {{\bf{a}}_{n + 1}}} \right\|}} = \sqrt {\frac{3}{2}} \frac{{{{\bf{s}}_{n + 1}} - {{\bf{a}}_{n + 1}}}}{{{Y_{n + 1}}}}.
\end{equation}
Back stress:
\begin{equation}
\label{Equ:an+1}
{{\bf{a}}_{n + 1}} = \sum\limits_{k = 1}^M {r_{n + 1}^k{\bf{b}}_{n + 1}^k},
\end{equation}
\begin{equation}
{{\bf{b}}_{n + 1}^k = {\bf{b}}_n^k + {\zeta ^k}\left( {\frac{2}{3}\Delta {\bm{\upepsilon }}_{n + 1}^{\rm{p}} - {\bf{b}}_{n + 1}^k\Delta p_{n + 1}^k} \right) + \left[ {H\left( {f_{n + 1}^k} \right) - 1} \right]{\frac{{\Delta r_{n + 1}^k}}{{r_{n + 1}^k}}{\bf{b}}_{n + 1}^k}},
\end{equation}
\begin{equation}
\Delta p_{n + 1}^k = \left[ {\mu  + H\left( {f_{n + 1}^k} \right)\left\langle {\sqrt {\frac{3}{2}} {{\bf{n}}_{n + 1}}:{\bf{b}}_{n + 1}^k - \mu } \right\rangle } \right]\Delta {p_{n + 1}}.
\end{equation}
Non-proportionality parameter:
\begin{equation}
{\phi _{n + 1}} = \sqrt {1 - \frac{{{{\bf{n}}_{n + 1}}:{\mathbb{C}_{n + 1}}:{\mathbb{C}_{n + 1}}:{{\bf{n}}_{n + 1}}}}{{{\mathbb{C}_{n + 1}}::{\mathbb{C}_{n + 1}}}}},
\end{equation}
\begin{equation}
\Delta {\mathbb{C}_{n + 1}} = {c_c}\left( {{{\bf{n}}_{n + 1}} \otimes {{\bf{n}}_{n + 1}} - {\mathbb{C}_{n + 1}}} \right)\Delta {p_{n + 1}}.
\end{equation}
Dynamic recovery:
\begin{equation}
{g_{n + 1}} = \sqrt {\frac{2}{3}\left( {{\bm{\upepsilon }}_{n + 1}^{\rm{p}} - {{\bm{\upbeta }}_{n + 1}}} \right):\left( {{\bm{\upepsilon }}_{n + 1}^{\rm{p}} - {{\bm{\upbeta }}_{n + 1}}} \right)}  - {q_{n + 1}},
\end{equation}
\begin{equation}
\Delta {q_{n + 1}} = \eta H\left( {{g_{n + 1}}} \right)\left\langle {{{\bf{n}}_{n + 1}}:{\bf{n}}_{n + 1}^*} \right\rangle \Delta {p_{n + 1}},
\end{equation}
\begin{equation}
\Delta {{\bm{\upbeta }}_{n + 1}} = \left( {1 - \eta } \right)H\left( {{g_{n + 1}}} \right)\left\langle {{{\bf{n}}_{n + 1}}:{\bf{n}}_{n + 1}^*} \right\rangle \sqrt {\frac{3}{2}} \Delta {p_{n + 1}}{\bf{n}}_{n + 1}^*,
\end{equation}
\begin{equation}
{\bf{n}}_{n + 1}^* = \frac{{{\bm{\upepsilon }}_{n + 1}^{\rm{p}} - {{\bm{\upbeta }}_{n + 1}}}}{{\left\| {{\bm{\upepsilon }}_{n + 1}^{\rm{p}} - {{\bm{\upbeta }}_{n + 1}}} \right\|}}.
\end{equation}

\subsection{Implicit stress integration}
\noindent
The problem considered here is stated as follows: given all constitutive variables at step $n$ and also $\Delta {{\bm{\upepsilon }}_{n + 1}}$, find $\bm{\upsigma}_{n+1}$  satisfying the discretized constitutive relations. Let us use return mapping, which consists of an elastic predictor and a plastic corrector. The elastic predictor is taken to be an elastic tentative stress:
\begin{equation}
{\bm{\upsigma }}_{n + 1}^{\rm{tr}} = {\mathbb{D}^{\rm{e}}}:\left( {{{\bm{\upepsilon }}_{n + 1}} - {\bm{\upepsilon }}_n^{\rm{p}}} \right)
\end{equation}
where ${\mathbb{D}^{\rm{e}}} = \kappa {\bf{I}} \otimes {\bf{I}} + 2G\mathbb{I}$, $G$ is shear modules of material and $\kappa$ is Lame constant $\kappa  = K - \frac{2}{3}G$, ${\bf{I}}$ and $\mathbb{I}$ respectively indicate the second and fourth order identity, given in component form as ${I_{ij}} = {\delta _{ij}}$ and ${\mathbb{I}_{ijkl}} = {\delta _{ik}}{\delta _{jl}}$.

Also we can derive the relation between stress and trial stress at state $n+1$ as:
\begin{equation}
\label{Equ:sigma1}
{{\bm{\upsigma }}_{n + 1}^{} = {\mathbb{D}^{\rm{e}}}:\left( {{\bm{\upepsilon }}_{n + 1}^{} - {\bm{\upepsilon }}_{n + 1}^{\rm{p}}} \right)} = {\bm{\upsigma }}_{n + 1}^{\rm{tr}} - 2G\Delta {\bm{\upepsilon }}_{n + 1}^{\rm{p}}
\end{equation}

Note that, because of elastic isotropy and plastic incompressibility, $\Delta {\bm{\upepsilon }}_{n + 1}^{\rm{p}}$ is a deviatoric tensor with $Tr\left( {\Delta {\bm{\upepsilon }}_{n + 1}^{\rm{p}}} \right)=0$.

The stress deviator is given by:
\begin{equation}
{\bf{s}} \equiv {\bm{\upsigma }} - \frac{1}{3}\left( {{\bm{\upsigma }}:{\bf{I}}} \right){\bf{I}} = {\mathbb{I}_{\rm{d}}}:{\bm{\upsigma }}
\end{equation}
where ${\mathbb{I}_{\rm{d}}} = \mathbb{I} - \frac{1}{3}{\bf{I}} \otimes {\bf{I}}$ is the deviatoric projection tensor.

Taking the deviatoric part of Equation (\ref{Equ:sigma1}), we have:
\begin{equation}
{{\bf{s}}_{n + 1}} = {\bf{s}}_{n + 1}^{\rm{tr}} - 2G\Delta {\bm{\upepsilon }}_{n + 1}^{\rm{p}}
\end{equation}

Both sides minus ${{\bf{a}}_{n + 1}}$, we obtain:
\begin{equation}
\label{Equ:sminusa1}
{{\bf{s}}_{n + 1}} - {{\bf{a}}_{n + 1}} = {\bf{s}}_{n + 1}^{\rm{tr}} - 2G\Delta {\bm{\upepsilon }}_{n + 1}^{\rm{p}} - {{\bf{a}}_{n + 1}}
\end{equation}

A two steps radial return map method (see \ref{Fig:radial_return_map}) proposed by Mineo Kobayashi and Nobutada Ohno (2002) \cite{kobayashi2002implementation} are employed for integrating the present constitutive model. The backstress is written as:
\begin{equation}
{{\bf{a}}_{n + 1}} = \sum\limits_{k = 1}^M {r_{n + 1}^k{\bf{b}}_{n + 1}^k}
\end{equation}
\begin{figure}[!htp]
	\centering
	\includegraphics[width=16cm]{radial_return_map.pdf}
	\caption{Schematics of the two steps radial return map method.}
	\label{Fig:radial_return_map}
\end{figure}

Then the tentative form of the backstress is determined from:
\begin{equation}
\label{Equ:btrn+1}
{\bf{b}}_{n + 1}^{k,{\rm{tr}}} = {\bf{b}}_n^k + \frac{2}{3}{\zeta ^k}\Delta {\bm{\upepsilon }}_{n + 1}^{\rm{p}}
\end{equation}

Therefore, the normalized backstress ${\bf{b}}_{n + 1}^k$ can be written as:
\begin{equation}
\label{Equ:bn+1}
{\bf{b}}_{n + 1}^k = \theta _{n + 1}^k\left( {{\bf{b}}_n^k + \frac{2}{3}{\zeta ^k}\Delta {\bm{\upepsilon }}_{n + 1}^{\rm{p}}} \right) = \theta _{n + 1}^k{\bf{b}}_{n + 1}^{k,{\rm{tr}}}
\end{equation}
where $\theta _{n + 1}^k$ is defined by the following equation and satisfies $0 < \theta _{n + 1}^k \leqslant 1$:
\begin{equation}
\label{Equ:thetan+1}
\theta _{n + 1}^k = c_{n + 1}^k + H\left( {\bar f_{n + 1}^k} \right)\left( {\frac{1}{{\bar b_{n + 1}^{k,{\rm{tr}}}}} - c_{n + 1}^k} \right)
\end{equation}
$c_{n + 1}^k$ and ${\bar f_{n + 1}^k}$ are given as:
\begin{equation}
c_{n + 1}^k = \frac{1}{{1 + {\zeta ^k}\mu \Delta {p_{n + 1}} - \frac{{\Delta r_{n + 1}^k}}{{r_{n + 1}^k}}}}
\end{equation}
\begin{equation}
\bar f_{n + 1}^k = {\left( {c_{n + 1}^k\bar b_{n + 1}^{k,{\rm{tr}}}} \right)^2} - 1
\end{equation}
where
\begin{equation}
\bar b_{n + 1}^{k,{\rm{tr}}} = \sqrt {\frac{3}{2}{\bf{b}}_{n + 1}^{k,{\rm{tr}}}:{\bf{b}}_{n + 1}^{k,{\rm{tr}}}}
\end{equation}

Substituting ${{\bf{a}}_{n + 1}}$ into Equation (\ref{Equ:sminusa1}), we have:
\begin{equation}
\label{Equ:sminusa2}
\begin{aligned}
{{{\bf{s}}_{n + 1}} - {{\bf{a}}_{n + 1}}}=
{\bf{s}}_{n + 1}^{\rm{tr}} - \sum\limits_{k = 1}^M {r_{n + 1}^k\theta _{n + 1}^k{\bf{b}}_n^k}
- \left( {2G + \frac{2}{3}\sum\limits_{k = 1}^M {r_{n + 1}^k\theta _{n + 1}^k{\zeta ^k}} } \right)\Delta {\bm{\upepsilon }}_{n + 1}^{\rm{p}}
\end{aligned}
\end{equation}

Substitution of Equations (\ref{Equ:Fn+1}) to (\ref{Equ:nn+1}) and (\ref{Equ:sminusa2}) into the yield function ${F_{n + 1}} = 0$ provides:
\begin{equation}
\Delta {p_{n + 1}} = \frac{{\sqrt {\frac{3}{2}} \left\| {{\bf{s}}_{n + 1}^{\rm{tr}} - \sum\limits_{k = 1}^M {r_{n + 1}^k\theta _{n + 1}^k{\bf{b}}_n^k} } \right\| - {Y_n}}}{{{{\left( {\frac{{\partial Y}}{{\partial p}}} \right)}_{n + 1}} + 3G + \sum\limits_{k = 1}^M {r_{n + 1}^k\theta _{n + 1}^k{\zeta ^k}} }}
\end{equation}

\begin{equation}
{{\bf{s}}_{n + 1}} - {{\bf{a}}_{n + 1}} = \frac{{{Y_{n + 1}}\left( {{\bf{s}}_{n + 1}^{\rm{tr}} - \sum\limits_{k = 1}^M {r_{n + 1}^k\theta _{n + 1}^k{\bf{b}}_n^k} } \right)}}{{\sqrt {\frac{3}{2}} \left\| {{\bf{s}}_{n + 1}^{\rm{tr}} - \sum\limits_{k = 1}^M {r_{n + 1}^k\theta _{n + 1}^k{\bf{b}}_n^k} } \right\|}}
\end{equation}

\begin{figure}[ht]
	\centering
	\includegraphics[width=16cm]{successive_substitution.pdf}
	\caption{Successive substitution return mapping algorithm.}
	\label{Fig:radial_return_map}
\end{figure}

\subsection{Consistent tangent moduli}
\noindent
To complete the algorithmic scheme discussed above we derive in the following the tangent moduli consistent with the proposed algorithm. The tangent moduli determine the incremental change of the stresses $\Delta {\bm{\upsigma }}$ along with the incremental change of the total strains $\Delta {\bm{\upepsilon }}$ determined through the global iteration algorithm according to:
\begin{equation}
\Delta {\bm{\upsigma }} = {\mathbb{D}^{\rm{ep}}}:\Delta {\bm{\upepsilon }}
\end{equation}

Considering Equation (\ref{Equ:dotak3}), we assume that the increment of deviatoric back stress has the following differential:
\begin{equation}
\label{Equ:dan+1}
{\text{d}}\Delta {{\bf{a}}_{n + 1}} = \sum\limits_{k = 1}^M {\mathbb{H}_{n + 1}^k} :{\text{d}}\Delta {\bm{\upepsilon }}_{n + 1}^{\rm{p}}
\end{equation}
where $\mathbb{H}_{n + 1}^k( k = 1,2,3,...,M )$ are fourth-rank constitutive parameters.

It is noted that, when ${F_{n + 1}} = 0$, ${\bf{n}}_{n + 1}$ defined by Equation (\ref{Equ:nn+1}) satisfies:
\begin{equation}
\label{Equ:nn1}
{{\bf{n}}_{n + 1}}:{{\bf{n}}_{n + 1}} = 1
\end{equation}

On the assumption that all constitutive variables are known at state $n$, and that $F_{n+1}=0$, let us differentiate Equations (\ref{Equ:sigman+1}), (\ref{Equ:epsilonpn+1}), (\ref{Equ:nn+1}) and (\ref{Equ:nn1}) to obtain:
\begin{equation}
\label{Equ:dsigman+1}
{\text{d}}\Delta {{\bm{\upsigma }}_{n + 1}} = {\mathbb{D}^{\rm{e}}}:\left( {{\text{d}}\Delta {{\bm{\upepsilon }}_{n + 1}} - {\text{d}}\Delta {\bm{\upepsilon }}_{n + 1}^{\rm{p}}} \right)
\end{equation}
\begin{equation}
\label{Equ:depsilon+1}
{\text{d}}\Delta {\bm{\upepsilon }}_{n + 1}^{\rm{p}} = \sqrt {\frac{3}{2}} \left( {{\text{d}}\Delta {p_{n + 1}}{{\bf{n}}_{n + 1}} + \Delta {p_{n + 1}}{\text{d}}{{\bf{n}}_{n + 1}}} \right)
\end{equation}
\begin{equation}
\label{Equ:dnn+1}
{\text{d}}{{\bf{n}}_{n + 1}} = \sqrt {\frac{3}{2}} \frac{{{\text{d}}\Delta {{\bf{s}}_{n + 1}} - {\text{d}}\Delta {{\bf{a}}_{n + 1}}}}{{{Y_{n + 1}}}} - \frac{{{{\bf{n}}_{n + 1}}}}{{{Y_{n + 1}}}}{\left( {\frac{{\partial Y}}{{\partial p}}} \right)_{n + 1}}{\text{d}}\Delta {p_{n + 1}}
\end{equation}
\begin{equation}
{{\bf{n}}_{n + 1}}:{\text{d}}{{\bf{n}}_{n + 1}} = 0
\end{equation}

From Equation (\ref{Equ:sigma1}) we obtain:
\begin{equation}
\label{Equ:sn+1}
{{\bf{s}}_{n + 1}} = 2G{\mathbb{I}_{\rm{d}}}:\left( {{{\bm{\upepsilon }}_{n + 1}} - {\bm{\upepsilon }}_{n + 1}^{\rm{p}}} \right)
\end{equation}

Differentiating Equation (\ref{Equ:sn+1}), we have:
\begin{equation}
\label{Equ:dsn+1}
{\text{d}}\Delta {{\bf{s}}_{n + 1}} = 2G{\mathbb{I}_{\rm{d}}}:{\text{d}}\Delta {{\bm{\upepsilon }}_{n + 1}} - 2G{\text{d}}\Delta {\bm{\upepsilon }}_{n + 1}^{\rm{p}}
\end{equation}

Substitution of Equations (\ref{Equ:dan+1}) and (\ref{Equ:dsn+1}) into (\ref{Equ:dnn+1}) provides:
\begin{equation}
\label{Equ:dnn+12}
\begin{aligned}
{\text{d}}{{\bf{n}}_{n + 1}} = \sqrt {\frac{3}{2}} \frac{1}{{{Y_{n + 1}}}} \left[ {2G{\mathbb{I}_{\rm{d}}}:{\text{d}}\Delta {{\bm{\upepsilon }}_{n + 1}} - \left( {2G\mathbb{I} + \sum\limits_{k = 1}^M {\mathbb{H}_{n + 1}^k} } \right):{\text{d}}\Delta {\bm{\upepsilon }}_{n + 1}^{\rm{p}}} \right.\\
\left. { - \frac{2}{3}{{\left( {\frac{{\partial Y}}{{\partial p}}} \right)}_{n + 1}}{{\bf{n}}_{n + 1}} \otimes {{\bf{n}}_{n + 1}}:{\text{d}}\Delta {\bm{\upepsilon }}_{n + 1}^{\rm{p}}}\right]
\end{aligned}
\end{equation}
noting that
\begin{equation}
\label{Equ:dpn+1}
{\text{d}}\Delta {p_{n + 1}} = \sqrt {\frac{2}{3}} {{\bf{n}}_{n + 1}}:{\text{d}}\Delta {\bm{\upepsilon }}_{n + 1}^{\rm{p}}
\end{equation}

Substituting Equations (\ref{Equ:dnn+12}) and (\ref{Equ:dpn+1}) into Equation (\ref{Equ:depsilon+1}) to eliminate ${\text{d}}{{\bf{n}}_{n + 1}}$ and ${\text{d}}\Delta {p_{n + 1}}$, and rearranging the resulting equation, we obtain:
\begin{equation}
{\mathbb{L}_{n + 1}}:{\text{d}}\Delta {\bm{\upepsilon }}_{n + 1}^{\rm{p}} = 2G{\mathbb{I}_{\rm{d}}}:{\text{d}}\Delta {{\bm{\upepsilon }}_{n + 1}}
\end{equation}
where
\begin{equation}
\label{Equ:Ln+1}
\begin{aligned}
\mathbb{L}_{n + 1} = \frac{2}{3}\frac{Y_{n + 1}}{\Delta {p_{n + 1}}} \left( \mathbb{I} - \bf{n}_{n + 1} \otimes \bf{n}_{n + 1} \right) + 2G\mathbb{I} + \sum \limits_{k = 1}^M {\mathbb{H}_{n + 1}^k} \\
   + \frac{2}{3}{{\left( {\frac{{\partial Y}}{{\partial p}}} \right)}_{n + 1}}{{\bf{n}}_{n + 1}} \otimes {{\bf{n}}_{n + 1}}
\end{aligned}
\end{equation}

Elimination of ${\text{d}}\Delta {\bm{\upepsilon }}_{n + 1}^{\rm{p}}$ in Equation (\ref{Equ:dsigman+1}) by use of the Equation (\ref{Equ:Ln+1}) gives:
\begin{equation}
{\text{d}}\Delta {{\bm{\upsigma }}_{n + 1}} = \left( {{\mathbb{D}^{\rm{e}}} - 4{G^2}\mathbb{L}_{n + 1}^{ - 1}:{\mathbb{I}_{\rm{d}}}} \right):{\text{d}}\Delta {{\bm{\upepsilon }}_{n + 1}}
\end{equation}

Therefore, consistent tangent modulus $\frac{{{\text{d}}\Delta {{\bm{\upsigma }}_{n + 1}}}}{{{\text{d}}\Delta {{\bm{\upepsilon }}_{n + 1}}}}$ is derived as:
\begin{equation}
{\mathbb{D}^{\rm{ep}}} = {\mathbb{D}^{\rm{e}}} - 4{G^2}\mathbb{L}_{n + 1}^{ - 1}:{\mathbb{I}_{\rm{d}}}
\end{equation}

In order to obtain $\mathbb{H}_{n + 1}^k$, we differentiate Equations (\ref{Equ:an+1}) and (\ref{Equ:bn+1}), so that:
\begin{equation}
{\text{d}}{{\bf{a}}_{n + 1}} = \sum\limits_{k = 1}^M {\left( {r_{n + 1}^k{\text{d}}{\bf{b}}_{n + 1}^k + {\bf{b}}_{n + 1}^k{\text{d}}r_{n + 1}^k} \right)}
\end{equation}
\begin{equation}
\label{Equ:dbn+1}
{\text{d}}{\bf{b}}_{n + 1}^k = \frac{{{\bf{b}}_{n + 1}^k}}{{\theta _{n + 1}^k}}{\text{d}}\theta _{n + 1}^k + \frac{2}{3}\theta _{n + 1}^k{\zeta ^k}{\text{d}}\Delta {\bm{\upepsilon }}_{n + 1}^{\rm{p}}
\end{equation}

Considering Equation (\ref{Equ:thetan+1}), we may derive $\theta _{n + 1}^k$ in two cases.

I. When $H\left( {\bar f_{n + 1}^k} \right) = 0$, we have $\theta _{n + 1}^k = c_{n + 1}^k$, then:
\begin{equation}
\begin{aligned}
{\text{d}}\theta _{n + 1}^k = {\text{d}}c_{n + 1}^k =  - {\left( {\theta _{n + 1}^k} \right)^2} \left[ {\zeta ^k}\mu {\text{d}}\Delta {p_{n + 1}} - \frac{1}{{r_{n + 1}^k}}\left( {1 - \frac{{\Delta r_{n + 1}^k}}{{r_{n + 1}^k}}} \right)\right.\\
\left.{{\left( {\frac{{\partial {r^k}}}{{\partial p}}} \right)}_{n + 1}}{\text{d}}\Delta {p_{n + 1}} \right]
\end{aligned}
\end{equation}

Therefore, $\mathbb{H}_{n + 1}^k$ in Equation (\ref{Equ:dan+1}) has an expression:
\begin{equation}
\begin{aligned}
\mathbb{H}_{n + 1}^k = \frac{2}{3}r_{n + 1}^k\theta _{n + 1}^k{\zeta ^k}\mathbb{I} + \sqrt {\frac{2}{3}} \left[ {{\left( {\frac{{\partial {r^k}}}{{\partial p}}} \right)}_{n + 1}} - r_{n + 1}^k\theta _{n + 1}^k{\zeta ^k}\mu \right.\\
\left. + \theta _{n + 1}^k\left( {1 - \frac{{\Delta r_{n + 1}^k}}{{r_{n + 1}^k}}} \right){{\left( {\frac{{\partial {r^k}}}{{\partial p}}} \right)}_{n + 1}} \right]{\bf{b}}_{n + 1}^k \otimes {{\bf{n}}_{n + 1}}
\end{aligned}
\end{equation}

II. When $H\left( {\bar f_{n + 1}^k} \right) = 1$, Equation (\ref{Equ:thetan+1}) becomes $\theta _{n + 1}^k = \frac{1}{{\bar b_{n + 1}^{k,{\rm{tr}}}}}$, then:
\begin{equation}
\label{Equ:thetan+1}
{\text{d}}\theta _{n + 1}^k =  - \frac{3}{2}{\left( {\theta _{n + 1}^k} \right)^3}{\bf{b}}_{n + 1}^{k,{\rm{tr}}}:{\text{d}}{\bf{b}}_{n + 1}^{k,{\rm{tr}}}
\end{equation}

We differentiate Equation (\ref{Equ:btrn+1}), so that:
\begin{equation}
\label{Equ:dbtrn+1}
{\text{d}}{\bf{b}}_{n + 1}^{k,{\rm{tr}}} = \frac{2}{3}{\zeta ^k}{\text{d}}\Delta {\bm{\upepsilon }}_{n + 1}^{\rm{p}}
\end{equation}

Eliminating ${\bf{b}}_{n + 1}^{k,{\rm{tr}}}$ and ${\text{d}}{\bf{b}}_{n + 1}^{k,{\rm{tr}}}$ in Equation (\ref{Equ:thetan+1}) by using Equations (\ref{Equ:btrn+1}) and (\ref{Equ:dbtrn+1}), we obtain:
\begin{equation}
\label{Equ:thetan+1}
{\text{d}}\theta _{n + 1}^k =  - {\left( {\theta _{n + 1}^k} \right)^2}{\zeta ^k}{\bf{b}}_{n + 1}^k:{\text{d}}\Delta {\bm{\upepsilon }}_{n + 1}^{\rm{p}}
\end{equation}

Substitution of Equation (\ref{Equ:thetan+1}) into (\ref{Equ:dbn+1}) provides:
\begin{equation}
\label{Equ:dbn+12}
{\text{d}}{\bf{b}}_{n + 1}^k = \frac{2}{3}\theta _{n + 1}^k{\zeta ^k}{\text{d}}\Delta {\bm{\upepsilon }}_{n + 1}^{\rm{p}} - {\zeta ^k}\theta _{n + 1}^k{\bf{b}}_{n + 1}^k \otimes {\bf{b}}_{n + 1}^k:{\text{d}}\Delta {\bm{\upepsilon }}_{n + 1}^{\rm{p}}
\end{equation}

Therefore, $\mathbb{H}_{n + 1}^k$ in Equation (\ref{Equ:dan+1}) has an expression:
\begin{equation}
\begin{aligned}
\mathbb{H}_{n + 1}^k = \frac{2}{3}r_{n + 1}^k\theta _{n + 1}^k{\zeta ^k}\mathbb{I} - r_{n + 1}^k{\zeta ^k}\theta _{n + 1}^k{\bf{b}}_{n + 1}^k \otimes {\bf{b}}_{n + 1}^k + \\
\sqrt {\frac{2}{3}} {\left( {\frac{{\partial {r^k}}}{{\partial p}}} \right)_{n + 1}}{\bf{b}}_{n + 1}^k \otimes {{\bf{n}}_{n + 1}}
\end{aligned}
\end{equation}