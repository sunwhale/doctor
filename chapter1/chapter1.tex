\chapter{Introduction}
\section{Background and motivations}
\noindent
\subsection{Thermomechanical fatigue}
\noindent
Gas turbine components are exposed in high temperature and experience cyclic multi-axial thermo-mechanical loadings. The increasing performance requirements to modern gas turbines challenge the material operating limits. Quantified characterization of deformation and fatigue behavior of materials under realistic conditions becomes increasingly of importance for industrial application \cite{harrison1996modelling}, including variations of structure temperature, mechanical loads as well as environmental conditions. 

Inconel 718 is the most popular nickel-based superalloy in turbine industry. It is an oxidation- and corrosion-resistant material possessing optimal thermal and mechanical property. 
It is used for both rotors and casings, so that the low cycle fatigue characterization of the material is of great significance for mechanical design. The operating temperature of Inconel 718 is limited to 650$^\circ$C for the life limited parts, such as discs and shafts.
A turbine suffers from high temperature low cycle fatigue (LCF) as well as thermomechanical fatigue (TMF).
The LCF and TMF behaviors of the nickel-based superalloy are the major concern for the safety and reliability of gas turbine engines.

% Gas turbine components experience severe cyclic multiaxial mechanical and thermal loadings. The increasing operating temperature in gas turbines is pushing materials closer to their operating limits. Quantifying mechanical behavior and fatigue performance of the components under the more realistic conditions becomes necessary to design reliable long-term components \cite{Harrison1996}.

In the past decades the Nickel-based superalloy Inconel 718 was extensively tested under isothermal loading conditions, especially by different aero engine makers. Engineering design was essentially based on uniaxial fatigue models, so that the influence of the loading multiaxiality was not clarified.
For engineering applications numerous investigations on the isothermal low cycle fatigue of the nickel-based superalloy were performed \cite{Koch85, Morrow88, Mahobia2014, Chen2016175, William1995, kim1988elevated, nelson1992creep}.
Many studies on the isothermal low cycle fatigue of the nickel-based superalloy were published in past years.
Ye et al. (2004) \cite{ye2004low} investigated LCF behavior of nickel-based superalloy GH4145/SQ. The tests were under fully reversed total strain amplitude control conditions at 538$^{\circ}$C in laboratory air. The alloy exhibits a pronounced initial fatigue hardening followed by continuous fatigue softening to failure at high strain amplitudes. Bilinear behavior with a change of slope at a plastic strain amplitude of about $0.2\%$ was observed in cyclic stress-strain, Coffin-Manson and plastic strain energy-life plots.
Mahobia et al. (2014) \cite{mahobia2014effect} studied LCF behavior of the alloy Inconel 718, with salt coating, showed drastic reduction in fatigue life at 650$^{\circ}$C. Fatigue life of the salt coated specimens was found to be drastically reduced at all the total strain amplitudes. In general there was cyclic softening both in the bare as well as salt coated specimens. Variation of fatigue life with plastic strain amplitude followed Coffin-Manson relationship.
Chen et al. (2016) \cite{Chen2016175} operated total strain-controlled low cycle fatigue (LCF) tests of a nickel based superalloy GH4169 at 650$^{\circ}$C. Cyclic softening occurred during the LCF process and it was related to the total strain amplitude. The relationship between life-time and total strain amplitude was obtained by combining Basquin equation and Coffin-Manson equation.

Generally life design of the turbine components is based on the isothermal fatigue concept, although the real loading history in turbine is essentially under varying temperature. It is assumed that the material fatigue at the higher temperature is more critical than that under the thermomechanical condition. From material testing view point, isothermal fatigue can be conducted more easily than the thermomechanical fatigue. However, experiments reveal that the varying temperature changes fatigue damage mechanisms and may accelerate material failure significantly. Recently quantifying thermomechanical fatigue damage becomes an important design issue, especially for high performance components.
Thermomechanical fatigue (TMF) means fatigue under both cyclic mechanical loads as well as cyclic temperature. Under thermomechanical fatigue the material experiences different damage processes and the fatigue life is sensitive to the loading configuration. In past years many TMF results were published mainly for uniaxial loading \cite{Evans2008, Kulawinski2015, remy2003thermal, Bauer2009}. The major goal was to demonstrate effects of the loading phase angles and to correlate fatigue life with the phase angle. Phenomenological fatigue models were proposed in \cite{Vose2013} to predict the material fatigue life under isothermal and thermomechanical loading conditions. 

% Uniaxial thermomechanical fatigue (TMF) tests \cite{remy2003thermal} were published and phenomenological fatigue models \cite{Vose2013} were proposed to predict the material fatigue life under isothermal and thermo-mechanical loading conditions.
Xiao (2006) \cite{Xiao2006157} studied the stress-controlled thermomechanical fatigue (TMF) behavior of Inconel 718 superalloy with different concentrations of boron (B) and carbon (C) with temperature varying between 350$^{\circ}$C and 650$^{\circ}$C at a stress ratio of 0.1. The IP-TMF life was observed to be much shorter than that of the OP-TMF, with a crossover occurring at the higher stress range.
Evans (2008) \cite{evans2008thermo} described the development of a thermo-mechanical fatigue test facility, the coil design and specimen gripping arrangement was modified from its original design to improve the thermal gradient.
The experimental results shows that the in-phase cycle (300$^{\circ}$C-680$^{\circ}$C) compared favourably with tests at 680$^{\circ}$C.
The out-of-phase test gave shorter lives on a strain range basis than either 300$^{\circ}$C or 680$^{\circ}$C isothermal tests.
These TMF failures were dominated by fatigue crack growth with evidence of closure.

However, these works are limited to the uniaxial loading condition.
It is well known the deformation and damage mechanisms under multiaxial loads can significantly differ from those under uniaxial loading \cite{fang2015cyclic, kang2004uniaxial, chen2004modified}. Most components in turbine engines typically experience significant variations in multiaxial states of stress, strain and temperatures under non-isothermal conditions. There are few works on the multiaxial TMF fatigue life published \cite{Brookes2010}.
% Most components in aero gas turbine engines, in particular in turbine blades, typically experience significant variations in multiaxial states of stress, strain and temperatures under non-isothermal conditions.

There are few works on the multiaxial TMF fatigue life published.
Brookes (2010) \cite{brookes2010axial} investigated the uniaxial, torsional and axial-torsional thermomechanical fatigue (TMF) behavior of the near-$\gamma$ TiAl-alloy TNB-V5 with temperature varying between 400 and 800$^{\circ}$C and mechanical strain amplitudes ranging from 0.15\% to 0.7\%. The non-proportional OP-TMF test has a shorter lifetime than the uniaxial OP-TMF test with the same Mises equivalent mechanical strain amplitude.
Kulawinski (2015) \cite{Kulawinski201521} investigated the thermo-mechanical uniaxial and biaxial-planar fatigue behavior of a nickel base superalloy for in-phase and out-of-phase loading between 400$^{\circ}$C and 650$^{\circ}$C.

Therefore, this study focuses on the thermomechanical fatigue under tension-torsion loading conditions and discusses multiaxial TMF fatigue. Both elastic-plastic behavior and fatigue life under multiaxial TMF loading condition are investigated. It is shown that the TMF affects fatigue performance of the material significantly and the TMF fatigue has to be considered with more information about the thermomechanical loads.

% \subsection{old}
% Gas turbine components experience severe cyclic multi-axial mechanical loadings and thermal loadings in the high pressure compressor and turbine modules.
% The increasing operating temperatures in gas turbines are pushing materials closer to their operating limits.
% Therefore, it is necessary to characterize the deformation and fatigue behavior of materials under the more realistic conditions \cite{harrison1996modelling}.
% Inconel 718 is a nickel-base superalloy and possesses attractive high temperature mechanical properties such as exceptionally high temperature strength and good resistance to oxidation and creep.
% The alloy has been extensively used in aircraft engines like the fabrication of turbine discs because of these overall properties. Its working temperature ranges from 600$^{\circ}$C to 700$^{\circ}$C as a key hot-end component in the aero-engines.
% High pressure turbine disc inevitably withstands hostile environment with respect to load and temperature during its service. Therefore, the material is inclined to suffer strain-controlled high temperature LCF and TMF.
% These two factors may reduce the service life of the working components significantly.
% Therefore, the LCF behavior and TMF of the nickel-base superalloy are of major concern considering the safety and reliability of gas turbine engines.


\section{Development of cyclic plasticity}
\noindent
Predicting fatigue performance of mechanical components needs a reliable constitutive description of the material. There were numerous studies on low cycle fatigue of the nickel-based superalloy published. In the past decades numerous constitutive models for cyclic inelasticity have been published \cite{ohno1993kinematic, Pun2014138, AbdelKarim2000225, Kang2004299}, based on the concept introduced by Armstrong and Frederick \cite{armstrong1966mathematical}. Kobayashi et al. \cite{Kobayashi2008389} and Pereira and Lerch \cite{Pereira2001715} studied Inconel 718 behavior using the Johnson-Cook constitutive relation and, however, did not take the cyclic softening behavior into account. 

Bai and Wierzbicki \cite{Bai20081071} modified the classical metal plasticity by taking the dependency of the Lode angle and stress triaxiality into account, which was extended by Algarni et al. \cite{Algarni2015140} to describe the evolution of yield surface under monotonic loading conditions. Becker and Hackenberg \cite{Becker2011596} suggested a limit surface concept and described the material behavior for a wide temperature range under monotonic and cyclic loading.
Farrahi \cite{Farrahi2014245} applied two plasticity approaches including the spring-slider rule of Nagode and the hardening rule of Chaboche to simulate cyclic behaviors.
Ohmenh\"{a}user (2014) \cite{Ohmenhauser2014631} adopted a two-layer rheological constitutive model for cyclic thermo-mechanical loading conditions.
However, neither non-proportionality nor multi-axiality of the applied loads was considered under thermo-mechanical cyclic loading conditions.
The alloy Inconel 718 exhibited a pronounced initial cyclic hardening then became continuous cyclic softening at high strain amplitudes. Experiments revealed that the conventional constitutive equations are not suitable to give correct stress and strain predictions for Inconel 718. The conventional constitutive models have to be improved to take the multi-axial, thermo-mechanical and cyclic loading conditions into account.

Zhu et al. \cite{ZHU2016} introduced a cyclic elasto-viscoplastic constitutive model for thermo-mechanically coupled problems. The model introduced the energy conservation in viscoplastic deformations and considered heating exchange due to high strain rate as well as its effects to mechanical behavior of the material. The work is interesting for metal forming, but not related to thermo-mechanical fatigue, in which the description of materials mechanical behavior under complex loading condition is of importance. 
Development of a reliable cyclic plasticity with experimental verification is necessary for predicting thermo-mechanical fatigue performance.

% The present papers introduce the experimental facility of a non-proportional thermomechanical fatigue test and discusses results of multiaxial TMF fatigue life.
% Finally both elastic-plastic behavior and fatigue life under TMF loading condition are investigated.
% It is shown that the TMF affects fatigue performance of the material significantly.

% Depending on the dominant parameters, fatigue models can be divided into several categories, stress-based models, stain-based models and energy-based models.
% Stress-based models are commonly for high cycle fatigue analysis.
% Stain and energy based models are usually used to describe low cycle fatigue.
% In addition, some models incorporate the critical plane in the fatigue criteria and the critical plane is dependent on the material, stress state and strain amplitude.
% Because of the different possible failure modes, no single analysis method is appropriate for all loading conditions.
% Accurate stress and strain distributions of the structural components are necessary for the fatigue parameter calculation.
% Therefore, the classical constitutive models need to be improved to take the multiaxial, thermomechanical and cyclic loading conditions into account.

% Many constitutive models for the description of cyclic inelasticity have been developed in the literatures over the past few decades: Prager (1955) \cite{prager1955new}; Besseling (1959) \cite{besseling1959theory}; Mroz (1967) \cite{mroz1967description}; Chaboche (1991) \cite{Chaboche1991661}; Ohno and Wang (1993) \cite{Ohno1993375}; McDowell (2000) \cite{McDowell2000293}; Kobayashi and Ohno (2002) \cite{kobayashi2002implementation}; Kang et al. (2003) \cite{Kang20031801}; Chen and Jiao (2004) \cite{Chen2004871}; Hassan et al. (2008) \cite{Hassan20081863}; Abdel-Karim (2010b) \cite{AbdelKarim2010170}; Becker and Hackenberg (2011) \cite{Becker2011596}; Yu et al. (2012) \cite{Yu201288}; Khutia et al. (2013) \cite{Khutia201388}; Safari et al. (2013) \cite{Safari2013402}. Also some of the reviews are presented by Chaboche (2008) \cite{Chaboche20081642} and Abdel-Karim (2010a) \cite{AbdelKarim2010711} for a detailed discussion of some other models.

Most of the models mentioned above are developed phenomenologically based on the macroscopic experimental results.
For general application of a stable material, Armstrong and Frederick, Chaboche and Ohno-Wang models are capable to capture the essential phenomena.
For more complex behavior, like the ratcheting, cyclic hardening/softening, non-proportionality, non-Masing behavior, additional equations must be incorporated.
But more equations will bring difficulties in numerical solving.
Implementation of an advanced plasticity theory to capture the mechanic behavior in general loading is still a challenge.
Cyclic plastic deformation is an essential component of the fatigue damage process.
Therefore, an understanding of multi-axial cyclic plastic deformation is often necessary, particularly in situations where significant plasticity exists such as in notches and in low cycle fatigue.
However, none of the models can be generally and precisely enough to predict all of the features during cyclic deformation owing to the complexity of phenomena in cyclic deformation.

% The present work focuses on the multi-axial cyclic behavior and the stress relaxations of Inconel 718 at varying temperatures between 300$^\circ$C and 650$^\circ$C. Effects of non-proportional cyclic loading as well as thermo-mechanical coupling are studied. Inconel 718 was proven to be hardly influenced by the strain rate below 650$^\circ$C under realistic loading speed \cite{kim1988elevated, Schlesinger2017}, and the constitutive model is assumed to be rate-independent. However, experiments reveal complex constitutive behavior of the material, such as cyclic hardening/softening, non-proportional hardening, thermo-mechanical phase effect, non-masing effect etc. Quantitative description of the material behavior needs a constitutive model based on extensive experimental investigation. In the present work a modified cyclic plasticity model is suggested for the nickel-based superalloy Inconel 718 under multi-axial thermo-mechanical cyclic loading conditions and can give a uniform description of the material modeling for both isothermal and thermo-mechanical fatigue loading components.

\section{Thermal gradient mechanical fatigue}
\noindent
% 飞机发动机推力的提高很大程度上依赖于涡轮前总温的提高。
% 对于高温所带来的一系列问题,解决的办法主要有以下三个:
% (1)提高材料的耐热性,发展高性能耐热合金,制造单晶叶片;
% (2)采用热障涂层,对基底材料起到隔热作用,降低基底温度;
% (3)采用先进的冷却技术,以少量的冷却空气获得更好的降温效果。
% 高温合金及单晶材料耐热性的提高远远无法满足目前的温度设计需求,即使采用陶瓷基复合材料等耐高温材料,也不能完全取消冷却,先进的冷却可使高温部件承受更高的工作温度,使发动机寿命更长、可靠性更高。
The improvement of aircraft engine thrust is highly dependent on the increment of the turbine inlet temperature.
The solutions of the problems caused by the high temperature are mainly as the following:
(1) improving heat resistance of materials, developing high-performance superalloy and manufacturing single crystal blades;
(2) the thermal barrier coating is used to protect the base material and reduce the substrate temperature;
(3) the use of advanced cooling technology, with a small amount of cooling air to achieve better cooling effect.
The improvement on heat resistance of high temperature alloy and single crystal materials can not meet the current temperature design requirement. It is cannot completely eliminate cooling even if using the ceramic matrix composites, which is high-temperature-resistant. The effective cooling method makes it possible that the high-temperature parts are capable of withstanding high operation temperature, so that the engine life will be longer and the reliability will be more higher.

% 发动机涡轮叶片主要采用气膜冷却和内部流冷却,轮盘通常采用内部二次流冷却。
% 服役过程中,涡轮叶片不仅受到较大的交变载荷,而且在叶片表面和内部分别受到高温高压燃气的冲击和冷却气体的作用,这样涡轮叶片就遭受载荷和温度同时变化带来的热机械疲劳损伤。
% 此外,为了增强发动机冷却效果,提高发动机效率,先进的航空发动机和燃气轮机热端涡轮叶片多为薄壁多孔结构。
% 因此,我们设计了薄壁圆管试件来模拟零件的冷却结构,同时试件外壁涂覆有热障涂层。
% 在内部冷却气体作用下,试件内表面与外表面之间会产生很大的温度梯度,
% 实验过程中,采用的常规感应加热设备只会对内部金属层加热,使得内部金属层温度高于外部陶瓷层温度,这不符合热障涂层构件实际工作状态下的温度分布。
% 实际发动机启动阶段升温过程只需要几秒钟的时间,而且降温也相当迅速,这些都对涂覆热障涂层的热梯度机械疲劳试验设备提出了更高的要求,同时也制约了这方面的研究。
% 由于热梯度机械疲劳是试验室中最接近涡轮叶片服役状态的模拟试验,因而这方面的研究对于理解涂覆热障涂层的叶片损伤机理具有重要意义。
The turbine blades are mainly cooled by air film and internal flow, and the turbine disk is usually cooled by the secondary flow.
In the course of service, the turbine blades are not only subjected to the large alternating loads, but also subjected to the impact of high temperature and high pressure gas and the effect of cooling air on the surface and inside of the blades, respectively. So the turbine blades suffer from the thermal mechanical fatigue damage caused by the change of load and temperature.
In addition, in order to enhance the engine cooling effect and improve the engine efficiency, most of the advanced aero-engine and gas turbine blades are thin-walled porous structures.
Therefore, we designed a thin-walled tube specimen to simulate the cooling structure of the parts, and the outer wall of the specimen was coated with the thermal barrier coating.
Under the action of the internal cooling air, a large temperature gradient between the inner surface and the outer surface of the specimen was produced.
In the experiment, the conventional induction heating equipment only heated the inner metal layer, which maked the temperature of inner metal layer higher than the external ceramic layer temperature. It did not conform to the temperature distribution of the thermal barrier coating components under the actual working state.
Actually, the engine start-up phase heating process only takes a few seconds, and the cooling stage is also very rapid. these are coating thermal barrier coatings on the heat gradient mechanical fatigue test equipment has put forward higher requirements, but also restricted the research.
Because the thermal gradient mechanical fatigue test is the closest to the service state of turbine blades in the laboratory, the research has an important sense to understand the blade damage mechanism of the coated thermal barrier coating.

% 同时,温度梯度会导致零件承受多轴载荷。
% 对于内部冷却的零件,温度梯度会产生额外的应力,在热表面上表现为多轴压缩载荷,而在冷却表面上表现为多轴拉伸载荷。
% 常规的热机械疲劳试验无法模拟这些应力条件并达到合适的温度分布,因此我们的试验系统是为热梯度机械疲劳(TGMF)测试而设计开发的。
% 该系统可以在空心试件表上实现受控的温度梯度循环,同时施加机械载荷。
% 通过聚光辐射的方法加热试件外表面,同时内表面通过压缩空气冷却来实现温度梯度。
% 该试验系统需要实现较高的加热和冷却速度,因此加热系统的功率和冷却气体的流量需要进行详细设计。
Moreover, the temperature gradient brings about multiaxial loads to the parts.
For internally cooled parts, the temperature gradient can produce additional stress, which is expressed as a multiaxial compressive load on the hotter surface, while a multiaxial tensile load is displayed on the cooler surface.
Conventional thermomechanical fatigue tests are not able to simulate these stress conditions and achieve a temperature gradient in the radial direction. So we designed this test system for thermal gradient mechanical fatigue (TGMF) testing.
The system can realize the controlled thermal cycle with temperature gradient on the hollow specimen and apply the mechanical load at the same time.
The external surface of the specimen is heated by the method of concentrating radiation, while the inner surface is cooled by compressed air to achieve the temperature gradient.
The test system needs to achieve high heating and cooling speed, so the power of heating system and the volume flow of the cooling air should be designed in detail.


%Thermal gradients cause multiaxial loads in cooled components.
%In internal cooling, for example of rotors in the first stage of a jet engine, the thermal gradient induced stresses can not be relaxed by macroscopic deformations.
%The stresses occurring at the component lead to multiaxial pressure loads on the hot surface and to multiaxial tensile loads on the cooled surface.
%Since conventional thermomechanical tests do not simulate these stress conditions and achieve a homogeneous temperature distribution, our test system was designed and developed for the Thermal Gradient Mechanical Fatigue (TGMF) tests.
%The system allows cyclic and simultaneously thermal and mechanical stress with controlled temperature gradients on the wall of hollow test specimens.
%The temperature gradient is achieved by heating the outer surface with a furnace which emits concentrated radiation and the inner surface is simultaneously cooled with compressed air.
%
%These realistic tests have the advantage that they can transfer data from laboratory tests to the operating conditions.
%In addition, the heating and cooling rates achieved in the TGMF test apparatus allow very short test cycles so that the fatigue load of an entire flight can be applied to a test body within three to five minutes.



\section{Research objective}
\noindent
The research objectives of the study is as follows,
\newcounter{Lcount}
\begin{list}{\arabic{Lcount}.}
  {\usecounter{Lcount}
  \setlength{\rightmargin}{\leftmargin}}
    \item To perform the strain controlled monotonic, cyclic isothermal and multi-axial thermo-mechanical fatigue (TMF) tests of the nickel-base superalloy Inconel 718.
    \item To derive a constitutive model of the superalloy Inconel 718 under multi-axial thermo-mechanical loading conditions.
    \item To implement the constitutive model into ABAQUS UMAT.
    \item To extend the constitutive model from small strain to finite deformation framework.
    \item To establish an efficacious multi-axial fatigue damage parameter to predict the fatigue life under non-proportional TMF loading histories.
    \item To design and develop a radiation furnace for thermal gradient mechanical fatigue (TGMF) tests.
    \item To propose a fatigue model for TGMF assessment.
\end{list}

\section{Outline of the thesis}
The thesis contains nine chapters which together with a review of relevant previous research, development of test facility, details the results of experimental, computational, analytical and microstructural work.
Because of the different testing procedures, the thesis is divided into two portions.

\textbf{The first portion is the thermomechanical fatigue (TMF) of the nickel-base superalloy Inconel 718.}
\begin{list}{\arabic{Lcount}.}
  {\usecounter{Lcount}
  \setlength{\rightmargin}{\leftmargin}}
    \item Development of a TMF test facility. The thermo-mechanical fatigue (TMF) test facility was developed to match the practice standards.
        Temperature control was modified from welded thermocouples to wrapped thermocouples.
        Coil design and specimen gripping arrangement was modified from its original design to improve the thermal gradient in the gauge length.
        The high temperature biaxial extensometer was tuned and calibrated specially for the Inconel 718 specimen and CMSX-4 specimen.
        The cooling water of the extensometer is optimized to ensure it confirm stable during the stain controlled thermomechanical fatigue tests.
    \item Multiaxial TMF tests. The Inconel 718 material behaviors was investigated under uniaxial, biaxial loading, isothermal and thermomechanical conditions.
    \item Constitutive models. The consecutive model is modified from the Fang's Model.
        Implementation of a suitable constitutive model into the commercial FEM code ABAQUS, which can give an accurate description of the essential cyclic mechanical behaviors.
    \item Fatigue models.
\end{list}

\textbf{The second portion is the thermal gradient mechanical fatigue (TGMF) of the single crystal material CMSX-4.}

\begin{list}{\arabic{Lcount}.}
  {\usecounter{Lcount}
  \setlength{\rightmargin}{\leftmargin}}
    \item Development of a thermal gradient mechanical fatigue (TGMF) test facility. A thermal gradient mechanical fatigue (TGMF) test facility was designed and developed for the CMSX-4 specimen with thermal barrier coatings (TBC).
        The details of the testing system and the design procedure were presented in Chapter 8.
    \item Experimental results.
\end{list} 