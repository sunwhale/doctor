\addchap{Abstract}                 % include abstract in the table of contents (without number)
\sectionmark{Abstract}             % correct section headings
\noindent
Mechanical parts in aero engines are loaded by both mechanical loading and thermal loading simultaneously. Experiments revealed that the mechanical behavior as well as fatigue performance of the material depend on amplitudes and the phase angle of the thermo-mechanical loads. Description of the interdependence of the loads is one of the most intensively investigated topics in aero engine structures. In the present work, extensive experiments are performed for a nickel-based superalloy under both iso-thermal and thermo-mechanical loading conditions. Furthermore, fatigue tests reveal that the thermomechanical loading reduces fatigue life of the material significantly. Based on experimental data a thermomechanical loading parameter is introduced to assess TMF fatigue failure. The TMF fatigue life can be calibrated based on the present concept reasonably. 

The axial-torsional thermomechanical fatigue behavior for Ni-based superalloy Inconel 718 was investigated experimentally in the temperature interval from 300C to 650C. The experimental results showed that, in the same von Mises equivalent mechanical strain amplitude condition, the fatigue life can be seriously decreased when the axial mechanical strain and the temperature are in-phase, and the non-proportional loading of mechanical strains can result in the fatigue life to further decrease. The analysis results showed that the fatigue lives depend on the summation of fatigue, creep and oxidation damages. The interaction between dislocations and precipitations can induce the additional hardening under mechanical non-proportional loading, which can obviously increase the fatigue damage. The tensile mean stress and the non-proportional additional hardening can further increase the oxidation damage existed in all loading conditions, which result in the increase of inelastic strain at high temperature. Creep damage behavior on grain boundaries can be shown in the case of the high tensile stress and the high temperature acting simultaneously, and increased with shear stress at the same time. In addition, the tensile stress has an obvious effect on the nucleation of creep cavities contrasted to the shear stress at high temperature, and more creep damage can be caused by the non-proportional additional hardening.