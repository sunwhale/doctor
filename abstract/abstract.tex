\addchap{Abstract}                 % include abstract in the table of contents (without number)
\sectionmark{Abstract}             % correct section headings
\noindent
Mechanical parts in aero engines are loaded by both mechanical loading and thermal loading simultaneously. Generally, failure due to the contribution of both thermal and mechanical loading is termed thermomechanical fatigue (TMF). For aero engines, internal cooling is important for reducing the temperature of parts and ensuring the normal operation of the parts. The temperature gradient induced by the internal cooling has a significant influence on the stress distribution and fatigue life of parts. Thermal gradient mechanical fatigue (TGMF) is the failure caused by the thermomechanical loading combined with the thermal gradient. In past years, many TMF results were published mainly for uniaxial loading but there are few works were established on the multiaxial TMF and TGMF. In the present work, the life assessments of multiaxial TMF and TGMF are studied experimentally and analytically. Furthermore, the predicting fatigue performance of mechanical components needs a reliable constitutive description of the material. A constitutive model was proposed and implemented. The main work of this study is summarized as follows:

(1) Thermomechanical and non-proportional loading affect mechanical behavior of metals and change the constitutive modeling. In the study, extensive experiments are systematically performed for a popular nickel-based superalloy Inconel 718 under both isothermal and thermo-mechanical loading conditions, to investigate the constitutive behavior and computational modeling. Within the frame of the Ohno-Wang cyclic plasticity a modified constitutive model has been suggested to meet  cyclic hardening/softening, non-proportional hardening, thermo-mechanical phase effect, non-masing effect observed in experiments. The suggested model describes both isothermal as well as thermo-mechanical experiments reasonably. Influences of the thermo-mechanical loading can be integrated into the non-proportional plasticity terms. The implicit integration algorithm of the constitutive model is developed and implemented into the commercial finite element code. Comparison between  experimental results and computations confirms that the model can predict the cyclic plastic behavior precisely under most different varying temperatures and multi-axial loading paths.

(2) Recent investigations reveal significant difference of thermomechanical fatigue (TMF) from isothermal fatigue. The influence of thermal phase angle and loading non-proportionality was investigated experimentally in the temperature interval from 573K to 923K. Various multiaxial fatigue life models show significant deviations due to different loading configurations and seem to not catch effects of the thermomechanical features in fatigue. Based on the experiments a thermomechanical loading parameter is introduced to assess fatigue failure. The new thermomechanical model can calibrate non-proportional thermomechanical multiaxial fatigue reasonably.

(3) A radiation furnace was developed for the thermal gradients mechanical fatigue (TGMF). The heat transfer behavior of the radiation furnace was systematically studied by the experimental and analytical methods, to determine the optical structure and optimieze the temperature cycling. The radiation furnace together with the loading, and cooling subsystems enables a cyclic and at the same time thermal and mechanical load with controlled temperature gradients on the wall of tubular specimens. The temperature gradient is achieved by heating the outer surface with the radiation furnace that emits concentrated radiation and simultaneously cooling the inner surface with compressed air. As a consequence, a TGMF experiment system on tubular specimen is constructed and the TGMF experimental methods are systematically studied. 

(4) TGMF experiments were performed under both in-phase and out-of-phase loading conditions to quantify the influence of the thermal gradient. Comparison of the TMF and TGMF experimental results showed that, in the same mechanical strain amplitude and thermal phase angle conditions, the fatigue life can be obviously decreased by the thermal gradient. The proposed TMF life model show seriously deviations and do not consider the effects of the thermal gradient in fatigue. Therefore, a correction term of the temperature gradient is suggested to assess the fatigue failure. The suggested TGMF life model was verified to be reasonably accurate when predicting the TGMF lifetime of Inconel 718 and most of the predicted fatigue lives are within the scatter band with a factor of 2.