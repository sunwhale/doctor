\addchap{Zusammenfassung}                 % include abstract in the table of contents (without number)
\sectionmark{Zusammenfassung}             % correct section headings
\noindent
Mechanische Teile in Triebwerken werden gleichzeitig mechanisch und thermisch belastet, die sogenannte thermomechanische Ermüdung (Thermo-Mechanical Fatigue, TMF). Im Triebwerk spielt die interne Kühlung eine große Rolle, um die Temperatur des Bauteils auf ein bestimmtes Niveau zu reduzieren und gleichzeitig die Maschine in einem Betriebszustand sicherzustellen. Die Untersuchung bestätigt, dass der durch die Innenkühlung induzierte Temperaturgradient einen signifikanten Einfluss auf die Spannungsverteilung und Lebensdauer des Bauteils hat. Die thermische gradienten-mechanische Ermüdung (Thermal Gradient Mechanical Fatigue, TGMF) ist eine wichtige Versagensform für die Bauteile, die durch thermomechanische Belastungen in Kombination mit dem Temperaturgradienten beschädigt werden. In den vergangenen Jahren wurden viele TMF-Untersuchungen hauptsächlich für uniaxiale Belastungen veröffentlicht, aber die mehrachsigen TMF sowie TGMF wurden jedoch wenig studiert. In der vorliegenden Arbeit wird die Lebensdauerabschätzung von mehrachsigen TMF und TGMF experimentell und numerisch untersucht. Darüber hinaus wird zur Vorhersage der Ermüdungsfestigkeit ein zuverlässiges Werkstoffmodell für TMF und TGMF Zustände eingeführt, das im Rahmen dieser Arbeit experimentell und rechnerisch untersucht und in eine Finite-Elemente-Software implementiert wird. Folgende Untersuchungen im Hinblick auf TMF/TGMF Lebensdauervorhersagen der Nickellegierung werden in dieser Arbeit durchgeführt und diskutiert:
\begin{enumerate}
\item Thermomechanische und nicht-proportionale Belastungen beeinflussen das Werkstoffverhalten und verändern die Werkstoffmodellierung. In diesem Zusammenhang werden umfangreiche Experimente für eine gängige Nickelbasislegierung Inconel 718 unter isothermischen und thermomechanischen Lastbedingungen durchgeführt, um das Werkstoffverhalten und die rechnerische Modellierung zu untersuchen. Im Rahmen vom Ohno-Wang Modell der zyklischen Plastizität wurde ein modifiziertes Stoffgesetz vorgeschlagen, um die in Experimenten beobachtete zyklische Verfestigung/Entfestigung, nichtproportionale Verfestigung, thermomechanischen Phaseneffekte und Nicht-Masing-Effekte usw. zu erfassen. Das vorgeschlagene Modell beschreibt sowohl isothermische als auch thermomechanische Experimente. Einflüsse der thermomechanischen Belastung können im nichtproportionalen Plastizitätsmodell integriert werden. Der implizite Integrationsalgorithmus des Stoffmodells wird entwickelt und in die kommerzielle Finite-Elemente-Software ABAQUS implementiert. Der Vergleich zwischen experimentellen Ergebnissen und Berechnungen bestätigt, dass das Modell das zyklische plastische Verhalten genau unter den verschiedensten Temperaturschwankungen und mehrachsigen Belastungspfaden zuverlässig vorhersagen kann.

\item Neue Untersuchungen zeigten den signifikanten Unterschied thermomechanischer Ermüdung (TMF) zu der isothermen Ermüdung. Der Einfluss des thermischen Phasenwinkels und der Lastnichtproportionalität wurde experimentell im Temperaturbereich von 300$^{\circ}$C bis 650$^{\circ}$C untersucht. Verschiedene mehrachsige Lebensdauermodelle zeigen signifikante Abweichungen aufgrund unterschiedlicher Lastkonfigurationen und können die thermomechanischen Auswirkungen auf Ermüdung nicht richtig beschreiben. Basierend auf den Experimenten wird ein thermomechanischer Lastparameter eingeführt, um Ermüdungsversagen besser zu bewerten. Das neue thermomechanische Modell kann die nichtproportionale thermomechanische mehrachsige Ermüdung kalibrieren.

\item Im Rahmen dieser Arbeit wurde ein Strahlungsofen für die thermische gradienten-mechanische Ermüdung (TGMF) entwickelt. Das Wärmeübertragungsverhalten des Strahlungsofens wurde systematisch experimentell und analytisch untersucht, um die optische Struktur zu bestimmen und den Temperaturzyklus zu optimieren. Der Strahlungsofen ermöglicht, zusammen mit dem mechanischen Teilsystem (die MTS Maschine) und einem Kühlungssystem, zyklische und thermo-mechanische Ermüdungsversuche mit kontrollierten Temperaturgradienten an rohrförmigen Proben. Der Temperaturgradient wird erreicht, indem die äußere Oberfläche durch die Strahlung erhitzt und die innere Oberfläche mit Druckluft gekühlt wird. Als Konsequenz wird ein TGMF-Versuchssystem an der röhrenförmigen Probe konstruiert, und die TGMF-experimentellen Methoden werden systematisch untersucht.

\item TGMF-Experimente wurden sowohl unter gleichphasigen als auch gegenphasigen Lastbedingungen durchgeführt, um den Einfluss des Temperaturgradienten zu quantifizieren. Ein Vergleich der TMF- und TGMF-Versuchsergebnisse zeigte, dass bei gleicher mechanischer Dehnungsamplitude und gleichen thermischen Phasenwinkelbedingungen die Ermüdungslebensdauer durch den thermischen Gradienten verringert werden konnte. Das vorgeschlagene TMF-Lebensdauermodell zeigt starke Abweichungen und berücksichtigt nicht die Auswirkungen des thermischen Gradienten auf die Ermüdung. Daher wird ein Korrekturterm des Temperaturgradienten vorgeschlagen, um die Ermüdung zu bewerten. Das vorgeschlagene TGMF-Lebensdauermodell wurde als hinreichend genau bestätigt, wenn die TGMF-Lebensdauer von Inconel 718 vorhergesagt wurde, und die meisten der vorhergesagten Ermüdungslebensdauer liegen innerhalb des Streubandes mit einem Faktor von 2.
\end{enumerate}


