\addchap{Zusammenfassung}                 % include abstract in the table of contents (without number)
\sectionmark{Zusammenfassung}             % correct section headings
\noindent
Mechanische Teile in Flugzeugtriebwerken werden gleichzeitig durch mechanische Belastung und thermische Belastung belastet. Experimente zeigten, dass das mechanische Verhalten sowie das Ermüdungsverhalten des Materials von den Amplituden und dem Phasenwinkel der thermomechanischen Lasten abhängen. Die Beschreibung der gegenseitigen Abhängigkeit der Lasten ist eines der am intensivsten untersuchten Themen in Flugzeugtriebwerken. In der vorliegenden Arbeit wurden umfangreiche Experimente für eine Nickelbasis-Superlegierung unter sowohl iso-thermischen als auch thermomechanischen Belastungsbedingungen durchgeführt. Weiterhin zeigen Ermüdungstests, dass die thermomechanische Belastung die Ermüdungslebensdauer des Materials signifikant reduziert. Basierend auf experimentellen Daten wird ein thermomechanischer Belastungsparameter eingeführt, um TMF Ermüdungsversagen zu bewerten. Die TMF-Ermüdungslebensdauer kann basierend auf dem vorliegenden Konzept vernünftig kalibriert werden.