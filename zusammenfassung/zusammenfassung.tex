\addchap{Zusammenfassung}                 % include abstract in the table of contents (without number)
\sectionmark{Zusammenfassung}             % correct section headings
\noindent
Mechanische Teile in Flugzeugtriebwerken werden gleichzeitig durch mechanische Belastung und thermische Belastung belastet. Im Allgemeinen wird Versagen aufgrund des Beitrags sowohl der thermischen als auch der mechanischen Belastung thermomechanische Ermüdung (TMF) genannt. Bei Flugzeugmotoren ist die interne Kühlung wichtig, um die Temperatur der Teile zu reduzieren und den normalen Betrieb der Teile sicherzustellen. Der durch die Innenkühlung induzierte Temperaturgradient hat einen signifikanten Einfluss auf die Spannungsverteilung und Lebensdauer der Teile. Thermische Gradienten-Ermüdung (TGMF) ist der Fehler, der durch die thermomechanische Belastung in Kombination mit dem thermischen Gradienten verursacht wird. In den vergangenen Jahren wurden viele TMF-Ergebnisse hauptsächlich für die uniaxiale Belastung veröffentlicht, aber es wurden nur wenige Arbeiten an der multiaxialen TMF und TGMF erstellt. In der vorliegenden Arbeit werden die Lebensdauerabschätzungen von multiaxialen TMF und TGMF experimentell und analytisch untersucht. Darüber hinaus benötigt die Vorhersage der Ermüdungsfestigkeit von mechanischen Komponenten eine zuverlässige konstitutive Beschreibung des Materials. Ein konstituierendes Modell wurde vorgeschlagen und umgesetzt. Die Hauptarbeit dieser Studie ist wie folgt zusammengefasst:

(1) Thermomechanische und nicht-proportionale Belastung beeinflussen das mechanische Verhalten von Metallen und verändern die konstitutive Modellierung. In der Studie werden umfangreiche Experimente für eine beliebte Nickelbasis-Superlegierung Inconel 718 unter isothermalen und thermomechanischen Belastungsbedingungen durchgeführt, um das konstitutive Verhalten und die computergestützte Modellierung zu untersuchen. Im Rahmen der zyklischen Plastizität von Ohno-Wang wurde ein modifiziertes konstitutives Modell vorgeschlagen, um die in Experimenten beobachtete zyklische Härtung / Erweichung, nicht-proportionale Härtung, thermomechanischen Phaseneffekt und Nicht-Masing-Effekt zu erfüllen. Das vorgeschlagene Modell beschreibt sowohl isotherme als auch thermo-mechanische Experimente einigermaßen. Einflüsse der thermomechanischen Belastung können in die nichtproportionalen Plastizitätsterme integriert werden. Der implizite Integrationsalgorithmus des konstitutiven Modells wird entwickelt und in den kommerziellen Finite-Elemente-Code implementiert. Der Vergleich zwischen experimentellen Ergebnissen und Berechnungen bestätigt, dass das Modell das zyklische plastische Verhalten genau unter den verschiedensten Temperaturschwankungen und multiaxialen Belastungspfaden vorhersagen kann.

(2) Neuere Untersuchungen zeigen den signifikanten Unterschied thermomechanischer Ermüdung (TMF) von isothermer Ermüdung. Der Einfluß des thermischen Phasenwinkels und der Belastungs-Nichtproportionalität wurde experimentell im Temperaturintervall von 300$^{\circ}$C bis 650$^{\circ}$C untersucht. Verschiedene multiaxiale Lebensdauermodelle zeigen signifikante Abweichungen aufgrund unterschiedlicher Belastungskonfigurationen und scheinen keine Auswirkungen der thermomechanischen Merkmale auf Ermüdung zu haben. Basierend auf den Experimenten wird ein thermomechanischer Belastungsparameter eingeführt, um Ermüdungsversagen zu bewerten. Das neue thermomechanische Modell kann die nicht-proportionale thermomechanische multiaxiale Ermüdung vernünftig kalibrieren.

(3) Ein Strahlungsofen wurde für die thermische Ermüdung (TGMF) entwickelt. Das Wärmeübertragungsverhalten des Strahlungsofens wurde systematisch mit den experimentellen und analytischen Methoden untersucht, um die optische Struktur zu bestimmen und den Temperaturzyklus zu optimieren. Der Strahlungsofen ermöglicht zusammen mit den Teilsystemen Beladung und Kühlung eine zyklische und gleichzeitig thermische und mechanische Belastung mit kontrollierten Temperaturgradienten an der Wand von rohrförmigen Proben. Der Temperaturgradient wird erreicht, indem die äußere Oberfläche mit dem Strahlungsofen, der konzentrierte Strahlung emittiert, erhitzt wird und gleichzeitig die innere Oberfläche mit Druckluft gekühlt wird. Als Konsequenz wird ein TGMF-Experimentsystem an der röhrenförmigen Probe konstruiert, und die TGMF-experimentellen Methoden werden systematisch untersucht.

(4) TGMF-Experimente wurden sowohl unter gleichphasigen als auch phasenverschobenen Beladungsbedingungen durchgeführt, um den Einfluss des thermischen Gradienten zu quantifizieren. Ein Vergleich der TMF- und TGMF-Versuchsergebnisse zeigte, dass bei gleicher mechanischer Dehnungsamplitude und gleichen thermischen Phasenwinkelbedingungen die Ermüdungslebensdauer durch den thermischen Gradienten verringert werden konnte. Das vorgeschlagene TMF-Lebensdauermodell zeigt starke Abweichungen und berücksichtigt nicht die Auswirkungen des thermischen Gradienten auf die Ermüdung. Daher wird ein Korrekturterm des Temperaturgradienten vorgeschlagen, um den Ermüdungsfehler zu bewerten. Das vorgeschlagene TGMF-Lebensdauermodell wurde als hinreichend genau bestätigt, wenn die TGMF-Lebensdauer von Inconel 718 vorhergesagt wurde, und die meisten der vorhergesagten Ermüdungslebensdauer liegen innerhalb des Streubandes mit einem Faktor von 2.