\chapter{Experimental investigation on Inconel 718}

\noindent
% 疲劳在结构及部件的安全评估中尤为重要,在实际工程中,很多疲劳损伤是由于机械载荷或热载荷引发的。传统的疲劳分析往往将热载荷转化为机械载荷加以分析,但这种方法无法考察温度和机械载荷叠加交互作用的影响。20世纪80年代后期,热机械疲劳试验系统的出现为该领域的发展提供了必要的条件,同时为了更好地定量研究材料在热机械疲劳下的损伤行为,国际上也普遍开展了热机械疲劳的试验研究工作。1995年5月在芬兰首次召开了国际热机械疲劳学术会议,就热机械疲劳试验方法、力学行为和寿命预测等问题进行了讨论。此后,ISO和ASTM等标准化组织也设置了专门的小组进行相关试验方法的研究,并且陆续推出了一些参考标准。
Fatigue is particularly crucial in the safety evaluation of structures and components. In practical projects, many fatigue damages are caused by mechanical or thermal loads. Traditional fatigue assessment often considers thermal loads as mechanical loads, but this method cannot describe the effects of temperature and mechanical load interactions. In the late 1980s, the development of the thermomechanical fatigue test system provided necessary conditions for the research of this field. At the same time, to study the damage behavior of materials quantitatively under thermomechanical fatigue conditions, thermomechanical fatigue tests are also commonly performed internationally. The first international conference on thermomechanical fatigue was held in Finland in May 1995. The issues of thermomechanical fatigue test methods, mechanical behavior and life prediction were discussed. Since then, ISO and ASTM and other standardization organizations have also set up special teams to research relevant test methods, and have successively introduced some reference standards.

% 本章对镍基高温合金Inconel 718开展了单轴拉伸、等温疲劳和热机械疲劳试验。其中单轴拉伸试验用于获取该合金材料在不同温度下的单调力学性能,等温疲劳试验用于获取该合金材料在不同温度下的循环力学性能。通过单轴拉伸和等温疲劳试验对该合金在高温下的低周疲劳性能和本构行为进行深入了解,为建立适用于多轴热机械疲劳载荷下的本构方程提供试验数据。在此基础上开展多轴热机械疲劳试验,获取材料在不同温度循环和不同机械应变路径下的变形行为及疲劳数据,用于验证本研究建议的本构模型和疲劳模型。

In this chapter, uniaxial tensile, isothermal fatigue and thermomechanical fatigue tests were performed on the nickel-base superalloy Inconel 718. The uniaxial tensile test is used to obtain the monotonic mechanical properties of the alloy at different temperatures, and the isothermal fatigue test is used to determine the cyclic mechanical properties of the alloy at different temperatures. The low-cycle fatigue behavior and constitutive behavior under high temperature were deeply understood by uniaxial tension and isothermal fatigue tests, and experimental data were provided for establishing a constitutive equation applicable to multiaxial thermomechanical fatigue loads. Based on the isothermal tests, multiaxial thermomechanical fatigue tests were carried out to obtain the deformation behavior and fatigue data of the material under different temperature cycles and different mechanical strain paths and used to validate the constitutive model and the fatigue model proposed in this study.

\section{Material specification}
\noindent
The investigated nickel-based superalloy Inconel 718 is manufactured by the company ThyssenKrupp, Germany, and provided in rods of 20 mm diameter. According to the standard ASTM B637 \cite{ASTMB63716}, the rods were solution-treated at 980$^{\circ}$C for one and a half hour then cooled to room temperature by water. Then they were aged for eight hours at 720$^{\circ}$C, furnace cooled at 56$^{\circ}$C/h to 621$^{\circ}$C, where they were held for eight hours and forced air cooling to room temperature.
The chemical compositions of the nickel-base superalloy are given in Table \ref{Tab:ChemicalCompositionofIN718}.

\begin{table}[htbp]
  \centering
  \caption{Chemical composition of Inconel (wt. \%) in the research.}
    \begin{tabular}{llllllllll}
    \toprule
    C     & S     & Cr    & Ni    & Mn    & Si    & Mo    & Ti    & Nb    & Cu \\
    \midrule
    0.02  & $<$0.001 & 18.53 & 53.44 & 0.05  & 0.06  & 3.06  & 0.99  & 5.30  & 0.04 \\
    \midrule
    Fe    & P     & Al    & Pb    & Co    & B     & Ta    & Se    & Bi    &  \\
    \midrule
    17.71 & 0.007 & 0.56  & 0.0002 & 0.13  & 0.004 & $<$0.01 & $<$0.0003 & $<$0.00003 &  \\
    \bottomrule
    \end{tabular}%
  \label{Tab:ChemicalCompositionofIN718}%
\end{table}%

The microstructure of the heat-treated alloy was characterized by an optical microscope. The typical optical micrographs are shown in \ref{Fig:MicrostructureofInconel718}. The material shows an isotropic grain distribution and the average grain size is about 15 $\rm{\upmu m}$.

\begin{figure}
  \begin{minipage}[t]{0.5\linewidth}
    \centering
    \begin{overpic}[width=8.0cm]{200X.png}
    \put(0,65){\fcolorbox{white}{white}{(a)}}
    \end{overpic}
  \end{minipage}%
  \begin{minipage}[t]{0.5\linewidth}
    \centering
    \begin{overpic}[width=8.0cm]{1000X.png}
    \put(0,65){\fcolorbox{white}{white}{(b)}}
    \end{overpic}
  \end{minipage}
  \caption{Microstructures of the investigated nickel-based superalloy Inconel 718 after solution treatment: (a) 200 times, (b) 1000 times.}
  \label{Fig:MicrostructureofInconel718}
\end{figure}

\section{Specimen specification}
\noindent
% 疲劳试验可以采用包括圆柱、空心圆管以及矩形等多种式样类型,从减少径向温度梯度角度出发,热机械疲劳试验方法标准中主要推荐采用圆管空心试样,其次为圆柱实心试样,而矩形试样相对较少。虽然圆管试样具有一定的优点,但其容易失稳屈曲以及内壁难以抛光的缺点限制了该类型试样的使用。此外,试样的加工工艺对试验结果也有直接的影响,试样的残余应力和粗糙度是影响其疲劳寿命的主要因素。
Monotonic tensile and uniaxial isothermal fatigue tests were performed by using the solid cylindrical specimen.
Its geometry is shown in \ref{Fig:Specimen}(a), a total length of 185 mm, an outer diameter of the gauge section of 10 mm and 35 mm in gauge length.
The outer surface roughness values of $R_{\rm{a}}=0.1$ $\rm{\upmu m}$.
The advantage of the solid specimen is easy to be manufactured.

\begin{figure}[!htp]
  \centering
  \begin{overpic}[width=12.0cm]{IN718_Axial_Specimen.pdf}
  \put(-10,20){\fcolorbox{white}{white}{(a)}}
  \end{overpic}
  \begin{overpic}[width=12.0cm]{IN718_Multiaxial_Specimen.pdf}
  \put(-10,20){\fcolorbox{white}{white}{(b)}}
  \end{overpic}
  \caption{Dimensions of (a) the solid specimen for uniaxial tests and (b) the tubular specimen for axial-torsional tests.}
  \label{Fig:Specimen}
\end{figure}

All multiaxial fatigue testing specimens were in thin-walled tubular shape.
The geometry of the thin-walled tubular specimen is shown in \ref{Fig:Specimen}(b), 185 mm in the total length, a gauge length of 35 mm, an outer diameter of the gage section of 12 mm and an inner diameter of the specimen of 10 mm.
Thus, the wall thickness of the tubular specimen in the gauge section is 1 mm.
The tubular specimen offers a uniform stress in the gage length, and the stress can be calculated directly from the applied loads.
Moreover, uniform temperature distributions in the gage length of the tubular specimens are easily achieved by induction heating, since relatively simple shape induction coils can be employed.

Both solid and thin-walled tubular specimens were designed by taking into the account of the available testing equipment and loading conditions, e.g., the gauge length of the extensometer, the height of the support frame of the extensometer, the height of the furnace and coil, and so on.
The experiments confirmed that the specimen geometry influenced the temperature distribution in the gauge length.
Consequently, the final geometries which presented in the above figures were determined by experimental and numerical analysis.
The dimensions of solid and tubular specimens are adapt to the dimension ratios suggested by the standards ASTM E606 \cite{astm1998standard} and E2207 \cite{ASTME2207}, respectively.
A discussion of different multiaxial testing techniques can be found in Socie (2000) \cite{Socie2000}.
All specimens were manufactured by a CNC machining center, and the surfaces were polished before testing.


%Axial and biaxial loading tests were respectively conducted on the solid and thin-walled tubular specimen.
%Dimensions of the specimens are shown in Figure \ref{Fig:IN718_Axial_Specimen} and \ref{Fig:IN718_Multiaxial_Specimen}.
%The dimensions of both type specimens have to fulfill the standards, e.g. ASTM E606 \cite{astm1998standard} and E2207 \cite{standard2007e2207}, and match the extensometer, the furnace height and the induction coil shape.

%Several requirements should be considered to determine the dimensions of the two type specimens.
%Firstly, ASTM E606\cite{astm1998standard} and E2207\cite{standard2007e2207} standards should be fulfilled.
%Secondly, the high temperature extensometers and furnace or induction coil need enough space to be installed.
%Thirdly, we notice that the specimen geometries influence the temperature distribution of the whole specimen.
%Consequently the final geometries are determined by the numerical analysis and temperature measurement.

\section{Experimental apparatus}
\noindent
% 本章的单调拉伸试验、等温疲劳试验和热机械疲劳试验均是在MTS Model 809液压伺服疲劳试验机进行的,如图\ref{Fig:MTS809}所示。MTS Model 809为轴向/扭转复合疲劳材料试验机,力加载量程为+-250 kN,扭矩加载量程为2200Nm,在软件系统的配合下可以实现复杂的加载路径。
Experiments were conducted on a servo controlled hydraulic system (MTS Model 809) with both electric resistance furnace (MTS Model 653) and induction heating system (TruHeat HF 3010), as shown in \ref{Fig:MTS809}.
The system is capable of applying axial and biaxial loads with temperature controlled simultaneously.
The MTS Model 809 is an axial/torsional testing rig for fatigue tests. It has the axial and rotational actuator which are independent on each other.
An axial/torsional coupler can ensure precise alignment during testing, and its testing capacity is $\pm$250 kN in the axial direction and $\pm$2200 N$\cdot$m in the torsional direction.
The testing rig is controlled by the MTS Flextest 40 digital controller. Moreover, the controller is connected to a host computer, which provides commands and acquires data.
The crosshead of the test rig can be moved to accommodate the dimensions of the heating device and strain measuring equipment.
During the test, the axial actuator can be controlled in displacement, load or strain mode and the torsional actuator can be controlled in torque, twist angle or angle strain mode.
At the same time, only one control mode can be active for each actuator.
Load cells are attached to the crosshead, and it can measure the axial force as well as the rotational torque.
The axial displacement and twist angle are measured by the linear-variable displacement transducers (LVDT) which are attached on the actuators.

\begin{figure}[htbp]
  \centering
  \includegraphics[width=14cm]{MTS809.pdf}
\caption{MTS Model 809 axial torsional materials testing system with the induction heating device.}
\label{Fig:MTS809}
\end{figure}

During the uniaxial test, the strain within the gauge length of the specimen was measured using an axial extensometer (see \ref{Fig:extensometer}(a)).
The gauge length of the extensometer is 12 mm, and the extensometer was calibrated in accordance with the recommendations of ASTM E83.
The extensometer has two ceramic rods which contact the specimen.
The contact force of the extensometer has to be as low as possible to avoid damage of the specimen surface leading to premature crack initiation at the contact edges of the extensometer.

\begin{figure}[!htp]
  \centering
    \begin{overpic}[width=8.0cm]{uniaxial_extensometer.jpg}
      \put(90,5){\fcolorbox{white}{white}{(a)}}
    \end{overpic}
    \begin{overpic}[width=8.0cm]{biaxial_extensometer.jpg}
      \put(90,5){\fcolorbox{white}{white}{(b)}}
    \end{overpic}
\caption{The strain within the gauge length of the specimen was measured by the high-temperature extensometers: (a) axial extensometer, (b) axial/torsional extensometer.}
\label{Fig:extensometer}
\end{figure}

\begin{figure}[!htp]
  \centering
  \includegraphics[width=15cm]{equipment.pdf}
  \caption{Locations of the induction heating coil, thermocouple and extensometer.}
  \label{Fig:Equipment}
\end{figure}

A water-cooled axial/torsional extensometer (see \ref{Fig:extensometer}(b)) was used to measure the deformation in the gage section of the specimen at elevated temperature.
The axial/torsional extensometer can measure the axial and shear strain simultaneously without the coupling effect.

The control loop feedback mechanism of the test system is a proportional–integral–derivative controller (PID controller).
The PID controller continuously calculates an error value as the difference between a desired setpoint (SP) and a measured process variable (PV) and applies a correction based on proportional, integral, and derivative terms.
The test system has to be tuned to achieve the experimental requirements so that its servo loop responds accurately its command signal.
The values of proportional, integral, and derivative gains of the PID controller can be determined by the PID tuning process.
Especially, for the axial/torsional extensometer, the PID gains have to be carefully tuned.
It is found that the stability of the strain control loop is dependent on the mass flow of the cooling water.
The mass flow of the cooling water has to be less than 1.5 l/min. Otherwise, the strain control loop will be instability.

The induction heating device is used for isothermal and thermomechanical fatigue tests.
As shown in \ref{Fig:Equipment}, the inductor is made of copper coil.
When the power supply sends an alternating current through the inductor, an alternating magnetic field is created.
When the metal specimen is placed within the inductor and enters the magnetic field, circulating eddy currents are induced within the specimen.
These eddy currents flow against the electrical resistivity of the metal, generating precise and localized heat without any direct contact between the specimen and the inductor.
Therefore, a certain temperature field can be generated in the gauge length of the specimen.
However, the geometry of the induction coil will influence the temperature distribution of the specimen.
The efficiency of induction heating is dependent on the coil density and the distance between the coil and specimen surface.
Concentrated coil density and a short distance to the specimen surface will result in high heat flux, vice versa.
Meanwhile, the installation of the extensometer has to be considered.
We need to adjust the coil to optimize the axial temperature distribution.
The strains were measured by the high-temperature axial and biaxial extensometers which contact the specimen surface with two ceramic rods.
The gauge lengths of the axial and biaxial extensometers are 12 mm and 25 mm, respectively.
\ref{Fig:Equipment} shows the locations of the induction heating device, thermocouple and extensometer.

% The maximum temperature deviations along the 12 mm gauge length were $\pm5^{\circ}$C and along the 25 mm gauge length were $\pm8^{\circ}$C.

\section{Monotonic tensile tests}
\noindent
The monotonic tensile test is the fundamental material test in which a specimen is subjected to a controlled tension until failure.
The purpose of the monotonic tensile tests is to determine the fundamental mechanical properties of Inconel 718 for using in the finite element calculations.
The monotonic tensile tests were performed at 20$^{\circ}$C, 300$^{\circ}$C, 550$^{\circ}$C and 650$^{\circ}$C.
The experimental procedure was in accordance with the standard ASTM E21.
The heating device was an electric resistance furnace (MTS Model 653), and the temperature was measured by the type K thermocouple which was sufficiently sensitive and reliable.
The specimens were in the air at atmospheric pressure.
Before the start of the test, the specimen was heated to the predetermined temperature and held on for an hour under zero force condition.
The aim is to ensure the temperature distribution of the specimen has reached equilibrium.

% The temperature measurement sensor was the type K thermocouple which was sufficiently sensitive and reliable to ensure that the temperature of the specimen is within the limits specified in the gauge length.
% At the beginning of each test with force kept constant at zero and the specimen was heated to the intended temperature and held on for an hour to ensure the temperature distribution of the specimen became stabilization.

\begin{figure}[!htp]
\centering{\includegraphics[width=14cm]{Furnace653_1.jpg}}
\caption{MTS Model 653 resistance furnace.}
\label{Fig:Furnace653_1}
\end{figure}

%Before the mechanical tension load applied, all specimens were heated to the set temperature by the resistance furnace and held the temperature for an hour until the temperature distribution of the specimen became to the steady state.
% For the duration of each test, the temperature deviations between the indicated and nominal values were less than $\pm1.5^{\circ}$C.
For the duration of each test, the temperature deviations between the predetermined and measured values were less than $\pm1.5^{\circ}$C.
The tensile tests were performed under displacement-controlled and the strain rates were about $1.0\times 10^{-4}$ $\rm{s}^{-1}$.
The extensometer was removed after the axial strain had reached 6\%.
The monotonic tensile test conditions are listed in Table \ref{tab:TensionLoadingConditions}.
\begin{table}[htbp]
  \centering
  \caption{The monotonic tensile test program.}
    \begin{tabular}{cccc}
    \toprule
    Temperature & Temperature deviation & Remove point of extensometer  & Strain rate  \\
    \midrule
    20$^{\circ}$C  & 0.1$^{\circ}$C & $6.0\%$ & $1.0\times 10^{-4}$ $\rm{s}^{-1}$ \\
    300$^{\circ}$C & 0.6$^{\circ}$C & $6.0\%$ & $1.0\times 10^{-4}$ $\rm{s}^{-1}$ \\
    550$^{\circ}$C & 1.3$^{\circ}$C & $6.0\%$ & $1.0\times 10^{-4}$ $\rm{s}^{-1}$ \\
    650$^{\circ}$C & 1.5$^{\circ}$C & $6.0\%$ & $1.0\times 10^{-4}$ $\rm{s}^{-1}$ \\
    \bottomrule
    \end{tabular}%
  \label{tab:TensionLoadingConditions}%
\end{table}%

%The test process involves placing the specimen in the testing machine, calibrating the zero point of force and extensometer , heating the specimen to the indicated temperature and slowly extending it until it fractures.
%Finally, the specimen was carefully removed and examined.
%During this process, The axial strain rate keeps constant as $1 \times 10^{-4}$s$^{-1}$.
%Extensometer measurement values of the gauge section is recorded against the applied force.
The elongation and force measurements are used to calculate the engineering stress $\sigma_{\rm eng}$ and engineering strain $\varepsilon_{\rm eng}$, using the following equation:
\begin{equation}
\sigma_{\rm eng}=\frac{F}{A_0},
\end{equation}
\begin{equation}
\varepsilon_{\rm eng}=\frac{l-l_0}{l_0},
\end{equation}
where $F$ is the axial force, $A_0$ is the specimen initial cross-sectional area, $\Delta l$ is the change of gauge length, $l_0$ and $l$ are the initial and final gauge lengths of the specimen.
Noting that, it is recommended to use true stress $\sigma$ and logarithmic strain $\varepsilon$ in large deformation condition, and they can be expressed as:
\begin{equation}
\sigma=\sigma_{\rm eng}(1+\varepsilon_{\rm eng}),
\end{equation}
\begin{equation}
\varepsilon=\ln(1+\varepsilon_{\rm eng}).
\end{equation}

\ref{Fig:plot_monotonic_tensile} shows the variation of the Inconel 718 tension stress-strain curves with different temperatures 20$^{\circ}$C, 300$^{\circ}$C, 550$^{\circ}$C and 650$^{\circ}$C.
The fundamental material properties of Inconel 718 at different temperatures can be determined from the experimental data, such as Young's modulus, yield strength, ultimate stress and fracture elongation (see Table \ref{tab:General_material_mechanical_properties}).
There is a considerable decline in Young's modulus, yield strength and ultimate stress with the increase of temperature.
%With the increase of temperature, the Young's modulus and yield strength show a certain decline.
However, the fracture elongation was insensitivity with the temperature.
As a result, the material still retains reasonable strength at 650$^{\circ}$C.

\begin{figure}[!htp]
  \centering
  \includegraphics[width=15cm]{plot_monotonic_tensile.pdf}
  \caption{Stress-strain response of Inconel 718 specimens under monotonic tensile loading at different temperatures.}
  \label{Fig:plot_monotonic_tensile}
\end{figure}

\begin{table}[htbp]
  \centering
  \caption{Results summary of Inconel 718 tensile tests.}
    \begin{tabular}{lcccc}
    \toprule
    Temperature         & Young's modulus   & Yield stress            & Ultimate stress     & Fracture elongation\\
    $T$ [$^{\circ}$C]   & $E$ [GPa]         & $\sigma_{p=0.2}$ [MPa]  & $\sigma_{\rm ult}$ [MPa]    & $\varepsilon_{\rm f}$ [\%]\\
    \midrule
    20    & 206.3 & 1192 & 1433 & 26.4 \\
    300   & 190.2 & 1140 & 1334 & 21.2 \\
    550   & 180.2 & 1090 & 1282 & 21.5 \\
    650   & 171.6 & 1064 & 1255 & 23.4 \\
    \bottomrule
    \end{tabular}%
  \label{tab:General_material_mechanical_properties}%
\end{table}%

%\begin{figure}[!htp]
%\centering\scalebox{0.5}{\includegraphics{plot_elastic_by_temperature_in718.pdf}}
%\caption{Elastic properties of Inconel 718 at different temperatures.}
%\label{Fig:plot_elastic_by_temperature_in718}
%\end{figure}
%
%% Table generated by Excel2LaTeX from sheet 'table'
%\begin{table}[htbp]
%  \centering
%  \caption{Elastic properties of Inconel 718 at different temperatures.}
%    \begin{tabular}{cccc}
%    \toprule
%    Temperature         & Young's Modulus   & Shear Modulus & Poisson's Ratio \\
%    $T$ [$^{\circ}$C]   & $E$ [GPa]         & $G$ [GPa]     & $\nu$\\
%    \midrule
%    16    & 208.21  & 80.79  & 0.289  \\
%    100   & 205.71  & 79.74  & 0.290  \\
%    200   & 198.57  & 77.07  & 0.288  \\
%    300   & 193.26  & 75.33  & 0.283  \\
%    400   & 185.01  & 72.27  & 0.280  \\
%    500   & 177.69  & 69.44  & 0.280  \\
%    600   & 170.84  & 66.46  & 0.285  \\
%    700   & 157.35  & 60.76  & 0.295  \\
%    \bottomrule
%    \end{tabular}%
%  \label{tab:addlabel}%
%\end{table}%
%
%\begin{equation}
%E=206308.7-51.2T+0.01109T^2-3.8439\times10^{-5}T^3 \rm{[MPa]}
%\end{equation}
%
%\begin{equation}
%\nu=0.29+1.458\times10^{-5}T-2.067\times10^{-7}T^2+2.780\times10^{-10}T^3
%\end{equation}

\section{Isothermal fatigue tests}
% \noindent
% In this section, both uniaxial and axial-torsional isothermal fatigue tests are introduced.
\subsection{Introduction of axial-torsional fatigue tests}
\noindent
Axial-torsional fatigue tests involved axial and torsional loading simultaneously with the predetermined phase relationship.
The test procedure was in accordance with the standard ASTM E2207 \cite{ASTME2207}.
The standard deals with strain-controlled, axial, torsional, and combined in-phase and out-of-phase axial/torsional fatigue tests with thin-walled, circular cross-section, tubular specimens at isothermal, ambient and elevated temperature.
Some concepts are defined as follows:

Axial strain $\varepsilon$, refers to engineering axial strain, and is defined as the change in length divided by the original length $\Delta L_{\rm{g}}/L_{\rm{g}}$.

Shear strain $\gamma$, refers to engineering shear strain, resulting from the application of a torsional moment to a cylindrical specimen.
Such a torsional shear strain is simple shear and is defined similar to axial strain with the exception that the shearing displacement, $\Delta L_{\rm{s}}$ is perpendicular to rather than parallel to the gage length, $L_{\rm{g}}$, that is, $\gamma=\Delta L_{\rm{s}}/L_{\rm{g}}$.
As shown in \ref{Fig:Shear_Strain}, $\varphi$ is the angle of twist along the gage length of the cylindrical specimen, $\theta$ expressed in radians is the angle of twist between the planes defining the gage length of the cylindrical specimen.
$\theta$ is measurable directly as angle using specially calibrated torsional extensometers.
Thus, we have $\gamma = (d/2)\theta/ L_{\rm{g}}$.

\begin{figure}[!htp]
  \centering
  \includegraphics[width=12cm]{shear_strain.pdf}
  \caption{Twisted gage section of a cylindrical specimen due to a torsional moment.}
  \label{Fig:Shear_Strain}
\end{figure}

Biaxial strain amplitude ratio $\lambda$, is defined as the ratio of the shear strain amplitude $\gamma_{\rm{a}}$ to the axial strain amplitude $\varepsilon_{\rm{a}}$, that is, $\gamma_{\rm{a}}/\varepsilon_{\rm{a}}$.

Axial stress $\sigma$, refers to the engineering stress, which is the ratio of force, $F$, to specimen original cross-sectional area, A, that is, $\sigma=F/A$.

Shear stress $\tau$, refers to engineering shear stress, acting in the orthogonal tangential and axial directions of the gage section and is a result of the applied torsional moment, (Torque) $T$, to the thin-walled tubular specimen. The shear stress, like the shear strain, is always the greatest at the outer diameter. Under elastic loading conditions, shear stress also varies linearly through the thin wall of the tubular specimen. However, under elasto-plastic loading conditions, shear stress tends to vary in a nonlinear fashion. Most strain-controlled axial-torsional fatigue tests are conducted under elasto-plastic loading conditions. Therefore, an assumption of a uniformly distributed shear stress is recommended. The relationship between such shear stress applied to the mean diameter of the gage section and the torsional moment, $T$, is
\begin{equation}
\tau=\frac{16T}{\pi(d_{\rm{o}}^2-d_{\rm{i}}^2)(d_{\rm{o}}+d_{\rm{i}})},
\label{Equ:shear_stress}
\end{equation}
where, $\tau$ is the shear stress, $d_{\rm{o}}$ and $d_{\rm{i}}$ are the outer and inner diameters of the tubular specimen, respectively.

\subsection{Strain paths}
\noindent
\ref{Fig:LoadPath} shows schematically and exemplary six fully reversed mechanical strain paths for uniaxial tension-compression (\ref{Fig:LoadPath}(a)), pure torsional (\ref{Fig:LoadPath}(b)), and axial–torsional (\ref{Fig:LoadPath}(c,d,e,f)) tests.
For convenience, hereafter, the six strain path which shown in \ref{Fig:LoadPath} from (a) to (f) are denoted as 'TC', 'TOR', 'CRO', 'PRO', 'NPR' and 'CIR', respectively.
\begin{figure}[ht]
  \begin{minipage}[t]{0.33\linewidth}
    \centering
    \includegraphics[width=4.5cm]{load_path_1.pdf}
    \centerline{\small (a) Tension-Compression(TC) path.}
  \end{minipage}
  \begin{minipage}[t]{0.33\linewidth}
    \centering
    \includegraphics[width=4.5cm]{load_path_5.pdf}
    \centerline{\small (b) Torsional(TOR) path.}
  \end{minipage}
  \begin{minipage}[t]{0.33\linewidth}
    \centering
    \includegraphics[width=4.5cm]{load_path_2.pdf}
    \centerline{\small (c) Proportional(PRO) Path.}
  \end{minipage}%

  \begin{minipage}[t]{0.33\linewidth}
    \centering
    \includegraphics[width=4.5cm]{load_path_4.pdf}
    \centerline{\small (d) Cross(CRO) Path.}
  \end{minipage}
  \begin{minipage}[t]{0.33\linewidth}
    \centering
    \includegraphics[width=4.5cm]{load_path_3.pdf}
    \centerline{\small (e) Diamond(NPR) Path.}
  \end{minipage}
  \begin{minipage}[t]{0.33\linewidth}
    \centering
    \includegraphics[width=4.5cm]{load_path_6.pdf}
    \centerline{\small (f) Circle(CIR) Path.}
  \end{minipage}%
  \caption{Schematic representation of different strain paths.}
  \label{Fig:LoadPath}
\end{figure}
The strain paths are described by the mechanical strain $\varepsilon_{\rm{mech}}$ and equivalent shear strain $\gamma/\sqrt{3}$ in a cycle.
\ref{Fig:LoadPath}(c) represents a 45$^\circ$ proportional loading with the ratio $\varepsilon_{\rm{mech}}/(\gamma/\sqrt{3})=1$. \ref{Fig:LoadPath}(e) shows a diamond non-proportional strain path (i.e. the ratio of mechanical strain amplitude and shear strain amplitude is 1). \ref{Fig:LoadPath}(f) shows a circle non-proportional strain path (i.e. the phase between mechanical and torsional strain is 90$^\circ$).
Loading ratio $R_\varepsilon$ means the ratio of minimum mechanical strain to the maximum mechanical strain in a loading cycle. Under the axial-torsional condition, the loading ratio is calculated from the principal strains.
All isothermal LCF tests were carried out with the given loading ratio $R_{\varepsilon}=\varepsilon_{{\rm{mech}},\min}/\varepsilon_{{\rm{mech}},\max}=-1$.
The tension-compression tests were performed on solid specimens (see \ref{Fig:Specimen}(a)), while the pure torsional, and axial–torsional tests were performed on tubular specimens (see \ref{Fig:Specimen}(b)).

\subsection{Process of isothermal fatigue tests}
\noindent
Both uniaxial and axial-torsional strain controlled isothermal fatigue (IF) tests were performed with the induction heating device.
The temperature was measured by the thermocouple wrapped around the surface at the center of the gauge section of the specimen.
For the duration of each IF test, the temperature deviations between the indicated and nominal values were less than $\pm$3$^\circ$C.
During the uniaxial and axial-torsional IF tests, the strains were measured by the axial extensometer (MTS Model 632.53) and the axial/torsional extensometer (MTS Model 632.68), respectively.

% The isothermal LCF tests were conducted under six fully reversed strain paths. As introduced above, the strain paths are shown in \ref{Fig:LoadPath}. The test conditions are summarized in Table \ref{tab:Loading-Conditions}.

The isothermal LCF tests were conducted under five fully reversed strain paths which include tension-compression (\ref{Fig:LoadPath}(a)), pure torsional (\ref{Fig:LoadPath}(b)) and non-proportional (\ref{Fig:LoadPath}(d,e,f)) strain paths.
% The isothermal LCF experimental program consisted of axial, pure torsional and axial-torsional tests within the temperature range from 300$^\circ$C to 650$^\circ$C.
The test conditions are summarized in Table \ref{tab:Loading-Conditions}.
In Table \ref{tab:Loading-Conditions}, the test type consists of the strain path and temperature condition. For example, TC-IF means the isothermal fatigue test with the tension-compression strain path, TOR-IF means the isothermal fatigue test with the pure torsional strain, and so on.
All tests were strain controlled, and the stress-strain hysteresis loops were recorded for each cycle.
% The strain path is defined as the combination of axial strain $\varepsilon$ and shear strain $\gamma$ in a cycle under the biaxial loaded strain-controlled conditions, such as tension-compression (\ref{Fig:LoadPath}(a)), pure torsional (\ref{Fig:LoadPath}(b)), proportional (\ref{Fig:LoadPath}(c)) and non-proportional (\ref{Fig:LoadPath}(d,e,f)) strain paths.

\begin{table}[ht]
  \centering
  \caption{Temperature and loading conditions of the isothermal tests program.}
    \begin{tabular}{p{2cm}<{\centering}p{2cm}<{\centering}p{3.5cm}<{\centering}p{3.5cm}<{\centering}p{2cm}<{\centering}}
    \toprule
    Test type & Temperature & Axial strain range & Shear strain range & Strain rate \\
          & $T$ [$^\circ$C] & $\pm \varepsilon$ [\%] & $\pm \gamma/ \sqrt 3$ [\%] & $\dot \varepsilon_{\rm{eq}}$ [s$^{-1}$] \\
    \midrule
    TC-IF    & 300   & 1.00  & -     & $1.00\times 10^{-3}$ \\
          & 400   & 1.00  & -     & $1.00\times 10^{-3}$ \\
          & 550   & 1.00  & -     & $1.00\times 10^{-3}$ \\
          & 650   & 1.00  & -     & $1.00\times 10^{-3}$ \\
          & 650   & 0.80  & -     & $1.00\times 10^{-3}$ \\
          & 650   & 0.70  & -     & $1.00\times 10^{-3}$ \\
          & 650   & 0.60  & -     & $1.00\times 10^{-3}$ \\
          & 650   & 0.50  & -     & $1.00\times 10^{-3}$ \\
          & 650   & 0.45  & -     & $1.00\times 10^{-3}$ \\
          & 650   & 0.40  & -     & $6.40\times 10^{-3}$ \\
    \midrule
    TOR-IF   & 300   & -     & 0.60  & $6.00\times 10^{-4}$ \\
          & 300   & -     & 0.73  & $7.30\times 10^{-4}$ \\
          & 300   & -     & 0.85  & $8.50\times 10^{-4}$ \\
          & 300   & -     & 1.00  & $1.00\times 10^{-3}$ \\
          & 550   & -     & 0.85  & $8.50\times 10^{-4}$ \\
          & 650   & -     & 0.85  & $8.50\times 10^{-4}$ \\
    \midrule
    CIR-IF  & 300   & 0.60  & 0.73  & $0.94\times 10^{-4}$ \\
          & 550   & 0.60  & 0.73  & $0.94\times 10^{-4}$ \\
          & 650   & 0.60  & 0.73  & $0.94\times 10^{-4}$ \\
          & 650   & 0.60  & 0.60  & $0.85\times 10^{-4}$ \\
    \midrule
    NPR-IF  & 300   & 1.00  & 1.00  & $1.44\times 10^{-4}$ \\
          & 550   & 1.00  & 1.00  & $1.44\times 10^{-4}$ \\
          & 650   & 1.00  & 1.00  & $1.44\times 10^{-4}$ \\
    \midrule
    CRO-IF  & 300   & 1.00  & 1.00  & $1.44\times 10^{-4}$ \\
          & 550   & 1.00  & 1.00  & $1.44\times 10^{-4}$ \\
          & 650   & 1.00  & 1.00  & $1.44\times 10^{-4}$ \\
    \bottomrule
    \end{tabular}%
  \label{tab:Loading-Conditions}%
\end{table}%

% \begin{table}[ht]
%   \centering
%   \caption{Temperature and loading conditions of the isothermal tests program.}
%     \begin{tabular}{p{2cm}p{2cm}p{3cm}p{3cm}p{2cm}}
%     \toprule
%     Test type & Temperature & Axial strain range & Shear strain range & Strain rate \\
%           & $T$ [K] & $\pm \varepsilon _{\rm mech}$ [\%] & $\pm \gamma/ \sqrt 3$ [\%] & $\dot \varepsilon_{\rm{eq}}$ [s$^{-1}$] \\
%     \midrule
%     TC-IF    & 573   & 1.00  & -     & $1.00\times 10^{-3}$ \\
%           & 673   & 1.00  & -     & $1.00\times 10^{-3}$ \\
%           & 823   & 1.00  & -     & $1.00\times 10^{-3}$ \\
%           & 923   & 1.00  & -     & $1.00\times 10^{-3}$ \\
%           & 923   & 0.80  & -     & $1.00\times 10^{-3}$ \\
%           & 923   & 0.70  & -     & $1.00\times 10^{-3}$ \\
%           & 923   & 0.60  & -     & $1.00\times 10^{-3}$ \\
%           & 923   & 0.50  & -     & $1.00\times 10^{-3}$ \\
%           & 923   & 0.45  & -     & $1.00\times 10^{-3}$ \\
%           & 923   & 0.40  & -     & $6.40\times 10^{-3}$ \\
%     \midrule
%     TOR-IF   & 573   & -     & 0.60  & $6.00\times 10^{-4}$ \\
%           & 573   & -     & 0.73  & $7.30\times 10^{-4}$ \\
%           & 573   & -     & 0.85  & $8.50\times 10^{-4}$ \\
%           & 573   & -     & 1.00  & $1.00\times 10^{-3}$ \\
%           & 823   & -     & 0.85  & $8.50\times 10^{-4}$ \\
%           & 923   & -     & 0.85  & $8.50\times 10^{-4}$ \\
%     \midrule
%     CIR-IF  & 573   & 0.60  & 0.73  & $0.94\times 10^{-4}$ \\
%           & 823   & 0.60  & 0.73  & $0.94\times 10^{-4}$ \\
%           & 923   & 0.60  & 0.73  & $0.94\times 10^{-4}$ \\
%           & 923   & 0.60  & 0.60  & $0.85\times 10^{-4}$ \\
%     \midrule
%     NPR-IF  & 573   & 1.00  & 1.00  & $1.44\times 10^{-4}$ \\
%           & 823   & 1.00  & 1.00  & $1.44\times 10^{-4}$ \\
%           & 923   & 1.00  & 1.00  & $1.44\times 10^{-4}$ \\
%     \midrule
%     CRO-IF  & 573   & 1.00  & 1.00  & $1.44\times 10^{-4}$ \\
%           & 823   & 1.00  & 1.00  & $1.44\times 10^{-4}$ \\
%           & 923   & 1.00  & 1.00  & $1.44\times 10^{-4}$ \\
%     \bottomrule
%     \end{tabular}%
%   \label{tab:Loading-Conditions}%
% \end{table}%

The uniaxial tension-compression tests were performed at 300$^\circ$C, 400$^\circ$C, 550$^\circ$C and 650$^\circ$C.
Triangular waveforms were applied at the predetermined strain rates over the strain range between $\pm 0.4\%$ and $\pm 1.0\%$.
For uniaxial tests, the strain rates were below $1.0\times 10^{-3}$ $\rm{s}^{-1}$.
The pure torsional and axial-torsional tests were performed at 300$^\circ$C, 550$^\circ$C and 650$^\circ$C.
% A axial-torsional test involved axial and torsional loading simultaneously with a predetermined phase relationship.
As introduced above, the procedure of axial-torsional test was in accordance with the standard ASTM E2207 \cite{ASTME2207}.
The process of the LCF tests was defined by using an MTS Multipurpose software (MPE).
The strain path, temperature, strain rate were given to the software.
Furthermore, a peak/valley compensation (PVC) method was used to obtain an accurate feedback value.
Failure was deemed to have occurred if the peak load dropped 15\% from its stabilized value. A 30\% load drop in comparing with the stabilized peak value was defined as failure, corresponding to nucleation of macrocracks.
Finally, the tests were stopped at the 200,000 cycle if no failure was detected.


% The procedure of axial-torsional test was in accordance with the standard ASTM E2207 which deals with strain-controlled, axial, torsional and combined in-phase and out-of-phase axial torsional fatigue testing with thin-walled tubular specimens under isothermal condition.
% The axial strain $\varepsilon$ refers to engineering axial strain, defined as $\varepsilon=\Delta L_{\rm{g}}/L_{\rm{g}}$. The engineering shear strain $\gamma$ is defined as twist of the specimen.
\subsection{Experimental results}
\noindent
% 当进行疲劳试验时,在循环载荷作用下,循环开始的应力应变滞后回线是不封闭的,封闭的滞后回线只有经过一定循环次数后才形成。金属材料在由循环开始状态变成循环稳定状态的过程,与其在循环应力及应变作用下的形变抗力变化有关。
% 在低周疲劳的循环加载初期,材料对循环加载的响应是一个由不稳定向稳定过渡的过程。此过程可分别用应力控制下的应变-时间曲线和应变控制下的应力-时间曲线描述。循环硬化是材料在循环过程中变形抗力不断提高、应变逐渐减小的现象。循环软化则是材料在循环过程中变形抗力不断减小而应变逐渐增加的现象。
% 在低周疲劳的循环加载初期,材料对循环加载的响应是一个由不稳定向稳定过渡的过程。
At the initial stage of cyclic loading of low cycle fatigue, the material's response to cyclic loading is a transition from instability to stability.
% 此过程可分别用应力控制下的应变-时间曲线和应变控制下的应力-时间曲线描述。
The strain-time curve can describe this process under stress control or stress-time curve under strain control, respectively.
% 金属材料在循环加载过程中的应力应变曲线通常有两种类型:循环硬化及循环软化。根据加载控制方式的不同,循环软化存在两种表现形式:当保持应力幅值恒定时,应变响应随循环次数不断增大;当保持应变幅值恒定时,应力响应随循环次数不断减小。
There are usually two types of stress-strain responses for metal materials during cyclic loading: cyclic hardening and cyclic softening.
According to different loading control methods, cycle softening has two manifestations: when the stress amplitude is kept constant, the strain response increases with the number of cycles; when the strain amplitude is kept constant, the stress response decreases with the number of cycles.

\ref{Fig:plot_exp_TC-IF_2} shows the stress-strain responses of tension-compression fatigue tests at the temperatures 300$^\circ$C, 550$^\circ$C and 650$^\circ$C with the mechanical strain range $\varepsilon=\pm1.0\%$. \ref{Fig:plot_exp_TOR-IF} shows the stress-strain responses of pure torsional tests at temperatures 300$^\circ$C, 550$^\circ$C and 650$^\circ$C with the shear strain range $\gamma=\pm1.47\%$. It can be observed that the cyclic stress-strain response is closely related to the temperature. For the same strain amplitude, the cycle softening is greater with increasing temperature. This is similar to the phenomenon observed by \cite{Fournier1977, Xiao2005,kim1988elevated, Schlesinger2017} when studying Inconel 718 under elevated temperature.
% 其中等温拉压试验的应变幅值为+-1\%,扭转试验的剪应变幅值为+-1.47\%。可以看出,循环应力响应与温度密切相关,对于相同的应变幅值,材料的循环软化程度更大。
% 导致材料出现循环软化现象的微观因素有很多,比如沉淀剪切、析出相的溶解、共格的丧失以及位错退化与重排等。
Many microstructural evolutions cause cyclic softening of materials, such as shearing of precipitate phases, dissolution of precipitates, degradation, and rearrangement of dislocations.
% 对于镍基高温合金Inconel 718循环软化现象的解释多为沉淀相的剪切机制。
The explanation for the cyclic softening phenomenon of the nickel-based superalloy Inconel 718 is mostly the shearing of the $\gamma ''$ precipitates.

As shown in \ref{Fig:plot_exp_TC-IF_650C}, the saturated stress-strain hysteresis loops are presented for all performed tests at 650$^\circ$C.
It can be observed that Inconel 718 occurred cyclic softening at 650$^\circ$C for all investigated strain amplitudes.

% \cite{Fournier1977}研究Inconel 718分别在293 K和550$^\circ$C下低周疲劳性能,发现在常温下材料先出现硬化而后变为软化,在高温下仅出现循环软化现象。他们得出的结论是在循环应变过程中沉淀相被剪切。类似的,Xiao\cite{Xiao2005}等针对Inconel 718在室温以及高温920 K下观察到了明显的循环软化以及平面滑移带,且共格和非共格的沉淀相$\gamma ''$移动的位错在{111}滑移面上剪切共格和非共格的沉淀相γ’’。此外,Xiao[46]等还研究了650oC时Inconel718中不同含量的硼元素对于材料的低周疲劳性能影响,结果发现四种不同含量硼元素的Inconel718合金均表现为循环软化,对此他们沿用之前的解释,沉淀相的剪切在开始阶段建立了位错移动的优先路径,这是Inconel718在650oC出现循环软化和平面滑移带的主要原因。E.HariKrishna[44]等研究了传统的和改进之后的Inconel718合金高温650oC下的低周疲劳性能,同样发现两种合金均出现了循环软化现象。他们指出,在载荷开始循环的几圈中由于短程有序的破坏而形成滑移带,且对于镍基高温合金随应变幅值增大滑移带数目增多。在γ’和γ’’作为强化相的镍基高温合金中通常会形成平面滑移带,这导致强化相受到剪切,推动了滑移带中的应变局部化。由于高应变幅值下平面滑移带数目更多,更多的强化相受到剪切而导致软化程度越高。

\begin{figure}[!htp]
  \centering
  \begin{overpic}[width=8.0cm]{plot_exp_half_life_cycle_TC-IF_2.pdf}
  \put(84,13){(a)}
  \end{overpic}
  \begin{overpic}[width=8.0cm]{plot_exp_pv_TC-IF_2.pdf}
  \put(84,13){(b)}
  \end{overpic}
  \caption{Experimental results of isothermal tension-compression tests with the axial strain range $\varepsilon = \pm1.0\%$ at temperatures 300$^\circ$C, 550$^\circ$C, and 650$^\circ$C: (a) stress-strain loops, (b) peak-valley stresses.}
  \label{Fig:plot_exp_TC-IF_2}
\end{figure}

\begin{figure}[!htp]
  \centering
  \begin{overpic}[width=8.0cm]{plot_exp_half_life_cycle_TOR-IF.pdf}
  \put(84,13){(a)}
  \end{overpic}
  \begin{overpic}[width=8.0cm]{plot_exp_pv_TOR-IF.pdf}
  \put(84,13){(b)}
  \end{overpic}
  \caption{Experimental results of isothermal pure torsional tests with the shear strain range $\gamma = \pm1.47\%$ at temperatures 300$^\circ$C, 550$^\circ$C, and 650$^\circ$C: (a) stress-strain loops, (b) peak-valley stresses.}
  \label{Fig:plot_exp_TOR-IF}
\end{figure}

\begin{figure}[!htp]
  \centering
  \begin{overpic}[width=8.0cm]{plot_exp_half_life_cycle_TCIF_650C.pdf}
  \put(84,13){(a)}
  \end{overpic}
  \begin{overpic}[width=8.0cm]{plot_exp_pv_TCIF_650C.pdf}
  \put(84,13){(b)}
  \end{overpic}
  \caption{Experimental results of isothermal tension-compression fatigue tests with different strain amplitude at 650$^\circ$C: (a) stress-strain loops, (b) peak-valley stresses.}
  \label{Fig:plot_exp_TC-IF_650C}
\end{figure}

\begin{figure}[!htp]
  \centering
  \begin{overpic}[width=8.0cm]{plot_exp_half_life_cycle_CIRC-IF_axial.pdf}
  \put(84,13){(a)}
  \end{overpic}
  \begin{overpic}[width=8.0cm]{plot_exp_pv_CIRC-IF_axial.pdf}
  \put(84,13){(b)}
  \end{overpic}
  \begin{overpic}[width=8.0cm]{plot_exp_half_life_cycle_CIRC-IF_torsional.pdf}
  \put(84,13){(c)}
  \end{overpic}
  \begin{overpic}[width=8.0cm]{plot_exp_pv_CIRC-IF_torsional.pdf}
  \put(84,13){(d)}
  \end{overpic}
  \caption{Experimental results of isothermal circle path tests at temperatures 300$^\circ$C, 550$^\circ$C, and 650$^\circ$C: (a) axial stress-strain loops, (b) axial peak-valley stresses, (c) torsional stress-strain loops, (d) torsional peak-valley stresses.}
  \label{Fig:plot_exp_pv_CIRC-IF}
\end{figure}

% \begin{figure}
%  \begin{minipage}[t]{0.5\linewidth} % 如果一行放2个图,用0.5,如果3个图,用0.33\
%  \nonumber
%    \centering
%    \includegraphics[width=3.5in]{Axial+-1D0_300C.pdf}
%    \centerline{(a) 300$^{\circ}$C.}
%  \end{minipage}%
%  \begin{minipage}[t]{0.5\linewidth}
%    \centering
%    \includegraphics[width=3.5in]{Axial+-1D0_400C.pdf}
%    \centerline{(b) 400$^{\circ}$C.}
%  \end{minipage}

%  \begin{minipage}[t]{0.5\linewidth} % 如果一行放2个图,用0.5,如果3个图,用0.33\
%  \nonumber
%    \centering
%    \includegraphics[width=3.5in]{Axial+-1D0_550C.pdf}
%    \centerline{(c) 550$^{\circ}$C.}
%  \end{minipage}%
%  \begin{minipage}[t]{0.5\linewidth}
%    \centering
%    \includegraphics[width=3.5in]{Axial+-1D0_650C.pdf}
%    \centerline{(d) 650$^{\circ}$C.}
%  \end{minipage}
%  \caption{Isothermal tension-compression stress-strain loops with strain range +-1\% at different temperatures.}
%  \label{Fig:LoadPath}
% \end{figure}

\section{Thermomechanical fatigue tests}
\subsection{Introduction of thermomechanical fatigue tests}
\noindent
Thermomechanical fatigue testing is a variant of fatigue testing designed to simulate real-world service conditions of engineered components.
Specifically, TMF tests characterize the response of materials to both cyclic mechanical loading and fluctuating temperature, which produce a synergistic response that may not be well-predicted with isothermal fatigue testing.
TMF tests are most frequently conducted on high-temperature alloys used in jet engines, using a combination of in-phase and out-of-phase mechanical and thermal cycling.

% 与高温等温疲劳试验不同,TMF试验过程中,需要同时控制温度和机械载荷两个变量,温度处于循环状态。
% 目前常用的加热方式包括induction, direct resistance, radiant, and forced air heating等。
% 电阻式加热炉难以满足试验对于升温和降温速率的要求,主流疲劳试验设备制造商(如MTS和INSTRON等)多数采用感应加热方式升温。
% 此外,在TMF试验中还涉及到降温过程,本文所采用的方式为借助压缩空气通过强制对流对试件进行冷却。
% 因此,一个完整的TMF试验系统包括两个子试验系统:力加载系统和温度控制系统,并且温度控制系统可分为升温和降温两部分。
% 由于感应线圈在试验过程中时自身也会发热,因此也需要额外的水冷设备用于感应线圈的冷却。
Different from isothermal fatigue test at elevated temperature, during the TMF test process, it is necessary to control both temperature and mechanical load at the same time, and the temperature is circulating.
Commonly heating methods for elevated temperature fatigue test include induction, direct resistance, radiant, and forced air heating.
Resistance heating furnaces are difficult to meet the test requirements for temperature increase and decrease rate. Major fatigue testing equipment manufacturers (such as MTS and INSTRON) mostly use induction heating for TMF tests.
Also, the cooling process is also involved in the TMF test. The method used in this paper is to cool the test specimen by forced convection with compressed air.
Therefore, a complete TMF test system includes two sub-test systems: force loading system and temperature control system, furthermore the temperature control system can be divided into two parts: heating and cooling.
Since the induction coil also heats itself by the high-frequency current during the test, an additional water cooling device is also required for the cooling of the induction coil.

% 大多数研究中TMF试验采用基于应变控制的加载模式,但也有少数采用基于应力控制的情况,其目的是研究相应工况下的裂纹扩展行为。
% 其次,TMF试验的升降温速率一般控制在10℃/s以内,温度循环变化的范围基本也在200℃以上,因此,一个即使不带任何载荷保持时间的TMF周期都将明显长于LCF周期。同时,在TMF试验中需要控制温度与机械载荷保持协调变化,满足特定的相位关系,这也是该试验最难之处。根据机械载荷和温度之间相位角的不同,TMF试验可分为多种情况,被广泛采用的相位角有同相位(in-phase,简称IP)、反相位(out-of-phase,简称OP)和-90°、+90°以及-135°、+135°。在IP条件下,机械载荷和温度同时达到各自的最大值或最小值;而在OP条件下,情况正好与IP相反,机械载荷和温度存在180°的相位差,二者异步达到各自的最大和最小值;而在其他相位条件下,情况依次类推。
TMF tests were mainly performed under the strain control mode, but in a few studies, stress-controlled TMF tests are carried out to study the crack propagation behavior.
The temperature rate (heating and cooling) of TMF tests is generally within 10$^\circ$C/s, and the temperature range of each cycle is basically above 200$^\circ$C. Therefore, the TMF cycle without any holding time will be significantly longer than the LCF cycle. At the same time, in the TMF tests, it is necessary to keep a specific phase angle between the temperature and mechanical load, which is the most difficult part of the tests. According to the difference of the phase angle between the mechanical load and the temperature, the TMF tests can be divided into several cases. The widely used phase angles are in-phase (0$^\circ$, IP), out-of-phase (180$^\circ$, OP), -90$^\circ$, +90$^\circ$, -135$^\circ$, and 135$^\circ$. Under IP conditions, the mechanical load and temperature simultaneously reach their respective maximum or minimum values; while under OP conditions, the situation is exactly the opposite of IP conditions, the mechanical load and temperature have a phase difference of 180$^\circ$, and both asynchronously reach their respective maximum and the minimum. \ref{Fig:thermomechanical_phase} shows schematically and exemplary three thermomechanical loading conditions.

\begin{figure}[!htp]
  \centering
  \includegraphics[width=16cm]{thermomechanical_phase.pdf}
  \caption{Schematics of mechanical strain and temperature for in-phase, out-of-phase and $90^\circ$-phase TMF Tests.}
  \label{Fig:thermomechanical_phase}
\end{figure}

\subsection{Process of thermomechanical fatigue tests}
\noindent
ISO 12111 and ASTM E2368 \cite{ASTME2005} are the standard practices for thermomechanical fatigue tests under uniaxial strain-controlled conditions.
Unfortunately, there are no relevant standards for multiaxial thermomechanical fatigue tests.
Therefore, multiaxial TMF tests were conducted with reference to ASTM E2207 \cite{ASTME2207}, which is the standard practice for strain-controlled axial-torsional fatigue tests with thin-walled tubular specimens.
According to these standards, some concepts are defined as follows:

% a thermomechanical fatigue cycle is defined as a condition where uniform temperature and strain fields over the specimen gage section are simultaneously varied and independently controlled.
% The test procedure was adapt to the standard ASTM E2207-15 which deals with
% strain-controlled, axial, torsional, and combined in-phase and out-of-phase axial torsional fatigue
% testing with thin-walled, circular cross-section, tubular specimens at isothermal, ambient and
% elevated temperatures.


\begin{itemize}
  \item {\em Coefficient of thermal expansion} $\alpha$, the fractional change in free expansion strain for a unit change in temperature, as measured on the test specimen.
  % \item {\em Thermal strain} $\varepsilon_{\rm{th}}$, the strain component resulting from a change in temperature under free expansion conditions,
  % \begin{equation}
  % \varepsilon_{\rm{th}}=\alpha\Delta T.
  % \label{Equ:thermal_strain}
  % \end{equation}
  % \item {\em Mechanical strain} $\varepsilon_{\rm{mech}}$, the strain component resulting when the free expansion thermal strain specimen is subtracted from the total strain.
  % \item {\em Total strain} $\varepsilon_t$, the strain component measured on the test specimen, and is the sum of the thermal strain and the mechanical strain,
  % \begin{equation}
  % \varepsilon_t=\varepsilon_{\rm{mech}}+\varepsilon_{\rm{th}}.
  % \label{Equ:total_strain}
  % \end{equation}
  % \item {\em Strain ratio} $R_{\varepsilon}$, the ratio of minimum mechanical strain to the maximum mechanical strain in a strain cycle,
  % \begin{equation}
  % R_{\varepsilon}=\varepsilon_{\rm{mech,min}}/\varepsilon_{\rm{mech,max}}.
  % \end{equation}
  % \item {\em Mechanical strain/temperature phase angle} $\varphi$, for the purpose of assessing phasing accuracy, this is defined as the waveform shift (expressed in degrees) between the maximum temperature response as measured on the specimen and the maximum mechanical strain response. For reference purpose, the angle $\varphi$ is considered positive if the temperature response maximum leads the mechanical strain response maximum by $180^\circ$ or less, otherwise the phase angle is considered to be negative.
  %     The phase $\varphi$ between temperature and mechanical strain can be chosen between $-180^\circ$ and $180^\circ$.
  %     One of the special case is $\varphi=0^\circ$ where the maximum and minimum of temperature and strain are reached at the same time. This is called in-phase TMF.
  %     The other special case is $\varphi=180^\circ$ where the maximum value of temperature leads the maximum value of mechanical strain by a time value equal to $1/2$ the cycle period. This is called out-of-phase (anti-phase) TMF.
  %     As shown in \ref{Fig:Thermomechanical_phase}, the vertical coordinates are strain $\varepsilon$ and temperature $T$.
  \item {\em Thermal strain}, $\varepsilon_{\rm{th}}$, is induced by temperature and assumed to be linear proportional to temperature increment,
  \begin{equation}
  \varepsilon_{\rm{th}}=\alpha\Delta T.
  \label{Equ:thermal_strain}
  \end{equation}
  The thermal strain is generally isotropic.
  \item {\em Mechanical strain}, $\varepsilon_{\rm{mech}}$, is resulting from applied load and directly related with material deformation and damage. Under tension-torsion loading conditions, the mechanical deformation in testing is characterized by the maximum principal strain.
  \item {\em Total strain}, $\varepsilon_{\rm{tot}}$, is measured by the extensometer and denotes the sum of the thermal and mechanical strains,
  \begin{equation}
  \varepsilon_{\rm{tot}}=\varepsilon_{\rm{mech}}+\varepsilon_{\rm{th}}.
  \label{Equ:total_strain}
  \end{equation}
  Since the fatigue loading amplitudes are generally small, strains are additive.
  \item {\em Loading ratio}, $R_{\varepsilon}$, denotes the ratio of the minimum mechanical strain over the maximum mechanical strain within a loading cycle,
  \begin{equation}
  R_{\varepsilon}=\varepsilon_{\rm{mech,min}}/\varepsilon_{\rm{mech,max}}.
  \end{equation}
  Under axial-torsional loading condition, the mechanical strains are the principal strains.
  \item {\em Phase angle of the thermal loading and mechanical loading}, $\theta_{T-\varepsilon}$, represents the time difference between maximum values of temperature and mechanical strain.
  \item {\em Phase angle of the thermal loading and torsional loading}, $\theta_{T-\gamma}$, represents the time difference between maximum values of temperature and shear strain.
  \item {\em Phase angle of the mechanical loading and torsional loading}, $\theta_{\varepsilon-\gamma}$, represents the time difference between maximum values of mechanical strain and shear strain.
  \item {\em Equivalent mechanical strain}, $\varepsilon_{\rm{eq}}$, is calculated by the von Mises relation\cite{Pol1991Cyclic},
  \begin{equation}
  \varepsilon_{\rm{eq}}=\sqrt {\varepsilon_{\rm{mech}}^2 + {\gamma ^2}/3}.
  \end{equation}
\end{itemize}

According to the standards discussed above, the implemented steps during the process of carrying out strain-controlled multiaxial TMF tests is shown in \ref{Fig:tmf_code}, in the form of a flow diagram.
The implemented program consist of three steps:
%calibration of the testing system, the setting up and performing procedures of the strain-controlled test.
\begin{enumerate}
  \item Calibration,
  \item TMF test setup,
  \item TMF test.
\end{enumerate}

The calibration of the testing system is necessary to ensure the accuracy and repeatability of the test results.
As shown in \ref{Fig:tmf_code}, force transducer, extensometer and the temperature measurement are the three parts which have to be calibrated.
The accuracy of the force transducer shall meet the ASTM E4 and E467 requirements ($F_{\rm error} \leq \pm 1\%$) during the TMF test duration.
The axial and shear strains in the gage section of the specimen are measured with the high-temperature extensometers.
The extensometers are calibrated in accordance with the recommendations of ASTM E83.
The specimen temperature was measured by the thermocouples in contact with the specimen surface, and the temperature measurement system is calibrated in accordance with the recommendations of ASTM E220.

After the step of calibration, the setting up procedures of TMF test is conducted.
The TMF test setup consists of six steps: data acquisition, optimization of thermal stability, command/feedback phasing compensation, thermal strain compensation, thermal strain validation and preliminary measurements.

As mentioned above, the MTS Model 809 servohydraulic testing machine is used in conjunction with the computer and controller to control the test and log the data obtained.
The data is collected by the computerized system during the TMF test duration, such as cycle count, force, displacement, torque, rotation, mechanical strain, shear strain, temperature and so on. The sampling frequency is determined by the testing frequency to ensure more than 200 sampling points per cycle. At the same time, it is necessary to ensure that the temperature deviation between two sampling points is lower than 5$^\circ$C.

\begin{figure}[!htp]
  \centering
  \includegraphics[width=16.0cm]{temperature_monitors.pdf}
  \caption{Locations of the temperature monitors.}
  \label{Fig:temperature_monitors}
\end{figure}

\begin{figure}[!htp]
  \centering
  \begin{overpic}[width=15.0cm]{plot_thermal_stability_tmf.pdf}
    \put(86,65){\fcolorbox{white}{white}{(a)}}
  \end{overpic}
  \begin{overpic}[width=15.0cm]{plot_thermal_stability_deviation_tmf.pdf}
    \put(86,65){\fcolorbox{white}{white}{(b)}}
  \end{overpic}
  \caption{Varying temperatures in a tubular specimen with 25 mm of the gauge length, heated by an optimized induction coil. (a) Temperature variations of three thermal elements during a loading cycle. (b) Temperature deviations in the axial direction.}
  \label{Fig:thermal_stability}
\end{figure}

Throughout the test, the specimen should be heated uniformly in the gauge section by the specified temperature command. In accordance with the recommendations of ASTM E2368 \cite{ASTME2005}, the axial temperature gradient in the gauge section of the specimen not exceed $\pm$3 K or $\pm$1\% of the
maximum cyclic temperature given in K, and the maximum allowable transverse temperature gradient is $\pm$7 K or $\pm$1\% of maximum cyclic temperature given in K.
% 对于感应加热方式,其感应线圈的布置方式会对试样表面温度分布产生很大的影响。因为线圈的布置不仅需要考虑加热效果,而且要方便引伸计的安装。\cite{Hahner2006}。
Axial temperature gradients of the specimen are influenced mainly by surface convection, heat conduction, electrical conductivity, induction frequency, water-cooled specimen grips, specimen geometry and coil configuration. Generally, only the coil configuration and the specimen geometry are variable. As shown in \ref{Fig:Specimen}(b), tubular specimens are used in combination with the induction heating device. The coil configuration has a significant influence on the temperature distribution of the specimen, at the same time, the installation of the extensometer has to be considered \cite{Hahner2006}.
% 因此,需要在试验初期首先对试样上、中、下及环向位置测温,并通过调整线圈形状、位置及冷却角度来降低试样的温度梯度,以满足ASTM或ISO热机械疲劳试验方法标准的相关规定。
In comparison with isothermal fatigue test, the thermomechanical fatigue test is carried out under varying temperature.
In order to optimize the thermal stability during the TMF testing, temperatures at the upper, center, lower, and circumferential positions in the gauge section of the specimen were measured by the chromel-alumel (type K) thermocouples wrapped around the specimen. \ref{Fig:temperature_monitors} shows the locations of the thermocouples.
The axial and transverse temperature gradients of the specimen can be reduced by adjusting the diameter, screw pitch, installation position of the induction coil.
Consequently, the configuration of the coil was tuned iteratively to obtain the minimum temperature gradients.
Furthermore, heating and cooling rates also influence the temperature gradients.
In order to avoid considerable axial and transverse temperature gradients within the strain gauge of the specimen, in the present work, the heating and cooling rates were chosen about 2.92$^\circ$C/s with a triangular shape temperature cycle, thus the time of one cycle is about 180 s, and the frequency is about 0.0056 Hz.
The temperature range for all TMF tests in the study is from 300$^\circ$C to 650$^\circ$C. \ref{Fig:thermal_stability}(a) shows the dynamic temperature measurement after the thermal stability optimization during a conducted TMF cycle. \ref{Fig:thermal_stability}(b) shows the temperature deviations between the upper/lower/middle monitor and the predetermined temperature command. It is concluded that the axial temperature gradients during the TMF test are between -10$^\circ$C and 15$^\circ$C.

It should be noted that the thin-walled tubular specimen was used in the TMF tests. In the tubular specimen, the shear strain and shear stress are not constant in the cross-section, and the maximum shear stress and shear strain always occur at the outer diameter. Under elasto-plastic loading conditions, the shear stress varies nonlinearly along the radial direction of the tubular specimen. As introduced above, according to ASTM E2207 \cite{ASTME2207}, the assumption of a uniformly distributed shear stress is recommended. The shear stresses reported in the TMF tests are calculated by Equation (\ref{Equ:shear_stress}). The effect of temperature variation on shear strain measurement can be neglected \cite{Bakis2014}. Consequently, the compensation of torsional thermal strain is not performed in the present work. The shear strains are obtained directly from the measurement results by the extensometer.

% 本研究的TMF试验包括IP,OP和90$^\circ$相位,其中IP和OP条件下的载荷加载波形如图\ref{Fig:Thermomechanical_phase}所示。
% 对于应变控制模式下的TMF试验,为了控制相位恒定并避免试样产生热应力,因此,在试验中的任意加载时刻都需要进行热应变补偿。补偿的方法分为两种:a)基于温度的应变补偿;b)基于时间的应变补偿。其中,基于温度的补偿方式为目前试验所普遍采用。
Temperature induced thermal expansion strain is generally called thermal strain, which is not constant during temperature cycling. Thermal strain will cause additional stress in the strain-controlled TMF test. Therefore, the thermal strain was actively compensated during the strain-controlled TMF test. Two common methods can be employed to compensate for the thermal strain: time-based and temperature-based compensation. In this study, temperature-based compensation was used to achieve the desired mechanical strain. The thermal strain is assumed as a function of specimen temperature:
\begin{equation}
\varepsilon_{\rm{th}}=A_0+A_1 T+A_2 T^2,
\label{Equ:thermal_strain_fitting}
\end{equation}
where the coefficients $A_0, A_1, A_2$ of the function can be fitted by the least squares method. Therefore, the data of thermal strain vs. temperature is needed. The thermal strain is measured during an intended temperature cycling under zero force condition, once the specimen and immediate environment have achieved a state of dynamic temperature equilibrium. The intended temperature cycles were identical to the temperature-time cycles which used in the subsequent TMF test.
% 在试验过程中,计算机根据实测的温度即可以算出当时的热应变量,代入式\ref{Equ:mech_strain}即可得到机械应变,并可由此实现机械应变控制,如图\ref{Fig:plot_schematic_thermal_strain}所示。
% 试验采用机械应变$\varepsilon_{\rm{mech}}$控制,根据方程\ref{Equ:thermal_strain}和\ref{Equ:total_strain}可知,机械应变$\varepsilon_{\rm{mech}}$可采用下式计算:
In the present work, the TMF tests were carried out by mechanical strain control mode.
During the test, the thermal strain was calculated by Equation (\ref{Equ:thermal_strain_fitting}) with the measured temperature. Thus, the mechanical strain can be obtained by
\begin{equation}
\varepsilon_{\rm{mech}}=\varepsilon_{\rm{tot}}-\varepsilon_{\rm{th}}=\varepsilon_{\rm{tot}}-(A_0+A_1 T+A_2 T^2).
\label{Equ:mech_strain}
\end{equation}
\ref{Fig:plot_schematic_thermal_strain} shows the relationship between total strain, mechanical strain and thermal strain during the TMF tests.

\begin{figure}
  \centering
  \begin{overpic}[width=8.0cm]{plot_schematic_thermal_strain_IP.pdf}
    \put(0,65){\fcolorbox{white}{white}{(a)}}
  \end{overpic}
  \begin{overpic}[width=8.0cm]{plot_schematic_thermal_strain_OP.pdf}
    \put(0,65){\fcolorbox{white}{white}{(b)}}
  \end{overpic}
  \caption{Strain decomposition of thermomechanical fatigue tests in terms of total strain, mechanical strain and thermal strain. (a) TMF-IP. (b) TMF-OP.}
  \label{Fig:plot_schematic_thermal_strain}
\end{figure}

The accuracy of thermal strain compensation has to be checked before TMF-testing by a zero stress test. The zero stress test is performed under mechanical strain control at $\varepsilon_{\rm{mech}}=0$ by subjecting the specimen to thermal cycling. Here, the thermal strain compensation method will be used to actively compensate for the induced thermal expansion strain of the specimen. During this cycle, the resulting extreme values of stress occurring during zero stress test shall not exceed $\pm$2\% of $\Delta \sigma$ occurring during the TMF test, which means the mechanical strain range controlled during the actual test is within $\pm$2\% of the desired mechanical strain range.

After the setting up procedure, the strain tests were started after several thermal cycles (i.e., in the condition with zero force) in order to ensure the stability of temperature along the gauge section. The criterion for failure is corresponding to the decrease of 25\% from the stabilized peak load.

\begin{figure}[!htp]
\centering
\includegraphics[height=22.0cm]{tmf_code.pdf}
\caption{A flow diagram outlining the various steps of the strain-controlled TMF test.}
\label{Fig:tmf_code}
\end{figure}

In the study, the TMF tests were performed within the temperature interval from 300$^\circ$C to 650$^\circ$C under mechanical strain controlled with fully reversed mechanical strain ratio $R_{\varepsilon}=-1$. The thermomechanical fatigue test matrix consisted of tension-compression in-phase (TC-IP), tension-compression out-of-phase (TC-OP), tension-compression 90$^\circ$ phase (TC-90), proportional in-phase (PRO-IP) and non-proportional in-phase (NPR-IP) tests. The experimental conditions of the TMF tests are summarized in Table \ref{tab:test_program_tmf} with respect to testing parameters such as test type, mechanical and torsional strain amplitudes, equivalent strain amplitude and rates as well as phase shifts between mechanical strain, shear strain, and temperature. All tests were carried out in the air.

In \ref{Fig:LoadPath}, three strain loading paths are shown schematically for uniaxial tension/compression (\ref{Fig:LoadPath}(a)), proportional (\ref{Fig:LoadPath}(c)), and non-proportional (\ref{Fig:LoadPath}(e)) TMF tests, respectively. \ref{Fig:LoadPath}(c) represents a 45$^\circ$ proportional loading with the ratio $\varepsilon_{\rm mech}/(\gamma/\sqrt{3})=1$. While \ref{Fig:LoadPath}(e) shows a diamond non-proportional strain path (i.e. the ratio of mechanical strain amplitude and shear strain amplitude is 1). Loading ratio $R_\varepsilon$ means the ratio of minimum mechanical strain to the maximum mechanical strain in a loading cycle. Under the axial-torsional condition, the loading ratio is calculated from the principal strains.

% 试验采用机械应变$\varepsilon_{\rm{mech}}$控制,根据方程\ref{Equ:thermal_strain}和\ref{Equ:total_strain}可知,机械应变$\varepsilon_{\rm{mech}}$可采用下式计算:
% \begin{equation}
% \varepsilon_{\rm{mech}}=\varepsilon_{t}-\varepsilon_{\rm{th}}=\varepsilon_{t}-\alpha\Delta T,
% \label{Equ:mech_strain}
% \end{equation}
% 式中$\varepsilon_{t}$为引伸计测得的总应变,$\varepsilon_{\rm{th}}$为热应变,$\alpha$为材料热膨胀系数,对于金属材料来说热膨胀系数$\alpha$是温度的函数,因此在试验中采用下式对热应变进行计算,
% \begin{equation}
% \varepsilon_{\rm{th}}=a_0+a_1(T-T_0)+a_2(T-T_0)^2,
% \label{Equ:thermal_strain_fitting}
% \end{equation}
% 式中$T$为试件温度,$T_0$为参考温度,这里我们认为热应变是温度的二次多项式函数,$a_0,a_1,a_2$为用于曲线拟合的参数。
% 试验的温度循环为300-650$^\circ$C,我们选取循环的平均温度475$^\circ$C作为参考温度$T_0$。
% 根据胡克定律可知,如果试件的载荷为零,则有$\varepsilon_{\rm{mech}}=0$,即$\varepsilon_{t}=\varepsilon_{\rm{th}}$。
% 试验过程中我们保持试件的载荷为零,将试件加热至475$^\circ$C并保温一个小时,使试件的温度场进入稳态,记录此时的热应变$\varepsilon_{th,T_0}$为热应变零点。
% 然后在零载荷的作用下,对试样进行300-650$^\circ$C的温度循环,测量并记录试样的热应变随温度的变化关系。
% 将试验结果带入式\ref{Equ:thermal_strain_fitting}拟合,求出参数$a_0,a_1,a_2$,并将此参数输入计算机的实验控制程序中。
% 在试验过程中,计算机根据实测的温度即可以算出当时的热应变量,代入式\ref{Equ:mech_strain}即可得到机械应变,并可由此实现机械应变控制,如图\ref{Fig:plot_schematic_thermal_strain}所示。

% 热机械疲劳试验系统之中有两个相互独立的控制闭合回路, 即机械加载和温度循环加载闭合回路,对于机械加载系统, 力值(载荷)偏差和同轴度是试验机最重要的两个性能指标,以ASTM E2368-2017为例,对应的热机械疲劳试验方法标准引用了专门的检定标准:
% (1)力传感器需要满足ASTM E4和E467要求;
% (2)引伸计需要满足ASTM E83要求;
% (3)加热方式包括induction, direct resistance, radiant, or forced air heating等, 对于感应加热, 建议选择低频发热装置, 以降低集肤效应;
% (4)温度控制测温系统的标定满足ASTM E220规定, 控温变化≤±2 ℃
% (5)温度梯度

% In this section we will introduce the multiaxial thermomechanical fatigue testing process.
% As the name implies the multiaxial thermomechanical fatigue testing is combined with the multiaxial mechanical testing and thermomechanical fatigue (TMF) testing.

\begin{table}[htbp]
  \centering
  \caption{Temperature and loading conditions of the thermomechanical fatigue tests program.}
    \begin{tabular}{p{2cm}<{\centering}p{1.5cm}<{\centering}p{1.5cm}<{\centering}p{1.5cm}<{\centering}p{2.5cm}<{\centering}p{1cm}<{\centering}p{1cm}<{\centering}p{1cm}<{\centering}}
    \toprule
    Test Type & $\pm \varepsilon_{\rm{mech}}$ & $\pm \gamma/ \sqrt 3$ & $\varepsilon_{\rm{eq}}$ & $\dot \varepsilon_{\rm{eq}}$ & $\theta_{T-\varepsilon}$ & $\theta_{T-\gamma}$ & $\theta_{\varepsilon-\gamma}$ \\
          & [\%]  & [\%]  & [\%]  & [s$^{-1}$] & [$^\circ$] & [$^\circ$] & [$^\circ$] \\
    \midrule
    TC-IP & 1.00  & -     & 1.00  & $2.22\times 10^{-4}$ & 0     & -     & - \\
          & 0.80  & -     & 0.80  & $1.78\times 10^{-4}$ & 0     & -     & - \\
          & 0.70  & -     & 0.70  & $1.56\times 10^{-4}$ & 0     & -     & - \\
          & 0.60  & -     & 0.60  & $1.33\times 10^{-4}$ & 0     & -     & - \\
    \midrule
    TC-OP & 1.00  & -     & 1.00  & $2.22\times 10^{-4}$ & 180   & -     & - \\
          & 0.80  & -     & 0.80  & $1.78\times 10^{-4}$ & 180   & -     & - \\
          & 0.70  & -     & 0.70  & $1.56\times 10^{-4}$ & 180   & -     & - \\
          & 0.65  & -     & 0.65  & $1.44\times 10^{-4}$ & 180   & -     & - \\
    \midrule
    TC-90 & 1.00  & -     & 1.00  & $2.22\times 10^{-4}$ & 90    & -     & - \\
    \midrule
    PRO-IP & 0.71  & 0.71  & 1.00  & $2.22\times 10^{-4}$ & 0     & 0     & 0 \\
          & 0.57  & 0.57  & 0.80  & $1.78\times 10^{-4}$ & 0     & 0     & 0 \\
          & 0.57  & 0.57  & 0.80  & $1.78\times 10^{-4}$ & 0     & 0     & 0 \\
          & 0.42  & 0.42  & 0.60  & $1.33\times 10^{-4}$ & 0     & 0     & 0 \\
    \midrule
    NPR-IP & 1.00  & 1.00  & 1.00  & $2.22\times 10^{-4}$ & 0     & 90    & 90 \\
          & 0.80  & 0.80  & 0.80  & $1.78\times 10^{-4}$ & 0     & 90    & 90 \\
          & 0.70  & 0.70  & 0.70  & $1.56\times 10^{-4}$ & 0     & 90    & 90 \\
          & 0.70  & 0.70  & 0.70  & $1.56\times 10^{-4}$ & 0     & 90    & 90 \\
          & 0.60  & 0.60  & 0.60  & $1.33\times 10^{-4}$ & 0     & 90    & 90 \\
          & 0.50  & 0.50  & 0.50  & $1.11\times 10^{-4}$ & 0     & 90    & 90 \\
    \bottomrule
    \end{tabular}%
  \label{tab:test_program_tmf}%
\end{table}%

\subsection{Experimental results}
\noindent
Fatigue tests with three different phase angles of the thermal loading and mechanical loading should give a systematical overview of the thermomechanical behavior of the material. \ref{Fig:plot_exp_TCTMF} illustrates half-life hysteresis loops and the evolution of the peak, valley and mean stresses for TC-IP, TC-OP, and TC-90 TMF tests. As expected, under strain-controlled uniaxial cyclic loading with mean mechanical strain, all fatigue tests show the cyclic softening behavior, and the hysteresis loops reach a stable state with increasing number of cycles. The decreases of mean stresses were observed during the TC-IP loading tests, whereas for TC-OP loading the opposite is the case. Furthermore, due to the symmetry of temperature in the TC-90 loading (i.e., the temperatures are the same at maximum and minimum axial strain), the mean axial stress is constant during the entire life-cycle process.

As shown in \ref{Fig:plot_exp_PROTMF}, the experimental results of the PRO-IP TMF tests show the similar cyclic softening behavior of the uniaxial TC-IP tests and the shear stress response is consistent with the axial stress response. In \ref{Fig:plot_exp_PROTMF}(d) and (f), a decrease of mean stress is present in both axial and hoop directions.

\ref{Fig:plot_exp_NPRTMF} shows the experimental results of NPR-IP TMF tests. Unlike the results of PRO-IP TMF tests, for equivalent strain amplitude $\Delta \varepsilon_{\rm eq}/2\geqslant$0.8\%, cyclic hardening was present in the first three cycles in both axial and torsional directions. Afterward, continuous cyclic softening follows for the major part of the fatigue life. Furthermore, compared to the PRO-IP data, both axial and shear stress amplitudes due to the same equivalent strain amplitudes are higher than the case of PRO-IP loading tests, as shown in \ref{Fig:plot_exp_NPRTMF}(d) and (f). Consequently, it was concluded that the additional non-proportional hardening occurred in the mechanical non-proportional loading condition. In \ref{Fig:plot_exp_NPRTMF}(f), dissimilarly to the axial stress cases, the mean shear stress under NPR-IP loading hardly changes.

\begin{figure}[!htp]
  \centering
  \begin{overpic}[width=8.0cm]{plot_exp_half_life_cycle_TCIP.pdf}
    \put(84,13){\fcolorbox{white}{white}{(a)}}
  \end{overpic}
  \begin{overpic}[width=8.0cm]{plot_exp_pv_TCIP.pdf}
    \put(84,13){\fcolorbox{white}{white}{(b)}}
  \end{overpic}

  \begin{overpic}[width=8.0cm]{plot_exp_half_life_cycle_TCOP.pdf}
    \put(84,13){\fcolorbox{white}{white}{(c)}}
  \end{overpic}
  \begin{overpic}[width=8.0cm]{plot_exp_pv_TCOP.pdf}
    \put(84,13){\fcolorbox{white}{white}{(d)}}
  \end{overpic}

  \begin{overpic}[width=8.0cm]{plot_exp_half_life_cycle_TC90.pdf}
    \put(84,13){\fcolorbox{white}{white}{(c)}}
  \end{overpic}
  \begin{overpic}[width=8.0cm]{plot_exp_pv_TC90.pdf}
    \put(84,13){\fcolorbox{white}{white}{(d)}}
  \end{overpic}
  \caption{Experimental results of the uniaxial thermomechanical fatigue tests in the temperature interval from 300$^\circ$C to 650$^\circ$C.
  (a) Half life stable hysteresis loops of TC-IP tests.
  (b) Peak, valley and mean stresses of TC-IP tests.
  (c) Half life stable hysteresis loops of TC-OP tests.
  (d) Peak, valley and mean stresses of TC-OP tests.
  (e) Half life stable hysteresis loops of TC-90 tests.
  (f) Peak, valley and mean stresses of TC-90 tests.}
  \label{Fig:plot_exp_TCTMF}
\end{figure}

\begin{figure}[!htp]
  \centering
  \begin{overpic}[width=8.0cm]{plot_exp_half_life_cycle_PROIP_strain.pdf}
    \put(84,13){\fcolorbox{white}{white}{(a)}}
  \end{overpic}
  \begin{overpic}[width=8.0cm]{plot_exp_half_life_cycle_PROIP.pdf}
    \put(84,13){\fcolorbox{white}{white}{(b)}}
  \end{overpic}

  \begin{overpic}[width=8.0cm]{plot_exp_half_life_cycle_PROIP_axial.pdf}
    \put(84,13){\fcolorbox{white}{white}{(c)}}
  \end{overpic}
  \begin{overpic}[width=8.0cm]{plot_exp_pv_PROIP_axial.pdf}
    \put(84,13){\fcolorbox{white}{white}{(d)}}
  \end{overpic}

  \begin{overpic}[width=8.0cm]{plot_exp_half_life_cycle_PROIP_torsional.pdf}
    \put(84,13){\fcolorbox{white}{white}{(e)}}
  \end{overpic}
  \begin{overpic}[width=8.0cm]{plot_exp_pv_PROIP_torsional.pdf}
    \put(84,13){\fcolorbox{white}{white}{(f)}}
  \end{overpic}
  \caption{Experimental results of the PRO-IP tests with equivalent strain amplitudes $\Delta\varepsilon_{\rm eq}/2$= 0.6\%, 0.8\% and 1.0\%: (a) strain paths, (b) cyclic stress responses of the cycle at $N_{\rm f}/2$, (c) half-life hysteresis loops correspond to axial components, (d) peak, valley and mean values of shear stresses, (e) half-life hysteresis loops correspond to shear components, (f) peak, valley and mean values of shear stresses.
  }
  \label{Fig:plot_exp_PROTMF}
\end{figure}

\begin{figure}[!htp]
  \centering
  \begin{overpic}[width=8.0cm]{plot_exp_half_life_cycle_NPRIP_strain.pdf}
    \put(84,13){\fcolorbox{white}{white}{(a)}}
  \end{overpic}
  \begin{overpic}[width=8.0cm]{plot_exp_half_life_cycle_NPRIP.pdf}
    \put(84,13){\fcolorbox{white}{white}{(b)}}
  \end{overpic}

  \begin{overpic}[width=8.0cm]{plot_exp_half_life_cycle_NPRIP_axial.pdf}
    \put(84,13){\fcolorbox{white}{white}{(c)}}
  \end{overpic}
  \begin{overpic}[width=8.0cm]{plot_exp_pv_NPRIP_axial.pdf}
    \put(84,13){\fcolorbox{white}{white}{(d)}}
  \end{overpic}

  \begin{overpic}[width=8.0cm]{plot_exp_half_life_cycle_NPRIP_torsional.pdf}
    \put(84,13){\fcolorbox{white}{white}{(e)}}
  \end{overpic}
  \begin{overpic}[width=8.0cm]{plot_exp_pv_NPRIP_torsional.pdf}
    \put(84,13){\fcolorbox{white}{white}{(f)}}
  \end{overpic}
  \caption{Experimental results of the NRP-IP tests with equivalent strain amplitudes $\Delta\varepsilon_{\rm eq}/2$=0.5\%, 0.6\%, 0.7\%, 0.8\% and 1.0\%: (a) strain paths, (b) cyclic stress responses of the cycle at $N_{\rm f}/2$, (c) half-life hysteresis loops correspond to axial components, (d) peak, valley and mean values of shear stresses, (e) half-life hysteresis loops correspond to shear components, (f) peak, valley and mean values of shear stresses.}
  \label{Fig:plot_exp_NPRTMF}
\end{figure}

%\begin{figure}
%  \begin{minipage}[t]{0.5\linewidth} % 如果一行放2个图,用0.5,如果3个图,用0.33\
%  \nonumber
%    \centering
%    \includegraphics[width=3.2in]{plot_exp_half_life_cycle_TCOP.pdf}
%    \centerline{(a)}
%  \end{minipage}%
%  \begin{minipage}[t]{0.5\linewidth}
%    \centering
%    \includegraphics[width=3.2in]{plot_exp_pv_TCOP.pdf}
%    \centerline{(b)}
%  \end{minipage}
%  \caption{Out of phase (OP) TMF tests of Inconel 718 with temperature range 350-600$^{\circ}$C. (a)Half life stable hysteresis loops. (b)Peak, valley and mean stresses.}
%  \label{Fig:plot_exp_TCOP}
%\end{figure}
%
%\begin{figure}
%  \begin{minipage}[t]{0.5\linewidth} % 如果一行放2个图,用0.5,如果3个图,用0.33\
%  \nonumber
%    \centering
%    \includegraphics[width=3.2in]{plot_exp_half_life_cycle_TC90.pdf}
%    \centerline{(a)}
%  \end{minipage}%
%  \begin{minipage}[t]{0.5\linewidth}
%    \centering
%    \includegraphics[width=3.2in]{plot_exp_pv_TC90.pdf}
%    \centerline{(b)}
%  \end{minipage}
%  \caption{90 $^\circ$ phase TMF tests of Inconel 718 with temperature range 350-600$^{\circ}$C. (a)Half life stable hysteresis loops. (b)Peak, valley and mean stresses.}
%  \label{Fig:plot_exp_TC90}
%\end{figure}

%\subsection{1}
%Experiments are conducted on the MTS model 809 axial/torsional servohydraulic testing system.
%% with the model 647 hydraulic wedge grips.
%The system can provide the maximum axial load to 250 kN and the maximum torque to 2200N$\cdot$m.
%The controller of our test system is the FlexTest Models 40 Controller.
%It provides real-time closed-loop control, with transducer conditioning and function generation to drive various types of servoactuators.
%The FlexTest 40 Controller consists of the Model 494.04 chassis and a computer workstation that runs MTS controller applications.
%
%The Model 494.04 Chassis includes three VMEbus(Versa Module Europa bus) slots as shown in \ref{Fig:Typical_Model_494_Chassis_Connections}.
%Slot 1 is processor board which provides PIDF(Proportional, Integral, Derivative, Feed forward) processing and it is an interface between the controller and the computer workstation.
%Slots 2, 3 are I/O carrier which support up to four mezzanine cards that can condition transducers, drive servovalves, provide A-to-D inputs, and interface to various digital transducers (such as encoders and Temposonics transducers).
%Slots 4, 5 are system board which provides control of the test system hydraulics, including hydraulic power unit (HPU) and hydraulic service manifold (HSM) control.
%\begin{figure}[!htp]
%\centering\scalebox{1.0}{\includegraphics{Typical_Model_494_Chassis_Connections.pdf}}
%\caption{Typical Model 494.04 Chassis Connections.}
%\label{Fig:Typical_Model_494_Chassis_Connections}
%\end{figure} 