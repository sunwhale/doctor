\chapter{Conclusion}

\begin{figure}
  \begin{minipage}[t]{0.5\linewidth}
  \nonumber
    \centering
    \includegraphics[width=3.0in]{7033-1.jpg}
    \centerline{(a)500X.}
  \end{minipage}%
  \begin{minipage}[t]{0.5\linewidth}
    \centering
    \includegraphics[width=3.0in]{7033-12.jpg}
    \centerline{(b)850.}
  \end{minipage}
  \caption{TC-OP.}
  \label{Fig:MicrostructureofInconel718}
\end{figure}

\begin{figure}
  \begin{minipage}[t]{0.5\linewidth}
  \nonumber
    \centering
    \includegraphics[width=3.0in]{7036-1.jpg}
    \centerline{(a)$\Delta \varepsilon_{m}/2=0.5\%$.}
  \end{minipage}%
  \begin{minipage}[t]{0.5\linewidth}
    \centering
    \includegraphics[width=3.0in]{7036-7.jpg}
    \centerline{(b)$\Delta \varepsilon_{m}/2=0.5\%$.}
  \end{minipage}
  
  \begin{minipage}[t]{0.5\linewidth}
  \nonumber
    \centering
    \includegraphics[width=3.0in]{7046-1.jpg}
    \centerline{(C)$\Delta \varepsilon_{m}/2=0.7\%$.}
  \end{minipage}%
  \begin{minipage}[t]{0.5\linewidth}
    \centering
    \includegraphics[width=3.0in]{7046-6.jpg}
    \centerline{(D)$\Delta \varepsilon_{m}/2=0.7\%$.}
  \end{minipage}
  
  \caption{NRP-IP.}
  \label{Fig:MicrostructureofInconel718}
\end{figure}

\begin{figure}
  \begin{minipage}[t]{0.5\linewidth}
  \nonumber
    \centering
    \includegraphics[width=3.0in]{7047-1.jpg}
    \centerline{(a)500X.}
  \end{minipage}%
  \begin{minipage}[t]{0.5\linewidth}
    \centering
    \includegraphics[width=3.0in]{7047-12.jpg}
    \centerline{(b)1000X.}
  \end{minipage}
  \caption{TC-IP.}
  \label{Fig:MicrostructureofInconel718}
\end{figure}

\begin{figure}
  \begin{minipage}[t]{0.5\linewidth}
  \nonumber
    \centering
    \includegraphics[width=3.0in]{7112-1.jpg}
    \centerline{(a)500X.}
  \end{minipage}%
  \begin{minipage}[t]{0.5\linewidth}
    \centering
    \includegraphics[width=3.0in]{7112-3.jpg}
    \centerline{(b)1000X.}
  \end{minipage}
  \caption{TC-IF.}
  \label{Fig:MicrostructureofInconel718}
\end{figure}

\begin{figure}
  \begin{minipage}[t]{0.5\linewidth}
  \nonumber
    \centering
    \includegraphics[width=3.0in]{7040-3.jpg}
    \centerline{(a)$\Delta \varepsilon_{eq}/2=0.6\%$.}
  \end{minipage}%
  \begin{minipage}[t]{0.5\linewidth}
    \centering
    \includegraphics[width=3.0in]{7040-5.jpg}
    \centerline{(b)$\Delta \varepsilon_{eq}/2=0.6\%$.}
  \end{minipage}
  \caption{PRO-IP.}
  \label{Fig:MicrostructureofInconel718}
\end{figure}

\begin{figure}
  \begin{minipage}[t]{0.5\linewidth}
  \nonumber
    \centering
    \includegraphics[width=3.0in]{7206-1.jpg}
    \centerline{(a)$\Delta \varepsilon_{eq}/2=0.55\%$.}
  \end{minipage}%
  \begin{minipage}[t]{0.5\linewidth}
    \centering
    \includegraphics[width=3.0in]{7206-3.jpg}
    \centerline{(b)$\Delta \varepsilon_{eq}/2=0.55\%$.}
  \end{minipage}
  \caption{TC-IP-TGMF.}
  \label{Fig:MicrostructureofInconel718}
\end{figure}

\begin{figure}
  \begin{minipage}[t]{0.5\linewidth}
  \nonumber
    \centering
    \includegraphics[width=3.0in]{720719.jpg}
    \centerline{(a)$\Delta \varepsilon_{eq}/2=0.55\%$.}
  \end{minipage}%
  \begin{minipage}[t]{0.5\linewidth}
    \centering
    \includegraphics[width=3.0in]{720721.jpg}
    \centerline{(b)$\Delta \varepsilon_{eq}/2=0.55\%$.}
  \end{minipage}
  
  \begin{minipage}[t]{0.5\linewidth}
  \nonumber
    \centering
    \includegraphics[width=3.0in]{720932.jpg}
    \centerline{(a)$\Delta \varepsilon_{eq}/2=0.45\%$.}
  \end{minipage}%
  \begin{minipage}[t]{0.5\linewidth}
    \centering
    \includegraphics[width=3.0in]{720935.jpg}
    \centerline{(b)$\Delta \varepsilon_{eq}/2=0.45\%$.}
  \end{minipage}
  
  \caption{TC-OP-TGMF.}
  \label{Fig:MicrostructureofInconel718}
\end{figure}

\begin{figure}
  \begin{minipage}[t]{0.5\linewidth}
  \nonumber
    \centering
    \includegraphics[width=3.0in]{7301-7.jpg}
    \centerline{(a)$\Delta \varepsilon_{eq}/2=0.55\%$.}
  \end{minipage}%
  \begin{minipage}[t]{0.5\linewidth}
    \centering
    \includegraphics[width=3.0in]{7301-4.jpg}
    \centerline{(b)$\Delta \varepsilon_{eq}/2=0.55\%$.}
  \end{minipage}
  
  \begin{minipage}[t]{0.5\linewidth}
  \nonumber
    \centering
    \includegraphics[width=3.0in]{730109.jpg}
    \centerline{(a)$\Delta \varepsilon_{eq}/2=0.55\%$.}
  \end{minipage}%
  \begin{minipage}[t]{0.5\linewidth}
    \centering
    \includegraphics[width=3.0in]{730116.jpg}
    \centerline{(b)$\Delta \varepsilon_{eq}/2=0.55\%$.}
  \end{minipage}
  
  \caption{TC-IP-TGMF-TBC.}
  \label{Fig:MicrostructureofInconel718}
\end{figure}









%This thesis has presented computational results of heat transfer in a heated multiple cavity rig with axial-flow-through.
%The rig comprised an encased rotor with two internal cavities from a typical compressor disc-shroud stack and a central shaft.
%The inlet and rotating conditions are $Re_z$ up to $1.4 \times 10^5$ and $Re_{\phi}$ up to $1.53 \times 10^7$.
%Conclusions are dependent on the computational results:
%
%%    ���������
%\newcounter{Lcount}
%%   ��ǩ������ʾΪ��������
%\begin{list}{\arabic{Lcount}.}
%%    ʹ�ü�����
%  {\usecounter{Lcount}
%%    ���ҶԳ�
%  \setlength{\rightmargin}{\leftmargin}}
%%    ��ʼ
%    \item \textbf{Pressure distribution. }Air pressure in the rotating cavities increases with the normalized radial coordinate and it is a linear relation between the pressure and radial coordinate. A pressure coefficient is suggested to describe the pressure distribution.
%        The pressure coefficient, which is assumed by pressure and radial coordinate, grows squarely with the rotational Reynolds number.
%        Gap ratio influences the pressure coefficient.
%        With increasing of gap ratio, the pressure coefficient decreases.
%        However, the pressure coefficient is insensible with discs' thickness.
%    \item \textbf{Average Nusselt number of disc. }Average Nusselt number from the inner surface of the disc has a linear relation with the square root of rotational Reynolds number.
%        Results with different rotational Reynolds numbers but the fixed inlet speed and gap ratio are chosen to determine the relation.
%        Comparison of the results with different gap ratios shows that geometry effect of the average Nusselt number is not evident.
%        Generally with increasing of gap ratio, the average Nusselt number from disc decreases.
%        The small gap ratio relates to the reduction of convective heat transfer.
%        Average Nusselt number from disc also has a relation with Rossby number.
%        Quotient of average Nusselt number and rotational Reynolds number is a linear function of Rossby number.
%    \item \textbf{Local Nusselt number of disc. }The expression of local Nusselt number is suggested.
%        The local Nusselt number from the inner surface of the disk is suggested as a function of rotational Reynolds number, axial Reynolds number, local temperature and radial coordinate.
%        Parameters of the expression are determined by the computational results.
%        The expression can be used to estimate the discs' convective heat transfer coefficient.
%    \item \textbf{Average Nusselt number of shroud. }Average Nusselt number from the inner surface of cavities' shroud shows a linear relation with rotational Reynolds number.
%        Gap ratio influences the linear relation.
%        As the gap ratio becomes larger, the slope of the linear relation increases.
%    \item \textbf{Prediction of temperature. }Comprising with the experimental data, CFD and FEM computations shows that the suggested thermal model can estimate good temperature prediction of rotating axial-flow-through cavities.
%        With the expression of local Nusselt number, the convective heat transfer coefficients are calculated.
%        The convective heat transfer coefficients are used as the boundary conditions of the inner surface of the cavities in the structural thermal computations.
%        The difference between the results of FEM and CFD simulations is less than $\pm 3\%$.
%
%\end{list}
%
%
%Further experimental and numerical study is needed to improve the prediction of the complex flow and heat transfer in the heated rotating cavities with axial-through-flow. Transient simulation is needed to study the transient heat transfer behavior instead of the steady state simulation. 