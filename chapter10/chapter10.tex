\chapter{Conclusions and outlooks}
\noindent
镍基单晶涡轮叶片TMF 是制约先进航空发动机研制的瓶颈难题。为解决这一问题,本文以Inconel 718为研究对象,开展了单调拉伸、等温疲劳、热机疲劳和热梯度机械疲劳的试验空心气冷涡轮叶片的多层次
TMF 试验,完成了镍基单晶及TMF 损伤机理、本构建模和寿命建模研究,主要结论有:
% 热机疲劳和热机械梯度疲劳在涡轮发动机中是非常重要的疲劳失效模式。
Thermomechanical fatigue and thermal gradient mechanical fatigue are very important fatigue failure modes in turbine engines.
Recently, lots of investigations about life estimation model under uniaxial TMF loading were reported. This thesis focuses on the cyclic plastic deformation and life prediction under multiaxial TMF loading. Furthermore, a radiation furnace was designed and developed for the thermal gradient mechanical fatigue test. A life prediction model was proposed for the TGMF loading.

In the thesis, the multiaxial thermomechanical behavior of the Nickel-based superalloy Inconel 718 has been investigated experimentally and computationally.
Based on the the Ohno-Wang's model, a constitutive model has been developed for multiaxial thermomechanical cyclic plasticity.
An implicit computational integration algorithm for the constitutive model has been developed and implemented into the general purpose commercial finite element code ABAQUS. The computational predictions confirm that the model can describe elastic-plastic mechanical behavior under most different thermomechanical loading conditions.
From the present experimental and computational investigation the following conclusions can be drawn:
\begin{itemize}

\item {The kinematic hardening can represent the cyclic loading and the isotropic hardening can consider the non-proportional loading path effects in the constitutive model.
Complex variations in the peak and valley stresses can be described by the kinematic hardening model properly. The numerical predictions presents good agreement with the experiments in varying temperatures.}

\item {Computations confirm the significance of the non-proportional hardening under multiaxial loading conditions and agree with experiments under different loading conditions. Generally, the prediction of the constitutive model is reasonable under both proportional and non-proportional loadings.}

\item {The temperature-dependent material parameters are determined under isothermal conditions. The results reveal the constitutive model can approach thermomechanical behavior reasonably under both axial and multiaxial thermomechanical loadings.}

\item{The complex loading path may induce additional strain hardening in the compressive normal stress, which cannot be caught by the present constitutive model. To quantify this effect, more detailed experiments are necessary.}

\end{itemize}

This paper has described the experimental facility of non-proportional thermomechanical fatigue tests. The machine had to be carefully calibrated from its original design to improve the thermal gradient and ensure the control stability of the axial-torsional extensometers. The Nickel-based superalloy Inconel 718 was investigated experimentally and computationally under 650$^\circ$C isothermal and 300$^\circ$C to 650$^\circ$C thermomechanical loading conditions with proportional and non-proportional mechanical loading paths. The main conclusions are as follows:

\begin{itemize}
\item Experimental results shows that the TC-OP tests have a longer lifetime than the TC-IP tests within the Mises equivalent mechanical strain amplitude range from 0.6\% to 1\%.

\item TMF lifetimes are strongly influenced by the strain path. At the same equivalent mechanical strain amplitude and the same in-phase temperature conditions, the NPR-IP tests have a shorter lifetime than the lifetime of TC-IP tests. However, the PRO-IP tests have the longest lifetime.

\item Manson-Coffin fatigue parameters are determined on the isothermal fatigue tests. The life predictions from the known fatigue models are unsatisfactory. The plots contains generally large scatterings, which implies improper fatigue variables in the models.

\item A TMF life prediction model is proposed to consider influences from varying temperature as well as the non-proportional loads.
\end{itemize}
