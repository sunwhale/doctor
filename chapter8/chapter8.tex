\chapter{Development of the radiation furnace and testing procedure of TGMF}
\section{Introduction}
\noindent
Thermal gradients cause multiaxial loads in cooled components.
In internal cooling, for example of rotors in the first stage of a jet engine, the thermal gradient induced stresses can not be relaxed by macroscopic deformations.
The stresses occurring at the component lead to multiaxial pressure loads on the hot surface and to multiaxial tensile loads on the cooled surface.
Since conventional thermomechanical tests do not simulate these stress conditions and achieve a homogeneous temperature distribution, our test system was designed and developed for the Thermal Gradient Mechanical Fatigue (TGMF) tests.
The system allows cyclic and simultaneously thermal and mechanical stress with controlled temperature gradients on the wall of hollow test specimens.
The temperature gradient is achieved by heating the outer surface with a furnace which emits concentrated radiation and the inner surface is simultaneously cooled with compressed air.

These realistic tests have the advantage that they can transfer data from laboratory tests to the operating conditions.
In addition, the heating and cooling rates achieved in the TGMF test apparatus allow very short test cycles so that the fatigue load of an entire flight can be applied to a test body within three to five minutes.

In addition, in order to enhance the engine cooling effect and improve the engine efficiency, most of the advanced aero-engine and gas turbine blades are thin-walled porous structures.
Therefore, we designed a thin-walled tube specimen to simulate the cooling structure of the parts, and the outer wall of the specimen was coated with the thermal barrier coating.
Under the action of the internal cooling air, a large temperature gradient between the inner surface and the outer surface of the specimen was produced.
In the experiment, the conventional induction heating equipment only heated the inner metal layer, which maked the temperature of inner metal layer higher than the external ceramic layer temperature. It did not conform to the temperature distribution of the thermal barrier coating components under the actual working state.
Actually, the engine start-up phase heating process only takes a few seconds, and the cooling stage is also very rapid. these are coating thermal barrier coatings on the heat gradient mechanical fatigue test equipment has put forward higher requirements, but also restricted the research.


Conventional thermomechanical fatigue tests are not able to simulate these stress conditions and achieve a temperature gradient in the radial direction. So we designed this test system for thermal gradient mechanical fatigue (TGMF) testing.
The system can realize the controlled thermal cycle with temperature gradient on the hollow specimen and apply the mechanical load at the same time.
The external surface of the specimen is heated by the method of concentrating radiation, while the inner surface is cooled by compressed air to achieve the temperature gradient.
The test system needs to achieve high heating and cooling speed, so the power of heating system and the volume flow of the cooling air should be designed in detail.




% 对于薄壁圆管试件,我们可选择的加热方式有电阻炉、电磁感应、火焰喷射和辐射。
For thin-walled tube specimen, the heating methods include resistance furnace, induction, flame, radiation, etc.
% 高温炉可在炉体内达到均匀的温度场,具有较高的温度精度,被广泛应用于等温试验,如蠕变、高温低循环、蠕变-疲劳等。然而高温炉的加热速度较慢,无法满足TMF试验快速升温的需求。并且高温炉体为封闭结构,难以对试件进行强制冷却,因而不适用于温度快速变化的TMF试验。
The resistance furnace (see \ref{Fig:heating_methods}(a)) can reach a uniform temperature field in the furnace cavity. It has a high temperature accuracy, and is widely used in isothermal tests, such as creep, low cycle fatigue and creep-fatigue. However, the heating rate of the resistance furnace is slow and can not meet the requirement of rapid temperature varying during the TMF testing. In addition, the resistance furnace is an enclosed cavity and it is difficult to perform forced air cooling of the specimen, and thus it is not suitable for the TMF test.
% 电阻炉的优点在于温度稳定性好,但实际发动机启动阶段升温过程只需要几十秒钟的时间,而且降温也相当迅速,采用电阻炉无法实现试件的快速升温和降温。
% Resistance furnaces have the advantage of good temperature stability, but the actual temperature rise during the start-up phase of the engine only takes tens of seconds, and cooling is also very rapid. 
% The use of resistance furnaces cannot achieve rapid temperature rise and cooling of the test pieces.

% 电磁感应加热的优点在于加热效率高,试件升温迅速,通常热机械疲劳试验采用高频电磁感应设备进行加热,其中高频电磁感应加热厚度约为1-2mm,这意味着试件的内外表面是同时加热的,在内部冷却过程中,不利于产生内外表面的温度梯度,同时电磁感应方式只会对内部金属层加热,使得内部金属层温度高于外部陶瓷层温度,这不符合热障涂层构件实际工作状态下的温度分布\cite{BRENDEL2008234}。

As introduced in chapter 2, induction heating (see \ref{Fig:heating_methods}(b)) has the advantages of high heating rate and high efficiency. It is widely used in TMF testing.
% 感应加热具有加热能力强、加热速度快,且具有低成本的优势,可以通过感应线圈的设计模拟复杂温度场,被广泛应用于金属材料高温疲劳试验。
% 由于感应电流的趋肤效应,感应加热通常只能对工件表层进行加热,且感应频率越高,加热深度越浅。工件内部温度通过热传导实现升温。
Due to the skin effect of the induced current, induction heating usually only heats the surface of the specimen, and the higher the induction frequency, the shallower the heating depth.
In the present study, the heating depth of the induced current is about 1-2mm, and thus the inner and outer surfaces of the specimen is heated simultaneously.
Considering the internal cooling process, the induction heating is not conducive to produce the temperature gradient between internal and external surface of the specimen. Furthermore, considering a specimen covered with a thermal barrier coating (TBC) on its outer surface, the induction heating can only heat the metal part of the specimen, and the TBC (typically a ceramic material) will not be heated. This will cause the temperature inside the specimen to be higher than the outside temperature, which does not meet the temperature distribution in actual working condition \cite{BRENDEL2008234}.

% 燃气加热是利用燃烧室中燃油燃烧时所产生的高温燃气对试件进行加热,在满足试件加热条件的同时,还可模拟燃气与零部件的热量交换、氧化、腐蚀等过程,并且对于试件的材料无特殊要求(金属、非金属复合材料等)。燃气加热能力强,可以将试件加热到很高温度。但是燃气加热需要搭建复杂的燃烧室设备及其控制系统,火焰加热受环境影响较大,试件的温度场均匀性较差,远远达不到TMF试验的要求。火焰喷射的优点在于接近真实发动机涡轮叶片的工作环境,但火焰的稳定性差,火焰形状难以控制,很难形成均匀的温度场\cite{MAUGET2017225}。
Flame heating (see \ref{Fig:heating_methods}(c)) is the use of high-temperature gas generated in the combustion of fuel in the combustion chamber to heat the specimen. It can simulate the heat exchange, oxidation, corrosion and other processes of the gas and specimen. Flame heating has no special requirements for the material of the specimen (metal, non-metal composites, etc.) and allows the specimen to be heated to very high temperature. However, it requires the combustion chamber and complex control system. The flame heating is greatly affected by the environment, and the uniformity of the temperature field of the specimen is poor, which is far from the requirements of the TMF test \cite{MAUGET2017225}.

% 石英灯辐射加热是利用石英灯管照射试件,实现试件升温的加热方式。辐射加热可加热金属、非金属和复合材料等多种材料,可通过多温区控制实现复杂的温度场。
Quartz lamp radiant heating is the use of quartz lamp to irradiate the specimen or workpiece to achieve the predetermined temperature field. Radiant heating can heat metals, non-metals, composites, etc., and can obtain complex temperature fields through multi-temperature zone control.
% 德国航空航天中心(Geman Aerospace Center, DLR)[36]通过镜面反射16 根1kW 的石英灯管设计了一种石英灯辐射加热炉(图 8),该加热炉加热能力较高,可在15s 内将试件表面温度从室温升至1000℃。由于加热时需要灯光照射在试件表面,因而难以安装引伸计等应变测量设备。但照射灯加热通常需要对多根灯管进行同步反馈控制,对控制系统的软硬件要求较高。并且在考核段的温度均匀性尚未得到很好的解决,限制了其在材料高温疲劳试验领域的应用。
Deutsches Zentrum f\"{u}r Luft- und Raumfahrt (DLR) \cite{BAUFELD2008219} has developed a thermal gradient mechanical fatigue (TGMF) test facilities (see \ref{Fig:heating_methods}(d)) which allow simultaneous cyclic thermal and mechanical loading with controlled thermal gradients over the wall of hollow specimens. The thermal gradient is obtained by heating of the outer surface with a concentrating radiation furnace and simultaneous internal cooling with pressurized air. The high heat flux of the radiation furnace allows heating rates comparable to those in real turbine blades of a jet engine. The radiation furnace has a power of 16 kW and consists of 16 quartz lamps. A hollow cylindrical specimen from nickel base super alloy can be heated from 100°C to 1000°C in ca. 20 seconds. 


\begin{figure}[!htp]
  \centering
  \begin{overpic}[width=8.0cm]{Resistance.jpg}
    \put(84,13){\fcolorbox{white}{white}{(a)}}
  \end{overpic}
  \begin{overpic}[width=8.0cm]{Induction.jpg}
    \put(84,13){\fcolorbox{white}{white}{(b)}}
  \end{overpic}

  \begin{overpic}[width=8.0cm]{Flame.png}
    \put(84,13){\fcolorbox{white}{white}{(c)}}
  \end{overpic}
  \begin{overpic}[width=8.0cm]{Radiation.png}
    \put(84,13){\fcolorbox{white}{white}{(d)}}
  \end{overpic}

  \caption{TGMF tests of Inconel 718 with temperature range 350-600$^{\circ}$C.
  (a),(c),(e) Half life stable hysteresis loops.
  (b),(d),(f) Peak, valley and mean stresses.}
  \label{Fig:heating_methods}
\end{figure}

% 因此,参考德宇航的工作\cite{BAUFELD2008219},我们为热梯度机械疲劳(TGMF)测试设计开发了聚光辐射加热系统,如\ref{Fig:Radiation_Furnace2}所示。
% 该系统包括16根卤素灯管,灯丝的直径小于0.5mm,因此每根灯丝可以被抽象为一个线光源。
% 每根灯管对应一面反射镜,反射镜的几何形状为椭圆柱面,每根灯管位于其对应的椭圆柱面其中一个焦点,试件位于所有椭圆柱面的公共焦点,通过镜面反射将光线聚焦在试件表面进行加热。
% 通过聚光辐射的方法加热试件外表面,同时内表面通过压缩空气冷却来实现温度梯度。
% 该系统可以在空心试件表上实现受控的温度梯度循环,同时施加机械载荷,适用于金属和非金属材料。

% 在TMF和TGMF测试中,温度控制是实现每个测试的可重复加载条件的关键问题,两种试验采用相似的温度控制方法。
% 在测试过程中,使用接触式K型电偶测量试件外表面的温度,考虑到试验过程中温度变化速率很快,需要传感器对温度变化有迅速的响应,传感器体积越小,对温度的响应越迅速,因此我们选取测量点直径为0.25mm的热电偶,将试件中心点的温度作为控制量,以实现良好的温度控制,精度为±5°C。
% 对于试件内表面,考虑到气体流动,在TGMF期间,内部点焊热电偶是不可能的,因为它们的信号受到内部冷却的影响。
% 同时由于试件的尺寸很小,内孔直径只有6.5mm,也无法使用热像仪进行测量,因此在这里采用计算的方法来确定时间内表面的温度。


\begin{figure}[!htp]
\centering{\includegraphics[width=12cm]{IN718_Axial_Specimen_TGMF.pdf}}
\caption{Dimensions of the specimen for TGMF testing.}
\label{Fig:IN718_Axial_Specimen_TGMF}
\end{figure}

% \noindent
% % 飞机发动机推力的提高很大程度上依赖于涡轮前总温的提高。
% % 对于高温所带来的一系列问题,解决的办法主要有以下三个:
% % (1)提高材料的耐热性,发展高性能耐热合金,制造单晶叶片;
% % (2)采用热障涂层,对基底材料起到隔热作用,降低基底温度;
% % (3)采用先进的冷却技术,以少量的冷却空气获得更好的降温效果。
% % 高温合金及单晶材料耐热性的提高远远无法满足目前的温度设计需求,即使采用陶瓷基复合材料等耐高温材料,也不能完全取消冷却,先进的冷却可使高温部件承受更高的工作温度,使发动机寿命更长、可靠性更高。
% The improvement of aircraft engine thrust is highly dependent on the increment of the turbine inlet temperature.
% The solutions of the problems caused by the high temperature are mainly as the following:
% (1) improving heat resistance of materials, developing high-performance superalloy and manufacturing single crystal blades;
% (2) the thermal barrier coating is used to protect the base material and reduce the substrate temperature;
% (3) the use of advanced cooling technology, with a small amount of cooling air to achieve better cooling effect.
% The improvement on heat resistance of high temperature alloy and single crystal materials can not meet the current temperature design requirement. It is cannot completely eliminate cooling even if using the ceramic matrix composites, which is high-temperature-resistant. The effective cooling method makes it possible that the high-temperature parts are capable of withstanding high operation temperature, so that the engine life will be longer and the reliability will be more higher.

% % 发动机涡轮叶片主要采用气膜冷却和内部流冷却,轮盘通常采用内部二次流冷却。
% % 服役过程中,涡轮叶片不仅受到较大的交变载荷,而且在叶片表面和内部分别受到高温高压燃气的冲击和冷却气体的作用,这样涡轮叶片就遭受载荷和温度同时变化带来的热机械疲劳损伤。
% % 此外,为了增强发动机冷却效果,提高发动机效率,先进的航空发动机和燃气轮机热端涡轮叶片多为薄壁多孔结构。
% % 因此,我们设计了薄壁圆管试件来模拟零件的冷却结构,同时试件外壁涂覆有热障涂层。
% % 在内部冷却气体作用下,试件内表面与外表面之间会产生很大的温度梯度,
% % 实验过程中,采用的常规感应加热设备只会对内部金属层加热,使得内部金属层温度高于外部陶瓷层温度,这不符合热障涂层构件实际工作状态下的温度分布。
% % 实际发动机启动阶段升温过程只需要几秒钟的时间,而且降温也相当迅速,这些都对涂覆热障涂层的热梯度机械疲劳试验设备提出了更高的要求,同时也制约了这方面的研究。
% % 由于热梯度机械疲劳是试验室中最接近涡轮叶片服役状态的模拟试验,因而这方面的研究对于理解涂覆热障涂层的叶片损伤机理具有重要意义。
% The turbine blades are mainly cooled by air film and internal flow, and the turbine disk is usually cooled by the secondary flow.
% In the course of service, the turbine blades are not only subjected to the large alternating loads, but also subjected to the impact of high temperature and high pressure gas and the effect of cooling air on the surface and inside of the blades, respectively. So the turbine blades suffer from the thermal mechanical fatigue damage caused by the change of load and temperature.
% In addition, in order to enhance the engine cooling effect and improve the engine efficiency, most of the advanced aero-engine and gas turbine blades are thin-walled porous structures.
% Therefore, we designed a thin-walled tube specimen to simulate the cooling structure of the parts, and the outer wall of the specimen was coated with the thermal barrier coating.
% Under the action of the internal cooling air, a large temperature gradient between the inner surface and the outer surface of the specimen was produced.
% In the experiment, the conventional induction heating equipment only heated the inner metal layer, which maked the temperature of inner metal layer higher than the external ceramic layer temperature. It did not conform to the temperature distribution of the thermal barrier coating components under the actual working state.
% Actually, the engine start-up phase heating process only takes a few seconds, and the cooling stage is also very rapid. these are coating thermal barrier coatings on the heat gradient mechanical fatigue test equipment has put forward higher requirements, but also restricted the research.
% Because the thermal gradient mechanical fatigue test is the closest to the service state of turbine blades in the laboratory, the research has an important sense to understand the blade damage mechanism of the coated thermal barrier coating.

% % 同时,温度梯度会导致零件承受多轴载荷。
% % 对于内部冷却的零件,温度梯度会产生额外的应力,在热表面上表现为多轴压缩载荷,而在冷却表面上表现为多轴拉伸载荷。
% % 常规的热机械疲劳试验无法模拟这些应力条件并达到合适的温度分布,因此我们的试验系统是为热梯度机械疲劳(TGMF)测试而设计开发的。
% % 该系统可以在空心试件表上实现受控的温度梯度循环,同时施加机械载荷。
% % 通过聚光辐射的方法加热试件外表面,同时内表面通过压缩空气冷却来实现温度梯度。
% % 该试验系统需要实现较高的加热和冷却速度,因此加热系统的功率和冷却气体的流量需要进行详细设计。
% Moreover, the temperature gradient brings about multiaxial loads to the parts.
% For internally cooled parts, the temperature gradient can produce additional stress, which is expressed as a multiaxial compressive load on the hotter surface, while a multiaxial tensile load is displayed on the cooler surface.
% Conventional thermomechanical fatigue tests are not able to simulate these stress conditions and achieve a temperature gradient in the radial direction. So we designed this test system for thermal gradient mechanical fatigue (TGMF) testing.
% The system can realize the controlled thermal cycle with temperature gradient on the hollow specimen and apply the mechanical load at the same time.
% The external surface of the specimen is heated by the method of concentrating radiation, while the inner surface is cooled by compressed air to achieve the temperature gradient.
% The test system needs to achieve high heating and cooling speed, so the power of heating system and the volume flow of the cooling air should be designed in detail.


% %Thermal gradients cause multiaxial loads in cooled components.
% %In internal cooling, for example of rotors in the first stage of a jet engine, the thermal gradient induced stresses can not be relaxed by macroscopic deformations.
% %The stresses occurring at the component lead to multiaxial pressure loads on the hot surface and to multiaxial tensile loads on the cooled surface.
% %Since conventional thermomechanical tests do not simulate these stress conditions and achieve a homogeneous temperature distribution, our test system was designed and developed for the Thermal Gradient Mechanical Fatigue (TGMF) tests.
% %The system allows cyclic and simultaneously thermal and mechanical stress with controlled temperature gradients on the wall of hollow test specimens.
% %The temperature gradient is achieved by heating the outer surface with a furnace which emits concentrated radiation and the inner surface is simultaneously cooled with compressed air.
% %
% %These realistic tests have the advantage that they can transfer data from laboratory tests to the operating conditions.
% %In addition, the heating and cooling rates achieved in the TGMF test apparatus allow very short test cycles so that the fatigue load of an entire flight can be applied to a test body within three to five minutes.

\section{Mechanisms of heat transfer}
\noindent
%\subsection{Mechanisms of heat transfer}
Heat transfer describes the exchange of thermal energy and it is normally from a high temperature object to a lower temperature object.
%The exchange of kinetic energy of particles through the boundary between two systems which are at different temperatures from each other or from their surroundings.
%Heat transfer always occurs from a region of high temperature to another region of lower temperature.
Heat transfer changes the internal energy of both systems involved according to the First Law of Thermodynamics.
And the Second Law of Thermodynamics defines the concept of thermodynamic entropy, by measurable heat transfer.
Thermal equilibrium is reached when all involved bodies and the surroundings reach the same temperature.
%Thermal expansion is the tendency of matter to change in volume in response to a change in temperature.
The fundamental modes of heat transfer are conduction, convection and radiation.
In the engineering sciences, heat transfer includes the processes of thermal conduction, radiation, convection, and sometimes mass transfer.
Usually more than one of these processes occurs in a given situation as shown in \ref{Fig:HeatTransfer}.
% shows a high temperature object cooling in air,
%Common case is transfer of heat by a combination of the three modes.
\begin{figure}[!htp]
	\centering
	\includegraphics[width=16cm]{Heat_Convection.pdf}
	\caption{The fundamental modes of heat transfer.}
	\label{Fig:HeatTransfer}
\end{figure}

\subsection{Thermal conduction}
\noindent
Thermal conduction is the transfer of internal energy by microscopic diffusion and collisions of particles or quasi-particles within a body.
%The microscopically diffusing and colliding objects include molecules, atoms, and electrons.
%They transfer disorganized microscopic kinetic and potential energy, which are jointly known as internal energy.
Conduction can only take place within an object or material, or between two objects that are in contact with each other.
%Conduction takes place in all phases of ponderable matter, such as solids, liquids, gases and plasmas, but it is distinctly recognizable only when the matter is undergoing neither chemical reaction nor differential local internal flows of distinct chemical constituents.
%In the presence of such chemically defined contributory sub-processes, only the flow of internal energy is recognizable, as distinct from thermal conduction. When the processes of conduction yield a net flow of energy across a boundary because of a temperature gradient, the process is characterized as a flow of heat.
%Heat spontaneously flows from a hotter to a colder body. In the absence of external drivers, temperature differences decay over time, and the bodies approach thermal equilibrium.

%In conduction, the heat flow is within and through the body itself.
%In contrast, in heat transfer by thermal radiation, the transfer is often between bodies, which may be separated spatially.
%Also possible is transfer of heat by a combination of conduction and thermal radiation.
%In convection, internal energy is carried between bodies by a material carrier. In solids, conduction is mediated by the combination of vibrations and collisions of molecules, of propagation and collisions of phonons, and of diffusion and collisions of free electrons.
%In gases and liquids, conduction is due to the collisions and diffusion of molecules during their random motion. Photons in this context do not collide with one another, and so heat transport by electromagnetic radiation is conceptually distinct from heat conduction by microscopic diffusion and collisions of material particles and phonons. In condensed matter, such as a solid or liquid, the distinction between conduction and radiative transfer of heat is clear in physical concept, but it is often not phenomenologically clear, unless the material is semi-transparent.

%In the engineering sciences, heat transfer includes the processes of thermal radiation, convection, and sometimes mass transfer. Usually more than one of these processes occurs in a given situation. The conventional symbol for the material property, thermal conductivity, is $k$.

%The law of heat conduction, also known as Fourier's law, states that the time rate of heat transfer through a material is proportional to the negative gradient in the temperature and to the area, at right angles to that gradient, through which the heat flows.

An empirical relationship between the conduction rate in a material and the temperature gradient in the direction of energy flow, first formulated by Fourier who concluded that the heat flux resulting from thermal conduction is proportional to the magnitude of the temperature gradient and opposite to it in sign.
For a three directional conduction, the differential form of Fourier's Law of thermal conduction may be expressed as:
\begin{equation}
{\bf{q}} =  - k\nabla T
\end{equation}
where $\bf{q}$ is the local heat flux density(W/m$^2$), $k$ is the material's conductivity(W/mK), $\nabla T$ is the temperature gradient(K/m).
The heat flux density is the amount of energy that flows through a unit area per unit time.
The thermal conductivity, k, is often treated as a constant, though this is not always true.
While the thermal conductivity of a material generally varies with temperature, the variation can be small over a significant range of temperatures for some common materials.
In anisotropic materials, the thermal conductivity typically varies with orientation; in this case k is represented by a second-order tensor. In non uniform materials, k varies with spatial location.

\subsection{Thermal convection}
\noindent
Convective heat transfer, often referred to simply as convection, is the transfer of heat from one place to another by the movement of fluids.
Two types of convective heat transfer may be distinguished: free convection and forced convection.
Free convection describes when fluid motion is caused by buoyancy forces that result from the density variations due to variations of thermal temperature in the fluid.
Forced convection describes when a fluid is forced to flow over the surface by an external source such as fans, by stirring, and pumps, creating an artificially induced convection current.

\begin{figure}[!htp]
	\centering
	\includegraphics[width=16cm]{Thermal_boundary_layer.pdf}
	\caption{Thermal boundary layer.}
	\label{Thermal_boundary_layer}
\end{figure}

Convection is sometimes assumed to be described by Newton's law.
Newton's law, which requires a heat transfer coefficient, states that the rate of heat loss of a body is proportional to the difference in temperatures between the body and its surroundings. %The constant of proportionality is the heat transfer coefficient.
As shown in \ref{Thermal_boundary_layer}, the basic relationship for heat transfer by convection is:
%The heat transfer rate can be written as,
    \begin{equation}
    q_y=hA(T_{\rm{s}}-T_{\infty})
    \end{equation}
where $q_y$ is the heat transferred per unit time, $h$ is the heat transfer coefficient, $T_{\rm{s}}$ is the object's surface temperature and $T_{\infty}$ is the fluid temperature.

%And because heat transfer at the surface is by conduction,
%    \begin{equation}
%    q_y=-kA \frac{\partial}{\partial y} (T-T_{s})|_{y=0}
%    \end{equation}
%These two terms are equal; thus
%    \begin{equation}
%    -kA \frac{\partial}{\partial y} (T-T_{s})|_{y=0} = hA(T_s-T_{\infty})
%    \end{equation}
%Rearranging,
%    \begin{equation}
%    \frac{h}{k}=\frac{\frac{\partial (T_s-T)}{\partial y}|_{y=0}}{\frac{T_s-T_{\infty}}{L}}
%    \end{equation}
%Making it dimensionless by multiplying by representative length L,
%    \begin{equation}
%    \frac{hL}{k}=\frac{\frac{\partial (T_s-T)}{\partial y}|_{y=0}}{T_s-T_{\infty}}
%    \label{tab:Nusselt1}
%    \end{equation}


%Convective heat transfer is one of the major modes of heat transfer and convection is also a major mode of mass transfer in fluids.  Therefore, influence of fluid field factors will impact the convective heat transfer. There are five aspects:
%
%
%1.Natural convection and forced convection.
%
%
%$\bullet$ Natural Convection.
%
%The onset of natural convection is determined by the Rayleigh number (Ra). This dimensionless number is given by:
%
%    \begin{equation}
%        Ra=\frac{\Delta \rho g L^3}{D \mu}
%        \label{tab:RayleighNumber}
%    \end{equation}
%
%where
%
%    $\Delta \rho$ is the difference in density between the two parcels of material that are mixing,\par
%    g is the local gravitational acceleration,\par
%    L is the characteristic length-scale of convection,\par
%    D is the diffusivity of the characteristic that is causing the convection,\par
%    $\mu$ is the dynamic viscosity.\par
%
%$\bullet$ Forced convection.
%
%In general, natural convection flow rate is low, the natural convection heat transfer coefficient is usually lower than the forced convection heat transfer coefficient.
%
%2.Laminar flow and turbulent flow.
%
%$\bullet$ Limina flow.
%
%Laminar flow occurs when a fluid flows in parallel layers, with no disruption between the layers.In the perpendicular direction, the heat transfer depends mainly on molecular diffusion (ie, thermal conductivity).
%
%$\bullet$ Turbulent flow.
%
%There are strong pulsation and vortex. The fluids between various parts mix rapidly, so turbulent convection effect is stronger than laminar flow.
%
%3.Phase transition.
%
%$\bullet$ Boiling.
%
%$\bullet$ Condensation.
%
%4.The physical properties of the fluid.
%
%$\bullet$ Thermal conductivity k,$[W/(m \cdot K)]$. If the fluid thermal conductivity is larger, thermal resistance become smaller. Convective heat transfer is more intense.
%
%$\bullet$ Density $\rho$, $[kg/m^3]$.
%
%$\bullet$ Specific heat capacity c,$[J/(kg \cdot K)]$.
%
%$\rho c$ reflects the size of the heat capacity of unit volume of fluid. The greater its value is, the more heat is transferred by convection, and the convective heat transfer is more intense.
%
%
%$\bullet$ Dynamic viscosity $\eta$, $[Pa \cdot s]$. Kinematic viscosity $\nu=\eta /\rho$, $[m^2/s]$. The viscosity of the fluid affect the velocity distribution and flow pattern, so it affects the convective heat transfer.
%
%$\bullet$ Volume expansion coefficient $\alpha_V$, $[K^{-1}]$. In the general case of a gas, liquid, or solid, the volumetric coefficient of thermal expansion is given by
%
%    \begin{equation}
%        \alpha_V=\frac{1}{V}\left(\frac{\partial V}{\partial T}\right)_p
%        \label{tab:ExpansionCoefficient}
%    \end{equation}
%
%For an ideal gas, the volumetric thermal expansivity (i.e. relative change in volume due to temperature change) depends on the type of process in which temperature is changed. Two known cases are isobaric change, where pressure is held constant, and adiabatic change, where no work is done and no change in entropy occurs.
%
%In an isobaric process, the volumetric thermal expansivity, which we denote $\beta_p$,(The index p denotes an isobaric process) is:
%
%    \begin{equation}
%        PV = nRT
%    \end{equation}
%    \begin{equation}
%        \ln\left(V\right) = \ln \left(T\right) + \ln\left(nR/P\right)
%    \end{equation}
%    \begin{equation}
%        \beta_p = \bigg(\frac{1}{V} \frac{dV}{dT}\bigg)_p = \bigg(\frac{d(ln V)}{d T}\bigg)_p = \frac{d(ln T)}{d T} = \frac{1}{T}
%    \end{equation}
%
%5.The geometric factors of the heat transfer surface.
%
%Geometry, size, relative position, roughness and other geometric factors will affect the flow of fluid, and therefore affect the fluid velocity distribution and the convective heat transfer. The surface heat transfer coefficient is a function of many variables.


\subsection{Thermal radiation}
\noindent
Thermal radiation is the emission of electromagnetic waves from all matter that has a temperature greater than absolute zero.
It represents a conversion of thermal energy into electromagnetic energy.
All matter with a temperature greater than absolute zero emits thermal radiation. When the temperature of the body is greater than absolute zero, inter-atomic collisions cause the kinetic energy of the atoms or molecules to change. This results in charge-acceleration and/or dipole oscillation which produces electromagnetic radiation, and the wide spectrum of radiation reflects the wide spectrum of energies and accelerations that occur even at a single temperature.

Thermal radiation power of a black body per unit area of radiating surface per unit of solid angle and per unit frequency $\nu$ is given by Planck's law as:
\begin{equation}
u(\nu,T)=\frac{2 h\nu^3}{c^2}\cdot\frac1{e^{h\nu/k_{\rm{B}}T}-1},
\end{equation}
or in terms of wavelength
\begin{equation}
u(\lambda,T)=\frac{\beta}{\lambda^5}\cdot\frac1{e^{hc/k_{\rm{B}}T\lambda}-1},
\end{equation}
where $h=6.626\times10^{-34}{\rm{J}}\cdot {\rm{s}}$ is Planck's constant, $b=2.898\times10^{-3}{\rm{m}} \cdot {\rm{K}}$ is Wien's displacement constant, $k_{\rm{B}}=1.381\times10^{-23}{\rm{J}} \cdot {\rm{K}}^{-1}$ is Boltzmann constant and $c$ is the speed of light.

This formula mathematically follows from calculation of spectral distribution of energy in Quantization (physics)|quantized electromagnetic field which is in complete thermal equilibrium with the radiating object. The equation is derived as an infinite sum over all possible frequencies. The energy, $E=h \nu$, of each photon is multiplied by the number of states available at that frequency, and the probability that each of those states will be occupied.

Integrating the above equation over $\nu$ the power output given by the Stefan-Boltzmann law is obtained, as:
\begin{equation}
P = \sigma \cdot A \cdot T^4,
\label{Equ:IntegratingOfStefanBoltzmann}
\end{equation}
where the constant of proportionality $\sigma=5.670\times10^{-8}\rm{W}\cdot \rm{m}^{-2}\cdot \rm{K}^{-1}$ is the Stefan-Boltzmann constant and $A$ is the radiating surface area.

The wavelength $\lambda \,$, for which the emission intensity is highest, is given by Wien's displacement law as
\begin{equation}
\lambda_{\max} = \frac{b}{T}.
\end{equation}
For surfaces which are not black bodies, one has to consider the (generally frequency dependent) emissivity factor $\epsilon(\nu)$. This factor has to be multiplied with the radiation spectrum formula before integration. If it is taken as a constant, the resulting formula for the power output can be written in a way that contains $\epsilon$ as a factor:
\begin{equation}
P = \epsilon \cdot \sigma \cdot A \cdot T^4.
\end{equation}

This type of theoretical model, with frequency-independent emissivity lower than that of a perfect black body, is often known as a 'gray body'. For frequency-dependent emissivity, the solution for the integrated power depends on the functional form of the dependence, though in general there is no simple expression for it. Practically speaking, if the emissivity of the body is roughly constant around the peak emission wavelength, the gray body model tends to work fairly well since the weight of the curve around the peak emission tends to dominate the integral.

\subsection{Thermocouple}
\noindent
A thermocouple is a sensor for measuring temperature.
It consists of two dissimilar metal wires, joined at one end.
%When properly configured, thermocouples can provide temperature measurements over a wide range of temperatures.
In contrast to most other methods of temperature measurement, thermocouples are self powered and require no external form of excitation.
The physical principle of the thermocouple is Seebeck effect.

\begin{figure}[!htp]
	\centering
	\includegraphics[width=6cm]{Seebeck_Voltage.pdf}
	\caption{Schematic of Seebeck effect.}
	\label{Fig:Seebeck_Voltage}
\end{figure}
In 1821 the German physicist Thomas Johann Seebeck discovered the continuous current flow in the thermoelectric circuit when two wires of dissimilar metals are joined at both ends and one of the ends is heated, and that is the Seebeck effect.
If this circuit is broken at the center (see \ref{Fig:Seebeck_Voltage}), the net open circuit voltage $e_{\rm AB}$ (the Seebeck voltage) is a function of the junction temperature and the composition of the two metals.
This was because the electron energy levels in each metal shifted differently and a voltage difference between the junctions.
We can't measure the Seebeck voltage directly because we must first connect a voltmeter to the thermocouple, and the voltmeter leads themselves create a new thermoelectric circuit.

A thermocouple is an electrical device consisting of two different conductors forming electrical junctions at differing temperatures. A thermocouple produces a temperature-dependent voltage as a result of the thermoelectric effect, and this voltage can be interpreted to measure temperature.
The thermoelectric effect is the direct conversion of temperature differences to electric voltage and vice versa. A thermoelectric device creates voltage when there is a different temperature on each side. Conversely, when a voltage is applied to it, it creates a temperature difference. At the atomic scale, an applied temperature gradient causes charge carriers in the material to diffuse from the hot side to the cold side.
\begin{figure}[!htp]
	\centering
	\includegraphics[width=10cm]{Thermocouple_circuit_Ktype_including_voltmeter_temperature.pdf}
	\caption{K-type thermocouple in the standard thermocouple measurement configuration.}
	\label{Fig:Thermocouple_circuit_Ktype_including_voltmeter_temperature}
\end{figure}

Once we obtain the voltage from the thermocouple we have to convert the voltage to a temperature.
As shown in \ref{Fig:plot_thermocouples}, output voltages for the type K, R and S thermocouples are plotted as a function of temperature.
Unfortunately, the temperature-versus-voltage relationship of the thermocouple is not linear.

\begin{figure}[htbp]
	\centering
	\includegraphics[width=14cm]{plot_thermocouples.pdf}
	\caption{Characteristic functions for thermocouples types K, R, S.}
	\label{Fig:plot_thermocouples}
\end{figure}

\begin{table}[ht]
  \centering
  \small
  \caption{Direct polynomial coefficients of type R thermocouple.}
    \begin{tabular}{llll}
    \toprule
    Range & -50 to 1064.18$^{\circ}$C & 1064.18 to 1664.5$^{\circ}$C & 1664.5 to 1768.1$^{\circ}$C \\
    \midrule
    $c_0=$ & 0     & 2.95157925316$\times10^{3}$ & 1.52232118209$\times10^{5}$ \\
    $c_1=$ & 5.289617298 & -2.520612513 & -2.68819888545$\times10^{2}$ \\
    $c_2=$ & 1.39166589782$\times10^{-2}$ & 1.59564501865$\times10^{-2}$ & 1.71280280471$\times10^{-1}$ \\
    $c_3=$ & -2.38855693017$\times10^{-5}$ & -7.64085947576$\times10^{-6}$ & -3.45895706453$\times10^{-5}$ \\
    $c_4=$ & 3.56916001063$\times10^{-8}$ & 2.05305291024$\times10^{-9}$ & -9.34633971046$\times10^{-12}$ \\
    $c_5=$ & -4.62347666298$\times10^{-11}$ & -2.93359668173$\times10^{-13}$ &  \\
    $c_6=$ & 5.00777441034$\times10^{-14}$ &       &  \\
    $c_7=$ & -3.73105886191$\times10^{-17}$ &       &  \\
    $c_8=$ & 1.57716482367$\times10^{-20}$ &       &  \\
    $c_9=$ & -2.81038625251$\times10^{-24}$ &       &  \\
    \bottomrule
    \end{tabular}%
  \label{tab:TypeRDirectPolynomial}%
\end{table}%


\begin{table}[ht]
  \centering
  \small
  \caption{Inverse polynomial coefficients of type R thermocouple.}
    \begin{tabular}{lllr}
    \toprule
    Temperature range & -50 to 250$^{\circ}$C & 250 to 1200$^{\circ}$C & \multicolumn{1}{l}{1064 to 1664.5$^{\circ}$C} \\
    \midrule
    Voltage range & -226 to 1923$\mu$V & 1923 to 13228$\mu$V & \multicolumn{1}{l}{11361 to 19739$\mu$V} \\
    \midrule
    $k_{0}=$ & 0.0000000 & 1.334584505$\times 10^{1}$ & \multicolumn{1}{l}{-8.199599416$\times 10^{1}$} \\
    $k_{1}=$ & 1.8891380$\times 10^{-1}$ & 1.472644300$^\circ$C$\times 10^{-1}$ & 1.553962042$\times 10^{-1}$ \\
    $k_{2}=$ & -9.3835290$\times 10^{-5}$ & -1.844024844$\times 10^{-5}$ & -8.342197663$\times 10^{-6}$ \\
    $k_{3}=$ & 1.3068619$\times 10^{-7}$ & 4.031129726$\times 10^{-9}$ & 4.279433549$\times 10^{-10}$ \\
    $k_{4}=$ & -2.2703580$\times 10^{-10}$ & -6.249428360$\times 10^{-13}$ & -1.19157791$\times 10^{-14}$ \\
    $k_{5}=$ & 3.5145659$\times 10^{-13}$ & 6.468412046$\times 10^{-17}$ & 1.492290091$\times 10^{-19}$ \\
    $k_{6}=$ & -3.8953900$\times 10^{-16}$ & -4.458750426$\times 10^{-21}$ & \multicolumn{1}{l}{} \\
    $k_{7}=$ & 2.823.9471$\times 10^{-19}$ & 1.994710146$\times 10^{-25}$ & \multicolumn{1}{l}{} \\
    $k_{8}=$ & -1.2607281$\times 10^{-22}$ & -5.313401790$\times 10^{-30}$ & \multicolumn{1}{l}{} \\
    $k_{9}=$ & 3.1353611$\times 10^{-26}$ & 6.481976217$\times 10^{-35}$ & \multicolumn{1}{l}{} \\
    $k_{10}=$ & -3.3187769$\times 10^{-30}$ &       & \multicolumn{1}{l}{} \\
    \midrule
    Error & 0.02 to -0.02$^{\circ}$C & 0.005 to -0.005$^{\circ}$C & \multicolumn{1}{l}{0.001 to -0.0005$^{\circ}$C} \\
    \bottomrule
    \end{tabular}%
  \label{tab:TypeRInversePolynomial}%
\end{table}%

Therefore, polynomials are used to present the temperature-versus-voltage and the inverse relationship.
Direct polynomials (see Equation (\ref{Equ:Direct_polynomials})) provide the thermoelectric voltage from a known temperature and inverse polynomials (see Equation (\ref{Equ:Inverse_polynomials})) provide the temperature from a known thermoelectric voltage.
\begin{equation}
\label{Equ:Direct_polynomials}
E = {c_0} + {c_1}T + {c_2}{T^2} + ... + {c_n}{T^n} = \sum\limits_{i = 0}^n {{c_i}{T^i}}
\end{equation}
\begin{equation}
\label{Equ:Inverse_polynomials}
T = {k_0} + {k_1}E + {k_2}{E^2} + ... + {k_n}{E^n} = \sum\limits_{i = 0}^n {{k_i}{E^i}}
\end{equation}
Where, $T$ is temperature in degrees Celsius ($^{\circ} \rm{C}$), $E$ is thermocouple electric potential in microvolts (${\rm{\mu V}}$), $c_i,k_i$ are polynomial coefficients unique to each type thermocouple, $n$ is the maximum order of the polynomial.
As n increases, the accuracy of the polynomial improves.
A representative number is $n = 9$ for $\pm 1 ^{\circ} \rm{C}$ accuracy.

In order to obtain the higher accuracy and system speed, the polynomials may be divided into several small sectors over a narrow temperature range.
In the software for our data acquisition system, the thermocouple characteristic curve is divided into three sectors, and each sector is approximated by a nine or ten order
polynomial.
%as well as used in our temperature measurement system.
For example, type R thermocouples coefficients of the approximate direct/inverse polynomials are given in Table \ref{tab:TypeRDirectPolynomial} and \ref{tab:TypeRInversePolynomial}.
%Further more, the similar coefficients wtype K and type S thermocouples are also used in our temperature measurement system.
% Table generated by Excel2LaTeX from sheet 'R'

Three types (K, R, S) of thermocouples are used in our temperature measurement system for different temperature ranges.



%\section{Heat and heat transfer}
%The internal, thermal energy of a body or of a component is referred to as its "heat" in general terms and in a manner of speaking. In fact, heat is the thermal energy, the exchange of which is preceded by the temperature difference alone. The thermal energy of a body, which is stored as a kinetic energy in the non-directed particle motion at the molecular level, is changed by heat input and exhaust.
%The storage capacity for thermal energy of a substance is expressed by its specific heat capacity $c$.
%This characteristic value indicates the thermal energy related to the mass, which is necessary to heat up the material by 1 K.
%The heat capacity depends on the temperature.
%The relationship between mass and volume of the substance is described by the density $\rho$.
%As a result of the thermal expansion of a material, its density normally decreases with increasing temperature.
%Heat transfer is carried out by heat transport.
%Heat transport always takes place at right angles to the levels of the same temperature (isotherms) in the direction of the negative temperature gradient.
%Under constant boundary conditions, the temperature field approaches a stationary, stable state with increasing time.
%The case of the stationary state thus denotes a temperature field that no longer has any temporal dependence.
%According to Baehr and Stephan [8], three types of heat transport can be distinguished: heat conduction, heat radiation and convection.
%These species may occur isolated or combined.
%
%\subsection{Heat conduction}
%
%The energy transport between adjacent molecules of a system is referred to as heat conduction.
%The relevant parameter for this mechanism is the thermal conductivity k of the material.
%For simplified calculations, this characteristic value is often defined as being independent of temperature, however, the value of the thermal conductivity is to be regarded as temperature-dependent.
%For thermally isotropic material behavior, this scalar proportionality factor describes the relationship between the gradient of the scalar temperature field $T$ and the heat flux in the heat conduction law according to Fourier \cite{fourier1822theorie}:
%
%
%\begin{equation}
%\dot{q} =  - k\nabla T
%\end{equation}
%
%
%The minus sign in equation (2.1) takes account of the fact that the heat flow points in the direction of the negative temperature gradient.
%
%\subsection{Heat radiation}
%Heat transport by electromagnetic waves is called heat radiation. In contrast to the other types of heat transport, heat radiation can also take place in a vacuum, so no medium needs to be used. The heat flow due to heat radiation is according to Stefan [61] and Boltzmann [11]:
%
%
%\subsection{Convection}
%
%If the heat transport is carried out by transfer from one medium to another as a result of particle movement at the macroscopic level, this is referred to as a convective heat transfer. The characteristic feature here is that the heat transport takes place at a boundary layer and with the participation of a flowing medium (fluid or gas).
%
%Due to the difference between the temperatures of the flowing medium $T_{\infty}$ and the body $T_s$ under consideration, a heat flow occurs.
%The heat transfer coefficient is the defined proportionality factor by which the relationship between the temperature difference and the heat flow density is described:
%
%Since the heat transfer coefficient as a single characteristic covers a large number of influences which characterize the processes in the boundary layer flow between the medium and the body (viscosity, flow properties, etc.), these values are usually either determined experimentally or recalculated from fluid dynamics by approximation methods from the heat flow.
%The heat transfer coefficients can be exported directly as a time- or temperature-dependent variable from flow-mechanical calculations.
%In order to recalculate these transition coefficients from the thermal current density determined in a flow simulation, knowledge of the temperature-dependent material parameters of the materials involved is necessary.
%For turbulent heat transitions, case-related correlations exist (eg according to Dittus and Boelter [17] for turbulent tube flows) which are used for these purposes.
%
%In the case of natural convection due to the free flow of water to steel, ecological values of 100 w / m2K to 600 w / m2K are customary for the transition coefficient. For forced convection, the flow velocity is determined by external influences. In these cases, the transition coefficient can increase to several tens of thousands of w / m2K. Even higher values are possible if condensation and evaporation occurs in the region of the boundary layer. More detailed information can be found in Herwig [32]. Due to the dependency of the transition coefficient on both the temperature and the temperature difference, the thermal current density in this case is not proportional to the temperature difference. For forced convection, the heat transfer coefficient can be regarded as independent of the temperature difference (see Baehr and Stephan [8]).
%
%For applications in which both convection and heat radiation have to be taken into account, the heat transfer coefficient uk is often modified to cover both effects [8].
%
%For the calculations carried out in this work, the differential equation of the heat conduction (2.1) is solved using the FE program package Ansys. The thermal boundary conditions are defined by convection boundary conditions. The calculation procedure for the determination of the temperature field solution is presented in chapter 3 in detail.



%\section{Plasticity}
%
%If the stress state of a material exceeds a limit state, a disproportionate increase in the strain occurs.
%This limit state is usually formulated via a flow condition set with the stress tensor.
%If it is fulfilled by the current stress tensor, plastic strains occur.
%If this is fulfilled by the current stress tensor, plastic strains occur.
%These plastic strains are described by material models which represent the elastic-plastic deformation behavior by suitable flow and hardening rules.
%A basic assumption is that for small strains and isotropic material, the strain tensor can be additively split into an elastic and plastic part:
%
%Even in the presence of plastic strains, the elastic part of the strain tensor also follows Hooke's law:
%
%For isotropic material, the elastic tensor is dependent on two free parameters, the Lame constants.
%The relationship between them and the Young's modulus E and Poisson's ratio u by:
%
%The components of the elastic tensor are
%
%\section{Yield criterion}
%The yield criterion is defined as the limit state of stress from which plastic material behavior occurs.
%For the material models used in this work, the yield criterion of Mises \cite{mises1928mechanik} is used.
%It is formulated as a function of the deviatoric stress $\bm{s}$.
%Thus, only the shape-changing part of the stress tensor has an influence on the yield behavior, the volume-changing hydrostatic part is not taken into account in the yield criterion.
%For metallic materials, this yield is considered to be usable and generally accepted.
%The deviatoric stress $\bm{s}$ is defined as:
%\begin{equation}
%\bm{s} = \bm{\upsigma}-\frac{1}{3}\rm{tr}(\bm{\upsigma}),
%\end{equation}
%and the yield criterion according to von Mises is:
%\begin{equation}
%F(\bm{s},{\bf{a}},k) = (\bm{s}-{\bf{a}}):(\bm{s}-{\bf{a}})-k^2.
%\label{Equ:MisesCriterion}
%\end{equation}
%In the main stress space, the area defined by Equation \ref{Equ:MisesCriterion} describes a cylinder surface with a radius $k$, the center axis of which is along the hydrostatic axis (equisetrix).
%The pure deviatoric tensor $\bf{a}$ allows the consideration kinematic hardening by a shift of the middle or anchor point of the yield surface.
%The deviatoric yield stress is determined by the uniaxial yield stress according to:
%\begin{equation}
%k = \sqrt{\frac{2}{3}} \sigma_y.
%\end{equation}
%
%\section{Flow rule}
%The flow rule is used to determine the plastic strain increments.
%This is a evolution equation which establishes the relation between the plastic potential and the plastic multiplier to the increment of the plastic strain.
%If the yield criterion is the same used for the potential, it is an associated flow rule, otherwise it is called non-associated.
%The associated flow rule is:
%
%For the associated flow rule, the increments of the plastic strain can only occur in the direction of the normal tensor on the yield surface.
%This normal tensor is made up of:
%
%Correspondingly, the associated flow rule is also referred to as a normal rule. If the normalized direction tensor n in equation (2.13) is used:
%
%The plastic multiplier just corresponds to the accumulated plastic strain:
%
%
%\section{Material model}
%
%The characteristics of the material models used for numerical simulations are presented below. The first of the listed material models (after Armstrong / Frederick) is not used in the calculations carried out for this work; the descriptions of the models according to Chaboche and Ohno / Wang are the basic points defined by Armstrong / Frederick. The descriptions provide an overview of the formulation of the relevant development equations under isothermal conditions. The specifics of the implementation of the material models according to Chaboche and Ohno / Wang under thermal transient conditions are discussed in detail by Willuweit [66].
%
%\section{Armstrong and Frederick model}
%The Armstrong and Frederick model [7] considered a non-linear kinematic hardening rule.
%The evolution equation used in this model for the increment of the back stress tensor is:
%\begin{equation}
%{\rm{d}}{\bf{a}} = (h_0 \cdot {\bf{n}} - c \cdot {\bf{a}}){\rm{d}} p.
%\label{Equ:ArmstrongFrederick}
%\end{equation}
%$h_0$ corresponds to the initial value of the plastic tangent module, $c$ is a material constant.
%Transformation of Equation \ref{Equ:ArmstrongFrederick} leads to:
%\begin{equation}
%{\rm{d}}{\bf{a}} = c \cdot (r \cdot {\bf{n}} - {\bf{a}}){\rm{d}} p,
%\label{Equ:ArmstrongFrederick2}
%\end{equation}
%with
%\begin{equation}
%r=\frac{h_0}{c}.
%\end{equation}
%In this representation, $r$ corresponds to a limiting radius for the back stress tensor.
%The back stress tensor ${\bf{a}}$ is not permitted to leave the yield surface.
%The more the value of ${\bf{a}}$ approaches the boundary radius, the more the value of the plastic tangent module decreases until the stress-strain curve ends in a horizontal.
%This hardening rule can be integrated analytically integrated

\section{Design and development of the radiation furnace for TGMF tests}
\noindent
% http://www.dlr.de/wf/en/desktopdefault.aspx/tabid-2192/3209_read-5876/
% Thermal gradients cause multiaxial loads in cooled components. With internal cooling, for example rotors in the first stage of a jet engine, the thermally induced stresses can not relax by macroscopic deformations. The stresses occurring on the component lead to multiaxial pressure loads on the hot surface and to multiaxial tensile loads on the cooled surface.

% Since it is not possible to simulate these load conditions and to achieve a homogeneous temperature distribution in conventional thermomechanical tests, the Institute for Materials Research of the German Aerospace Center (DLR) has developed two test rigs for mechanical fatigue by thermal gradients (TGMF). These systems allow a cyclic and at the same time thermal and mechanical load with controlled temperature gradients on the wall of hollow test specimens. The temperature gradient is achieved by heating the outer surface with an oven that emits concentrated radiation and simultaneously cooling the inner surface with compressed air.

% A stationary temperature gradient is reached, usually after 20 to 40 seconds. Due to the high heat flux generated by the radiant furnace, heating rates can be achieved that correspond to those achieved in real, turbine jet turbine blades., Realistic cooling rates can be achieved by forced air cooling on the outside surface. During the cooling sequence, the test cycle, the specimen is cooled via cooling holes in the shutters.

% These realistic tests have the advantage that they can be used to transfer data from laboratory tests to the conditions of use. In addition, the heating and cooling rates achieved in the TGMF test apparatus allow very short test cycles so that the fatigue load of one entire flight can be applied to one specimen within three to five minutes.
\noindent
In this section, the high temperature radiation furnace is developed.
It is used to heat the specimens with thermal barrier coating up to a high temperature.
The furnace is matching with the MTS 809 axial/torsional fatigue testing machine.
Specimen heating can be accomplished by various techniques including induction, direct resistance, radiant, or forced air heating.
%There are several techniques to heat the specimen such as

%The mainly heating methods of high temperature material experiments are: environmental high temperature furnace, high frequency electromagnetic induction heating and radiation heating.
Forced air heating furnace can be used for all kinds of material. It has good temperature stability and mostly used in isothermal material tests, but it is not suitable for the experiments which need rapid temperature change.

Induction heating is the process of heating an electrically conducting object (usually a metal) by electromagnetic induction, through heat generated in the object by eddy currents. An induction heater consists of an electromagnet, and an electronic oscillator which passes a high-frequency alternating current (AC) through the electromagnet. The rapidly alternating magnetic field penetrates the object, generating electric currents inside the conductor called eddy currents. The eddy currents flowing through the resistance of the material heat it by Joule heating.

An important feature of the induction heating process is that the heat is generated inside the object itself, instead of by an external heat source via heat conduction. Thus objects can be very rapidly heated.
Induction heating is convenient to match up an additional air cooling system, which commonly used in thermal mechanical fatigue test, but induction heating method is not suitable for heating non-metallic materials.
%光加热高温炉的原理是利用镜面反射,将光线汇聚在试件表面,通过光辐射传递能量,达到加热试件表面的目的。
%光源通常选取卤素灯。

A halogen lamp, also known as a quartz-halogen, is an incandescent lamp that has a small amount of a halogen such as iodine or bromine added.
%The combination of the halogen gas and the tungsten filament produces a halogen cycle chemical reaction which redeposits evaporated tungsten back onto the filament, increasing its life and maintaining the clarity of the envelope.
%Because of this, a halogen lamp can be operated at a higher temperature than a standard gas-filled lamp of similar power and operating life, producing light of a higher luminous efficacy and color temperature.
The small size of halogen lamps permits their use in compact optical systems for projectors and illumination.
% 聚光辐射加热系统包括16根卤素灯管,灯丝的直径小于0.5mm,因此每根灯管可以被抽象为一个线光源。
The radiation furnace consists of 16 halogen lamps and the filament diameter of the lamp is about 0.5 mm. In the simulation, the filament can be abstracted as a cylindrical surface, due to its geometry.

% 每根灯管对应一面反射镜,反射镜的几何形状为椭圆柱面,每根灯管位于其对应的椭圆柱面其中一个焦点,试件位于所有椭圆柱面的公共焦点,通过镜面反射将光线聚焦在试件表面进行加热。
% 根据传热学,要确定这样一个系统的温度状态,需要已知卤素灯的辐射效率,镜面的反射效率,试件表面的辐射吸收率,以及试件内外表面的热对流系数。
Each lamp corresponds to a reflector. The geometry of the reflector is an elliptic cylinder. Each lamp is located at one of the focus
lines of its corresponding elliptic cylinder. The specimen is located at the common focus lines of all elliptic cylinders. The light is reflected by the mirror and focus on the surface of the specimen for heating.
According to the heat transfer theory, to determine the temperature distribution of such a system, it is necessary to know the radiation efficiency of the halogen lamp, the reflection efficiency of the mirror, the radiation emissivity and the convective heat transfer coefficient of the inner and outer surfaces of the specimen.
% 我们的研究分为三个步骤:
% 1.单盏卤素灯密封腔辐射试验,确定单盏卤素灯的辐射效率,及物体表面的辐射吸收率;
% 2.单盏卤素灯镜面聚光加热试验,确定镀金镜面的反射效率,及物体表面的对流换热系数;
% 3.整体系统的聚光加热试验,验证计算结果。
% 本节介绍了单盏卤素灯自由辐射的试验过程及数值计算结果。

Three tests were carried out to determine the heat transfer parameters:

1. Single lamp radiation test in a sealed cavity to determine the radiation efficiency of the single halogen lamp;

2. Elliptical cylinder mirror reflection test, to determine the reflection efficiency of the gold-plated mirror, and the convection heat transfer coefficient of the surface;

3. Heating test of the radiation furnace to verify the calculation results.
This section describes the test procedure and numerical results of the free radiation of a single halogen lamp.


\section{Experiments and simulations of the single lamp radiation in the cylindrical cavity}
\noindent
% 单盏卤素灯密封腔辐射试验装置设计如\ref{Fig:OneLightRadiation}(a)所示,包括圆柱形不锈钢壁面、上下盖板和灯头固定部件等,构成了一个圆柱形密封腔。
% 所有零件都使用SS304不锈钢通过数控加工中心制作,空腔内壁采用耐高温黑色涂料均匀喷涂,确保空腔内壁各处的辐射发射率相同。
% 我们采用K型热电偶测量圆柱壁面的温度分布。
% 根据该试验装置的对称性,选取其中一个轴对称截面放置热电偶,内壁面和外壁面在垂直方向各五个,热电偶的位置及编号如\ref{Fig:OneLightRadiation}(b)所示。
\noindent
The test rig of the single lamp radiation in a cylindrical cavity is shown in \ref{Fig:OneLightRadiation}(a).
% It consists of a cylindrical wall, upper and lower cover plates, and lamp holders to form a cylindrical sealed chamber.
It consists of a cylindrical wall, upper and lower cover plates, and lamp holders.
The cylindrical wall, upper and lower cover plates constructed a sealed cylindrical cavity and their material was the stainless steel SS304. The inner walls of the cavity were coated with a high-temperature resistant black paint to ensure that the emissivity is the same throughout the inner walls of the cavity.
K-type thermocouples were used to measure the temperature distribution of the cylindrical wall.
According to the symmetry of the test rig, an axisymmetric section was chosen to weld the thermocouples. The positions of the thermocouples are shown in \ref{Fig:OneLightRadiation}(b), with five thermocouples welded on the inner surface and the other five thermocouples welded on the outer surface. For simplicity

\begin{figure}[!htp]
  \centering
  \begin{overpic}[width=8.0cm]{OneLightRadiationPhoto.jpg}
    \put(0,90){\fcolorbox{white}{white}{(a)}}
  \end{overpic}
  \begin{overpic}[width=8.0cm]{OneLightRadiationShow.pdf}
    \put(0,90){\fcolorbox{white}{white}{(b)}}
  \end{overpic}
  \caption{The test rig of the single halogen lamp radiation in a cylindrical cavity. (a) The infrared temperature measurement point. (b) Thermocouple locations.}
  \label{Fig:OneLightRadiation}
\end{figure}

\begin{figure}[!htp]
  \centering
  \begin{overpic}[width=8.0cm]{NI9213.jpg}
    \put(84,65){\fcolorbox{white}{white}{(a)}}
  \end{overpic}
  \begin{overpic}[width=8.0cm]{IR.jpg}
    \put(84,65){\fcolorbox{white}{white}{(b)}}
  \end{overpic}
  \caption{Temperature measurement devices: (a) NI-9213  temperature input module. (b) Infrared thermometer.}
  \label{Fig:temperature_measurement_devices}
\end{figure}

% \begin{figure}
%   \begin{minipage}[t]{0.5\linewidth} % 如果一行放2个图,用0.5,如果3个图,用0.33\
%   \nonumber
%     \centering
%     \includegraphics[width=3.0in]{OneLightRadiationPhoto.jpg}
%     \centerline{(a) Cylindrical cavity.}
%     \label{Fig:OneLightRadiationShow}
%   \end{minipage}%
%   \begin{minipage}[t]{0.5\linewidth}
%     \centering
%     \includegraphics[width=3.0in]{OneLightRadiationShow.pdf}
%     \centerline{(b) Thermocouple locations.}
%     \label{Fig:OneLightRadiationPhoto}
%   \end{minipage}

%   \begin{minipage}[t]{0.5\linewidth} % 如果一行放2个图,用0.5,如果3个图,用0.33\
%   \nonumber
%     \centering
%     \includegraphics[width=3.0in]{NI9213.jpg}
%     \centerline{(c) NI9213 thermocouple signal transmitter }
%     \centerline{and the data acquisition device. }
%     \label{Fig:NI9213}
%   \end{minipage}%
%   \begin{minipage}[t]{0.5\linewidth}
%     \centering
%     \includegraphics[width=3.0in]{IR.jpg}
%     \centerline{(d) The infrared temperature measurement }
%     \centerline{device.}
%     \label{Fig:IR}
%   \end{minipage}

% %  \begin{minipage}[t]{1.0\linewidth} % 如果一行放2个图,用0.5,如果3个图,用0.33\
% %  \nonumber
% %    \centering
% %    \includegraphics[width=5.0in]{BlockDiagram.png}
% %    \centerline{(e) Program block diagram of data collection, temperature transform and computation.}
% %    \label{Fig:FrontPanel}
% %  \end{minipage}

%   \caption{Test rig of the single halogen lamp radiation in a cylindrical cavity.}
%   \label{Fig:OneLightRadiation}
% \end{figure}

% 使用National Instruments公司的NI 9213型16通道热电偶数据采集器(见\ref{Fig:OneLightRadiation}(c))对热电偶的测量结果进行采集,该数据采集器内置CJC冷端温度补偿,通过在PC端安装NI-DAQmx驱动软件实现与电脑之间的通信。 National Instruments公司同时提供了LabVIEW可编程软件,根据Equation \ref{Equ:Inverse_polynomials}编写程序,将K型热电偶的电压信号转换为温度值,并将试验结果按照合适的文件格式存储。
The NI-9213 temperature input module (see \ref{Fig:temperature_measurement_devices}(a)) was used to acquire the input signals from the thermocouples. It is a high-density thermocouple input module that is designed for higher channel count systems and it includes anti-aliasing filters, open-thermocouple detection, and cold-junction compensation for high accuracy thermocouple measurements. 
The NI-9213 temperature input module was connected to a host computer by the USB interface. Based on the LabVIEW software platform, a program was developed to converted the voltage signal of the thermocouple to the temperature value according to Equation (\ref{Equ:Inverse_polynomials}).

% 为了保证实验的精度,我们在圆柱密封腔外表面中心位置(如\ref{Fig:OneLightRadiation}(a)所示)选取了一个矩形区域进行喷涂,并将矩形区域的中心作为红外温度测量点。
% 采用Voltcraft IR 2200型红外测温仪(见\ref{Fig:OneLightRadiatiosoftware platformn}(d))测量该点的温度。
% 其中,该型号红外测温仪的温度测量范围是-50至2000$^{\circ}$C,测量精度为$\pm2^{\circ}$C。
In order to improve the accuracy of the experiment, an infrared thermometer (see \ref{Fig:temperature_measurement_devices}(b)) was also used for the temperature measurement.
The measurement range of the infrared thermometer is from -50 to 2000$^{\circ}$C, and the accuracy is $\pm2^{\circ}$C.
A rectangular region on the outer surface of the cylinderical wall was coated by the black paint, and the infrared temperature measurement point was shown in \ref{Fig:OneLightRadiation}(a).

% 试验开始前,将所有热电偶与数据采集器连接,标定每个热电偶的温度值并进行冷端补偿。
% 将热电偶采集器和红外测温数据与PC端进行同步,定义数据采集程序。
% 开启卤素灯加热一段时间后关闭电源,试验装置自然冷却至室温,记录升温与降温全过程。
% 更换不同功率的卤素灯(240W,350W和400W),获取不同功率卤素灯的辐射加热和冷却曲线。
Prior to the testing, all thermocouples were connected to the data acquisition unit. Each thermocouple was calibrated and cold junction compensation was performed.
The data acquisition unit and infrared thermometer were connected to a host computer via the USB interface.
% A LabVIEW program was developed to acquire data, synchronize time and design custom display interfaces.
On the host computer, a LabVIEW program was developed for data acquisition, time synchronization and custom display interfaces.

When the test begins, the halogen lamp was turned on for a period of time. Then it was turned off and the device was cooled to room temperature by the natural convection.
During the test, the temperature of each thermocouple was recorded.
The above test process was repeated with different power halogen lamps to obtain the heating and cooling data.
Through the experiments, the temperature-time curves of the 11 temperature measurement points (10 thermocouples and 1 infrared thermometer) were obtained.

% 通过试验我们可以得到圆柱壁面上11个不同位置测温点的温度-时间曲线。
% 下面我们将建立有限元模型,结合数值计算的结果,确定壁面的对流换热系数及表面散射率。
% 这里我们使用商业有限元软件ABAQUS的Heat transfer模块进行计算。
% 对于含有辐射传热的计算模型,两个常数需要提前定义:
% (1)Stefan-Boltzmann常数为$5.670\times10^{-8}\rm{W}/(\rm{m}^{2}\cdot \rm{K})$,
% (2)绝对零度为-273.15$^{\circ}$C。
% 我们进行的是瞬态温度场计算,因此材料的热传导系数、密度和比热容需要被定义。
% 圆柱壁面材料为SS304不锈钢,随着温度的升高,SS304不锈钢的传热系数也会增大,不同温度下的材料传热系数如Table \ref{Tab:SS304HeatTransfer}所示。
% 忽略温度对密度和比热容的影响,定义SS304不锈钢的密度为7873kg/m$^3$,比热容(Specific Heat Capacity)为502J/(kg$\cdot$K)。

Heat transfer in the test rig is governed by three effects: conduction through the metal, convection from the air, and radiative exchange between the lamp and parts of the metal surface. 
For the transient thermal computation, the thermal conductivity, density, and specific heat capacity of the material should be defined.
The material of the test rig is the stainless steel SS304 with a density of 7873kg/m$^3$ and a specific heat of 502J/$({\rm{kg}} \cdot {}^ \circ {\rm{C}})$. The density and specific heat capacity of the stainless steel SS304 are approximately considered as temperature independent, but the temperature dependence of thermal conductivity can not be neglected. The thermal conductivity of the stainless steel SS304 at different temperatures are listed in Table \ref{Tab:SS304HeatTransfer}, and it is observed that the conductivity increases with the increasing temperature. 

\begin{figure}[!htp]
  \centering
  \begin{overpic}[width=8.0cm]{filament.pdf}
    \put(84,65){\fcolorbox{white}{white}{(a)}}
  \end{overpic}
  \begin{overpic}[width=8.0cm]{FEM_model_cavity_radiation.png}
    \put(84,65){\fcolorbox{white}{white}{(b)}}
  \end{overpic}

  \begin{overpic}[width=8.0cm]{cavity_radiation_NT11_2.png}
    \put(84,65){\fcolorbox{white}{white}{(c)}}
  \end{overpic}
  \begin{overpic}[width=8.0cm]{plot_cmp_cavity_radiation_240W.pdf}
    \put(84,65){\fcolorbox{white}{white}{(d)}}
  \end{overpic}

  \begin{overpic}[width=8.0cm]{plot_cmp_cavity_radiation_350W.pdf}
    \put(84,65){\fcolorbox{white}{white}{(e)}}
  \end{overpic}
  \begin{overpic}[width=8.0cm]{plot_cmp_cavity_radiation_400W.pdf}
    \put(84,65){\fcolorbox{white}{white}{(f)}}
  \end{overpic}

  \caption{Comparison of experimental and computational results of the single lamp radiation in the cylindrical cavity. (a) Double ended halogen lamp with tube base. (b) Mesh of the FE-model. (c) Mesh of the FE-model. (d) Lamp power of 240W. (e) Lamp power of 350W. (f) Lamp power of 400W. }
  \label{Fig:OneLightRadiationSimulation}
\end{figure}

A computational heat transfer model are created in the commercial finite element software ABAQUS.
According to the symmetry of the test rig, the middle cross section of the cylinder is taken as the symmetry plane, and the FE-model is shown in \ref{Fig:OneLightRadiationSimulation}(b).
The FE-model is dimensioned in meters, and the temperature is measured in $^\circ$C, so the Stefan Boltzmann constant is taken as $5.670\times10^{-8}{\rm{W/}}({\rm{m}^{2}} \cdot {}^ \circ {\rm{C}})$ and absolute zero is set at 273.15$^\circ$C below zero.
The procedure consists of two heat transfer steps in which the thermal loading conditions are heating and cooling down.
The boundary constraints on the cylindrical cavity
Convection due to heat transfer from the air is applied at the internal and external surfaces of the cylindrical cavity.
Radiation is modeled between the internal surfaces of the cavity and the filament of the lamp using the cavity radiation method.
The emissivities of the inner and outer surfaces of the cylindrical cavity are taken as variables. 

% In the following, we would establish a finite element model and combine the results of numerical calculations to determine the convection heat transfer coefficient and surface scattering rate of the wall surface.

\begin{table}[htbp]
  \centering
  \caption{Heat conductivity of 304 Stainless Steel.}
    \begin{tabular}{lcccccccccc}
    \toprule
    Temperature [$^{\circ}$C] & 100   & 200   & 300   & 400   & 500   & 600   & 700   & 800   & 900   & 1000 \\
    \midrule
    Conductivity [${\rm{W/}}({\rm{m}} \cdot {}^ \circ {\rm{C}})$] & 15.5  & 18.1  & 18.4  & 19.1  & 19.7  & 20.3  & 20.8  & 20.8  & 20.6  & 20.7 \\
    \bottomrule
    \end{tabular}%
  \label{Tab:SS304HeatTransfer}%
\end{table}%


% 单盏卤素灯密封腔辐射试验,可以简化为圆柱面(灯丝)在密闭空间内辐射传热模型。
% 根据结构对称性,以中心高度截面作为对称面,建立有限元计算模型,如\ref{Fig:OneLightRadiationSimulation}(c)所示。
% Single-halogen lamp sealed chamber radiation test can be simplified as cylindrical surface (filament) radiation heat transfer model in a confined space.


% 以400W卤素灯为例,经过测量灯芯长度为68mm(see \ref{Fig:OneLightRadiation}(a)),灯芯色温2950K。根据Equation \ref{Equ:IntegratingOfStefanBoltzmann}我们可以计算灯芯表面辐射面积为:
% 将灯芯等效为理想圆柱面,已知圆柱面的表面积93.16${\rm{mm}}^2$和长度68mm,因此,可以计算得出圆柱面的直径为0.436mm。

Taking the 400W halogen lamp as an example, the measured filament length is about 68mm (see \ref{Fig:OneLightRadiationSimulation}(a)), and the color temperature of the filament is 2950K. According to the Equation (\ref{Equ:IntegratingOfStefanBoltzmann}), the radiating surface area of the filament can be calculated by:
\begin{equation}
A = \frac{P}{\sigma \cdot T^4} = \frac{400}{5.670\times10^{-8}\times2950^4}=9.316\times10^{-5} {\rm{m }}^2.
\end{equation}
The shape of the filament is a spiral coil. 
For simplicity, the filament can be abstracted as a cylindrical surface with the area of 93.16mm$^2$ and the length of 68mm. Therefore, the diameter of the cylinder can be calculated as 0.436mm. The mesh of the filament is shown in \ref{Fig:OneLightRadiationSimulation}(a).



% 试验过程中环境温度为23$^{\circ}$C,空气自然对流换热系数的范围是5到25W/(${\rm{m}}^2\cdot$K)。
% 使用trial and error方法,可以得到内壁面(黑色喷涂)的表面散射率为0.95,外壁面(金属表面)的表面散射率为0.68,对流换热系数5.6W/(${\rm{m}}^2\cdot$K)。
% 使用得到的表面散射率和对流换热系数,分别对240W、350W和400W功率卤素灯的辐射传热进行计算,选取外壁面中心点的温度值进行对比,如\ref{Fig:OneLightRadiationSimulation}(b-d)所示。
% 数值模拟结果、热电偶测量值和红外温度测量仪所得到的温度值在加热和冷却两个过程中都得到了很好的吻合,这说明我们使用的材料常数和传热学参数是正确的。

During the test, the ambient temperature was measured as 23$^{\circ}$C, and the natural convection heat transfer coefficient of air ranged from 5 to 25 ${\rm{W/}}({\rm{m}^{2}} \cdot {}^ \circ {\rm{C}})$.
In the FE-model, several heat transfer coefficients are unknown.
Using the trial and error method, we determined the unknown coefficients with the emissivity of the inner wall surface (black coated) of 0.95, the emissivity of the outer wall surface (metallic surface) of 0.68, the radiation efficiency of 0.94, and the convective heat transfer coefficient of 5.6${\rm{W/}}({\rm{m}^{2}} \cdot {}^ \circ {\rm{C}})$.
Using the obtained heat transfer coefficients, the radiant heat transfer of the test rig with different power halogen lamps were computed through the finite element model. 
\ref{Fig:OneLightRadiationSimulation}(c) shows the nodal temperature field of the test rig. 
The temperature at the middle section of the outer wall surface are shown in \ref{Fig:OneLightRadiationSimulation}(d),(e),(f) with the lamp power of 240W, 350W and 400W, respectively.
We observed good agreement between the experimental and computational results.
Therefore, the emissivities and convective heat transfer coefficient determined from the trial and error method are appropriate.




\section{Experiments and simulations of the radiation with elliptical cylinder reflector}
\noindent
Elliptical cylinder reflectors have two conjugate focus lines. Light from one focus line passes through the other after reflection (see \ref{Fig:EllipsoidalReflectors}).
Deep elliptical cylinder surrounding a line source collect a much higher fraction of total emitted light than a spherical mirror or conventional lens system.
The effective collection are very small, and the geometry is well suited to the spatial distribution of the output from a line lamp.
Two elliptical cylinders can almost fully enclose a light source and target to provide nearly total energy transfer.
\begin{figure}[!htp]
	\centering
	\includegraphics[width=10cm]{Ellipsoidal_reflectors.png}
	\caption{A section of an ellipsoidal reflector. }
	\label{Fig:EllipsoidalReflectors}
\end{figure}
More significant perhaps is the elliptical cylinder's effect upon imaging of extended sources.
For a pure line source exactly at one focus line of the elliptical cylinder, almost all of the energy is transferred to the other focus line.
Unfortunately, every real light source, such as a halogen tube, has some finite extent; points of the source that are not exactly at the focus line of the elliptical cylinder will be magnified and defocused at the image.

\ref{Fig:EllipsoidalReflectors} illustrates how light from point S, off the focus F1, does not reimage near F2, but is instead spread along the axis.
Rays a, b and c, shown in the top half of the ellipse, are all from F1 and pass through F2, the second focus of the ellipse. Rays A, B and C, in the bottom half of the ellipse are exact ray traces for rays from a point close to F1. They strike the ellipse at equivalent points to a, b and c, but do not pass through F2. For a small spot (image) at F2 you need a very small source.
Because of this effect elliptical cylinder are most useful when coupled with a small source and a system that requires a lot of light without concern for particularly good imaging.

%根据灯丝形状不同光源可以被简化为点光源和直线光源两种。
%点光源的聚光采用椭圆球面镜,将点光源放置于椭圆球面镜的其中一个焦点,经过反射光线会聚焦至另一个焦点。
%参考德宇航的设计,我们采用管状卤素灯作为光源。

% 理想情况下,灯丝可简化为线光源,根据椭圆镜面的性质,如果线光源位于镜面其中一个焦点处,反射后的光线将全部通过另一个焦点。
% 但是根据上一节的计算,如果我们将灯丝近似为一个圆柱面,则该圆柱面的直径约为0.5mm,光线经过镜面反射后会有一定的发散。
% 真实的镜面无法保证完全光滑,光线会产生一定的漫反射,通过试验和计算,我们需要确定镜面的反射效率。

The filament can be simplified as a line light source, due to its small diameter. As discussed above, if a line light source is located at one focus line of the elliptical cylinder mirror, the reflected light will pass through the other focus line.
However, according to the computations in the previous section, if we consider the filament as a cylindrical surface, the diameter of the cylindrical surface is about 0.5mm. Not all of the reflected light is transferred to the other focus.
Meanwhile, the real mirror is not completely smooth and a part of the light is diffusely reflected. Therefore, we have to determine the reflection efficiency through the experiments and calculations.

% 单盏卤素灯镜面反射辐射传热试验装置,如\ref{Fig:OneLightWithEllipseMirrorReflectionShow}(a)所示,包括椭圆柱面镜、受光平板、温度传感器及数据采集系统等。
% 卤素灯位于椭圆柱面镜的其中一个焦点,受光平板的向光面垂直于椭圆柱面的长轴,并通过其另一个焦点,光线经过椭圆柱面镜反射后聚焦在受光平面上,如\ref{Fig:OneLightWithEllipseMirrorReflectionShow}(b)所示。
A reflected single halogen lamp radiation heat transfer testing device was developed to analyze the energy transformation during the light reflection.
As shown in \ref{Fig:OneLightWithEllipseMirrorReflectionShow}(a), the testing
device includes an elliptical cylinder mirror, a light-receiving plate, temperature sensors, and a data acquisition system.
The halogen lamp is located at one of the focus lines of the elliptical cylinder mirror. The light-receiving plate is perpendicular to the long axis of the elliptical cylinder mirror and its surface pass through the other focus line. During the test, the light is reflected by the elliptical cylinder mirror and focused on the light-receiving plate, as show in \ref{Fig:OneLightWithEllipseMirrorReflectionShow}(b).

\begin{figure}[ht]
  \begin{minipage}[t]{0.5\linewidth} % 如果一行放2个图,用0.5,如果3个图,用0.33\
  \nonumber
    \centering
    \includegraphics[width=2.5in]{OneLightWithEllipseMirrorReflectionShow.pdf}
    \centerline{(a) Schematic diagram of the test rig.}
    \label{Fig:OneLightWithEllipseMirrorReflectionShow}
  \end{minipage}%
  \begin{minipage}[t]{0.5\linewidth}
    \centering
    \includegraphics[width=3.5in]{OneLightWithEllipseMirrorReflectionShow2.pdf}
    \centerline{(b) A section of the ellipsoidal reflector.}
    \label{Fig:OneLightWithEllipseMirrorReflectionShow2}
  \end{minipage}

  \begin{minipage}[t]{0.5\linewidth} % 如果一行放2个图,用0.5,如果3个图,用0.33\
  \nonumber
    \centering
    \includegraphics[width=3.0in]{OneLightWithEllipseMirrorReflectionShow3.pdf}
    \centerline{(c) Thermocouple locations.}
    \label{Fig:OneLightWithEllipseMirrorReflectionShow3}
  \end{minipage}%
  \begin{minipage}[t]{0.5\linewidth}
    \centering
    \includegraphics[width=3.0in]{OneLightWithEllipseMirrorReflectionShow6.jpg}
    \centerline{(d) Experimental process.}
    \label{Fig:OneLightWithEllipseMirrorReflectionShow6}
  \end{minipage}

% \begin{minipage}[t]{0.5\linewidth} % 如果一行放2个图,用0.5,如果3个图,用0.33\
% \nonumber
%   \centering
%   \includegraphics[width=3.0in]{OneLightWithEllipseMirrorReflectionShow5.jpg}
%   \centerline{(e) Experimental process.}
%   \label{Fig:OneLightWithEllipseMirrorReflectionShow5}
% \end{minipage}%
% \begin{minipage}[t]{0.5\linewidth}
%   \centering
%   \includegraphics[width=3.0in]{OneLightWithEllipseMirrorReflectionShow6.jpg}
%   \centerline{(f) Focused light.}
%   \label{Fig:OneLightWithEllipseMirrorReflectionShow6}
% \end{minipage}

  \caption{Test rig of single halogen lamp radiation with the reflect mirror.}
  \label{Fig:OneLightWithEllipseMirrorReflectionShow}
\end{figure}

%灯丝的直径、灯丝与镜面的相对位置、镜面的表面粗糙度和镜面的轴线方向都会影响受光平面上的光强度分布。
% 与上一节相同,受光平板材料为SS304不锈钢,表面采用同样的耐高温黑色涂料喷涂,因此可以认为该受光平板的表面发射率与上一节中的圆柱空腔内壁相同,为0.95。
% 选择K型热电偶测量平板表面的温度分布,热电偶的位置分布如\ref{Fig:OneLightWithEllipseMirrorReflectionShow}(c)所示,共15个。
% 热电偶的数据采集,同样选取National Instruments公司的9213型16通道热电偶数据采集器,相应的编程和数据处理过程与上一节类似,采样频率为10Hz。

The material of the light-receiving plate is the stainless steel SS304, and the plate surface was coated with the high-temperature-resistant black paint. The paint was the same as the paint used in the cylindrical cavity radiation device. Therefore, the emissivity of the light-receiving plate was consider as 0.95, which was already determined in the previous section.
There are 15 type K thermocouples welded on the light-receiving plate to measure the temperature distribution of the plate during the test.
\ref{Fig:OneLightWithEllipseMirrorReflectionShow}(c) shows the positions of the thermocouples. 
The data acquisition system was also the NI-9213 thermocouple data collector, which was introduced in the previous section. The corresponding programming and data acquisition process was similar to the cylindrical cavity radiation testing and the sampling frequency was chosen as 10Hz.

The radiative transfer between the surfaces can be computed by ABAQUS through the cavity radiation method. Geometric view factors are computed in ABAQUS between each facet of the mesh on the surfaces. These view factors quantify the effect of radiative transfer between each facet and each of the other facets in the user-defined cavity. 
\ref{Fig:one_lamp_reflection_cavity_radiation} shows the FE-model and the computational result of the one lamp reflection test. The simulated temperature distribution of the light-receiving plate is much lower than the measured temperature distribution. The reason is the cavity radiation method can not simulate the specular reflection between the surfaces. 
\begin{figure}
  \centering
  \begin{overpic}[width=8.0cm]{FEM_model_single_lamp_reflection.png}
    \put(84,65){\fcolorbox{white}{white}{(a)}}
  \end{overpic}
  \begin{overpic}[width=8.0cm]{one_lamp_reflection_1.png}
    \put(84,65){\fcolorbox{white}{white}{(b)}}
  \end{overpic}

  \caption{FE-model (a) and computational result (b) of the one lamp reflection test.}
  \label{Fig:one_lamp_reflection_cavity_radiation}
\end{figure}

% \begin{figure}
%   \centering
%   \begin{overpic}[width=8.0cm]{one_lamp_reflection_1.png}
%     \put(84,65){\fcolorbox{white}{white}{(a)}}
%   \end{overpic}
%   \begin{overpic}[width=8.0cm]{one_lamp_reflection_2.png}
%     \put(84,65){\fcolorbox{white}{white}{(b)}}
%   \end{overpic}
%   \caption{.}
%   \label{Fig:one_lamp_reflection}
% \end{figure}

\begin{figure}
  \centering
  \begin{overpic}[width=8.0cm]{tracepro_iso.png}
    \put(84,65){\fcolorbox{white}{white}{(a)}}
  \end{overpic}
  \begin{overpic}[width=8.0cm]{tracepro_yz.png}
    \put(84,65){\fcolorbox{white}{white}{(b)}}
  \end{overpic}
  \caption{Computational model in TracePro: (a) isometric view, (b) view on $x$-$z$ plane.}
  \label{Fig:tracepro}
\end{figure}

\begin{figure}
  \centering
  \begin{overpic}[width=16.0cm]{irradiance_map_D1.0mm_S0.0mm_gray.png}
  \end{overpic}
  \caption{The irradiance map of the light-receiving plate.}
  \label{Fig:irradiance_map_D1.0mm_S0.0mm_gray}
\end{figure}


\begin{figure}
  \centering
  \begin{overpic}[width=8.0cm]{FEM_model_single_lamp_reflection_mesh.png}
    \put(84,65){\fcolorbox{white}{white}{(a)}}
  \end{overpic}
  \begin{overpic}[width=8.0cm]{one_lamp_reflection_2.png}
    \put(84,65){\fcolorbox{white}{white}{(b)}}
  \end{overpic}

  \begin{overpic}[width=8.0cm]{plot_compare_horizontal_reflection.pdf}
    \put(84,65){\fcolorbox{white}{white}{(c)}}
  \end{overpic}
  \begin{overpic}[width=8.0cm]{plot_compare_vertical_reflection.pdf}
    \put(84,65){\fcolorbox{white}{white}{(d)}}
  \end{overpic}

  \caption{(a) FE model of the light-receiving plate. (b) Temperature distribution on the light-receiving plate. (c) Temperature distribution along the horizontal direction. (d) Temperature distribution along the vertical direction.}
  \label{Fig:OneLightWithReflection}
\end{figure}

In order to obtain the optical path and radiation distribution on the light-receiving plate, a commercial optical engineering software TracePro was used to simulate the process. 
TracePro is the software program for designing and analyzing optical and illumination systems.
It has been used in many projects for designing and analyzing all types of optical/illumination systems ranging from stray light suppression in telescopes and cameras to biomedical applications to LED modeling and solar collector modeling.
A simulation model are created in TracePro, as shown in \ref{Fig:tracepro}.
The simulation model consists of three parts: the reflector, filament and light-receiving plate.
The upper and lower plate are ignored, because they are coated with the black paint, most of the light is absorbed when they irradiate on the plate.
For simplicity, the spiral coil shaped filament are abstracted as a cylinder with a diameter of 1mm.
The rays are traced through the systems to find energy distributions on the light-receiving plate.

The ray tracing results show that 27.7\% of the light emitted by the halogen lamp shined on the light-receiving plate and the irradiance distribution are plotted in \ref{Fig:irradiance_map_D1.0mm_S0.0mm_gray}.
The $x$ direction is the length of the plate and the $y$ direction is the height of the plate. Comparing \ref{Fig:irradiance_map_D1.0mm_S0.0mm_gray} with \ref{Fig:OneLightWithEllipseMirrorReflectionShow}(d), we observed similar irradiance distributions between the experimental and computational results.

The transient heat transfer model was created in ABAQUS, and the irradiance distributions obtained by TracePro are used as the heat flux boundary conditions.
The light-receiving plate was divided into 128$\times$128 meshes, as shown in \ref{Fig:OneLightWithReflection}(a). Thus, the irradiance distribution obtained by TracePro was also divided into 128$\times$128 areas with their geometries and positions corresponding to the meshes one by one. For simplicity, the heat flux on each area is assumed as a constant, then the heat flux on each area can be calculated by using its average value. Using this method, we can get the approximate heat flux boundary conditions. The larger the mesh density, the more accurate the calculation results. Through the calculations with different mesh densities, it is observed that the 128$\times$128 meshes are sufficiently accurate for our experiment. 

During the test, the halogen lamp was turned on for 200s then turned off. The transient heat transfer results at the running time 200s are shown in \ref{Fig:OneLightWithReflection}(b). \ref{Fig:OneLightWithReflection}(c) and (d) illustrate the comparison of the experimental and computational temperatures along the horizontal and vertical directions, with different reflection efficiencies. We observed the best agreement between the experimental and computational results with the reflection efficiency of 0.95. Therefore, the reflection efficiency of the gold coated reflector is determined as 0.95.

\section{Extensometer}
\noindent
An extensometer is a device that is used to measure changes in the length of an object.
It is useful for stress-strain measurements and tensile tests.
There are many different types of high temperature extensometers available.
The type of extensometer is dependent on the type of testing and the type of heating system used.

There are two observation windows at both side of the developed radiation furnace as shown in \ref{Fig:EpsilonExtensometer}(a).
The distance between each observation window and the specimen outer surface is 160mm.
Due to the long distance, Epsilon Model 3448 high temperature extensometer was selected.
% to mount with the radiation furnace
It has two ceramic rods which are made of high purity alumina.
They pass through the furnace and contact the specimen.
These are available in lengths as required to fit the furnace.
The extensometer is typically used for high temperature tension, compression and fatigue testing.
Its gauge length is 10mm with $\pm20\%$ full scale range.

\begin{figure}[!htp]
	\centering
	\begin{overpic}[width=8.0cm]{ObservationWindow.png}
		\put(84,65){\fcolorbox{white}{white}{(a)}}
	\end{overpic}
	\begin{overpic}[width=8.0cm]{Epsilon3448.png}
		\put(84,65){\fcolorbox{white}{white}{(b)}}
	\end{overpic}
\caption{(a) The observation window of radiation furnace. (b) Extensometer with self-supporting on the specimen.}
\label{Fig:EpsilonExtensometer}
\end{figure}

\begin{figure}[!htp]
	\centering
	\begin{overpic}[width=8.0cm]{Calibration.png}
		\put(0,90){\fcolorbox{white}{white}{(a)}}
	\end{overpic}
	\begin{overpic}[width=8.0cm]{GainDeltaK.pdf}
		\put(0,90){\fcolorbox{white}{white}{(b)}}
	\end{overpic}
\caption{Extensometer calibrator GWB-200JA.}
\label{Fig:Calibration}
\end{figure}

% \begin{figure}[!htp]
% 	\centering
% 	\includegraphics[width=8cm]{ObservationWindow.png}
% 	\caption{The observation window of radiation furnace.}
% 	\label{Fig:ObservationWindow}
% \end{figure}

The extensometer were held on the specimen by light, flexible ceramic fiber cords as shown in \ref{Fig:EpsilonExtensometer}(b).
These make the extensometer self-supporting on the specimen.
No furnace mounting brackets are required.
The side load on the test sample is greatly reduced because of the self-supporting design and light weight of the sensor.
The combination of radiant heat shields and convection cooling fins allow this model to be used at specimen temperatures up to 1200$^{\circ}$C without cooling.
% \begin{figure}[!htp]
% 	\centering
% 	\includegraphics[width=8cm]{Epsilon3448.png}
% 	\caption{Self-supporting on the specimen.}
% 	\label{Fig:Epsilon3448}
% \end{figure}

%An extensometer is a sensor attached to a specimen that measures a dimensional change (gage length or strain) that occurs in the specimen during testing.
The extensometer works by means of precision resistance-type strain gages bonded to a metallic element to form a Wheatstone bridge circuit.
It requires special test fixtures to aid in calibration.
%An extensometer requires DC excitation, which requires either a dedicated DC conditioner or a digital universal conditioner (DUC) configured in the DC mode.
In general, calibration is effected by connecting the extensometer into the extensometer socket on the loading frame and accurately setting the strain channel output, observed on a digital readout or recorder display against a precise, known displacement of the extensometer.
To assist in this operation, the high magnification extensometer calibrator GWB-200JA is available.
%The extensometer calibrator is used either where an extensometer does not feature electrical calibration or when a manual verification of the output is desired.

The calibration fixture, illustrated at \ref{Fig:Calibration}, consists of a precision micrometer head that drives a separate moveable shaft.
The micrometer head and shaft are mounted on a frame and the extensometer is attached between the moving and stationary shafts.
%Displacement is shown via a dial scale.
%It enables the extensometer to be exercised over its mechanical displacement range and permits the accurate adjustment of an extensometer over very small distances for tensile or compressive calibration.
%The extensometer is mounted to the separate moveable shaft.
%One of the shaft is adjustable, while the other remains stationary.
As the micrometer is adjusted, the lower shaft moves and displacement is applied to the extensometer.
The elongation is read from the dial scale of the micrometer head. This can be used to either verify the output of the extensometer shown on the test system or to set zero and full scale output to calibrate it initially.


\section{Design of the radiation furnace}
\noindent
% The radiant furnace is developed specifically for the TGMF tests.
% 辐射炉的设计过程如图所示,包括辐射传热试验/模拟、CAD设计、制作加工、装配、控制系统设计等步骤。
% 之间的章节已经介绍了椭圆柱面镜的辐射传热试验与数值计算结果。
% 我们获取了卤素灯的辐射效率,镜面反射效率以及试件表面的发射率等信息。
% 本研究中使用的试件标距段长度约为35cm,因此我们选取了Double ended halogen lamps with tube base.
The design process of the radiant furnace is shown in the figure, including radiative heat transfer test/simulation, CAD design, manufacturing, assembly, and control system design.
The results of radiative heat transfer experiments and numerical calculations for the single lamp have been introduced in the above sections.
The radiation efficiency, specular reflection efficiency, and emissivity of the surface are determined.
The length of the gauge section of the specimen used in TGMF test is 30cm, and the total length of the specimen is 120cm. Consequently, the double ended halogen lamps with a length of 115cm were chosen. According to the heat conduction equation, the temperature gradient of the material is proportional to the heat flux vector. In order achieve a large temperature gradient, a large heat flux is required. The developed radiant furnace consists of 16 halogen lamps, and the powers of the lamps can be chosen as 400W or 800W for different requirements of the TGMF tests.

CATIA is a multi-platform software suite for computer-aided design (CAD), computer-aided manufacturing (CAM) and computer-aided engineering (CAE).
The parametric design, simulation assembly and shape optimization of the radiation furnace were performed by using the software CATIA.
During the test, we must ensure that the relative position of the radiant furnace and the specimen remains unchanged. An aluminum alloy frame was used to create the support system of the radiation furnace. In order to reduce the effect of vibration, the support system is fixed on the column of the testing machine. The levelness of the support system is adjustable through a screw structure. \ref{Fig:radiation_furnace_show_2} shows the CAD model of the radiation furnace in CATIA. The dimension of the radiation furnace is about 380mm$\times$380mm$\times$210mm and the dimension of the support system is 600mm$\times$800mm. The radiation furnace can match with most of the fatigue testing machines.

\ref{Fig:radiation_furnace_show} shows that the radiation furnace includes two windows for observation. One window is for strain measurement and the other window is for temperature measurement. 
The strain measurement window can match with the high temperature extensometer or digital image correlation (DIC) equipment.
And the temperature measurement window can match with the thermocouple or infrared thermometer.
A quartz glass tube is an optional accessory, as shown in \ref{Fig:radiation_furnace_show}. The quartz glass tube can change the irradiance distribution through its refraction effect. When the diameter of the specimen is small, the quartz tube can be used to further focus the light. Meanwhile, the quartz glass tube can form a sealed cavity, in order to reduce the effect of external air flow during the test. As discussed in the above section, points of the source that are not exactly at the focus line of the elliptical cylinder will be magnified and defocused at the image. In practice, the filament is not completely on the central axis of the halogen lamp due to manufacturing errors. Therefore, an adjustable lamp holder are developed to finely tune the position of the lamp, in order to make the filament coaxial with the focus line of the elliptic cylindrical mirror.

In the present study, the lamp with the power of 400W was used. Thus, the radiation furnace has a total power of 6.4kW. A power regulator was used to control the voltage of the halogen lamps. The 16 lamps are connected in parallel to the power regulator to ensure the same voltage of each lamp at the same time. Therefore, a uniform radiation intensity in the circumferential direction of the specimen can be achieved.

Typical test data for a nickel-based superalloy tubular specimen are: heating, from 23$^\circ$C to 1400$^\circ$C in about 50 seconds, cooling from 1400$^\circ$C to 100$^\circ$C in about 20 seconds, stationary, temperature gradient of 30$^\circ$C/mm.


% It is complex to create the radiation heating system directly.
% Then the radiation furnace is suggested to divided into several subsystem as follows.
% The furnace includes the following subsystems:


\begin{figure}[!htp]
	\centering
	\includegraphics[width=16cm]{radiation_furnace_show_2.pdf}
	\caption{Design scheme of the radiation furnace with its support system.}
	\label{Fig:radiation_furnace_show_2}
\end{figure}

\begin{figure}[!htp]
	\centering
	% \includegraphics[width=16cm]{DesignWithCATIA.png}
	\includegraphics[width=16cm]{radiation_furnace_show.pdf}
	\caption{Schematic diagram of the radiation furnace.}
	\label{Fig:radiation_furnace_show}
\end{figure}

\begin{figure}[!htp]
\centering\scalebox{0.65}{\includegraphics{Radiation_Furnace2.pdf}}
\caption{Schematic diagram of the control system of the radiation furnace.}
\label{Fig:Radiation_Furnace}
\end{figure}

% \begin{figure}
%   \begin{minipage}[t]{0.5\linewidth} % 如果一行放2个图,用0.5,如果3个图,用0.33\
%   \nonumber
%     \centering
%     \includegraphics[width=3.0in]{MirrorBeforeOvergild.jpg}
%     \centerline{(a) 镀金前的镜面单元体.}
%     \label{fig:side:a}
%   \end{minipage}%
%   \begin{minipage}[t]{0.5\linewidth}
%     \centering
%     \includegraphics[width=3.0in]{MirrorAfterOvergild.jpg}
%     \centerline{(b) 镀金后的镜面单元体.}
%     \label{fig:side:b}
%   \end{minipage}

%   \caption{椭圆柱面镜的加工及镀金。}
%   \label{Fig:MirrorOvergild}
% \end{figure}

%\begin{figure}[!htp]
%\centering\scalebox{0.15}{\includegraphics{MTS370WithResistanceFurnace.jpg}}
%\caption{MTS-370 轴向疲劳材料试验机与电阻式加热炉.}
%\label{Fig:MTS370WithResistanceFurnace}
%\end{figure}

%\begin{figure}[!htp]
%\centering\scalebox{0.15}{\includegraphics{MTS370WithRadiationFurnace.jpg}}
%\caption{MTS-370 轴向疲劳材料试验机与辐射加热系统.}
%\label{Fig:MTS370WithRadiationFurnace}
%\end{figure}

\section{TGMF testing system}
\noindent
The TGMF testing system for the thin-walled tubular specimen is shown in \ref{Fig:TGMF_Test_System}. The testing system includes multiple subsystems such as loading, heating,  air cooling, and water cooling. These subsystems work cooperatively to ensure the mechanical load and temperature over the specimen gage section are simultaneously varied and independently controlled. The main facilities and their performance parameters of each subsystem are shown in Table \ref{Tab:TGMF_subsystem}.

\begin{table}[htbp]
  \centering
  \caption{Main equipments of TGMF testing system.}
    \begin{tabular}{p{3cm}p{4.5cm}p{6.5cm}}
    \toprule
    Subsystem & Facility & Comments \\
    \midrule
    Loading  & Fatigue testing machine & MTS 370, $\pm$250kN \\
          & Digtal controller & FlexTest 40 \\
          & Extensometer & gauge length 12mm, range $\pm$20\% \\
          & Host computer &  TestSuite$^{\rm TM}$ Multipurpose Software\\
    \midrule
    Heating & Halogen lamp & 400W/800W \\
          & Thermocouple & K/S/R-type \\
          & Temperature input module & 8-channel \\
          & Infrared thermometer & measuring range -50 to 2000$^\circ$C \\
          & Reflector & elliptic cylinders \\
          & Temperature controller & output 0-10V \\
          & Power Regulator & AC 180V-460V, 100A \\
          & Alarm & 4 temperature monitors \\
    \midrule
    Air cooling & Air compressor & 2.4kW \\
          & Air dryer & maximum allowable pressure 16bar \\
          & Metering valve & maximum volume flow 10m$^3$/h \\
          & Air pressure gauge & measuring range 0-14bar \\
          & Cooling passage & 4 manifolds, 32 lamp holders and flexible pipes \\
    \midrule
    Water cooling & Water-cooling machine & 10kW, outlet water temperature 18$^\circ$C \\
          & Cooling passage & fixtures, cooling passages of the radiation furnace and flexible pipes \\
    \bottomrule
    \end{tabular}%
  \label{Tab:TGMF_subsystem}%
\end{table}%

\begin{figure}[!htp]
	\centering
	\includegraphics[width=16cm]{Heating_Specimen.jpg}
	\caption{Specimen is heated by radiation.}
	\label{Fig:Heating_Specimen}
\end{figure}

\begin{figure}[!htp]
	\centering
	\includegraphics[width=16cm]{TGMF_Test_System.pdf}
	\caption{Experimental apparatus of TGMF test system.}
	\label{Fig:TGMF_Test_System}
\end{figure}

The loading subsystem consists of the MTS model 370 hydraulic servo fatigue testing systems, the digital controller, extensometer, and the host computer. The fatigue testing systems can provide a tension/compression mechanical load of ±250kN. The mechanical load is transmitted to specimen through a fixture. 
In order to meet the test requirements, the fixture for the thin-walled tubular specimen was designed as shown in FIG.
The cooling air can enter the internal of the tubular specimen through the fixture.
In order to reduce the additional bending moments and the bending stress in specimen, the fixture must provide a sufficiently high degree of coaxiality. Meanwhile, the frame of the MTS model 370 testing systems provides an alignment device, it can effectively reduce the additional bending moments in the specimen during the test.
When the radiation furnace heats the specimen, it inevitably heats the fixture. Therefore, the fixture was made of a directionally solidified nickel base superalloy to ensure the strength of the fixture during the whole test process. A thread paste was used to avoid the sintering between the fixture and specimen.
The extensometer is Epsilon-3448 with a gauge length of 10mm and a measuring range of $\pm$20\%, as introduced in the above section. The PID gains of the extensometer have to be carefully tuned for different materials.

The heating subsystem is used to apply a temperature load to the test specimen. The equipments include the radiation furnace, power supply, temperature monitors, temperature controller and so on. The heating subsystem requires sufficient heating power and the ability of dynamically adjust the heating power according to a given temperature.
The temperature of the specimen was measured by the thermocouples. In practice, the choice of the diameter of the thermocouple wire is very important. A thick thermocouple wire will cause a delay in the temperature measurement. It leads to a temperature deviation between the thermocouple and the specimen, and affects the response of the feedback control system. In order to improve the accuracy of temperature measurement and increase the response speed of thermocouples, the diameter of the thermocouple wire should be as smaller as possible. Therefore, the thermocouple wires with the diameter of 0.25mm were used.
However, K-type thermocouples will suffer from oxidation failure at high temperature over a long period of time. The smaller the diameter of the thermocouple, the shorter the oxidation failure time at high temperatures. Therefore, the K-type thermocouples with the diameter of 0.25mm should be used in the TGMF testing below 800$^\circ$C and the R and S-type thermocouples were used in the TGMF testing above 800$^\circ$C.

The temperature controller was used to control the temperature of the specimen.
The controller accepted the temperature sensor (thermocouple or infrared thermometer) as input and compare the actual temperature to the desired control temperature. It will then provide an output to the power regulator to control the luminance of the lamps. 
The temperature controller of the radiation furnace is a proportional–integral–derivative (PID) controller. The values of proportional, integral, and derivative gains of the PID controller can be determined by the PID tuning process.
However, the emissivity of the specimen and the temperature range will influence the PID gains of the temperature controller. Therefore, for the specimen with and without the TBC, the PID gains should be tuned respectively.

The air cooling subsystem consists of two air compressors, an air dryer, metering valves, air pressure gauges and cooling passages. 
The two air compressor were used to provide the cooling air of the specimen and the lamp holders, separately.
Each air compressor has a power of 2.4kW and provides compressed air of 0.4-0.8MPa. The compressed air with a pressure of 0.7MPa is filtered through the air dryer and output to the metering valve. The air dryer is used for removing water vapor from the compressed air.
During the TGMF testing, the pressure and the volume flow of the compressed air have to be kept as constants. The metering valve was used to control the volume flow of the compressed air.
The air cooling subsystem also provides compressed air for the lamp holders.
The duration of the TGMF test is usually long and the lamp holders have to be cooled to ensure that they can work properly.

The water cooling subsystem consists of a 10kW industrial water-cooling machine, valves, and corresponding pipes. 
The water was cooled by the machine and circulated through the pipes to cool the test equipment. 
The equipments of the TGMF testing system need to be cooled, such as the fixture system and the radiation furnace.



% \begin{itemize}
% \item \textbf{Quartz lamp alignment device} includes 16 halogen, its power supply circuit and position adjusting device.
% \item \textbf{Temperature control system} includes a controller, .
% \item \textbf{Switch and circuit} includes 16 halogen, its power supply circuit and position adjusting device.

% \item \textbf{Quartz lamp radiation heating system} includes 16 halogen, its power supply circuit and position adjusting device.
% \item \textbf{Air cooling system for the halogen lamps} includes an air compressor, breathable lamp holders and four air manifolds.
% \item \textbf{Air Cooling system for inner and outer surface of the specimen } is to realize the aim of fast cooling. Two hollow extend clamp are manufactured with cooling air pipe connected to each side.

% \item \textbf{Water cooling system for the mirrors.}
% \item \textbf{Temperature measurement system.}
% \item \textbf{Strain measurement system.}
% \item \textbf{Temperature control system.}
% \end{itemize}

% \section{Forced convection in turbulent pipe flow}

% \begin{figure}[!htp]
% \centering\scalebox{0.6}{\includegraphics{inner_cooling.pdf}}
% \caption{Schematic of inner cooling air system.}
% \label{Fig:}
% \end{figure}

% Because of the small inner diameter 6.5mm of the specimen, it is difficult to measure the temperature of the specimen inner surface during the tests.
% A variety of measurement options were tried to measure the inner surface temperature.
% It was found that the internal cooling air has a significant effect on the temperature measurement of the inner surface of the specimen.
% %内部冷却空气对试件内壁的温度测量有很大的影响。
% The measured temperature is not the temperature of the inner surface, but the average temperature of the wall and the cooling air flow.
% %测量得到的温度不是壁面的温度,而是壁面和冷却气流的平均温度。
% Therefore we tried to calculate the temperature distribution of the specimen inner surface using the FEM method.

% %The internal diameter of the specimen is 6.5 mm, and .
% %试件的内部直径为6.5mm,。

% The inner surface of the hollow specimen can be considers as a pipe.
% %空心试件可以简化为圆管.
% According to the formula of convection in turbulent pipe flow, the convective heat transfer coefficient of the inner surface of the specimen can be calculated.
% %根据圆管内部的强制对流换热公式,我们计算试件内表面的对流换热系数。

% Nusselt number (Nu) is a dimensionless number, defined as the ratio of convective to conductive heat transfer across (normal to) the boundary.
% The Nusselt number is given as:
% \[{\rm{N}}{{\rm{u}}_L} = \frac{{hL}}{k},\]
% where $h$ is the convective heat transfer coefficient of the fluid, $L$ is the characteristic length, $k$ is the thermal conductivity of the fluid.

% Gnielinski's correlation for turbulent flow in tubes is given as:
% \[{\rm{N}}{{\rm{u}}_D} = \frac{{\left( {f/8} \right)\left( {{\rm{R}}{{\rm{e}}_D} - 1000} \right){\rm{Pr}}}}{{1 + 12.7{{(f/8)}^{1/2}}\left( {{\rm{P}}{{\rm{r}}^{2/3}} - 1} \right)}},\]
% where $f$ is the Darcy friction factor that can either be obtained from the Moody chart or for smooth tubes from correlation developed by Petukhov:
% \[f = {\left( {0.79\ln \left( {{\rm{R}}{{\rm{e}}_D}} \right) - 1.64} \right)^{ - 2}},\]
% with
% \[0.5 \le {\rm{Pr}} \le 2000,\]
% \[3000 \le {\rm{R}}{{\rm{e}}_D} \le 5 \times {10^6},\]

% The Prandtl number (Pr) is a dimensionless number, defined as the ratio of momentum diffusivity to thermal diffusivity.
% The Prandtl number is given as:
% \[{\rm{Pr}} = \frac{{{c_p}\mu }}{k},\]
% where
% $c_{p}$ is specific heat, $\mu$ is dynamic viscosity and $k$ is thermal conductivity.

% The Reynolds number (Re) is a dimensionless number, defined as the ratio of inertial forces to viscous forces within a fluid.
% The Reynolds number is given as:
% \[{\rm{Re}} = \frac{{\rho uL}}{\mu },\]
% where
% $\rho$ is density of the fluid, $u$ is the velocity of the fluid with respect to the object, $\mu$ is the dynamic viscosity of the fluid and $L$ is a characteristic dimension.

% It is noted that the dynamic viscosity $\mu$ of an ideal gas is a function of the temperature. It can be derived as Sutherland's formula:
% \[\mu  = {\mu _0}\frac{{{T_0} + C}}{{T + C}}{\left( {\frac{T}{{{T_0}}}} \right)^{\frac{3}{2}}},\]
% where $\mu$ is dynamic viscosity at input temperature $T$,
% $\mu_0$ is reference viscosity at reference temperature $T_0$,
% $C$ is Sutherland's constant dependent on the gaseous material.
% Table \ref{tab:SutherlandConstant} shows the Sutherland's constant and reference values of the air.
% \begin{table}[htbp]
%   \centering
%   \caption{Sutherland's constant and reference values of the air.}
%     \begin{tabular}{p{2cm}p{2cm}p{2cm}p{3cm}}
%     \toprule
%     Gas   & $C$(K) & $T_0$(K) & $\mu_0$($\rm{Pa\cdot s}$) \\
%     \midrule
%     air   & 120   & 291.15 & $1.827\times 10^{-5}$ \\
%     \bottomrule
%     \end{tabular}%
%   \label{tab:SutherlandConstant}%
% \end{table}%

% The density of dry air $\rho$ can be calculated using the ideal gas law, expressed as a function of temperature and pressure:
% \begin{equation}
% \rho  = \frac{p}{{RT}},
% \label{Equ:AirDensity}
% \end{equation}
% where
% $p$ is absolute pressure,
% $T$ is absolute temperature,
% $R$ is specific gas constant.

% According to the measurement results of the sensors, the average pressure and temperature of the cooling air are 6.5 bar and 288.15K, respectively.

% \begin{table}[htbp]
%   \begin{threeparttable}
%   \centering
%   \caption{Heat convection coefficients of the inner surface of the specimen.}
%     \begin{tabular}{llrrrr}
%     \toprule
%     $T$   & ${\rm{K}}$ & 288.15  & 288.15  & 288.15  & 288.15  \\
%     $p$   & ${\rm{bar}}$ & 6.5   & 6.5   & 6.5   & 6.5  \\
%     $\rho  = \frac{p}{{RT}}$ \tnote{*1} & ${\rm{kg/}}{{\rm{m}}^{\rm{3}}}$ & 7.724 & 7.724 & 7.724 & 7.724 \\
%     $\mu  = {\mu _0}\frac{{{T_0} + C}}{{T + C}}{\left( {\frac{T}{{{T_0}}}} \right)^{\frac{3}{2}}}$ \tnote{*2} & -     & 1.81E-05 & 1.81E-05 & 1.81E-05 & 1.81E-05 \\
%     ${\dot V}$ & ${\rm{l/min}}$ & 10    & 20    & 30    & 40  \\
%     $u = \frac{{4\dot V}}{{\pi {D^2}}}$ \tnote{*3} & ${\rm{m/}}{{\rm{s}}^2}$ & 5.02  & 10.05  & 15.07  & 20.09  \\
%     ${\rm{Re}} = \frac{{\rho uL}}{\mu }$ \tnote{*4} & -     & 13906  & 27812  & 41718  & 55624  \\
%     ${\rm{Pr}} = \frac{{{c_p}\mu }}{k}$ \tnote{*5} & -     & 0.63  & 0.63  & 0.63  & 0.63  \\
%     $f = {\left( {0.79\ln \left( {{\rm{R}}{{\rm{e}}_D}} \right) - 1.64} \right)^{ - 2}}$ & -     & 0.0288  & 0.0241  & 0.0219  & 0.0205  \\
%     ${\rm{N}}{{\rm{u}}_D} = \frac{{\left( {f/8} \right)\left( {{\rm{R}}{{\rm{e}}_D} - 1000} \right){\rm{Pr}}}}{{1 + 12.7{{(f/8)}^{1/2}}\left( {{\rm{P}}{{\rm{r}}^{2/3}} - 1} \right)}}$ & -     & 36.76  & 62.61  & 85.39  & 106.50  \\
%     $h = \frac{k}{{D{\rm{N}}{{\rm{u}}_D}}}$ & ${\rm{W/(}}{{\rm{m}}^{\rm{2}}} \cdot {\rm{K)}}$ & 146.03  & 248.74  & 339.24  & 423.10  \\
%     \bottomrule
%     \end{tabular}%
%     \begin{tablenotes}
%     \item[*1] $R = 287.058 {\rm{J/(kg}} \cdot {\rm{K}})$ for dry air.
%     \item[*2] Sutherland's constant and reference values of the air (see Table \ref{tab:SutherlandConstant}).
%     \item[*3] $D=0.65$mm.
%     \item[*4] $L=D$.
%     \item[*5] $c_{p}=903.3 {\rm{J/(kg}} \cdot {\rm{K}})$ and $k=0.0258 {\rm{W/(m}} \cdot {\rm{K}})$ at $288.15{\rm{K}}$.
%     \end{tablenotes}
%     \end{threeparttable}
%   \label{tab:addlabel}%
% \end{table}%


\section{Process of TGMF testing}
\noindent
Recently, there are no relevant standards for TGMF testing. Because of the test processes of the TGMF and TMF are very similar, according to the testing process of TMF, the concepts in TGMF testing are defined as same as them in the TMF testing, such as:
\begin{itemize}
  \item {Thermal strain}, $\varepsilon_{\rm{th}}$,
  \item {Mechanical strain}, $\varepsilon_{\rm{mech}}$,
  \item {Total strain}, $\varepsilon_{\rm{tot}}=\varepsilon_{\rm{mech}}+\varepsilon_{\rm{th}}$,
  \item {Loading ratio}, $R_{\varepsilon}=\varepsilon_{\rm{mech,min}}/\varepsilon_{\rm{mech,max}}$.
\end{itemize}

According to the standards discussed above, 

The implemented steps of the strain-controlled TGMF testing is shown in \ref{Fig:tgmf_code}, in the form of a flow diagram. The flow diagram similar to the flow diagram of TMF (see \ref{Fig:tmf_code}). Comparing \ref{Fig:tgmf_code} with \ref{Fig:tmf_code}, based on the TMF test, the measurement and verification of the internal cooling air are added into the flow diagram, as well as the calibration of the radiation furnace.

\begin{figure}[!htp]
	\centering
	\includegraphics[width=16.0cm]{temperature_monitors_tgmf.pdf}
	\caption{Locations of the temperature monitors.}
	\label{Fig:temperature_monitors_tgmf}
\end{figure}

\begin{figure}[!htp]
	\centering
	\begin{overpic}[width=8.0cm]{plot_thermal_stability_tgmf_K.pdf}
		\put(84,65){\fcolorbox{white}{white}{(a)}}
	\end{overpic}
	\begin{overpic}[width=8.0cm]{plot_thermal_stability_tgmf_deviation_K.pdf}
		\put(84,65){\fcolorbox{white}{white}{(b)}}
	\end{overpic}
	\caption{Varying temperatures in a tubular specimen with the gauge length of 15mm, heated by the radiation furnace. (a) Temperature variations of three thermal elements during a loading cycle. (b) Temperature deviations in the axial direction.}
	\label{Fig:thermal_stability_TGMF}
\end{figure}

The light reflected off the upper and lower gold-coated walls leads to the highest intensity of radiation in the middle section of the specimen.
This will cause a temperature gradient along the axial direction of the specimen and the highest temperature will occur at the middle section of the specimen during the TGMF test.
Unlike the induction heating, the axial temperature gradient of the specimen is difficult to adjust by modifying the radiation furnace.
Therefore, the influence of the axial temperature gradient have to be considered in the computation.

In order to obtain the temperature distribution of the specimen during the TGMF testing, temperatures at the upper, center and lower positions in the gauge section of the specimen were measured by the thermocouples wrapped around the specimen. \ref{Fig:temperature_monitors_tgmf} shows the locations of the thermocouples. In the study, the temperature range of the TGMF testing is from 300$^\circ$C to 650$^\circ$C. The cycle period was predetermined as 240s with a triangular shape temperature cycle, thus the heating and cooling rates were about 2.2$^\circ$C/s.
\ref{Fig:thermal_stability_TGMF}(a) shows the dynamic temperature measurement during a stable TGMF cycle and \ref{Fig:thermal_stability_TGMF}(b) shows the temperature deviations between each thermocouple and the temperature command.
It is observed that the maximum axial temperature gradients during the TMF test is less than $\pm30$$^\circ$C.

% Therefore, we decided to develop and build a radiant oven for our purposes. In general, a radiant oven is a furnace in which radiation is used to heat. The simplest case of a radiant furnace is shown in Figure 3.1.
% 通常意义上讲,热疲劳试验时被测试件的试验环境是高温且温度不恒定,甚至温度是急剧反复变化的,这样被测试件会由于温度的梯度循环引起热应力循环(或热应变循环),从而产生疲劳破坏现象。热疲劳试验中,试件的外形尺寸和试验温度是影响试件疲劳特性的主要因素。目前,常见的热疲劳试验箱的高温试验温度范围约为200~500℃,具备温度梯度循环变化能力,有些超高温疲劳试验箱的温度可达几千摄氏度。传统的疲劳分析往往将热载荷转化为机械载荷加以分析,但这种方法无法考察温度和机械载荷叠加交互作用的影响,直至20世纪80年代后期才出现了热/机械疲劳试验系统。热/机械疲劳试验系统将热疲劳和机械疲劳耦合在一起,根据试件材料不同的应用环境,模拟试件实际的工作环境,更好地研究材料在热机疲劳下的损伤行为,从而得出更为准确的疲劳特性。在热/机械疲劳试验中,加载应力的大小和频率会直接影响到试件的断裂寿命,通常在试验频率范围内,疲劳寿命随着试验频率的降低而减少,高频时加载循环快,更容易出现疲劳损伤;低频时加载周期慢,除疲劳损伤外还容易出现蠕变损伤[11]。所以在疲劳试验时应根据需要选择合适的加载工况。
% Generally, test environment of the specimen is high temperature but not constant in the thermal fatigue test. Even the temperature changes abruptly and repeatedly. So that the specimen would cause thermal stress cycle or thermal strain cycle due to gradient circulation of temperature, resulting in fatigue damage. In the thermal fatigue test, the dimensions of the specimen and the test temperature are the main factors affecting the fatigue characteristics of the specimen. At present, the common thermal fatigue test chamber has a high temperature range from 200°C to 500°C. It has a capability of temperature gradient cyclic change. Some ultra-high temperature fatigue test chambers can reach several thousand degrees Celsius. The thermal load was often translated into mechanical load in traditional fatigue analysis, but this method cannot examine the influence of the interaction between temperature and mechanical load until the late 1980s when the thermal/mechanical fatigue test system appeared. The thermal/mechanical fatigue test system couples thermal fatigue and mechanical fatigue, and simulates the actual working environment of the specimen according to the different application environments of the specimen material. Having a better study on the damage behavior of the material under thermal fatigue lead to more accurate fatigue characteristics. In the thermal/mechanical fatigue test, the magnitude and frequency of the loading stress will directly affect the fracture life of the specimen. Usually within the test frequency range, the fatigue life decreases with the decrease of the test frequency. The loading cycle becomes faster at high frequencies, which is prone to fatigue damage. But at low frequencies loading cycle becomes slow, which is prone to creep damage in except fatigue damage[11]. Therefore, in the fatigue test, suitable loading conditions should be selected according to the requirements.

% \begin{table}[htbp]
%   \centering
%   \caption{Add caption}
%     \begin{tabular}{p{4cm}p{10cm}}
%     \toprule
%     Item  & Requirement \\
%     \midrule
%     Force transducer & The force transducer shall comply with the specifications in Practices E4 and E467. \\
%     Extensometers & The extensometers should qualify as Class B-2 or better in accordance with Practice E83. \\
%     Heating device & Specimen heating can be accomplished by various techniques including induction, direct resistance, radiant, or forced air heating. \\
%     Temperature measurement & The specimen temperature shall be measured using thermocouples in contact with the specimen surface in conjunction with an appropriate temperature indicating device or non-contacting sensors that are adjusted for emisivity changes by comparison to a reference such as thermocouples. \\
%     Temperature command & Through out the duration of the test, the temperatures indicated by the control sensor; for example, thermocouples shall not vary by more than 6 2 K from the corresponding values of the initial temperature cycle. \\
%     Temperature gradient & The maximum allowable axial temperature gradient is $\pm1\% T_{max}$K or $\pm3$K  \\
%     \bottomrule
%     \end{tabular}%
%   \label{tab:addlabel}%
% \end{table}%

% 热梯度机械疲劳试验由机械载荷循环,温度循环和内部冷却空气组成。机械循环由MTS液压伺服试验机实现,试件的轴对称安装由对中系统来保证。温度循环由聚光辐射加热系统来实现,卤素灯管发出的光线通过椭圆柱状镜面反射到试件的工作段。温度循环冷却由自然对流方式来实现,它较之强风冷却系统引起的温度梯度要小。图1给出了试件安装在试验机夹具上的示意图。高温应变测量系统和热电偶测温系统组成两个闭合回路,通过计算机控制软件,自动测量与控制应变和温度循环的幅度、频率及相位关系。应变的测量和控制是由高精度高温应变引伸计来实现的,为了避免高温的影响,引伸计主体安装在加热炉外侧,引伸计上的两根头部为楔形的陶瓷探棒穿过炉体,夹持在试件上。要想测量温度变化的每一个瞬态值并加以控制,必须采用高灵敏度的热电偶探头。我们选用了直径仅为0.25mm的Cr-Al合金快速响应热电偶,通过缠绕的方式将热电偶固定在试件外表面,热电偶两端点由弹簧构成稳定的拉力,在试件承受拉压载荷,其直径不断变化的情况下,热电偶工作段仍能够压紧在试件上,得到试件表面的真实温度。应变循环和温度循环的叠加构成相同频率的“同相位”和“反相位”两种形式,在同相位实验中,最大拉应变对应于最高温度;而在反相位实验中,最大压应变对应于最高温度。为了保证机械应变循环和温度循环频率相位上的精度,在控制程序中编入了时钟校对系统,以消除频率积累误差。所采用的温度循环范围有两种:250~500℃和250~650℃。所有试验的总应变控制范围(机械应变与温度产生的应变之和)均为±0。8\%,应变变化率为0。0002/秒。为了消除试件内壁由于温度迟滞产生的温度梯度效应,在最大和最小应变处有10秒的保持时间。图2给出了计算机控制的温度循环和机械循环两个闭路系统的示意图。
% The thermal gradient mechanical fatigue test consists of mechanical loading cycle, temperature cycle, and internal cooling air. The mechanical load is supplied by the MTS hydraulic servo testing machine. The axisymmetric installation of the specimen is ensured by the centering system. The temperature cycle is achieved by a concentrating radiant heating system. The light emitted by the halogen lamp is reflected by the elliptical cylinder mirror to the working section of the specimen. The temperature cycle cooling is achieved by the natural convection method, which has a lower temperature gradient than the strong wind cooling system. Figure 1 shows a schematic view of the specimen mounted on the fixture of the testing machine. The high-temperature strain measurement system and the thermocouple temperature measurement system form two closed loops. Through the computer control software, the amplitude, frequency, and phase relationship of the strain and temperature cycle are automatically measured and controlled. Strain measurement and control are achieved by high-precision high-temperature strain extensometers. To avoid the effect of high temperature, the extensometer body is mounted on the outside of the furnace. Two wedge-shaped ceramic probes on the extensometer penetrate the furnace body, and they are clamped tn the specimen. To measure and control every transient temperature change, a highly sensitive thermocouple probe must be used. We chose a Cr-Al alloy rapid response thermocouple with a diameter of only 0.25mm. The thermocouple was fixed on the outer surface of the specimen by winding. The thermocouple was stabilized at both ends by a spring. when the tensile load is applied to the specimen, and the diameter of the thermocouple is constantly changing, the working section of the thermocouple can still be pressed against the specimen to obtain the true temperature of the surface of the specimen. The superposition of the strain cycle and the temperature cycle constitutes two forms of the "in-phase" and "anti-phase" with the same frequency. In the in-phase experiment, the maximum tensile strain corresponds to the highest temperature, but in the anti-phase experiment, the maximum compressive strain corresponds to the highest temperature. In order to ensure the accuracy of the frequency phase of the mechanical strain cycle and the temperature cycle, a clock calibration system was incorporated in the control program to eliminate the frequency accumulation error. There are two types of temperature cycling used: 250-500°C and 250-650°C. The total strain control range (the sum of mechanical strain and temperature-induced strain) of all tests was ±0.8\%, and the rate of strain change was 0.002/s. In order to eliminate the effect of temperature gradients on the inner wall of the specimen due to temperature hysteresis, there is a holding time of 10 seconds at the maximum and minimum strains. Figure 2 shows a schematic diagram of two closed circuit systems for computer-controlled temperature cycling and mechanical cycling.

\begin{figure}[!htp]
\centering{\includegraphics[height=22.0cm]{tgmf_code.pdf}}
\caption{The experimental procedures of thermal gradient mechanical fatigue test.}
\label{Fig:tgmf_code}
\end{figure}


%temperature $T=15+273.15=288.15\rm{K}$
%pressure $p=6.5\rm{bar}$
%density $\rho_0= 1.293\rm{kg/m^3}$
%The average pressure of inner cooling air is 6.5 bar. With Equation \ref{Equ:AirDensity}, we obtain the air density at 6.5 bar is 7.858 $\rm{kg/m^3}$.
%inner diameter of specimen $d_{in} = 0.0065\rm{mm}$
%volume flow ${\dot V} = 25\rm{l/min}$