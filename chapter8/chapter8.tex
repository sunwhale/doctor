\chapter{Experimental apparatus and procedure of TGMF}
% 飞机发动机推力的提高很大程度上依赖于涡轮前总温的提高。
% 对于高温所带来的一系列问题,解决的办法主要有以下三个:
% (1)提高材料的耐热性,发展高性能耐热合金,制造单晶叶片;
% (2)采用热障涂层,对基底材料起到隔热作用,降低基底温度;
% (3)采用先进的冷却技术,以少量的冷却空气获得更好的降温效果。
% 高温合金及单晶材料耐热性的提高远远无法满足目前的温度设计需求,即使采用陶瓷基复合材料等耐高温材料,也不能完全取消冷却,先进的冷却可使高温部件承受更高的工作温度,使发动机寿命更长、可靠性更高。
The improvement of aircraft engine thrust is largely dependent on the increment of the turbine inlet temperature.
For the high temperature caused by a series of problems, the solution is mainly the following three:
(1) improving heat resistance of materials, developing high-performance superalloy and manufacturing single crystal blades;
(2) the thermal barrier coating is used to insulate the basement material and reduce the substrate temperature;
(3) the use of advanced cooling technology, with a small amount of cooling air to achieve better cooling effect.
High temperature alloy and single crystal materials to improve the heat resistance cannot meet the current temperature design needs, even if the use of ceramic matrix composites, such as high-temperature-resistant materials, cannot completely eliminate cooling, advanced cooling can make high-temperature components to withstand higher operating temperature, so that the engine life longer, more reliable.

% 发动机涡轮叶片主要采用气膜冷却和内部流冷却,轮盘通常采用内部二次流冷却。
% 服役过程中,涡轮叶片不仅受到较大的交变载荷,而且在叶片表面和内部分别受到高温高压燃气的冲击和冷却气体的作用,这样涡轮叶片就遭受载荷和温度同时变化带来的热机械疲劳损伤。
% 此外,为了增强发动机冷却效果,提高发动机效率,先进的航空发动机和燃气轮机热端涡轮叶片多为薄壁多孔结构。
% 因此,我们设计了薄壁圆管试件来模拟零件的冷却结构,同时试件外壁涂覆有热障涂层。
% 在内部冷却气体作用下,试件内表面与外表面之间会产生很大的温度梯度,
% 实验过程中,采用的常规感应加热设备只会对内部金属层加热,使得内部金属层温度高于外部陶瓷层温度,这不符合热障涂层构件实际工作状态下的温度分布。
% 实际发动机启动阶段升温过程只需要几秒钟的时间,而且降温也相当迅速,这些都对涂覆热障涂层的热梯度机械疲劳试验设备提出了更高的要求,同时也制约了这方面的研究。
% 由于热梯度机械疲劳是试验室中最接近涡轮叶片服役状态的模拟试验,因而这方面的研究对于理解涂覆热障涂层的叶片损伤机理具有重要意义。
The turbine blades are mainly cooled by air film cooling and internal flow cooling, and the turbine disk is usually cooled by the secondary internal flow.
In the course of service, the turbine blades are not only subjected to large alternating loads, but also to the surface and interior of the blades under the influence of high temperature and high-pressure gas and the cooling air, so the turbine blades suffer from the thermal mechanical fatigue damage caused by the change of load and temperature.
In addition, in order to enhance the engine cooling effect and improve the engine efficiency, the advanced aero-engine and gas turbine blades are most of the thin-walled porous structures.
Therefore, we have designed a thin-walled tube specimen to simulate the cooling structure of the parts, while the outer wall of the specimen is coated with a thermal barrier coating.
Under the action of the internal cooling air, a large temperature gradient between the inner surface of the specimen and the outer surface is produced.
In the experiment, the conventional induction heating equipment only heats the inner metal layer, which makes the inner metal layer temperature higher than the external ceramic layer temperature, which does not conform to the temperature distribution under the actual working state of the thermal barrier coating member.
The actual engine start-up phase heating process only takes a few seconds, and the cooling is also very rapid, these are coating thermal barrier coatings on the heat gradient mechanical fatigue test equipment has put forward higher requirements, but also restricted the research.
Because the thermal gradient mechanical fatigue is the closest to the turbine blades in the laboratory, the research is important to understand the mechanism of the blade damage of the coated thermal barrier coating.

% 同时,温度梯度会导致零件承受多轴载荷。
% 对于内部冷却的零件,温度梯度会产生额外的应力,在热表面上表现为多轴压缩载荷,而在冷却表面上表现为多轴拉伸载荷。
% 常规的热机械疲劳试验无法模拟这些应力条件并达到合适的温度分布,因此我们的试验系统是为热梯度机械疲劳(TGMF)测试而设计开发的。
% 该系统可以在空心试件表上实现受控的温度梯度循环,同时施加机械载荷。
% 通过聚光辐射的方法加热试件外表面,同时内表面通过压缩空气冷却来实现温度梯度。
% 该试验系统需要实现较高的加热和冷却速度,因此加热系统的功率和冷却气体的流量需要进行详细设计。
At the same time, the temperature gradient will cause the parts to withstand multiaxial loads.
For internally cooled parts, the temperature gradient produces additional stress, which is expressed as a multiaxial compressive load on the hotter surface, while a multiaxial tensile load is displayed on the cooler surface.
Conventional thermo-mechanical fatigue tests are not able to simulate these stress conditions and achieve a temperature gradient in the radial direction, so our test system is designed for thermal gradient mechanical fatigue (TGMF) testing.
The system can realize the controlled thermal cycle with temperature gradient on the hollow specimen and apply the mechanical load at the same time.
The external surface of the specimen is heated by the method of concentrating radiation, while the inner surface is cooled by compressed air to achieve the temperature gradient.
The test system needs to achieve high heating and cooling speed, so the power of heating system and the volume flow of the cooling air need to be designed in detail.


%Thermal gradients cause multiaxial loads in cooled components.
%In internal cooling, for example of rotors in the first stage of a jet engine, the thermal gradient induced stresses can not be relaxed by macroscopic deformations.
%The stresses occurring at the component lead to multiaxial pressure loads on the hot surface and to multiaxial tensile loads on the cooled surface.
%Since conventional thermomechanical tests do not simulate these stress conditions and achieve a homogeneous temperature distribution, our test system was designed and developed for the Thermal Gradient Mechanical Fatigue (TGMF) tests.
%The system allows cyclic and simultaneously thermal and mechanical stress with controlled temperature gradients on the wall of hollow test specimens.
%The temperature gradient is achieved by heating the outer surface with a furnace which emits concentrated radiation and the inner surface is simultaneously cooled with compressed air.
%
%These realistic tests have the advantage that they can transfer data from laboratory tests to the operating conditions.
%In addition, the heating and cooling rates achieved in the TGMF test apparatus allow very short test cycles so that the fatigue load of an entire flight can be applied to a test body within three to five minutes.

\section{Design and development of the radiation furnace for TGMF tests}


% http://www.dlr.de/wf/en/desktopdefault.aspx/tabid-2192/3209_read-5876/

Thermal gradients cause multiaxial loads in cooled components. With internal cooling, for example rotors in the first stage of a jet engine, the thermally induced stresses can not relax by macroscopic deformations. The stresses occurring on the component lead to multiaxial pressure loads on the hot surface and to multiaxial tensile loads on the cooled surface.

Since it is not possible to simulate these load conditions and to achieve a homogeneous temperature distribution in conventional thermomechanical tests, the Institute for Materials Research of the German Aerospace Center (DLR) has developed two test rigs for mechanical fatigue by thermal gradients (TGMF). These systems allow a cyclic and at the same time thermal and mechanical load with controlled temperature gradients on the wall of hollow test specimens. The temperature gradient is achieved by heating the outer surface with an oven that emits concentrated radiation and simultaneously cooling the inner surface with compressed air.

A stationary temperature gradient is reached, usually after 20 to 40 seconds. Due to the high heat flux generated by the radiant furnace, heating rates can be achieved that correspond to those achieved in real, turbine jet turbine blades., Realistic cooling rates can be achieved by forced air cooling on the outside surface. During the cooling sequence, the test cycle, the specimen is cooled via cooling holes in the shutters.

These realistic tests have the advantage that they can be used to transfer data from laboratory tests to the conditions of use. In addition, the heating and cooling rates achieved in the TGMF test apparatus allow very short test cycles so that the fatigue load of one entire flight can be applied to one specimen within three to five minutes.

The radiant heater has a power of 16 kW. Typical test data for a nickel-based superalloy hollow cylindrical specimen are: heating, from 100 ° C to 1000 ° C in about 20 seconds, cooling from 1000 ° C to 100 ° C in about 15 seconds, stationary, temperature gradient of 100 ° C per mm., The maximum mechanical load is 25 kN or 60 kN.

Therefore, we decided to develop and build a radiant oven for our purposes. In general, a radiant oven is a furnace in which radiation is used to heat. The simplest case of a radiant furnace is shown in Figure 3.1.

% Table generated by Excel2LaTeX from sheet 'Sheet1'
\begin{table}[htbp]
  \centering
  \caption{Add caption}
    \begin{tabular}{p{4cm}p{10cm}}
    \toprule
    Item  & Requirement \\
    \midrule
    Force transducer & The force transducer shall comply with the specifications in Practices E4 and E467. \\
    Extensometers & The extensometers should qualify as Class B-2 or better in accordance with Practice E83. \\
    Heating device & Specimen heating can be accomplished by various techniques including induction, direct resistance, radiant, or forced air heating. \\
    Temperature measurement & The specimen temperature shall be measured using thermocouples in contact with the specimen surface in conjunction with an appropriate temperature indicating device or non-contacting sensors that are adjusted for emisivity changes by comparison to a reference such as thermocouples. \\
    Temperature command & Through out the duration of the test, the temperatures indicated by the control sensor; for example, thermocouples shall not vary by more than 6 2 K from the corresponding values of the initial temperature cycle. \\
    Temperature gradient & The maximum allowable axial temperature gradient is $\pm1\% T_{max}$K or $\pm3$K  \\
    \bottomrule
    \end{tabular}%
  \label{tab:addlabel}%
\end{table}%


通常意义上讲,热疲劳试验时被测试件的试验环境是高温且温度不恒定,甚至温度是急剧反复变化的,这样被测试件会由于温度的梯度循环引起热应力循环(或热应变循环),从而产生疲劳破坏现象。热疲劳试验中,试件的外形尺寸和试验温度是影响试件疲劳特性的主要因素。目前,常见的热疲劳试验箱的高温试验温度范围为200~500℃左右,具备温度梯度循环变化能力,有些超高温疲劳试验箱的温度可达几千摄氏度。传统的疲劳分析往往将热载荷转化为机械载荷加以分析,但这种方法无法考察温度和机械载荷叠加交互作用的影响,直至20世纪80年代后期才出现了热/机械疲劳试验系统。热/机械疲劳试验系统将热疲劳和机械疲劳耦合在一起,根据试件材料不同的应用环境,模拟试件实际的工作环境,更好地研究材料在热机疲劳下的损伤行为,从而得出更为准确的疲劳特性。在热/机械疲劳试验中,加载应力的大小和频率会直接影响到试件的断裂寿命,通常在试验频率范围内,疲劳寿命随着试验频率的降低而减少,高频时加载循环快,更容易出现疲劳损伤;低频时加载周期慢,除疲劳损伤外还容易出现蠕变损伤[11]。所以在疲劳试验时应根据需要选择合适的加载工况。

热梯度机械疲劳试验由机械载荷循环,温度循环和内部冷却空气组成。机械循环由MTS液压伺服试验机实现,试件的轴对称安装由对中系统来保证。温度循环由聚光辐射加热系统来实现,卤素灯管发出的光线通过椭圆柱状镜面反射到试件的工作段。温度循环冷却由自然对流方式来实现,它较之强风冷却系统引起的温度梯度要小。图1给出了试件安装在试验机夹具上的示意图。高温应变测量系统和热电偶测温系统组成两个闭合回路,通过计算机控制软件,自动测量与控制应变和温度循环的幅度、频率及相位关系。应变的测量和控制是由高精度高温应变引伸计来实现的,为了避免高温的影响,引伸计主体安装在加热炉外侧,引伸计上的两根头部为楔形的陶瓷探棒穿过炉体,夹持在试件上。要想测量温度变化的每一个瞬态值并加以控制,必须采用高灵敏度的热电偶探头。我们选用了直径仅为0.25mm的Cr-Al合金快速响应热电偶,通过缠绕的方式将热电偶固定在试件外表面,热电偶两端点由弹簧构成稳定的拉力,在试件承受拉压载荷,其直径不断变化的情况下,热电偶工作段仍能够压紧在试件上,得到试件表面的真实温度。应变循环和温度循环的叠加构成相同频率的“同相位”和“反相位”两种形式,在同相位实验中,最大拉应变对应于最高温度;而在反相位实验中,最大压应变对应于最高温度。为了保证机械应变循环和温度循环频率相位上的精度,在控制程序中编入了时钟校对系统,以消除频率积累误差。所采用的温度循环范围有两种:250~500℃和250~650℃。所有试验的总应变控制范围(机械应变与温度产生的应变之和)均为±0。8\%,应变变化率为0。0002/秒。为了消除试件内壁由于温度迟滞产生的温度梯度效应,在最大和最小应变处有10秒的保持时间。图2给出了计算机控制的温度循环和机械循环两个闭路系统的示意图。

\begin{figure}[!htp]
\centering{\includegraphics[height=22.0cm]{tgmf_procedure.pdf}}
\caption{The experimental procedures of thermal gradient mechanical fatigue test.}
\label{Fig:procedure}
\end{figure}

In this section, the high temperature radiation furnace is applied.
It is used to heat the single crystal specimens with thermal barrier coating up to a high temperature.
The furnace is matching with the MTS 809 Axial/Torsional fatigue testing machine.
Specimen heating can be accomplished by various techniques including induction, direct resistance, radiant, or forced air heating.
%There are several techniques to heat the specimen such as




%中文
%The mainly heating methods of high temperature material experiments are: environmental high temperature furnace, high frequency electromagnetic induction heating and radiation heating.
Forced air heating furnace can be used for all kinds of material. It has good temperature stability and mostly used in isothermal material tests, but it is not suitable for the experiments which need rapid temperature change.

Induction heating is the process of heating an electrically conducting object (usually a metal) by electromagnetic induction, through heat generated in the object by eddy currents. An induction heater consists of an electromagnet, and an electronic oscillator which passes a high-frequency alternating current (AC) through the electromagnet. The rapidly alternating magnetic field penetrates the object, generating electric currents inside the conductor called eddy currents. The eddy currents flowing through the resistance of the material heat it by Joule heating.

An important feature of the induction heating process is that the heat is generated inside the object itself, instead of by an external heat source via heat conduction. Thus objects can be very rapidly heated.
Induction heating is convenient to match up an additional air cooling system, which commonly used in thermal mechanical fatigue test, but induction heating method is not suitable for heating non-metallic materials.
%光加热高温炉的原理是利用镜面反射,将光线汇聚在试件表面,通过光辐射传递能量,达到加热试件表面的目的。
%光源通常选取卤素灯。




\section{Experiment and simulation of single halogen lamp radiation in a cylindrical cavity}
%A halogen lamp, also known as a quartz-halogen, is an incandescent lamp that has a small amount of a halogen such as iodine or bromine added.
%The combination of the halogen gas and the tungsten filament produces a halogen cycle chemical reaction which redeposits evaporated tungsten back onto the filament, increasing its life and maintaining the clarity of the envelope.
%Because of this, a halogen lamp can be operated at a higher temperature than a standard gas-filled lamp of similar power and operating life, producing light of a higher luminous efficacy and color temperature.
%The small size of halogen lamps permits their use in compact optical systems for projectors and illumination.

% 聚光辐射加热系统包括16根卤素灯管,灯丝的直径小于0.5mm,因此每根灯管可以被抽象为一个线光源。
The radiation furnace consists of 16 halogen lamps. The filament diameter of the lamp is about 0.3 mm, so each halogen lamp can be abstracted as a line light source.
% 每根灯管对应一面反射镜,反射镜的几何形状为椭圆柱面,每根灯管位于其对应的椭圆柱面其中一个焦点,试件位于所有椭圆柱面的公共焦点,通过镜面反射将光线聚焦在试件表面进行加热。
% 根据传热学,要确定这样一个系统的温度状态,需要已知卤素灯的辐射效率,镜面的反射效率,试件表面的辐射吸收率,以及试件内外表面的热对流系数。
Each lamp corresponds to a mirror. The mirror's geometry is an elliptic cylinder. Each lamp is located at one of the focal points of its corresponding elliptic cylinder. The sample is located at the common focal point of all elliptic cylinders. The light is reflected by the mirror and focus on the surface of the specimen for heating.
According to the heat transfer theory, to determine the temperature distribution of such a system, it is necessary to know the radiation efficiency of the halogen lamp, the reflection efficiency of the mirror, the radiation emissivity and the convective heat transfer coefficient of the inner and outer surfaces of the specimen.
% 我们的研究分为三个步骤:
% 1.单盏卤素灯密封腔辐射试验,确定单盏卤素灯的辐射效率,及物体表面的辐射吸收率;
% 2.单盏卤素灯镜面聚光加热试验,确定镀金镜面的反射效率,及物体表面的对流换热系数;
% 3.整体系统的聚光加热试验,验证计算结果。
% 本节介绍了单盏卤素灯自由辐射的试验过程及数值计算结果。

The study is divided into three steps:
1. Single lamp radiation test in a sealed cavity to determine the radiation efficiency of the single halogen lamp;
2. Single lamp mirror light concentrating heating test, to determine the reflection efficiency of the gold-plated mirror, and the convection heat transfer coefficient of the surface;
3. Condensation heating test of the overall system to verify the calculation results.
This section describes the test procedure and numerical results of the free radiation of a single halogen lamp.

单盏卤素灯密封腔辐射试验装置设计如\ref{Fig:OneLightRadiation}(a)所示,包括圆柱形不锈钢壁面、上下盖板和灯头固定部件等,构成了一个圆柱形密封腔。
所有零件都使用SS304不锈钢通过数控加工中心制作,空腔内壁采用耐高温黑色涂料均匀喷涂,确保空腔内壁各处的辐射发射率相同。
我们采用K型热电偶测量圆柱壁面的温度分布。
根据该试验装置的对称性,选取其中一个轴对称截面放置热电偶,内壁面和外壁面在垂直方向各五个,热电偶的位置及编号如\ref{Fig:OneLightRadiation}(b)所示。

使用National Instruments公司的NI 9213型16通道热电偶数据采集器(见\ref{Fig:OneLightRadiation}(c))对热电偶的测量结果进行采集,该数据采集器内置CJC冷端温度补偿,通过在PC端安装NI-DAQmx驱动软件实现与电脑之间的通信。 National Instruments公司同时提供了LabVIEW可编程软件,根据Equation \ref{Equ:Inverse_polynomials}编写程序,将K型热电偶的电压信号转换为温度值,并将试验结果按照合适的文件格式存储。

为了保证实验的精度,我们在圆柱密封腔外表面中心位置(如\ref{Fig:OneLightRadiation}(a)所示)选取了一个矩形区域进行喷涂,并将矩形区域的中心作为红外温度测量点。
采用Voltcraft IR 2200型红外测温仪(见\ref{Fig:OneLightRadiation}(d))测量该点的温度。
其中,该型号红外测温仪的温度测量范围是-50至2000$^{\circ}$C,测量精度为$\pm2^{\circ}$C。

\begin{figure}
  \begin{minipage}[t]{0.5\linewidth} % 如果一行放2个图,用0.5,如果3个图,用0.33\
  \nonumber
    \centering
    \includegraphics[width=3.0in]{OneLightRadiationPhoto.jpg}
    \centerline{(a) Cylindrical cavity.}
    \label{Fig:OneLightRadiationShow}
  \end{minipage}%
  \begin{minipage}[t]{0.5\linewidth}
    \centering
    \includegraphics[width=3.0in]{OneLightRadiationShow.pdf}
    \centerline{(b) Thermocouple locations.}
    \label{Fig:OneLightRadiationPhoto}
  \end{minipage}

  \begin{minipage}[t]{0.5\linewidth} % 如果一行放2个图,用0.5,如果3个图,用0.33\
  \nonumber
    \centering
    \includegraphics[width=3.0in]{NI9213.jpg}
    \centerline{(c) NI9213 thermocouple signal transmitter }
    \centerline{and the data acquisition device. }
    \label{Fig:NI9213}
  \end{minipage}%
  \begin{minipage}[t]{0.5\linewidth}
    \centering
    \includegraphics[width=3.0in]{IR.jpg}
    \centerline{(d) The infrared temperature measurement }
    \centerline{device.}
    \label{Fig:IR}
  \end{minipage}

%  \begin{minipage}[t]{1.0\linewidth} % 如果一行放2个图,用0.5,如果3个图,用0.33\
%  \nonumber
%    \centering
%    \includegraphics[width=5.0in]{BlockDiagram.png}
%    \centerline{(e) Program block diagram of data collection, temperature transform and computation.}
%    \label{Fig:FrontPanel}
%  \end{minipage}

  \caption{Test rig of the single halogen lamp radiation in a cylindrical cavity.}
  \label{Fig:OneLightRadiation}
\end{figure}

试验开始前,将所有热电偶与数据采集器连接,标定每个热电偶的温度值并进行冷端补偿。
将热电偶采集器和红外测温数据与PC端进行同步,定义数据采集程序。
开启卤素灯加热一段时间后关闭电源,试验装置自然冷却至室温,记录升温与降温全过程。
更换不同功率的卤素灯(240W,350W和400W),获取不同功率卤素灯的辐射加热和冷却曲线。

通过试验我们可以得到圆柱壁面上11个不同位置测温点的温度-时间曲线。
下面我们将建立有限元模型,结合数值计算的结果,确定壁面的对流换热系数及表面散射率。
这里我们使用商业有限元软件ABAQUS的Heat transfer模块进行计算。
对于含有辐射传热的计算模型,两个常数需要提前定义:
(1)Stefan-Boltzmann常数为$5.670\times10^{-8}\rm{W}/(\rm{m}^{2}\cdot \rm{K})$,
(2)绝对零度为-273.15$^{\circ}$C。
我们进行的是瞬态温度场计算,因此材料的热传导系数、密度和比热容需要被定义。
圆柱壁面材料为SS304不锈钢,随着温度的升高,SS304不锈钢的传热系数也会增大,不同温度下的材料传热系数如Table \ref{Tab:SS304HeatTransfer}所示。
忽略温度对密度和比热容的影响,定义SS304不锈钢的密度为7873kg/m$^3$,比热容(Specific Heat Capacity)为502J/(kg$\cdot$K)。

\begin{table}[htbp]
  \centering
  \caption{Heat conductivity of 304 Stainless Steel.}
    \begin{tabular}{llll}
    \toprule
    Temperature & Conductivity & Temperature & Conductivity \\
    $^{\circ}$C & ${\rm{W/(m}} \cdot {\rm{K)}}$ & $^{\circ}$C & ${\rm{W/(m}} \cdot {\rm{K)}}$ \\
    \midrule
    100   & 15.5  & 600   & 20.3 \\
    200   & 18.1  & 700   & 20.8 \\
    300   & 18.4  & 800   & 20.8 \\
    400   & 19.1  & 900   & 20.6 \\
    500   & 19.7  & 1000  & 20.7 \\
    \bottomrule
    \end{tabular}%
  \label{Tab:SS304HeatTransfer}%
\end{table}%


单盏卤素灯密封腔辐射试验,可以简化为圆柱面(灯丝)在密闭空间内辐射传热模型。
根据结构对称性,以中心高度截面作为对称面,建立有限元计算模型,如\ref{Fig:OneLightRadiationSimulation}(c)所示。

以400W卤素灯为例,经过测量灯芯长度为68mm(see \ref{Fig:OneLightRadiation}(a)),灯芯色温2950K。根据Equation \ref{Equ:IntegratingOfStefanBoltzmann}我们可以计算灯芯表面辐射面积为:
\begin{equation}
A = \frac{P}{\sigma \cdot T^4} = \frac{400}{5.670\times10^{-8}\times2950^4}=9.316\times10^{-5} {\rm{m}}^2,
\end{equation}
将灯芯等效为理想圆柱面,已知圆柱面的表面积93.16${\rm{mm}}^2$和长度68mm,因此,可以计算得出圆柱面的直径为0.436mm。

\begin{figure}
  \begin{minipage}[t]{0.5\linewidth} % 如果一行放2个图,用0.5,如果3个图,用0.33\
  \nonumber
    \centering
    \includegraphics[width=3.0in]{HalogenLamp1.png}
    \centerline{(a) Double ended halogen lamp with tube base.}
    \label{Fig:HalogenLamp1}
  \end{minipage}%
  \begin{minipage}[t]{0.5\linewidth}
    \centering
    \includegraphics[width=3.0in]{Filament.pdf}
    \centerline{(b) Mesh of the filament.}
  \end{minipage}
  \caption{Comparison of experimental and computational results of free radiation with single halogen lamp.}
  \label{Fig:Filament}
\end{figure}

试验过程中环境温度为23$^{\circ}$C,空气自然对流换热系数的范围是5到25W/(${\rm{m}}^2\cdot$K)。
使用trial and error方法,可以得到内壁面(黑色喷涂)的表面散射率为0.95,外壁面(金属表面)的表面散射率为0.68,对流换热系数5.6W/(${\rm{m}}^2\cdot$K)。
使用得到的表面散射率和对流换热系数,分别对240W、350W和400W功率卤素灯的辐射传热进行计算,选取外壁面中心点的温度值进行对比,如\ref{Fig:OneLightRadiationSimulation}(b-d)所示。
数值模拟结果、热电偶测量值和红外温度测量仪所得到的温度值在加热和冷却两个过程中都得到了很好的吻合,这说明我们使用的材料常数和传热学参数是正确的。

\begin{figure}
  \begin{minipage}[t]{0.5\linewidth} % 如果一行放2个图,用0.5,如果3个图,用0.33\
  \nonumber
    \centering
    \includegraphics[width=2.5in]{OneLightRadiationTransient.pdf}
    \centerline{(a) FEM model.}
    \label{Fig:OneLightRadiationTransient}
  \end{minipage}%
  \begin{minipage}[t]{0.5\linewidth}
    \centering
    \includegraphics[width=3.0in]{plot_cmp_cavity_radiation_240W.pdf}
    \centerline{(b) Halogen lamp power 240W.}
    \label{Fig:Compare240W}
  \end{minipage}

    \begin{minipage}[t]{0.5\linewidth} % 如果一行放2个图,用0.5,如果3个图,用0.33\
  \nonumber
    \centering
    \includegraphics[width=3.0in]{plot_cmp_cavity_radiation_350W.pdf}
    \centerline{(c) Halogen lamp power 350W.}
    \label{Fig:Compare350W}
  \end{minipage}%
  \begin{minipage}[t]{0.5\linewidth}
    \centering
    \includegraphics[width=3.0in]{plot_cmp_cavity_radiation_400W.pdf}
    \centerline{(d) Halogen lamp power 400W.}
    \label{Fig:Compare400W}
  \end{minipage}
  \caption{Comparison of experimental and computational results of single halogen lamp radiation in the cylindrical cavity.}
  \label{Fig:OneLightRadiationSimulation}
\end{figure}



\section{Experiment and simulation of single halogen lamp with mirror condenser}
%\begin{figure}[!htp]
%\centering\scalebox{0.3}{\includegraphics{haloline-for-mains-voltage.jpg}}
%\caption{Double ended halogen lamps with tube base.}
%\label{Fig:}
%\end{figure}

Ellipsoidal reflectors have two conjugate foci. Light from one focus passes through the other after reflection (see \ref{Fig:EllipsoidalReflectors}).
Deep ellipsoids of revolution surrounding a source collect a much higher fraction of total emitted light than a spherical mirror or conventional lens system.
The effective collection are very small, and the geometry is well suited to the spatial distribution of the output from an arc lamp.
Two ellipsoids can almost fully enclose a light source and target to provide nearly total energy transfer.
    \begin{figure}[!htp]
        \centering\scalebox{0.6}{\includegraphics{Ellipsoidal_reflectors.png}}
        \caption{A section of an ellipsoidal reflector. }
        \label{Fig:EllipsoidalReflectors}
    \end{figure}
More significant perhaps is the ellipsoid's effect upon imaging of extended sources.
For a pure point source exactly at one focus of the ellipse, almost all of the energy is transferred to the other focus.
Unfortunately, every real light source, such as a lamp arc, has some finite extent; points of the source that are not exactly at the focus of the ellipse will be magnified and defocused at the image.

\ref{Fig:EllipsoidalReflectors} illustrates how light from point S, off the focus F1, does not reimage near F2, but is instead spread along the axis.
Rays a, b and c, shown in the top half of the ellipse, are all from F1 and pass through F2, the second focus of the ellipse. Rays A, B and C, in the bottom half of the ellipse are exact ray traces for rays from a point close to F1. They strike the ellipse at equivalent points to a, b and c, but do not pass through F2. For a small spot (image) at F2 you need a very small source.
Because of this effect ellipsoids are most useful when coupled with a small source and a system that requires a lot of light without concern for particularly good imaging.

%根据灯丝形状不同光源可以被简化为点光源和直线光源两种。
%点光源的聚光采用椭圆球面镜,将点光源放置于椭圆球面镜的其中一个焦点,经过反射光线会聚焦至另一个焦点。
%参考德宇航的设计,我们采用管状卤素灯作为光源。

理想情况下,灯丝可简化为线光源,根据椭圆镜面的性质,如果线光源位于镜面其中一个焦点处,反射后的光线将全部通过另一个焦点。
但是根据上一节的计算,如果我们将灯丝近似为一个圆柱面,则该圆柱面的直径约为0.5mm,光线经过镜面反射后会有一定的发散。
真实的镜面无法保证完全光滑,光线会产生一定的漫反射,通过试验和计算,我们需要确定镜面的反射效率。

单盏卤素灯镜面反射辐射传热试验装置,如\ref{Fig:OneLightWithEllipseMirrorReflectionShow}(a)所示,包括椭圆柱面镜、受光平板、温度传感器及数据采集系统等。
卤素灯位于椭圆柱面镜的其中一个焦点,受光平板的向光面垂直于椭圆柱面的长轴,并通过其另一个焦点,光线经过椭圆柱面镜反射后聚焦在受光平面上,如\ref{Fig:OneLightWithEllipseMirrorReflectionShow}(b)所示。
%灯丝的直径、灯丝与镜面的相对位置、镜面的表面粗糙度和镜面的轴线方向都会影响受光平面上的光强度分布。
与上一节相同,受光平板材料为SS304不锈钢,表面采用同样的耐高温黑色涂料喷涂,因此可以认为该受光平板的表面发射率与上一节中的圆柱空腔内壁相同,为0.95。
选择K型热电偶测量平板表面的温度分布,热电偶的位置分布如\ref{Fig:OneLightWithEllipseMirrorReflectionShow}(c)所示,共15个。
热电偶的数据采集,同样选取National Instruments公司的9213型16通道热电偶数据采集器,相应的编程和数据处理过程与上一节类似,采样频率为10Hz。

观察受光平板上光斑的位置和
\begin{figure}
  \begin{minipage}[t]{0.5\linewidth} % 如果一行放2个图,用0.5,如果3个图,用0.33\
  \nonumber
    \centering
    \includegraphics[width=2.5in]{OneLightWithEllipseMirrorReflectionShow.pdf}
    \centerline{(a) Schematic diagram of the test rig.}
    \label{Fig:OneLightWithEllipseMirrorReflectionShow}
  \end{minipage}%
  \begin{minipage}[t]{0.5\linewidth}
    \centering
    \includegraphics[width=3.5in]{OneLightWithEllipseMirrorReflectionShow2.pdf}
    \centerline{(b) A section of the ellipsoidal reflector.}
    \label{Fig:OneLightWithEllipseMirrorReflectionShow2}
  \end{minipage}

  \begin{minipage}[t]{0.5\linewidth} % 如果一行放2个图,用0.5,如果3个图,用0.33\
  \nonumber
    \centering
    \includegraphics[width=3.0in]{OneLightWithEllipseMirrorReflectionShow3.pdf}
    \centerline{(c) Thermocouple locations.}
    \label{Fig:OneLightWithEllipseMirrorReflectionShow3}
  \end{minipage}%
  \begin{minipage}[t]{0.5\linewidth}
    \centering
    \includegraphics[width=3.0in]{OneLightWithEllipseMirrorReflectionShow6.jpg}
    \centerline{(d) Experimental process.}
    \label{Fig:OneLightWithEllipseMirrorReflectionShow6}
  \end{minipage}

%  \begin{minipage}[t]{0.5\linewidth} % 如果一行放2个图,用0.5,如果3个图,用0.33\
%  \nonumber
%    \centering
%    \includegraphics[width=3.0in]{OneLightWithEllipseMirrorReflectionShow5.jpg}
%    \centerline{(e) Experimental process.}
%    \label{Fig:OneLightWithEllipseMirrorReflectionShow5}
%  \end{minipage}%
%  \begin{minipage}[t]{0.5\linewidth}
%    \centering
%    \includegraphics[width=3.0in]{OneLightWithEllipseMirrorReflectionShow6.jpg}
%    \centerline{(f) Focused light.}
%    \label{Fig:OneLightWithEllipseMirrorReflectionShow6}
%  \end{minipage}

  \caption{Test rig of single halogen lamp radiation with the reflect mirror.}
  \label{Fig:OneLightWithEllipseMirrorReflectionShow}
\end{figure}

\begin{figure}
  \begin{minipage}[t]{0.5\linewidth} % 如果一行放2个图,用0.5,如果3个图,用0.33\
  \nonumber
    \centering
    \includegraphics[width=2.2in]{OneLightWithEllipseMirrorReflection.pdf}
    \centerline{(a) FEM model.}
  \end{minipage}%
  \begin{minipage}[t]{0.5\linewidth}
    \centering
    \includegraphics[width=3.0in]{OneLightWithEllipseMirrorReflectionSimulation.png}
    \centerline{(b) Simulation result.}
  \end{minipage}

  \begin{minipage}[t]{0.5\linewidth} % 如果一行放2个图,用0.5,如果3个图,用0.33\
  \nonumber
    \centering
    \includegraphics[width=3.0in]{plot_cmp_temperature_horizontal.pdf}
    \centerline{(c) Temperature distribution along }
    \centerline{the horizontal direction. }
  \end{minipage}%
  \begin{minipage}[t]{0.5\linewidth}
    \centering
    \includegraphics[width=3.0in]{plot_cmp_temperature_vertical.pdf}
    \centerline{(d) Temperature distribution along }
    \centerline{the vertical direction. }
  \end{minipage}

  \caption{Comparison of experimental and computational results of single halogen lamp radiation with the reflect mirror.}
  \label{Fig:OneLightWithReflection}
\end{figure}
\newpage
\subsection{Design of the radiation furnace}
A radiant furnace is developed specifically for the TGMF tests.
% It is complex to create the radiation heating system directly.
% Then the radiation furnace is suggested to divided into several subsystem as follows.
The furnace includes the following subsystems:
\begin{itemize}
\item \textbf{Quartz lamp alignment device} includes 16 halogen, its power supply circuit and position adjusting device.
\item \textbf{Temperature control system} includes a controller, .
\item \textbf{Switch and circuit} includes 16 halogen, its power supply circuit and position adjusting device.

\item \textbf{Quartz lamp radiation heating system} includes 16 halogen, its power supply circuit and position adjusting device.
\item \textbf{Air cooling system for the halogen lamps} includes an air compressor, breathable lamp holders and four air manifolds.
\item \textbf{Air Cooling system for inner and outer surface of the specimen } is to realize the aim of fast cooling. Two hollow extend clamp are manufactured with cooling air pipe connected to each side.

\item \textbf{Water cooling system for the mirrors.}
\item \textbf{Temperature measurement system.}
\item \textbf{Strain measurement system.}
\item \textbf{Temperature control system.}
\end{itemize}


\begin{figure}[!htp]
\centering\scalebox{0.65}{\includegraphics{Radiation_Furnace2.pdf}}
\caption{Sketch of radiation furnace.}
\label{Fig:Radiation_Furnace}
\end{figure}

\begin{figure}[!htp]
\centering\scalebox{0.25}{\includegraphics{DesignWithCATIA.png}}
\caption{Design scheme of the radiation furnace with CATIA.}
\label{Fig:DesignWithCAITA}
\end{figure}

\begin{figure}[!htp]
\centering\scalebox{0.13}{\includegraphics{Heating_Specimen.jpg}}
\caption{Specimen is heated by radiation.}
\label{Fig:Heating_Specimen}
\end{figure}

\begin{figure}[!htp]
\centering\scalebox{0.1}{\includegraphics{TGMF_Test_System.jpg}}
\caption{Experimental apparatus of TGMF test system.}
\label{Fig:TGMF_Test_System}
\end{figure}

\begin{figure}
  \begin{minipage}[t]{0.5\linewidth} % 如果一行放2个图,用0.5,如果3个图,用0.33\
  \nonumber
    \centering
    \includegraphics[width=3.0in]{MirrorBeforeOvergild.jpg}
    \centerline{(a) 镀金前的镜面单元体.}
    \label{fig:side:a}
  \end{minipage}%
  \begin{minipage}[t]{0.5\linewidth}
    \centering
    \includegraphics[width=3.0in]{MirrorAfterOvergild.jpg}
    \centerline{(b) 镀金后的镜面单元体.}
    \label{fig:side:b}
  \end{minipage}

  \caption{椭圆柱面镜的加工及镀金。}
  \label{Fig:MirrorOvergild}
\end{figure}

%\begin{figure}[!htp]
%\centering\scalebox{0.15}{\includegraphics{MTS370WithResistanceFurnace.jpg}}
%\caption{MTS-370 轴向疲劳材料试验机与电阻式加热炉.}
%\label{Fig:MTS370WithResistanceFurnace}
%\end{figure}

%\begin{figure}[!htp]
%\centering\scalebox{0.15}{\includegraphics{MTS370WithRadiationFurnace.jpg}}
%\caption{MTS-370 轴向疲劳材料试验机与辐射加热系统.}
%\label{Fig:MTS370WithRadiationFurnace}
%\end{figure}
\newpage
\section{Extensometer}
An extensometer is a device that is used to measure changes in the length of an object.
It is useful for stress-strain measurements and tensile tests.
There are many different types of high temperature extensometers available.
The type of extensometer is dependent on the type of testing and the type of heating system used.

There are two observation windows at both side of the developed radiation furnace as shown in \ref{Fig:ObservationWindow}.
The distance between each observation window and the specimen outer surface is 160mm.
Due to the long distance, Epsilon Model 3448 high temperature extensometer was selected.
% to mount with the radiation furnace
It has two ceramic rods which are made of high purity alumina.
They pass through the furnace and contact the specimen.
These are available in lengths as required to fit the furnace.
The extensometer is typically used for high temperature tension, compression and fatigue testing.
Its gauge length is 10mm with $\pm20\%$ full scale range.

\begin{figure}[!htp]
\centering\scalebox{0.3}{\includegraphics{ObservationWindow.png}}
\caption{The observation window of radiation furnace.}
\label{Fig:ObservationWindow}
\end{figure}

The extensometer were held on the specimen by light, flexible ceramic fiber cords as shown in \ref{Fig:Epsilon3448}.
These make the extensometer self-supporting on the specimen.
No furnace mounting brackets are required.
The side load on the test sample is greatly reduced because of the self-supporting design and light weight of the sensor.
The combination of radiant heat shields and convection cooling fins allow this model to be used at specimen temperatures up to 1200$^{\circ}$C without cooling.
\begin{figure}[!htp]
\centering\scalebox{0.12}{\includegraphics{Epsilon3448.png}}
\caption{Self-supporting on the specimen.}
\label{Fig:Epsilon3448}
\end{figure}

%An extensometer is a sensor attached to a specimen that measures a dimensional change (gage length or strain) that occurs in the specimen during testing.
The extensometer works by means of precision resistance-type strain gages bonded to a metallic element to form a Wheatstone bridge circuit.
It requires special test fixtures to aid in calibration.
%An extensometer requires DC excitation, which requires either a dedicated DC conditioner or a digital universal conditioner (DUC) configured in the DC mode.
In general, calibration is effected by connecting the extensometer into the extensometer socket on the loading frame and accurately setting the strain channel output, observed on a digital readout or recorder display against a precise, known displacement of the extensometer.
To assist in this operation, the high magnification extensometer calibrator GWB-200JA is available.
%The extensometer calibrator is used either where an extensometer does not feature electrical calibration or when a manual verification of the output is desired.

The calibration fixture, illustrated at \ref{Fig:Calibration}, consists of a precision micrometer head that drives a separate moveable shaft.
The micrometer head and shaft are mounted on a frame and the extensometer is attached between the moving and stationary shafts.
%Displacement is shown via a dial scale.
%It enables the extensometer to be exercised over its mechanical displacement range and permits the accurate adjustment of an extensometer over very small distances for tensile or compressive calibration.
%The extensometer is mounted to the separate moveable shaft.
%One of the shaft is adjustable, while the other remains stationary.
As the micrometer is adjusted, the lower shaft moves and displacement is applied to the extensometer.
The elongation is read from the dial scale of the micrometer head. This can be used to either verify the output of the extensometer shown on the test system or to set zero and full scale output to calibrate it initially.
\begin{figure}[!htp]
\centering{\includegraphics[width=6cm]{Calibration_out.jpg}}
\centering{\includegraphics[width=8cm]{GainDeltaK.pdf}}
\caption{Extensometer calibrator GWB-200JA.}
\label{Fig:Calibration}
\end{figure}

\section{Forced convection in turbulent pipe flow}

\begin{figure}[!htp]
\centering\scalebox{0.6}{\includegraphics{inner_cooling.pdf}}
\caption{Schematic of inner cooling air system.}
\label{Fig:}
\end{figure}

Because of the small inner diameter 6.5mm of the specimen, it is difficult to measure the temperature of the specimen inner surface during the tests.
A variety of measurement options were tried to measure the inner surface temperature.
It was found that the internal cooling air has a significant effect on the temperature measurement of the inner surface of the specimen.
%内部冷却空气对试件内壁的温度测量有很大的影响。
The measured temperature is not the temperature of the inner surface, but the average temperature of the wall and the cooling air flow.
%测量得到的温度不是壁面的温度,而是壁面和冷却气流的平均温度。
Therefore we tried to calculate the temperature distribution of the specimen inner surface using the FEM method.

%The internal diameter of the specimen is 6.5 mm, and .
%试件的内部直径为6.5mm,。

The inner surface of the hollow specimen can be considers as a pipe.
%空心试件可以简化为圆管.
According to the formula of convection in turbulent pipe flow, the convective heat transfer coefficient of the inner surface of the specimen can be calculated.
%根据圆管内部的强制对流换热公式,我们计算试件内表面的对流换热系数。

Nusselt number (Nu) is a dimensionless number, defined as the ratio of convective to conductive heat transfer across (normal to) the boundary.
The Nusselt number is given as:
\[{\rm{N}}{{\rm{u}}_L} = \frac{{hL}}{k},\]
where $h$ is the convective heat transfer coefficient of the fluid, $L$ is the characteristic length, $k$ is the thermal conductivity of the fluid.

Gnielinski's correlation for turbulent flow in tubes is given as:
\[{\rm{N}}{{\rm{u}}_D} = \frac{{\left( {f/8} \right)\left( {{\rm{R}}{{\rm{e}}_D} - 1000} \right){\rm{Pr}}}}{{1 + 12.7{{(f/8)}^{1/2}}\left( {{\rm{P}}{{\rm{r}}^{2/3}} - 1} \right)}},\]
where $f$ is the Darcy friction factor that can either be obtained from the Moody chart or for smooth tubes from correlation developed by Petukhov:
\[f = {\left( {0.79\ln \left( {{\rm{R}}{{\rm{e}}_D}} \right) - 1.64} \right)^{ - 2}},\]
with
\[0.5 \le {\rm{Pr}} \le 2000,\]
\[3000 \le {\rm{R}}{{\rm{e}}_D} \le 5 \times {10^6},\]

The Prandtl number (Pr) is a dimensionless number, defined as the ratio of momentum diffusivity to thermal diffusivity.
The Prandtl number is given as:
\[{\rm{Pr}} = \frac{{{c_p}\mu }}{k},\]
where
$c_{p}$ is specific heat, $\mu$ is dynamic viscosity and $k$ is thermal conductivity.

The Reynolds number (Re) is a dimensionless number, defined as the ratio of inertial forces to viscous forces within a fluid.
The Reynolds number is given as:
\[{\rm{Re}} = \frac{{\rho uL}}{\mu },\]
where
$\rho$ is density of the fluid, $u$ is the velocity of the fluid with respect to the object, $\mu$ is the dynamic viscosity of the fluid and $L$ is a characteristic dimension.

It is noted that the dynamic viscosity $\mu$ of an ideal gas is a function of the temperature. It can be derived as Sutherland's formula:
\[\mu  = {\mu _0}\frac{{{T_0} + C}}{{T + C}}{\left( {\frac{T}{{{T_0}}}} \right)^{\frac{3}{2}}},\]
where $\mu$ is dynamic viscosity at input temperature $T$,
$\mu_0$ is reference viscosity at reference temperature $T_0$,
$C$ is Sutherland's constant dependent on the gaseous material.
Table \ref{tab:SutherlandConstant} shows the Sutherland's constant and reference values of the air.
\begin{table}[htbp]
  \centering
  \caption{Sutherland's constant and reference values of the air.}
    \begin{tabular}{p{2cm}p{2cm}p{2cm}p{3cm}}
    \toprule
    Gas   & $C$(K) & $T_0$(K) & $\mu_0$($\rm{Pa\cdot s}$) \\
    \midrule
    air   & 120   & 291.15 & $1.827\times 10^{-5}$ \\
    \bottomrule
    \end{tabular}%
  \label{tab:SutherlandConstant}%
\end{table}%

The density of dry air $\rho$ can be calculated using the ideal gas law, expressed as a function of temperature and pressure:
\begin{equation}
\rho  = \frac{p}{{RT}},
\label{Equ:AirDensity}
\end{equation}
where
$p$ is absolute pressure,
$T$ is absolute temperature,
$R$ is specific gas constant.

According to the measurement results of the sensors, the average pressure and temperature of the cooling air are 6.5 bar and 288.15K, respectively.

\begin{table}[htbp]
  \begin{threeparttable}
  \centering
  \caption{Heat convection coefficients of the inner surface of the specimen.}
    \begin{tabular}{llrrrr}
    \toprule
    $T$   & ${\rm{K}}$ & 288.15  & 288.15  & 288.15  & 288.15  \\
    $p$   & ${\rm{bar}}$ & 6.5   & 6.5   & 6.5   & 6.5  \\
    $\rho  = \frac{p}{{RT}}$ \tnote{*1} & ${\rm{kg/}}{{\rm{m}}^{\rm{3}}}$ & 7.724 & 7.724 & 7.724 & 7.724 \\
    $\mu  = {\mu _0}\frac{{{T_0} + C}}{{T + C}}{\left( {\frac{T}{{{T_0}}}} \right)^{\frac{3}{2}}}$ \tnote{*2} & -     & 1.81E-05 & 1.81E-05 & 1.81E-05 & 1.81E-05 \\
    ${\dot V}$ & ${\rm{l/min}}$ & 10    & 20    & 30    & 40  \\
    $u = \frac{{4\dot V}}{{\pi {D^2}}}$ \tnote{*3} & ${\rm{m/}}{{\rm{s}}^2}$ & 5.02  & 10.05  & 15.07  & 20.09  \\
    ${\rm{Re}} = \frac{{\rho uL}}{\mu }$ \tnote{*4} & -     & 13906  & 27812  & 41718  & 55624  \\
    ${\rm{Pr}} = \frac{{{c_p}\mu }}{k}$ \tnote{*5} & -     & 0.63  & 0.63  & 0.63  & 0.63  \\
    $f = {\left( {0.79\ln \left( {{\rm{R}}{{\rm{e}}_D}} \right) - 1.64} \right)^{ - 2}}$ & -     & 0.0288  & 0.0241  & 0.0219  & 0.0205  \\
    ${\rm{N}}{{\rm{u}}_D} = \frac{{\left( {f/8} \right)\left( {{\rm{R}}{{\rm{e}}_D} - 1000} \right){\rm{Pr}}}}{{1 + 12.7{{(f/8)}^{1/2}}\left( {{\rm{P}}{{\rm{r}}^{2/3}} - 1} \right)}}$ & -     & 36.76  & 62.61  & 85.39  & 106.50  \\
    $h = \frac{k}{{D{\rm{N}}{{\rm{u}}_D}}}$ & ${\rm{W/(}}{{\rm{m}}^{\rm{2}}} \cdot {\rm{K)}}$ & 146.03  & 248.74  & 339.24  & 423.10  \\
    \bottomrule
    \end{tabular}%
    \begin{tablenotes}
    \item[*1] $R = 287.058 {\rm{J/(kg}} \cdot {\rm{K}})$ for dry air.
    \item[*2] Sutherland's constant and reference values of the air (see Table \ref{tab:SutherlandConstant}).
    \item[*3] $D=0.65$mm.
    \item[*4] $L=D$.
    \item[*5] $c_{p}=903.3 {\rm{J/(kg}} \cdot {\rm{K}})$ and $k=0.0258 {\rm{W/(m}} \cdot {\rm{K}})$ at $288.15{\rm{K}}$.
    \end{tablenotes}
    \end{threeparttable}
  \label{tab:addlabel}%
\end{table}%





%temperature $T=15+273.15=288.15\rm{K}$
%pressure $p=6.5\rm{bar}$
%density $\rho_0= 1.293\rm{kg/m^3}$
%The average pressure of inner cooling air is 6.5 bar. With Equation \ref{Equ:AirDensity}, we obtain the air density at 6.5 bar is 7.858 $\rm{kg/m^3}$.
%inner diameter of specimen $d_{in} = 0.0065\rm{mm}$
%volume flow ${\dot V} = 25\rm{l/min}$

