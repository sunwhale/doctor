\chapter{Literature review}

\section{Continuum plasticity}
Continuum plasticity is a solid mechanics theory that is used to describe the plastic behavior of materials.
It is characterized by the assumption that a flow rule exists that can be used to determine the amount of plastic strain in the material.
This section introduces the fundamentals of time-independent phenomenological plasticity, such as invariants of the stress tensor, strain decomposition, yield surface, flow rule, Drucker postulates, consistency condition, isotropic and kinematic hardening.

\subsection{Invariants of the stress tensor}
In continuum mechanics, the stress state at any material point can be represented by the stress tensor $\bm{\sigma}$.
The stress tensor can be represented by a matrix of second order in an arbitrary coordinate system as:
\begin{equation}
\bm{\sigma} =
  \begin{bmatrix}
    \sigma_{1\,1}& \sigma_{1\,2}& \sigma_{1\,3}\\
    \sigma_{2\,1}& \sigma_{2\,2}& \sigma_{2\,3}\\
    \sigma_{3\,1}& \sigma_{3\,2}& \sigma_{3\,3}
  \end{bmatrix}.
\end{equation}
Since the stress tensor is symmetric, only six stress components axe independent.
Thus, six independent stress components determine a stress state uniquely and visa versa.
Using the above stress tensor, the three principal stresses can be determined using the characteristic equation
\begin{equation}
\det \left( {{\bm{\sigma }} - \sigma {\bf{I}}} \right) = 0,
\end{equation}
which can be expanded to
\begin{equation}
{\sigma ^3} - {I_1}{\sigma ^2} + {I_2}\sigma  - {I_3} = 0.
\end{equation}
where $\bf{I}$ is the second order identity tensor, ${I_1}$, ${I_2}$, and ${I_3}$ are the invariants of the stress tensor.
In terms of the principal stresses, $\sigma_1$, $\sigma_2$ and $\sigma_3$, the invariants are
\begin{equation}
\begin{array}{*{20}{l}}
{{I_1} = {\sigma _1} + {\sigma _2} + {\sigma _3}},\\
{{I_2} = {\sigma _1}{\sigma _2} + {\sigma _2}{\sigma _3} + {\sigma _3}{\sigma _1}},\\
{{I_3} = {\sigma _1}{\sigma _2}{\sigma _3}}.
\end{array}
\end{equation}
In plasticity theory, it is customary to decompose the stress tensor as
\begin{equation}
{\bm{\sigma }} = {\bf{s}} + {\sigma _m}{\bf{I}},
\end{equation}
where $\sigma_m$ is the hydrostatic stress given by
\begin{equation}
{\sigma _m} = {\rm{Tr}}\left( {\bm{\sigma }} \right) = \frac{1}{3}\left( {{\sigma _1} + {\sigma _2} + {\sigma _3}} \right),
\end{equation}
and $\bf{s}$ is the deviatoric stress tensor.
Using the deviatoric stress tensor, the deviatoric stress invariants can be defined as
\begin{equation}
\begin{array}{*{20}{l}}
{{J_1} = {\rm{Tr}}\left( {\bf{s}} \right) = 0},\\
{{J_2} = \frac{1}{2}{\bf{s}}:{\bf{s}} = \frac{1}{6}\left[ {{{\left( {{\sigma _1} - {\sigma _2}} \right)}^2} + {{\left( {{\sigma _2} - {\sigma _3}} \right)}^2} + {{\left( {{\sigma _3} - {\sigma _1}} \right)}^2}} \right]},\\
{{J_3} = \det \left( {\bf{s}} \right) = {s_1}{s_2}{s_3}}.
\end{array}
\end{equation}

In plasticity mechanics, the first principal invariant $I_1$ of the Cauchy stress $\boldsymbol{\sigma}$, and the second and third principal invariants $J_2$, $J_3$ of the deviatoric part $\boldsymbol{s}$ of the Cauchy stress are usually used to express the yield surface.


\subsection{Strain decomposition}
It is shown in \ref{Fig:Strain_decomposition} that the stress-strain curve can be obtained from a uniaxial tensile test.
Both elastic and inelastic regions are indicated.
In elastic region, the stress is proportional to the strain if the stress in the specimen is below a certain value, i.e. the yield stress $\sigma_{y0}$, which is the elastic limit.
As a material is loaded beyond its elastic limit, Hooke's law does not apply, the material no longer exhibits elastic behavior and the stress-strain relation becomes nonlinear. The material yields, begins to flow and residual, permanent deformation results after unloading.

For strain hardening materials the yield stress increases with increasing plastic deformation to a value of $\sigma_y$.
It is called hardening because the stress is increasing relative to perfect plastic behavior, also shown in the figure.
\begin{figure}[!htp]
\centering\scalebox{1.0}{\includegraphics{Rock_plasticity_compression_plain.pdf}}
\caption{The classical decomposition of strain into elastic and plastic parts.}
\label{Fig:Strain_decomposition}
\end{figure}
If, at a strain of $\varepsilon$, the loading were to be reversed, the material would cease to deform plastically (at least in the absence of time-dependent effects) and would show a linearly decreasing stress with strain such that the gradient of this part of the stress-strain curve would again be the Young's modulus, E, shown in \ref{Fig:Strain_decomposition}.
Once a stress of zero is achieved (provided the material remains elastic on full reversal of the load), the strain remaining in the test specimen is the plastic strain, $\varepsilon_p$.
The recovered strain, $\varepsilon_e$, is the elastic strain and it can be seen that the total strain, $\varepsilon$, is the sum of the two
\begin{equation}
\varepsilon = \varepsilon^e + \varepsilon^p
\end{equation}
This is called the classical additive decomposition of strain.
It is also apparent from \ref{Fig:Strain_decomposition} that the stress achieved at a strain of $\varepsilon$ is given by
\begin{equation}
\sigma = E\varepsilon^e = E(\varepsilon - \varepsilon^p)
\end{equation}

\subsection{Yield criterion}
A yield surface is a super surface in the six-dimensional space of stresses.
The yield surface is usually convex and the state of stress of inside the yield surface is elastic.
When the stress state lies on the surface the material is said to have reached its yield point and the material is said to have become plastic.
Further deformation of the material causes the stress state to remain on the yield surface, even though the shape and size of the surface may change as the plastic deformation evolves.

Only the von Mises and Tresca yield criterion is considered here.
There are many others including that of Drucker and the Gurson model for porous materials.
\begin{figure}[!htp]
\centering\scalebox{0.6}{\includegraphics{yield_surface_2D.pdf}}
\caption{The von Mises and Tresca yield surfaces in principal stress coordinates.}
\label{Fig:YieldSurface3D}
\end{figure}

\begin{figure}[!htp]
\centering\scalebox{0.5}{\includegraphics{yield_surface_3D.pdf}}
\caption{The von Mises and Tresca yield surfaces in principal stress coordinates.}
\label{Fig:YieldSurface2D}
\end{figure}

Let $f$ be a yield function such that $f = 0$ is our yield criterion. Then:

1.Yield is independent of the hydrostatic stress.
Since $f$ is independent of hydrostatic stress, it must be expressible in terms of the deviatoric stresses alone.

2.Yield in polycrystalline metals can be taken to be isotropic (provided we are concerned with yield in a volume of material containing many grains) and must therefore be independent of the labelling of the axes.

3.Yield stresses measured in compression have the same magnitude as yield stresses measured in tension.

The von Mises yield function is defined by
\begin{equation}
f = \sigma_e-\sigma_y =\sqrt{3J_2}-\sigma_y
\end{equation}

The von Mises yield criterion suggests that the yielding of materials begins when the second deviatoric stress invariant $J_2$ reaches a critical value.
For this reason, it is sometimes called the $J_2$-plasticity or $J_2$ flow theory.
The yield criterion is given by
\[f<0\rm{: Elastic\ deformation}\]
\[f=0\rm{: Plastic\ deformation}\]

%Let us consider the yield function in two-dimensional principal stress space by putting $\sigma_3 = 0$ and so imposing conditions of plane stress.
%Geometrically, this corresponds to finding the intersection between the von Mises cylinder and the plane $\sigma_3 = 0$.



\subsection{Flow rule}
The general mathematical treatment of the constitutive equation for plastic deformation was proposed by von Mises in 1928.
He noticed that in elasticity theory, the strain tensor was related to the stress through an elastic potential function, the complementary strain energy $U$ such that
\begin{equation}
{\bm{\upepsilon }} = \frac{{\partial U}}{{\partial {\bm{\sigma }}}}
\end{equation}

By generalizing and applying this idea to plasticity theory, Mises proposed that there exists a plastic potential $g(\bm{\sigma})$, such that the plastic strain rate $\dot{{\bm{\upepsilon }}^p}$ could be derived from the following flow rule:
\begin{equation}
\dot{{\bm{\upepsilon }}^p} = \dot\lambda \frac{\partial g}{\partial \bm{\sigma}}
\label{Equ:dgdsigma}
\end{equation}
where $\dot\lambda$ is a proportional positive scalar factor which can be determined from the yield criterion. Plasticity theory based on the above flow rule is called plastic potential theory.
The following remarks should be noted about the above flow rule:

Geometrically, the plastic potential $g(\bm{\sigma})=0$, represents a surface in the stress space and $\dot{{\bm{\upepsilon }}^p}$ can be represented by a vector in this space.
The plastic strain rate vector is normal to $g(\bm{\sigma})=0$.
Therefore, Equation \ref{Equ:dgdsigma} is also referred to as the normality flow rule in plasticity theory.

Common approach in plasticity theory is to assume that the plastic potential function $g(\bm{\sigma})$ is the same as the yield function $f(\bm{\sigma})$.
It is typically assumed that the plastic strain increment and the normal to the yield surface have the same direction, so that the flow rule can be written as:

%In metal plasticity, the assumption that the plastic strain increment and deviatoric stress tensor have the same principal directions is encapsulated in a relation called the flow rule.

\begin{equation}
\dot{{\bm{\upepsilon }}^p} = \dot\lambda \frac{\partial f}{\partial \bm{\sigma}}
\end{equation}
This is called the associated flow rule.
On the other hand, if $g \neq f$, the flow rule is called nonassociated.
In general, experimental observations show that inelastic deformation of metals can be characterized quite well by an associated flow rule, but for some porous materials a nonassociated flow rule provides a better representation of inelastic deformation.

The above flow rule is easily justified for perfectly plastic deformations for which $d\boldsymbol{\sigma} = 0$  when $d{{\bm{\upepsilon }}^p} > 0$, i.e., the yield surface remains constant under increasing plastic deformation. This implies that the increment of elastic strain is also zero, $d{{\bm{\upepsilon }}^e} = 0$, because of Hooke's law.
Therefore,
\begin{equation}
d\boldsymbol{\sigma}:\frac{\partial f}{\partial \boldsymbol{\sigma}} = 0 \quad \text{and} \quad d\boldsymbol{\sigma}:d{{\bm{\upepsilon }}^p} = 0 \,.
\end{equation}
Hence, both the normal to the yield surface and the plastic strain tensor are perpendicular to the stress tensor and must have the same direction.
For a work hardening material, the yield surface can expand with increasing stress.
We assume Drucker's second stability postulate which states that for an infinitesimal stress cycle this plastic work is positive, i.e.,
\begin{equation}
d\boldsymbol{\sigma}: d{{\bm{\upepsilon }}^p} \ge 0 \,.
\end{equation}
The above quantity is equal to zero for purely elastic cycles.
Examination of the work done over a cycle of plastic loading-unloading can be used to justify the validity of the associated flow rule.
The Prager consistency condition is needed to close the set of constitutive equations and to eliminate the unknown parameter $d\lambda$ from the system of equations.
The consistency condition states that $df = 0$  at yield because  $f(\boldsymbol{\sigma},{{\bm{\upepsilon }}^p}) = 0$ , and hence
\begin{equation}
df = \frac{\partial f}{\partial \boldsymbol{\sigma}}:d\boldsymbol{\sigma} + \frac{\partial f}{\partial \boldsymbol{\varepsilon}_p}:d\boldsymbol{\varepsilon}_p = 0 \,.
\end{equation}

\subsection{Hardening rule}
The hardening rule describes how the yield surface changes in position and size.
Evolution equations of the back stress tensor allows by the change in position of the kinematic hardening into account.
With this shift in the yield surface, the size and shape of the yield surface remain unchanged.

In material models in which isotropic hardening is depicted, it is done by evolution equations describing an increase in the radius $k$ of the yield surface.
A modeling that represents isotropic hardening is not suitable for cyclic stresses.
In such cases, the yield surface is expanded as far as it finally only purely elastic state occurs (elastic "Shake Down").
The differences of the material models described in the next section consist kinematic hardening only in the respectively used hardening rules to take into account.


\section{Mechanisms of heat transfer}

%\subsection{Mechanisms of heat transfer}
Heat transfer describes the exchange of thermal energy and it is normally from a high temperature object to a lower temperature object.
%The exchange of kinetic energy of particles through the boundary between two systems which are at different temperatures from each other or from their surroundings.
%Heat transfer always occurs from a region of high temperature to another region of lower temperature.
Heat transfer changes the internal energy of both systems involved according to the First Law of Thermodynamics.
And the Second Law of Thermodynamics defines the concept of thermodynamic entropy, by measurable heat transfer.
Thermal equilibrium is reached when all involved bodies and the surroundings reach the same temperature.
%Thermal expansion is the tendency of matter to change in volume in response to a change in temperature.
The fundamental modes of heat transfer are conduction, convection and radiation.
In the engineering sciences, heat transfer includes the processes of thermal conduction, radiation, convection, and sometimes mass transfer.
Usually more than one of these processes occurs in a given situation as shown in \ref{Fig:HeatTransfer}.
% shows a high temperature object cooling in air,
%Common case is transfer of heat by a combination of the three modes.
\begin{figure}[!htp]
\centering\scalebox{0.8}{\includegraphics{Heat_Convection.pdf}}
\caption{The fundamental modes of heat transfer.}
\label{Fig:HeatTransfer}
\end{figure}

\subsection{Thermal conduction}
Thermal conduction is the transfer of internal energy by microscopic diffusion and collisions of particles or quasi-particles within a body.
%The microscopically diffusing and colliding objects include molecules, atoms, and electrons.
%They transfer disorganized microscopic kinetic and potential energy, which are jointly known as internal energy.
Conduction can only take place within an object or material, or between two objects that are in contact with each other.
%Conduction takes place in all phases of ponderable matter, such as solids, liquids, gases and plasmas, but it is distinctly recognizable only when the matter is undergoing neither chemical reaction nor differential local internal flows of distinct chemical constituents.
%In the presence of such chemically defined contributory sub-processes, only the flow of internal energy is recognizable, as distinct from thermal conduction. When the processes of conduction yield a net flow of energy across a boundary because of a temperature gradient, the process is characterized as a flow of heat.
%Heat spontaneously flows from a hotter to a colder body. In the absence of external drivers, temperature differences decay over time, and the bodies approach thermal equilibrium.

%In conduction, the heat flow is within and through the body itself.
%In contrast, in heat transfer by thermal radiation, the transfer is often between bodies, which may be separated spatially.
%Also possible is transfer of heat by a combination of conduction and thermal radiation.
%In convection, internal energy is carried between bodies by a material carrier. In solids, conduction is mediated by the combination of vibrations and collisions of molecules, of propagation and collisions of phonons, and of diffusion and collisions of free electrons.
%In gases and liquids, conduction is due to the collisions and diffusion of molecules during their random motion. Photons in this context do not collide with one another, and so heat transport by electromagnetic radiation is conceptually distinct from heat conduction by microscopic diffusion and collisions of material particles and phonons. In condensed matter, such as a solid or liquid, the distinction between conduction and radiative transfer of heat is clear in physical concept, but it is often not phenomenologically clear, unless the material is semi-transparent.

%In the engineering sciences, heat transfer includes the processes of thermal radiation, convection, and sometimes mass transfer. Usually more than one of these processes occurs in a given situation. The conventional symbol for the material property, thermal conductivity, is $k$.

%The law of heat conduction, also known as Fourier's law, states that the time rate of heat transfer through a material is proportional to the negative gradient in the temperature and to the area, at right angles to that gradient, through which the heat flows.

An empirical relationship between the conduction rate in a material and the temperature gradient in the direction of energy flow, first formulated by Fourier who concluded that the heat flux resulting from thermal conduction is proportional to the magnitude of the temperature gradient and opposite to it in sign.
For a three directional conduction, the differential form of Fourier's Law of thermal conduction may be expressed as:
\begin{equation}
{\bf{q}} =  - k\nabla T
\end{equation}
where $\bf{q}$ is the local heat flux density(W/m$^2$), $k$ is the material's conductivity(W/mK), $\nabla T$ is the temperature gradient(K/m).
The heat flux density is the amount of energy that flows through a unit area per unit time.
The thermal conductivity, k, is often treated as a constant, though this is not always true.
While the thermal conductivity of a material generally varies with temperature, the variation can be small over a significant range of temperatures for some common materials.
In anisotropic materials, the thermal conductivity typically varies with orientation; in this case k is represented by a second-order tensor. In non uniform materials, k varies with spatial location.

\subsection{Thermal convection}

Convective heat transfer, often referred to simply as convection, is the transfer of heat from one place to another by the movement of fluids.
Two types of convective heat transfer may be distinguished: free convection and forced convection.
Free convection describes when fluid motion is caused by buoyancy forces that result from the density variations due to variations of thermal temperature in the fluid.
Forced convection describes when a fluid is forced to flow over the surface by an external source such as fans, by stirring, and pumps, creating an artificially induced convection current.

    \begin{figure}[!htp]
        \centering\scalebox{0.8}{\includegraphics{Thermal_boundary_layer.pdf}}
        \caption{Thermal boundary layer.}
        \label{Thermal_boundary_layer}
    \end{figure}

Convection is sometimes assumed to be described by Newton's law.
Newton's law, which requires a heat transfer coefficient, states that the rate of heat loss of a body is proportional to the difference in temperatures between the body and its surroundings. %The constant of proportionality is the heat transfer coefficient.
As shown in \ref{Thermal_boundary_layer}, the basic relationship for heat transfer by convection is:
%The heat transfer rate can be written as,
    \begin{equation}
    q_y=hA(T_s-T_{\infty})
    \end{equation}
where $q_y$ is the heat transferred per unit time, $h$ is the heat transfer coefficient, $T_s$ is the object's surface temperature and $T_{\infty}$ is the fluid temperature.

%And because heat transfer at the surface is by conduction,
%    \begin{equation}
%    q_y=-kA \frac{\partial}{\partial y} (T-T_{s})|_{y=0}
%    \end{equation}
%These two terms are equal; thus
%    \begin{equation}
%    -kA \frac{\partial}{\partial y} (T-T_{s})|_{y=0} = hA(T_s-T_{\infty})
%    \end{equation}
%Rearranging,
%    \begin{equation}
%    \frac{h}{k}=\frac{\frac{\partial (T_s-T)}{\partial y}|_{y=0}}{\frac{T_s-T_{\infty}}{L}}
%    \end{equation}
%Making it dimensionless by multiplying by representative length L,
%    \begin{equation}
%    \frac{hL}{k}=\frac{\frac{\partial (T_s-T)}{\partial y}|_{y=0}}{T_s-T_{\infty}}
%    \label{tab:Nusselt1}
%    \end{equation}


%Convective heat transfer is one of the major modes of heat transfer and convection is also a major mode of mass transfer in fluids.  Therefore, influence of fluid field factors will impact the convective heat transfer. There are five aspects:
%
%
%1.Natural convection and forced convection.
%
%
%$\bullet$ Natural Convection.
%
%The onset of natural convection is determined by the Rayleigh number (Ra). This dimensionless number is given by:
%
%    \begin{equation}
%        Ra=\frac{\Delta \rho g L^3}{D \mu}
%        \label{tab:RayleighNumber}
%    \end{equation}
%
%where
%
%    $\Delta \rho$ is the difference in density between the two parcels of material that are mixing,\par
%    g is the local gravitational acceleration,\par
%    L is the characteristic length-scale of convection,\par
%    D is the diffusivity of the characteristic that is causing the convection,\par
%    $\mu$ is the dynamic viscosity.\par
%
%$\bullet$ Forced convection.
%
%In general, natural convection flow rate is low, the natural convection heat transfer coefficient is usually lower than the forced convection heat transfer coefficient.
%
%2.Laminar flow and turbulent flow.
%
%$\bullet$ Limina flow.
%
%Laminar flow occurs when a fluid flows in parallel layers, with no disruption between the layers.In the perpendicular direction, the heat transfer depends mainly on molecular diffusion (ie, thermal conductivity).
%
%$\bullet$ Turbulent flow.
%
%There are strong pulsation and vortex. The fluids between various parts mix rapidly, so turbulent convection effect is stronger than laminar flow.
%
%3.Phase transition.
%
%$\bullet$ Boiling.
%
%$\bullet$ Condensation.
%
%4.The physical properties of the fluid.
%
%$\bullet$ Thermal conductivity k,$[W/(m \cdot K)]$. If the fluid thermal conductivity is larger, thermal resistance become smaller. Convective heat transfer is more intense.
%
%$\bullet$ Density $\rho$, $[kg/m^3]$.
%
%$\bullet$ Specific heat capacity c,$[J/(kg \cdot K)]$.
%
%$\rho c$ reflects the size of the heat capacity of unit volume of fluid. The greater its value is, the more heat is transferred by convection, and the convective heat transfer is more intense.
%
%
%$\bullet$ Dynamic viscosity $\eta$, $[Pa \cdot s]$. Kinematic viscosity $\nu=\eta /\rho$, $[m^2/s]$. The viscosity of the fluid affect the velocity distribution and flow pattern, so it affects the convective heat transfer.
%
%$\bullet$ Volume expansion coefficient $\alpha_V$, $[K^{-1}]$. In the general case of a gas, liquid, or solid, the volumetric coefficient of thermal expansion is given by
%
%    \begin{equation}
%        \alpha_V=\frac{1}{V}\left(\frac{\partial V}{\partial T}\right)_p
%        \label{tab:ExpansionCoefficient}
%    \end{equation}
%
%For an ideal gas, the volumetric thermal expansivity (i.e. relative change in volume due to temperature change) depends on the type of process in which temperature is changed. Two known cases are isobaric change, where pressure is held constant, and adiabatic change, where no work is done and no change in entropy occurs.
%
%In an isobaric process, the volumetric thermal expansivity, which we denote $\beta_p$,(The index p denotes an isobaric process) is:
%
%    \begin{equation}
%        PV = nRT
%    \end{equation}
%    \begin{equation}
%        \ln\left(V\right) = \ln \left(T\right) + \ln\left(nR/P\right)
%    \end{equation}
%    \begin{equation}
%        \beta_p = \bigg(\frac{1}{V} \frac{dV}{dT}\bigg)_p = \bigg(\frac{d(ln V)}{d T}\bigg)_p = \frac{d(ln T)}{d T} = \frac{1}{T}
%    \end{equation}
%
%5.The geometric factors of the heat transfer surface.
%
%Geometry, size, relative position, roughness and other geometric factors will affect the flow of fluid, and therefore affect the fluid velocity distribution and the convective heat transfer. The surface heat transfer coefficient is a function of many variables.


\subsection{Thermal radiation}
Thermal radiation is the emission of electromagnetic waves from all matter that has a temperature greater than absolute zero.
It represents a conversion of thermal energy into electromagnetic energy.
All matter with a temperature greater than absolute zero emits thermal radiation. When the temperature of the body is greater than absolute zero, inter-atomic collisions cause the kinetic energy of the atoms or molecules to change. This results in charge-acceleration and/or dipole oscillation which produces electromagnetic radiation, and the wide spectrum of radiation reflects the wide spectrum of energies and accelerations that occur even at a single temperature.

Thermal radiation power of a black body per unit area of radiating surface per unit of solid angle and per unit frequency $\nu$ is given by Planck's law as:
\begin{equation}
u(\nu,T)=\frac{2 h\nu^3}{c^2}\cdot\frac1{e^{h\nu/k_BT}-1},
\end{equation}
or in terms of wavelength
\begin{equation}
u(\lambda,T)=\frac{\beta}{\lambda^5}\cdot\frac1{e^{hc/k_BT\lambda}-1},
\end{equation}
where $h=6.626\times10^{-34}J \cdot s$ is Planck's constant, $b=2.898\times10^{-3}m \cdot K$ is Wien's displacement constant, $k_{B}=1.381\times10^{-23}J \cdot K^{-1}$ is Boltzmann constant and $c$ is the speed of light.

This formula mathematically follows from calculation of spectral distribution of energy in Quantization (physics)|quantized electromagnetic field which is in complete thermal equilibrium with the radiating object. The equation is derived as an infinite sum over all possible frequencies. The energy, $E=h \nu$, of each photon is multiplied by the number of states available at that frequency, and the probability that each of those states will be occupied.

Integrating the above equation over $\nu$ the power output given by the Stefan�CBoltzmann law is obtained, as:
\begin{equation}
P = \sigma \cdot A \cdot T^4,
\label{Equ:IntegratingOfStefanBoltzmann}
\end{equation}
where the constant of proportionality $\sigma=5.670\times10^{-8}\rm{W}\cdot \rm{m}^{-2}\cdot \rm{K}^{-1}$ is the Stefan�CBoltzmann constant and $A$ is the radiating surface area.

The wavelength $\lambda \,$, for which the emission intensity is highest, is given by Wien's displacement law as
\begin{equation}
\lambda_{max} = \frac{b}{T}.
\end{equation}
For surfaces which are not black bodies, one has to consider the (generally frequency dependent) emissivity factor $\epsilon(\nu)$. This factor has to be multiplied with the radiation spectrum formula before integration. If it is taken as a constant, the resulting formula for the power output can be written in a way that contains $\epsilon$ as a factor:
\begin{equation}
P = \epsilon \cdot \sigma \cdot A \cdot T^4.
\end{equation}

This type of theoretical model, with frequency-independent emissivity lower than that of a perfect black body, is often known as a 'grey body'. For frequency-dependent emissivity, the solution for the integrated power depends on the functional form of the dependence, though in general there is no simple expression for it. Practically speaking, if the emissivity of the body is roughly constant around the peak emission wavelength, the gray body model tends to work fairly well since the weight of the curve around the peak emission tends to dominate the integral.

\subsection{Temperature Measurement}
\subsubsection{Thermocouple}
A thermocouple is a sensor for measuring temperature.
It consists of two dissimilar metal wires, joined at one end.
%When properly configured, thermocouples can provide temperature measurements over a wide range of temperatures.
In contrast to most other methods of temperature measurement, thermocouples are self powered and require no external form of excitation.
The physical principle of the thermocouple is Seebeck effect.

\begin{figure}[!htp]
\centering\scalebox{1.0}{\includegraphics{Seebeck_Voltage.pdf}}
\caption{Schematic of Seebeck effect.}
\label{Fig:Seebeck_Voltage}
\end{figure}
In 1821 the German physicist Thomas Johann Seebeck discovered the continuous current flow in the thermoelectric circuit when two wires of dissimilar metals are joined at both ends and one of the ends is heated, and that is the Seebeck effect.
If this circuit is broken at the center (see \ref{Fig:Seebeck_Voltage}), the net open circuit voltage $e_{AB}$ (the Seebeck voltage) is a function of the junction temperature and the composition of the two metals.
This was because the electron energy levels in each metal shifted differently and a voltage difference between the junctions.
We can't measure the Seebeck voltage directly because we must first connect a voltmeter to the thermocouple, and the voltmeter leads themselves create a new thermoelectric circuit.

A thermocouple is an electrical device consisting of two different conductors forming electrical junctions at differing temperatures. A thermocouple produces a temperature-dependent voltage as a result of the thermoelectric effect, and this voltage can be interpreted to measure temperature.
The thermoelectric effect is the direct conversion of temperature differences to electric voltage and vice versa. A thermoelectric device creates voltage when there is a different temperature on each side. Conversely, when a voltage is applied to it, it creates a temperature difference. At the atomic scale, an applied temperature gradient causes charge carriers in the material to diffuse from the hot side to the cold side.
\begin{figure}[!htp]
\centering\scalebox{1.2}{\includegraphics{Thermocouple_circuit_Ktype_including_voltmeter_temperature.pdf}}
\caption{K-type thermocouple in the standard thermocouple measurement configuration.}
\label{Fig:Thermocouple_circuit_Ktype_including_voltmeter_temperature}
\end{figure}

\subsubsection{Thermocouple direct and inverse polynomials}
Once we obtain the voltage from the thermocouple we have to convert the voltage to a temperature.
As shown in \ref{Fig:Thermocouples}, output voltages for the type K, R and S thermocouples are plotted as a function of temperature.
Unfortunately, the temperature-versus-voltage relationship of the thermocouple is not linear.
\begin{figure}[!htp]
\centering\scalebox{0.5}{\includegraphics{Thermocouples.pdf}}
\caption{Characteristic functions for thermocouples types K, R, S.}
\label{Fig:Thermocouples}
\end{figure}
Therefore, polynomials are used to present the temperature-versus-voltage and the inverse relationship.
Direct polynomials (see Equation \ref{Equ:Direct_polynomials}) provide the thermoelectric voltage from a known temperature and inverse polynomials (see Equation \ref{Equ:Inverse_polynomials}) provide the temperature from a known thermoelectric voltage.
\begin{equation}
\label{Equ:Direct_polynomials}
E = {c_0} + {c_1}T + {c_2}{T^2} + ... + {c_n}{T^n} = \sum\limits_{i = 0}^n {{c_i}{T^i}}
\end{equation}
\begin{equation}
\label{Equ:Inverse_polynomials}
T = {k_0} + {k_1}E + {k_2}{E^2} + ... + {k_n}{E^n} = \sum\limits_{i = 0}^n {{k_i}{E^i}}
\end{equation}
Where, $T$ is temperature in degrees Celsius ($^{\circ} \rm{C}$), $E$ is thermocouple electric potential in microvolts (${\rm{\mu V}}$), $c_i,k_i$ are polynomial coefficients unique to each type thermocouple, $n$ is the maximum order of the polynomial.
As n increases, the accuracy of the polynomial improves.
A representative number is $n = 9$ for $\pm 1 ^{\circ} \rm{C}$ accuracy.

In order to obtain the higher accuracy and system speed, the polynomials may be divided into several small sectors over a narrow temperature range.
In the software for our data acquisition system, the thermocouple characteristic curve is divided into three sectors, and each sector is approximated by a nine or ten order
polynomial.
%as well as used in our temperature measurement system.
For example, type R thermocouples coefficients of the approximate direct/inverse polynomials are given in Table \ref{tab:TypeRDirectPolynomial} and \ref{tab:TypeRInversePolynomial}.
%Further more, the similar coefficients wtype K and type S thermocouples are also used in our temperature measurement system.
% Table generated by Excel2LaTeX from sheet 'R'

Three types (K, R, S) of thermocouples are used in our temperature measurement system for different temperature ranges.

\begin{table}[htbp]
  \centering
  \small
  \caption{Direct polynomial coefficients of type R thermocouple.}
    \begin{tabular}{llll}
    \toprule
    Range & -50 to 1064.18$^{\circ}$C & 1064.18 to 1664.5$^{\circ}$C & 1664.5 to 1768.1$^{\circ}$C \\
    \midrule
    $c_0=$ & 0     & 2.95157925316$\times10^{3}$ & 1.52232118209$\times10^{5}$ \\
    $c_1=$ & 5.289617298 & -2.520612513 & -2.68819888545$\times10^{2}$ \\
    $c_2=$ & 1.39166589782$\times10^{-2}$ & 1.59564501865$\times10^{-2}$ & 1.71280280471$\times10^{-1}$ \\
    $c_3=$ & -2.38855693017$\times10^{-5}$ & -7.64085947576$\times10^{-6}$ & -3.45895706453$\times10^{-5}$ \\
    $c_4=$ & 3.56916001063$\times10^{-8}$ & 2.05305291024$\times10^{-9}$ & -9.34633971046$\times10^{-12}$ \\
    $c_5=$ & -4.62347666298$\times10^{-11}$ & -2.93359668173$\times10^{-13}$ &  \\
    $c_6=$ & 5.00777441034$\times10^{-14}$ &       &  \\
    $c_7=$ & -3.73105886191$\times10^{-17}$ &       &  \\
    $c_8=$ & 1.57716482367$\times10^{-20}$ &       &  \\
    $c_9=$ & -2.81038625251$\times10^{-24}$ &       &  \\
    \bottomrule
    \end{tabular}%
  \label{tab:TypeRDirectPolynomial}%
\end{table}%

% Table generated by Excel2LaTeX from sheet 'R'
\begin{table}[htbp]
  \centering
  \small
  \caption{Inverse polynomial coefficients of type R thermocouple.}
    \begin{tabular}{lllr}
    \toprule
    Temperature range & -50 to 250$^{\circ}$C & 250 to 1200$^{\circ}$C & \multicolumn{1}{l}{1064 to 1664.5$^{\circ}$C} \\
    \midrule
    Voltage range & -226 to 1923$\mu$V & 1923 to 13228$\mu$V & \multicolumn{1}{l}{11361 to 19739$\mu$V} \\
    \midrule
    $k_{0}=$ & 0.0000000 & 1.334584505$\times 10^{1}$ & \multicolumn{1}{l}{-8.199599416$\times 10^{1}$} \\
    $k_{1}=$ & 1.8891380$\times 10^{-1}$ & 1.472644573$\times 10^{-1}$ & 1.553962042$\times 10^{-1}$ \\
    $k_{2}=$ & -9.3835290$\times 10^{-5}$ & -1.844024844$\times 10^{-5}$ & -8.342197663$\times 10^{-6}$ \\
    $k_{3}=$ & 1.3068619$\times 10^{-7}$ & 4.031129726$\times 10^{-9}$ & 4.279433549$\times 10^{-10}$ \\
    $k_{4}=$ & -2.2703580$\times 10^{-10}$ & -6.249428360$\times 10^{-13}$ & -1.19157791$\times 10^{-14}$ \\
    $k_{5}=$ & 3.5145659$\times 10^{-13}$ & 6.468412046$\times 10^{-17}$ & 1.492290091$\times 10^{-19}$ \\
    $k_{6}=$ & -3.8953900$\times 10^{-16}$ & -4.458750426$\times 10^{-21}$ & \multicolumn{1}{l}{} \\
    $k_{7}=$ & 2.823.9471$\times 10^{-19}$ & 1.994710146$\times 10^{-25}$ & \multicolumn{1}{l}{} \\
    $k_{8}=$ & -1.2607281$\times 10^{-22}$ & -5.313401790$\times 10^{-30}$ & \multicolumn{1}{l}{} \\
    $k_{9}=$ & 3.1353611$\times 10^{-26}$ & 6.481976217$\times 10^{-35}$ & \multicolumn{1}{l}{} \\
    $k_{10}=$ & -3.3187769$\times 10^{-30}$ &       & \multicolumn{1}{l}{} \\
    \midrule
    Error & 0.02 to -0.02$^{\circ}$C & 0.005 to -0.005$^{\circ}$C & \multicolumn{1}{l}{0.001 to -0.0005$^{\circ}$C} \\
    \bottomrule
    \end{tabular}%
  \label{tab:TypeRInversePolynomial}%
\end{table}% 

\section{ABAQUS user interface UMAT}
ABAQUS user subroutines are usually used to extend the ABAQU FEM solver to include user-defined functionalities not implemented in the ABAQUS source code.
The UMAT (User MATerial)interface provided by the ABAQUS solver for user-defined material laws and plasticity models is described in detail in \cite{abaqus20106}. In the following, we briefly discuss important variables and functions.

A UMAT routine is called at least once per time step during an FEM calculation for each integration point.
Inputs of the UMAT routine are the stress-strain state at the current time, all state variables and constants of the plasticity model, and the strain increment to be applied in the current time step.
Within the UMAT routine, the constitutive equations of the plasticity model are solved in the form of a differential equation system using a corresponding numerical solution method.
Subsequently, the new stress-strain state and the updated state variables are returned to the ABAQUS solver.
If the expansion increment is too large, and a solution of the constitutive equations with the necessary accuracy is not possible, the variable PNEWDT is returned to a new smaller step.
A UMAT routine thus operates on an incremental basis.
When it is viewed as a "black box," the corresponding conversion increment is simply expressed for each expansion increment entered into the UMAT routine:

\begin{equation}
\Delta \sigma_{ij}=UMAT(\Delta \varepsilon_{ij}).
\end{equation}

Since the present UMAT routines are programmed in FORTRAN and are not based on internal functions of the ABAQUS solver, it is possible to use these UMAT routines and thus also the included plasticity models "stand-alone" without the ABAQUS solver in a separate FORTRAN based environment.
It is only necessary to ensure that the plasticity models receive the necessary input variables via the UMAT interface and the existing status variables and output variables are buffered accordingly.
%This possibility is used here for multi-axial non-proportional notch strain simulation.
Significant variables of the ABAQUS-UMAT interface used are listed in \ref{tab:ABAQUS-UMAT_interface}.

\begin{table}[htbp]
  \centering
  \caption{Used variables of the ABAQUS-UMAT interface}
    \begin{tabular}{p{2cm}p{2cm}p{2cm}p{8cm}}
    \toprule
    Variable & Type  & Dimension & Description \\
    \midrule
    NDI   & INTEGER & scalar & Number of direct stress components \\
    NSHR  & INTEGER & scalar & Number of engineering shear stress components \\
    NTENS & INTEGER & scalar & Size of the stress or strain component array (NDI+NSHR) \\
    NSTATV & INTEGER & scalar & Number of solution-dependent state variables \\
    NPROPS & INTEGER & scalar & User-defined number of material constants associated with this user material \\
    DTIME & DOUBLE & scalar & Time increment $\Delta t$ \\
    TEMP  & DOUBLE & scalar & Temperature at the start of the increment $T_n$ \\
    DTEMP & DOUBLE & scalar & Increment of temperature $\Delta T$ \\
    PNEWDT & DOUBLE & scalar & Ratio of suggested new time increment to the time increment being used \\
    STRAN & DOUBLE & (NTENS) & Total mechanical strains at the beginning of the increment $\bm{\upepsilon}_n$ \\
    DSTRAN & DOUBLE & (NTENS) & Mechanical strain increments $\Delta \bm{\upepsilon}$ \\
    DDSDDE & DOUBLE & (NTENS, NTENS) & Jacobian matrix of the constitutive model $\partial\Delta {\bm{\upsigma }} /\partial\Delta {\bm{\upepsilon }}$ \\
    STRESS & DOUBLE & (NTENS) & Stress tensor at the end of the increment $\bm{\upsigma}_{n+1}$ \\
    STATEV & DOUBLE & (NSTATV) & Internal state variables of the model \\
    PROPS & DOUBLE & (NPROPS) & Material constants and model parameters \\
    DFGRD0 & DOUBLE & (3,3) & Deformation gradient at the beginning of the increment $\bm{F}_n$ \\
    DFGRD1 & DOUBLE & (3,3) & Deformation gradient at the end of the increment $\bm{F}_{n+1}$ \\
    \bottomrule
    \end{tabular}%
  \label{tab:ABAQUS-UMAT_interface}%
\end{table}%

%\section{Heat and heat transfer}
%The internal, thermal energy of a body or of a component is referred to as its "heat" in general terms and in a manner of speaking. In fact, heat is the thermal energy, the exchange of which is preceded by the temperature difference alone. The thermal energy of a body, which is stored as a kinetic energy in the non-directed particle motion at the molecular level, is changed by heat input and exhaust.
%The storage capacity for thermal energy of a substance is expressed by its specific heat capacity $c$.
%This characteristic value indicates the thermal energy related to the mass, which is necessary to heat up the material by 1 K.
%The heat capacity depends on the temperature.
%The relationship between mass and volume of the substance is described by the density $\rho$.
%As a result of the thermal expansion of a material, its density normally decreases with increasing temperature.
%Heat transfer is carried out by heat transport.
%Heat transport always takes place at right angles to the levels of the same temperature (isotherms) in the direction of the negative temperature gradient.
%Under constant boundary conditions, the temperature field approaches a stationary, stable state with increasing time.
%The case of the stationary state thus denotes a temperature field that no longer has any temporal dependence.
%According to Baehr and Stephan [8], three types of heat transport can be distinguished: heat conduction, heat radiation and convection.
%These species may occur isolated or combined.
%
%\subsection{Heat conduction}
%
%The energy transport between adjacent molecules of a system is referred to as heat conduction.
%The relevant parameter for this mechanism is the thermal conductivity k of the material.
%For simplified calculations, this characteristic value is often defined as being independent of temperature, however, the value of the thermal conductivity is to be regarded as temperature-dependent.
%For thermally isotropic material behavior, this scalar proportionality factor describes the relationship between the gradient of the scalar temperature field $T$ and the heat flux in the heat conduction law according to Fourier \cite{fourier1822theorie}:
%
%
%\begin{equation}
%\dot{q} =  - k\nabla T
%\end{equation}
%
%
%The minus sign in equation (2.1) takes account of the fact that the heat flow points in the direction of the negative temperature gradient.
%
%\subsection{Heat radiation}
%Heat transport by electromagnetic waves is called heat radiation. In contrast to the other types of heat transport, heat radiation can also take place in a vacuum, so no medium needs to be used. The heat flow due to heat radiation is according to Stefan [61] and Boltzmann [11]:
%
%
%\subsection{Convection}
%
%If the heat transport is carried out by transfer from one medium to another as a result of particle movement at the macroscopic level, this is referred to as a convective heat transfer. The characteristic feature here is that the heat transport takes place at a boundary layer and with the participation of a flowing medium (fluid or gas).
%
%Due to the difference between the temperatures of the flowing medium $T_{\infty}$ and the body $T_s$ under consideration, a heat flow occurs.
%The heat transfer coefficient is the defined proportionality factor by which the relationship between the temperature difference and the heat flow density is described:
%
%Since the heat transfer coefficient as a single characteristic covers a large number of influences which characterize the processes in the boundary layer flow between the medium and the body (viscosity, flow properties, etc.), these values are usually either determined experimentally or recalculated from fluid dynamics by approximation methods from the heat flow.
%The heat transfer coefficients can be exported directly as a time- or temperature-dependent variable from flow-mechanical calculations.
%In order to recalculate these transition coefficients from the thermal current density determined in a flow simulation, knowledge of the temperature-dependent material parameters of the materials involved is necessary.
%For turbulent heat transitions, case-related correlations exist (eg according to Dittus and Boelter [17] for turbulent tube flows) which are used for these purposes.
%
%In the case of natural convection due to the free flow of water to steel, ecological values of 100 w / m2K to 600 w / m2K are customary for the transition coefficient. For forced convection, the flow velocity is determined by external influences. In these cases, the transition coefficient can increase to several tens of thousands of w / m2K. Even higher values are possible if condensation and evaporation occurs in the region of the boundary layer. More detailed information can be found in Herwig [32]. Due to the dependency of the transition coefficient on both the temperature and the temperature difference, the thermal current density in this case is not proportional to the temperature difference. For forced convection, the heat transfer coefficient can be regarded as independent of the temperature difference (see Baehr and Stephan [8]).
%
%For applications in which both convection and heat radiation have to be taken into account, the heat transfer coefficient uk is often modified to cover both effects [8].
%
%For the calculations carried out in this work, the differential equation of the heat conduction (2.1) is solved using the FE program package Ansys. The thermal boundary conditions are defined by convection boundary conditions. The calculation procedure for the determination of the temperature field solution is presented in chapter 3 in detail.



%\section{Plasticity}
%
%If the stress state of a material exceeds a limit state, a disproportionate increase in the strain occurs.
%This limit state is usually formulated via a flow condition set with the stress tensor.
%If it is fulfilled by the current stress tensor, plastic strains occur.
%If this is fulfilled by the current stress tensor, plastic strains occur.
%These plastic strains are described by material models which represent the elastic-plastic deformation behavior by suitable flow and hardening rules.
%A basic assumption is that for small strains and isotropic material, the strain tensor can be additively split into an elastic and plastic part:
%
%Even in the presence of plastic strains, the elastic part of the strain tensor also follows Hooke's law:
%
%For isotropic material, the elastic tensor is dependent on two free parameters, the Lame constants.
%The relationship between them and the Young's modulus E and Poisson's ratio u by:
%
%The components of the elastic tensor are
%
%\section{Yield criterion}
%The yield criterion is defined as the limit state of stress from which plastic material behavior occurs.
%For the material models used in this work, the yield criterion of Mises \cite{mises1928mechanik} is used.
%It is formulated as a function of the deviatoric stress $\bm{s}$.
%Thus, only the shape-changing part of the stress tensor has an influence on the yield behavior, the volume-changing hydrostatic part is not taken into account in the yield criterion.
%For metallic materials, this yield is considered to be usable and generally accepted.
%The deviatoric stress $\bm{s}$ is defined as:
%\begin{equation}
%\bm{s} = \bm{\upsigma}-\frac{1}{3}\rm{tr}(\bm{\upsigma}),
%\end{equation}
%and the yield criterion according to von Mises is:
%\begin{equation}
%F(\bm{s},{\bf{a}},k) = (\bm{s}-{\bf{a}}):(\bm{s}-{\bf{a}})-k^2.
%\label{Equ:MisesCriterion}
%\end{equation}
%In the main stress space, the area defined by Equation \ref{Equ:MisesCriterion} describes a cylinder surface with a radius $k$, the center axis of which is along the hydrostatic axis (equisetrix).
%The pure deviatoric tensor $\bf{a}$ allows the consideration kinematic hardening by a shift of the middle or anchor point of the yield surface.
%The deviatoric yield stress is determined by the uniaxial yield stress according to:
%\begin{equation}
%k = \sqrt{\frac{2}{3}} \sigma_y.
%\end{equation}
%
%\section{Flow rule}
%The flow rule is used to determine the plastic strain increments.
%This is a evolution equation which establishes the relation between the plastic potential and the plastic multiplier to the increment of the plastic strain.
%If the yield criterion is the same used for the potential, it is an associated flow rule, otherwise it is called non-associated.
%The associated flow rule is:
%
%For the associated flow rule, the increments of the plastic strain can only occur in the direction of the normal tensor on the yield surface.
%This normal tensor is made up of:
%
%Correspondingly, the associated flow rule is also referred to as a normal rule. If the normalized direction tensor n in equation (2.13) is used:
%
%The plastic multiplier just corresponds to the accumulated plastic strain:
%
%
%\section{Material model}
%
%The characteristics of the material models used for numerical simulations are presented below. The first of the listed material models (after Armstrong / Frederick) is not used in the calculations carried out for this work; the descriptions of the models according to Chaboche and Ohno / Wang are the basic points defined by Armstrong / Frederick. The descriptions provide an overview of the formulation of the relevant development equations under isothermal conditions. The specifics of the implementation of the material models according to Chaboche and Ohno / Wang under thermal transient conditions are discussed in detail by Willuweit [66].
%
%\section{Armstrong and Frederick model}
%The Armstrong and Frederick model [7] considered a non-linear kinematic hardening rule.
%The evolution equation used in this model for the increment of the back stress tensor is:
%\begin{equation}
%{\rm{d}}{\bf{a}} = (h_0 \cdot {\bf{n}} - c \cdot {\bf{a}}){\rm{d}} p.
%\label{Equ:ArmstrongFrederick}
%\end{equation}
%$h_0$ corresponds to the initial value of the plastic tangent module, $c$ is a material constant.
%Transformation of Equation \ref{Equ:ArmstrongFrederick} leads to:
%\begin{equation}
%{\rm{d}}{\bf{a}} = c \cdot (r \cdot {\bf{n}} - {\bf{a}}){\rm{d}} p,
%\label{Equ:ArmstrongFrederick2}
%\end{equation}
%with
%\begin{equation}
%r=\frac{h_0}{c}.
%\end{equation}
%In this representation, $r$ corresponds to a limiting radius for the back stress tensor.
%The back stress tensor ${\bf{a}}$ is not permitted to leave the yield surface.
%The more the value of ${\bf{a}}$ approaches the boundary radius, the more the value of the plastic tangent module decreases until the stress-strain curve ends in a horizontal.
%This hardening rule can be integrated analytically integrated