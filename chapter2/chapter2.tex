\chapter{Fundamental of plasticity}

\section{Continuum plasticity}
Continuum plasticity is a solid mechanics theory that is used to describe the plastic behavior of materials.
It is characterized by the assumption that a flow rule exists that can be used to determine the amount of plastic strain in the material.
This section introduces the fundamentals of time-independent phenomenological plasticity, such as invariants of the stress tensor, strain decomposition, yield surface, flow rule, Drucker postulates, consistency condition, isotropic and kinematic hardening.

\subsection{Invariants of the stress tensor}
In continuum mechanics, the stress state at any material point can be represented by the stress tensor $\bm{\sigma}$.
The stress tensor can be represented by a matrix of second order in an arbitrary coordinate system as:
\begin{equation}
\bm{\sigma} =
  \begin{bmatrix}
    \sigma_{1\,1}& \sigma_{1\,2}& \sigma_{1\,3}\\
    \sigma_{2\,1}& \sigma_{2\,2}& \sigma_{2\,3}\\
    \sigma_{3\,1}& \sigma_{3\,2}& \sigma_{3\,3}
  \end{bmatrix}.
\end{equation}
Since the stress tensor is symmetric, only six stress components axe independent.
Thus, six independent stress components determine a stress state uniquely and visa versa.
Using the above stress tensor, the three principal stresses can be determined using the characteristic equation
\begin{equation}
\det \left( {{\bm{\sigma }} - \sigma {\bf{I}}} \right) = 0,
\end{equation}
which can be expanded to
\begin{equation}
{\sigma ^3} - {I_1}{\sigma ^2} + {I_2}\sigma  - {I_3} = 0.
\end{equation}
where $\bf{I}$ is the second order identity tensor, ${I_1}$, ${I_2}$, and ${I_3}$ are the invariants of the stress tensor.
In terms of the principal stresses, $\sigma_1$, $\sigma_2$ and $\sigma_3$, the invariants are
\begin{equation}
\begin{array}{*{20}{l}}
{{I_1} = {\sigma _1} + {\sigma _2} + {\sigma _3}},\\
{{I_2} = {\sigma _1}{\sigma _2} + {\sigma _2}{\sigma _3} + {\sigma _3}{\sigma _1}},\\
{{I_3} = {\sigma _1}{\sigma _2}{\sigma _3}}.
\end{array}
\end{equation}
In plasticity theory, it is customary to decompose the stress tensor as
\begin{equation}
{\bm{\sigma }} = {\bf{s}} + {\sigma _m}{\bf{I}},
\end{equation}
where $\sigma_m$ is the hydrostatic stress given by
\begin{equation}
{\sigma _m} = {\rm{Tr}}\left( {\bm{\sigma }} \right) = \frac{1}{3}\left( {{\sigma _1} + {\sigma _2} + {\sigma _3}} \right),
\end{equation}
and $\bf{s}$ is the deviatoric stress tensor.
Using the deviatoric stress tensor, the deviatoric stress invariants can be defined as
\begin{equation}
\begin{array}{*{20}{l}}
{{J_1} = {\rm{Tr}}\left( {\bf{s}} \right) = 0},\\
{{J_2} = \frac{1}{2}{\bf{s}}:{\bf{s}} = \frac{1}{6}\left[ {{{\left( {{\sigma _1} - {\sigma _2}} \right)}^2} + {{\left( {{\sigma _2} - {\sigma _3}} \right)}^2} + {{\left( {{\sigma _3} - {\sigma _1}} \right)}^2}} \right]},\\
{{J_3} = \det \left( {\bf{s}} \right) = {s_1}{s_2}{s_3}}.
\end{array}
\end{equation}

In plasticity mechanics, the first principal invariant $I_1$ of the Cauchy stress $\boldsymbol{\sigma}$, and the second and third principal invariants $J_2$, $J_3$ of the deviatoric part $\boldsymbol{s}$ of the Cauchy stress are usually used to express the yield surface.


\subsection{Strain decomposition}
It is shown in \ref{Fig:Strain_decomposition} that the stress-strain curve can be obtained from a uniaxial tensile test.
Both elastic and inelastic regions are indicated.
In elastic region, the stress is proportional to the strain if the stress in the specimen is below a certain value, i.e. the yield stress $\sigma_{y0}$, which is the elastic limit.
As a material is loaded beyond its elastic limit, Hooke's law does not apply, the material no longer exhibits elastic behavior and the stress-strain relation becomes nonlinear. The material yields, begins to flow and residual, permanent deformation results after unloading.

For strain hardening materials the yield stress increases with increasing plastic deformation to a value of $\sigma_y$.
It is called hardening because the stress is increasing relative to perfect plastic behavior, also shown in the figure.
\begin{figure}[!htp]
\centering\scalebox{1.0}{\includegraphics{Rock_plasticity_compression_plain.pdf}}
\caption{The classical decomposition of strain into elastic and plastic parts.}
\label{Fig:Strain_decomposition}
\end{figure}
If, at a strain of $\varepsilon$, the loading were to be reversed, the material would cease to deform plastically (at least in the absence of time-dependent effects) and would show a linearly decreasing stress with strain such that the gradient of this part of the stress-strain curve would again be the Young's modulus, E, shown in \ref{Fig:Strain_decomposition}.
Once a stress of zero is achieved (provided the material remains elastic on full reversal of the load), the strain remaining in the test specimen is the plastic strain, $\varepsilon_p$.
The recovered strain, $\varepsilon_e$, is the elastic strain and it can be seen that the total strain, $\varepsilon$, is the sum of the two
\begin{equation}
\varepsilon = \varepsilon^e + \varepsilon^p
\end{equation}
This is called the classical additive decomposition of strain.
It is also apparent from \ref{Fig:Strain_decomposition} that the stress achieved at a strain of $\varepsilon$ is given by
\begin{equation}
\sigma = E\varepsilon^e = E(\varepsilon - \varepsilon^p)
\end{equation}

\subsection{Yield criterion}
A yield surface is a super surface in the six-dimensional space of stresses.
The yield surface is usually convex and the state of stress of inside the yield surface is elastic.
When the stress state lies on the surface the material is said to have reached its yield point and the material is said to have become plastic.
Further deformation of the material causes the stress state to remain on the yield surface, even though the shape and size of the surface may change as the plastic deformation evolves.

Only the von Mises and Tresca yield criterion is considered here.
There are many others including that of Drucker and the Gurson model for porous materials.
\begin{figure}[!htp]
\centering\scalebox{0.6}{\includegraphics{yield_surface_2D.pdf}}
\caption{The von Mises and Tresca yield surfaces in principal stress coordinates.}
\label{Fig:YieldSurface3D}
\end{figure}

\begin{figure}[!htp]
\centering\scalebox{0.5}{\includegraphics{yield_surface_3D.pdf}}
\caption{The von Mises and Tresca yield surfaces in principal stress coordinates.}
\label{Fig:YieldSurface2D}
\end{figure}

Let $f$ be a yield function such that $f = 0$ is our yield criterion. Then:

1.Yield is independent of the hydrostatic stress.
Since $f$ is independent of hydrostatic stress, it must be expressible in terms of the deviatoric stresses alone.

2.Yield in polycrystalline metals can be taken to be isotropic (provided we are concerned with yield in a volume of material containing many grains) and must therefore be independent of the labelling of the axes.

3.Yield stresses measured in compression have the same magnitude as yield stresses measured in tension.

The von Mises yield function is defined by
\begin{equation}
f = \sigma_e-\sigma_y =\sqrt{3J_2}-\sigma_y
\end{equation}

The von Mises yield criterion suggests that the yielding of materials begins when the second deviatoric stress invariant $J_2$ reaches a critical value.
For this reason, it is sometimes called the $J_2$-plasticity or $J_2$ flow theory.
The yield criterion is given by
\[f<0\rm{: Elastic\ deformation}\]
\[f=0\rm{: Plastic\ deformation}\]

%Let us consider the yield function in two-dimensional principal stress space by putting $\sigma_3 = 0$ and so imposing conditions of plane stress.
%Geometrically, this corresponds to finding the intersection between the von Mises cylinder and the plane $\sigma_3 = 0$.



\subsection{Flow rule}
The general mathematical treatment of the constitutive equation for plastic deformation was proposed by von Mises in 1928.
He noticed that in elasticity theory, the strain tensor was related to the stress through an elastic potential function, the complementary strain energy $U$ such that
\begin{equation}
{\bm{\upepsilon }} = \frac{{\partial U}}{{\partial {\bm{\sigma }}}}
\end{equation}

By generalizing and applying this idea to plasticity theory, Mises proposed that there exists a plastic potential $g(\bm{\sigma})$, such that the plastic strain rate $\dot{{\bm{\upepsilon }}^p}$ could be derived from the following flow rule:
\begin{equation}
\dot{{\bm{\upepsilon }}^p} = \dot\lambda \frac{\partial g}{\partial \bm{\sigma}}
\label{Equ:dgdsigma}
\end{equation}
where $\dot\lambda$ is a proportional positive scalar factor which can be determined from the yield criterion. Plasticity theory based on the above flow rule is called plastic potential theory.
The following remarks should be noted about the above flow rule:

Geometrically, the plastic potential $g(\bm{\sigma})=0$, represents a surface in the stress space and $\dot{{\bm{\upepsilon }}^p}$ can be represented by a vector in this space.
The plastic strain rate vector is normal to $g(\bm{\sigma})=0$.
Therefore, Equation \ref{Equ:dgdsigma} is also referred to as the normality flow rule in plasticity theory.

Common approach in plasticity theory is to assume that the plastic potential function $g(\bm{\sigma})$ is the same as the yield function $f(\bm{\sigma})$.
It is typically assumed that the plastic strain increment and the normal to the yield surface have the same direction, so that the flow rule can be written as:

%In metal plasticity, the assumption that the plastic strain increment and deviatoric stress tensor have the same principal directions is encapsulated in a relation called the flow rule.

\begin{equation}
\dot{{\bm{\upepsilon }}^p} = \dot\lambda \frac{\partial f}{\partial \bm{\sigma}}
\end{equation}
This is called the associated flow rule.
On the other hand, if $g \neq f$, the flow rule is called nonassociated.
In general, experimental observations show that inelastic deformation of metals can be characterized quite well by an associated flow rule, but for some porous materials a nonassociated flow rule provides a better representation of inelastic deformation.

The above flow rule is easily justified for perfectly plastic deformations for which $d\boldsymbol{\sigma} = 0$  when $d{{\bm{\upepsilon }}^p} > 0$, i.e., the yield surface remains constant under increasing plastic deformation. This implies that the increment of elastic strain is also zero, $d{{\bm{\upepsilon }}^e} = 0$, because of Hooke's law.
Therefore,
\begin{equation}
d\boldsymbol{\sigma}:\frac{\partial f}{\partial \boldsymbol{\sigma}} = 0 \quad \text{and} \quad d\boldsymbol{\sigma}:d{{\bm{\upepsilon }}^p} = 0 \,.
\end{equation}
Hence, both the normal to the yield surface and the plastic strain tensor are perpendicular to the stress tensor and must have the same direction.
For a work hardening material, the yield surface can expand with increasing stress.
We assume Drucker's second stability postulate which states that for an infinitesimal stress cycle this plastic work is positive, i.e.,
\begin{equation}
d\boldsymbol{\sigma}: d{{\bm{\upepsilon }}^p} \ge 0 \,.
\end{equation}
The above quantity is equal to zero for purely elastic cycles.
Examination of the work done over a cycle of plastic loading-unloading can be used to justify the validity of the associated flow rule.
The Prager consistency condition is needed to close the set of constitutive equations and to eliminate the unknown parameter $d\lambda$ from the system of equations.
The consistency condition states that $df = 0$  at yield because  $f(\boldsymbol{\sigma},{{\bm{\upepsilon }}^p}) = 0$ , and hence
\begin{equation}
df = \frac{\partial f}{\partial \boldsymbol{\sigma}}:d\boldsymbol{\sigma} + \frac{\partial f}{\partial \boldsymbol{\varepsilon}_p}:d\boldsymbol{\varepsilon}_p = 0 \,.
\end{equation}

\section{Kinematics} 